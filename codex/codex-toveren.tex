\beginsong{Toveren}[by={Herman van Veen}]

\beginchorus
Als hij kon toveren,
Kwam alles voor elkaar.
Als hij kon toveren,
Dan werd geen mens te zwaar
En iederen die zong er.
Als hij kon toveren, kon toveren,
Dan hielden alle meansen van elkaar.
\endchorus

\beginverse
Ieder huis had 100 kamers,
In elke kamer stond tv
En z'n ouders bleven eeuwig leven
En hij leefde met ze mee.
De rivier was niet van water,
Maar van sinaasappelsap
En hij zou niet hoeven leren
Wat hij eigenlijk niet snapt.
\endverse

\beginverse
Z'n vriendje zou ineens begrijpen
Waaro ie ruzie met 'm kreeg
En iedereen zou voor hem buigen
Als hij de troon besteeg
En 's winters lag er altijd sneeuw
En was het lekker warm
En niemand werd er rijk geboren
En niemand werd er arm.
\endverse

\beginverse
Maar voor een toverspreuk van kwaliteit
Ben je zomaar 1000 gulden kwijt
En naar een beetje toverboek
Ben je toch wel 50 jaar op zoek
En de hele cursus tovenaar
Duurt 125 jaar.

Dat brengt ie allemaal niet op.
Ik denk dat hij voor 't begin al stopt,
Want zelfs de oma van z'n oma
Had nooit een tovenaarsdiploma
\endverse
\endsong
