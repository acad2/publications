\beginsong{Juchheidi}[index={Juchheidi}]

\beginverse
De student is vrolijk man,
Juchheidi, juchheida,
Zingt en drinkt zo veel hij kan,
Juchheidi, heida,
Springt en lacht maar altijd voort,
En kent nergens droevig oord.
\endverse

\beginchorus
Juchheidi, heidi, heida
Juchheidi, juchheida,
Juchheidi, heidi, heida
Juchheidi, heida.
\endchorus

\beginverse
Komt hij enen herberg in,
Hij drinkt immer blij van zin
En is't met het geld gedaan,
Nog blijft zijne pret bestaan
\endverse

\beginverse
Er blijft hem zo menig woon,
Waar men bier schenkt zonder loon,
En daarbij nog menig vriend
Die hem graag tot gastheer dient.
\endverse

\beginverse
Daarom zingt hij op de straat,
Blijde zangen vroeg en laat,
Minnend elke schone maagd
Die hem om zijn hartje zaagt.
\endverse

\beginverse
Munich, Hop, Jack-op of wijn,
‘t Kan hem nooit te vele zijn,
Altijd heeft hij honger, dorst,
Wijl hij zingt uit volle borst.
\endverse

\beginverse
En zo leeft hij vrolijk voort,
In het schoon studentenoord,
Tussen boek en pijp en pint,
Waar elk meisje hem bemint.
\endverse

\beginverse
Overal de vlag in top!
Held're ogen, warme kop.
En de strijdzang langs de ree:
"Vliegt de Blauwvoet? Storm op Zee!"
\endverse

\beginverse
Leefden wij nog honderd jaar,
Nooit en rouwde 't onze schaar,
Al ons doen voor 't Vlaamse diet ,
't Gildenleven, gildenlied. 
\endverse

\endsong