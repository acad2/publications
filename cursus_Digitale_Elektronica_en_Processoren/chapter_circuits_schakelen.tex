\chapter{Circuits Schakelen}
\chplab{circuitsschakelen}
\begin{chapterintro}
Tot slot bieden we een laatste gedeelte over het bouwen van digitale schakelingen in de praktijk. Dit gedeelte is optioneel. Het is geen leerstof, staat niet in de presentaties en vorm zeker geen onderdeel tijdens het examen. Het is dan ook eerder tot stand gekomen voor ``enthousiastelingen'' die de opgedane kennis in de praktijk willen omzetten. In dit hoofdstuk beschrijven we welke ge\"integreerde circuits men zich dient aan te schaffen om de schakelingen te implementeren. In de rest van het deel worden enkele projecten besproken die men kan realiseren.
\end{chapterintro}
\section{Bouw van een elektronische schakeling}

\section{Oproep aan de lezers}
De auteur roept enthousiaste lezers op om projecten te delen zodat deze in deze cursus kunnen worden gepubliceerd als een hoofdstuk in dit deel. Men kan echter niet elk project als nuttig beschouwen. Ingediende projecten moeten aan enkele voorwaarden voldoen:
\begin{enumerate}
 \item De componenten in het project dienen in de cursus vermeld te worden. Het is niet de bedoeling om ``exotische componenten'' te introduceren, in het bijzonder denken we dan aan componenten uit de analoge elektronica (operationele versterker, spoel, ...). Sommige projecten kunnen een klein aantal van dit soort componenten bevatten. In dat geval dient men een korte beschrijving van de werking bij te voegen.
 \item Het project moet realiseerbaar zijn. Zowel op een op maat gemaakte printplaat als bijvoorbeeld een europrintplaat. Verder is het evenmin de bedoeling dat het project veel werk vereist en het resultaat weinig inzichten zal verwerken (hierbij denken we bijvoorbeeld aan een 1024-bit opteller).
 \item De effecten die in het project beschreven worden moeten te verklaren zijn, en dit op basis van de cursus.
\end{enumerate}
Indien het project aan deze voorwaarden voldoet maakt het kans om opgenomen te worden. Een lezer kan een project indienen op onderstaand adres: ??. Een ``aanvraag'' bestaat uit \'e\'en of meerdere schema's samen met een verslag. Dit verslag bevat een lijst van benodigde componenten, aanwijzingen bij de bouw van de schakeling en een tekst die de werking verklaart. Het verslag mag figuren bevatten die de werking verder uitleggen. Omdat een dergelijke aanvraag veel werk vraagt, kan men ook een ``voor-aanvraag'' indienen (op hetzelfde webadres). In een voor-aanvraag specificeert men kort het project in een tekst van maximaal een pagina. Op basis van de reactie van de auteur kan men dan beslissen om al dan niet een aanvraag in te dienen.