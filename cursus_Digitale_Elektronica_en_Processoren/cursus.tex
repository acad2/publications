\documentclass[a4paper,10pt,titlepage]{book}
\usepackage[dutch]{babel}
\usepackage{glossaries}
\usepackage{fullpage}
\usepackage{tikz}
\usepackage{index}
\usepackage{subfigure}
\usepackage{graphicx}
\usepackage{framed}
\usepackage{tabularx}
\usepackage{wrapfig}
\usepackage{listings}
\usepackage{multicol}
\usepackage{multicol}
\usepackage{multirow}
\usepackage{stmaryrd}
\usepackage{amsfonts}
\usepackage{amsmath}
\usepackage{bbding}
\usepackage{commusoftScripts}
\usetikzlibrary{circuits.logic.US}
\usetikzlibrary{DEPcomponents}
\usepgflibrary{shapes.geometric}
\usetikzlibrary{fit,calc,positioning,decorations.pathreplacing,matrix}
\usepackage{circuitikz}
\pdfinfo{
  /Title    (Cursus Digitale Elektronica en Processoren (DEP))
  /Author   (Willem Van Onsem)
  %/Creator  ()
  %/Producer ()
  /Subject  (Digitale Elektronica)
  /Keywords (Digitaal, Elektronica, Processoren, Schakelingen, KULeuven)
}
\ctikzset{bipoles/length=1cm}
\lstset{ %
language=VHDL,	                % choose the language of the code
basicstyle=\footnotesize,       % the size of the fonts that are used for the code
numbers=left,                   % where to put the line-numbers
numberstyle=\footnotesize,      % the size of the fonts that are used for the line-numbers
stepnumber=1,                   % the step between two line-numbers. If it's 1 each line 
                                % will be numbered
numbersep=5pt,                  % how far the line-numbers are from the code
backgroundcolor=\color{white},  % choose the background color. You must add \usepackage{color}
showspaces=false,               % show spaces adding particular underscores
showstringspaces=false,         % underline spaces within strings
showtabs=false,                 % show tabs within strings adding particular underscores
frame=tb,                       % adds a frame around the code
tabsize=2,                      % sets default tabsize to 2 spaces
%captionpos=n,                   % sets the caption-position to bottom
breaklines=true,                % sets automatic line breaking
breakatwhitespace=false,        % sets if automatic breaks should only happen at whitespace
title=\lstname,                 % show the filename of files included with \lstinputlisting;
                                % also try caption instead of title
}

\usepackage{minitoc}
\dominitoc
\setcounter{minitocdepth}{3}
\nomtcrule

\makeindex
\makeglossaries

\newenvironment{chapterintro}{\begin{it}\begin{large}}{\end{large}\end{it}}
\newcommand{\gtermen}[2]{\newglossaryentry{#1}{description={#2}}\termen{#1}}
\newcommand{\vhdltermen}[1]{\index{\texttt{VHDL}!\texttt{#1}}\texttt{#1}}
\newcommand{\vhdlref}[1]{\texttt{VHDL}-code \ref{#1}}
\newcommand{\NN}{\mathbb{N}}
\newcommand{\clrsin}{\texttt{Clr$^*$}}
\newcommand{\prsin}{\texttt{Pr$^*$}}
\newcommand{\pdot}[1]{\fill (#1) circle (0.4 mm);}
\newcommand{\stAs}[1]{$|$#1$|$}
\newcommand{\ndAs}[1]{\underline{#1}}
\title{\includegraphics[width=5cm]{sedesVcrop.pdf}\\Cursus:\\Digitale Elektronica \& Processoren\\\texttt{H01L1A}\\\texttt{\small versie 0.2.718}}
\author{Willem M. A. Van Onsem BSc\\\includegraphics[width=2.5cm]{kommusoftEmblema.pdf}}
\date{Katholieke Universiteit Leuven\\Academiejaar 2010-2011}
\begin{document}
\frontmatter
\begin{titlepage}
\maketitle
\end{titlepage}
\tableofcontents
\chapter*{Notities vooraf}
\begin{it}
Deze cursus omvat de volledige inhoud van het onderwerp ``Digitale Elektronica \& Processoren'' gegeven tijdens het academiejaar 2010-2011 door prof. Van Eycken. De cursus is dan ook hoofdzakelijk gebaseerd op de presentaties voor dit onderwerp. Andere bronnen worden vermeld in de referentielijst op pagina \pageref{reference}. Aanbevolen literatuur is \cite{brown2004fundamentals}, \cite{gajski1997principles}, \cite{wakerly2000digital} en \cite{ashenden2008designer}.
\paragraph{}
De cursus wordt uitgegeven onder de \texttt{CopyLeft} licentie, dit betekent dat iedereen de cursus vrij kan aanpassen en herverdelen.
\paragraph{}
De auteur garandeert de juistheid van deze cursus \underline{niet}. Hoewel deze cursus met de nodige zorg is samengesteld, is het niet ondenkbaar dat er fouten in staan. Errata/opmerkingen/suggesties kunnen altijd doorgestuurd worden naar \verb+vanonsem.willem@gmail.com+, deze worden dan in de volgende versie verbeterd.
\paragraph{Over de auteur}
valt eigenlijk niet veel te zeggen, behalve dat hij geen exemplaren signeert.
\paragraph{}
Speciale dank gaat naar (in alfabetische volgorde) ``Amy Winehouse'', ``Harold Budd'', ``Hooverphonic'', ``Moby'', ``Paul Simon'', ``Piknik'', ``Russkij Razmer'' en ``The Beatles'' voor de muziek tijdens de nachten waarin deze cursus tot stand kwam.
\section*{Betreft het examen}
Het examen bestond traditioneel tijdens het academiejaar 2010-2011 uit drie delen:
\begin{enumerate}
 \item Theorie: bestaat uit twee vragen, dit zijn meestal termen. Deze termen kunnen achteraan in de index op pagina \pageref{idx} worden teruggevonden. De meeste ook in de woordenlijst op pagina \pageref{glos}.
 \item Synthese van een datapad en controller: De student krijgt een \texttt{VHDL}-code, en wordt gevraagd naar een equivalent ASM-schema. Vervolgens dient hierbij het datapad en de controller gesynthetiseerd te worden. Soms volstaat het echter om bij de controller een toestandsdiagram weer te geven.
 \item Synthese van een toestandsdiagram: Er wordt een toestandsdiagram gegeven (meestal in tabelvorm). De student dient dit diagram te vertalen in hardware. Meestal worden er extra beperkingen opgelegd, zoals het gebruik van een bepaald type flip-flop, en poorten. Ook dient men vaak de Karnaugh-kaarten weer te geven.
\end{enumerate}
Het examen is volledig gesloten boek. De eerste twee vragen worden mondeling besproken, dit houdt echter in dat wat op papier staat de basis vormt. De laatste vraag is volledig schriftelijk. Enkel het examenmoment geldt als evaluatie. De student mag een referentieblad meebrengen over VHDL. Dit is te koop bij de VTK CuDi.
\section*{Layout en Stijl}
\paragraph{}
Alle pagina's zijn afdrukbaar met een zwart-wit printer. Dit drukt eventuele kosten bij een afdruk van deze cursus. Bovendien verhoogt het de leesbaarheid bij kleurenblinden.
\paragraph{}
Deze cursus wordt ge\"illustreerd met talloze afbeeldingen. Deze zijn allemaal tot stand gekomen met het grafisch pakket TikZ\footnote{TikZ: TikZ ist kein Zeichenprogramm.} samen met zelfgeschreven bibliotheken. Alle afbeeldingen zijn bijgevolg vectorieel. En kunnen dus eindeloos uitvergroot worden met een geschikte .pdf viewer.
\paragraph{}
\termenlayout{Terminologie} wordt in het vetjes en zonder schreven\footnote{De zogenoemde ``pootjes'' die sommige letters krijgen. (Engels: serifs)} gezet. Deze terminologie wordt ook herhaald op het einde van de cursus in de index op pagina \pageref{idx}.
\section*{Kudos}
Kudos gaan uit naar volgende personen/groepen (in alfabetische volgorde):
\begin{description}
 \item[\texttt{Conrad.be}] Leverancier van elektronische onderdelen om de circuits zelf te testen.
 \item[Prof. Christian Maes] Voor onvergetelijke lessen Statistische Thermodynamica over het grootcanonisch ensemble $\Xi\left(T,V,\mu\right)=e^{\beta PV}$.
 \item[Ingmar Dasseville] Voor enkele Haskell scripts in het kader van geautomatiseerde \LaTeX-code generatie.
 \item[Het revisoren-team] bestaande uit: Ingmar Dasseville, Jonas Vanthornhout, Katie Pauwelyn en Steven Roose.
 \item[Personen die errata indienden] (nog niemand)
\end{description}
\section*{Link naar deze cursus}
De meest recente versie van deze cursus is op onderstaande link te vinden:
\begin{figure}[H]
\centering
\begin{tikzpicture}
\draw (0,0) node{\verb+http://www.4shared.com/documents/??+};
\end{tikzpicture}
\caption{Link naar de meest recente versie van deze cursus.}
\end{figure}
Indien er fouten gerapporteerd worden, of toevoegingen gedaan worden zullen na verloop van tijd de aanpassingen daar te vinden zijn.
\paragraph{}Deze cursus is opgedragen aan leden van mijn familie:
\begin{itemize}
 \item \textsc{Louis Van Onsem (1931-1993)}: voor drie jaar peterschap en m'n derde voornaam: \textsc{Agnes}.% en vier maanden
 \item \textsc{Gabri\"ella Simons (1923-2012)}: voor haar humor, fijne herinneringen en 17 jaar plaatsvervangend peterschap.
 \item \textsc{Constant Soetewey (1907-1945)}: voor hoogstaande absurdistische literatuur, die dit jaar opnieuw uitgegeven werd\cite{Kohler} onder het pseudoniem \textsc{Kurt K\"ohler}.
\end{itemize}
\end{it}
\commusoftQuality{1}
\chapter*{Voorbeschouwing}
\mainmatter
%\part{Digitaal Ontwerp en Fysische Limieten}
%\chapter{De Basis van een Digitaal Ontwerp}
\label{ch:basis}
\chplab{basis}
\chapterquote{Eva werd niet als een soort accessoire van Adam geschapen, maar Adam was het eerste ontwerp voor Eva.}{Jeanne Moreau, Frans actrice (1928-)}
\begin{chapterintro}
Zoals de naam van deze cursus reeds doet vermoeden werken we met digitale schakelingen, digitale schakelingen onderscheiden zich van de traditionele analoge schakelingen omdat ze slechts een beperkt aantal waarden kunnen aannemen. In de praktijk werken we bijna uitsluitend met \termen{binaire signalen}. Deze waarden hoeven niet noodzakelijk een verschil in spanning aan te duiden. Sterker nog, het gebruik van elektronica is zelfs niet verplicht. Zo zouden we ook een verschil in stroomsterkte, druk, reflectie,... als parameter kunnen nemen. Het komt er enkel op aan om twee waarden te defini\"eren die we symbolisch zullen voorstellen als 0 en 1 en een reeks operaties te kunnen realiseren voor deze natuurfenomenen. Indien we echter meer dan twee verschillende waarden nodig hebben, kunnen we dit probleem oplossen door een aantal binaire waarden te groeperen. In dit eerste hoofdstuk zullen we een beperkte set aan basisoperaties op deze binaire signalen introduceren: de NOT, AND en OR. We formaliseren deze operaties in de booleaanse algebra. Ook bespreken we een methode om een booleaanse functie te synthetiseren. Tot slot bespreken we het verloop van een digitaal ontwerp en introduceren we een taal om hardware mee te beschrijven: VHDL.
\end{chapterintro}
\minitoc[n]
\section{Logische schakelingen}
\paragraph{}
Het definieren van binaire waarden op zich heeft niet veel nut: we willen met deze binaire waarden bepaalde operaties uitvoeren. Om berekeningen te kunnen uitvoeren maken we gebruik van \termen{logische schakelingen}. Logische schakelingen zijn een reeks operaties op \'e\'en of meer binaire parameters die resulteren in een uitgang. De waarde die op de uitgang staat is hierbij afhankelijk van de ingangen. De set van \gtermen{basisoperaties}{Set van operaties waaruit andere operaties zijn opgebouwd: in de booleaanse algebra zijn dit de NOT, AND en OR.} dient klein, simpel en tegelijk universi\"eel te zijn: elke complexe schakeling moet uit een combinatie van deze basisschakelingen kunnen worden opgebouwd.
\paragraph{}
Als basisschakelingen definieert men meestal de \gtermen{NOT}{Een populaire bassischakeling die $0$ afbeelt op $1$; en $1$ op $0$.}, \gtermen{AND}{Een populaire basisschakeling met twee ingangen. $\tupl{0,0}$, $\tupl{0,1}$ en $\tupl{1,0}$ worden afgebeeld op $0$; en $\tupl{1,1}$ wordt afgebeeld op $1$.} en \gtermen{OR}{Een populaire basisschakeling met twee ingangen. $\tupl{0,0}$ wordt afgebeeld op $0$; en $\tupl{0,1}$, $\tupl{1,0}$ en $\tupl{1,1}$ worden afgebeeld op $1$.} operaties. Deze operaties zijn eenvoudig en gemakkelijk te begrijpen en onthouden. Hoewel booleaanse algebra in heel wat andere cursussen reeds aan bod komt, zullen we er ook in deze cursus enkele secties aan besteden. We zullen de booleaanse algebra behandelen aan de hand van een model met lichtschakelaars.
\subsection{Logische schakelingen in huis}
We beschouwen een schakeling zoals op \figref{switchLight}.
\importtikzfigure{switchLight}{Basis van het lamp-model.}
Zoals we zien zal indien we de schakelaar $x$ indrukken, het lichtje branden. Indien we vervolgens de schakelaar loslaten zien we dat het lampje opnieuw dooft. We formaliseren dit door te schrijven:
\begin{equation}
L=x
\end{equation}
Waarbij we $L$ beschouwen als het al dan niet branden van het lampje, en $x$ aanduidt of de schakelaar al dan niet ingedrukt is. Indien we nu als ingang beschouwen of de schakelaar al dan niet ingedrukt is, en als uitgang of het lichtje al dan niet brandt, kunnen we hiermee functies gaan defini\"eren.
\paragraph{Not}
Indien we de implementatie van de schakelaar aanpassen zijn we in staat om een NOT poort te bouwen. Zoals op \figref{switchLightBasicGatesNot}. Deze schakelaar laat stroom door indien deze niet ingedrukt is. Bijgevolg kunnen we stellen dat het al dan niet branden van het lampje equivalent is met niet de schakelaar indrukken. We formaliseren dit als:
\begin{equation}
L=x'=\neg x=!x=\overline{x}=\mbox{NOT }x
\end{equation}
Zoals we zien zijn er in de loop der tijd nogal wat notaties ingevoerd; wat bovendien eigen is aan de volledige booleaanse algebra. In deze cursus zullen we enkel het accent ($x'$) als negatie gebruiken. Literatuur buiten deze cursus kan echter andere standaarden gebruiken. Een NOT poort wordt vaak ook een \termen{inverter} genoemd.
\paragraph{And}
Soms willen we dat het lampje pas gaat branden indien twee of meer schakelaars allemaal ingedrukt zijn. In dat geval spreken we van een AND. Een AND kunnen we implementeren volgens het lamp-model zoals in \figref{switchLightBasicGatesAnd}. We noteren:
\begin{equation}
L=x\cdot y=x\mbox{ AND }y
\end{equation}
\paragraph{Or}
Een andere basisbewerking is de OR. Hierbij willen we dat het lampje brandt bij minstens \'e\'en ingedrukte schakelaar. Of we noteren:
\begin{equation}
L=x+y=x\mbox{ OR } y
\end{equation}
Een implementatie in het lamp-model is te vinden op \figref{switchLightBasicGatesOr}.
\begin{figure}[htb]
\centering
\subfigure[Not]{\begin{tikzpicture}
\draw[thick] (-0.3,0.05) -- (0.3,0.05);
\draw (-0.15,-0.05) -- (0.15,-0.05);
\draw (0,0.05) -- (0,0.5) -- (0.5,0.5);
\fill (0.5,0.5) circle (0.05);
\fill (1,0.5) circle (0.05);
\draw (1,0.5) -- (1.5,0.5) -- (1.5,0.25);
\fill[black!20] (1.5,0) circle (0.25);
\draw[thick] (1.5,0) circle (0.25);
\draw[thick] (1.677,0.177) -- (1.323,-0.177);
\draw[thick] (1.677,-0.177) -- (1.323,0.177);
\draw (0,-0.05) -- (0,-0.5) -- (1.5,-0.5) -- (1.5,-0.25);
\draw (0.4,0.45) -- (1.1,0.45);
\draw (0.7,0.45) -- (0.7,0.65);
\draw (0.8,0.45) -- (0.8,0.65);
\draw (0.6,0.75) -- (0.75,0.6) -- (0.9,0.75);
\draw (0.75,0.75) node[anchor=south]{$x$};
\begin{scope}[xshift=2.5 cm]
\draw[thick] (-0.3,0.05) -- (0.3,0.05);
\draw (-0.15,-0.05) -- (0.15,-0.05);
\draw (0,0.05) -- (0,0.5) -- (0.5,0.5);
\fill (0.5,0.5) circle (0.05);
\fill (1,0.5) circle (0.05);
\draw (1,0.5) -- (1.5,0.5) -- (1.5,0.25);
\draw[thick] (1.5,0) circle (0.25);
\draw[thick] (1.677,0.177) -- (1.323,-0.177);
\draw[thick] (1.677,-0.177) -- (1.323,0.177);
\draw (0,-0.05) -- (0,-0.5) -- (1.5,-0.5) -- (1.5,-0.25);
\begin{scope}[yshift=-0.15 cm]
\draw (0.4,0.45) -- (1.1,0.45);
\draw (0.7,0.45) -- (0.7,0.65);
\draw (0.8,0.45) -- (0.8,0.65);
\draw (0.6,0.75) -- (0.75,0.6) -- (0.9,0.75);
\draw (0.75,0.75) node[anchor=south]{$x$};
\end{scope}
\end{scope}
\end{tikzpicture}
\figlab{switchLightBasicGatesNot}
}\\
\subfigure[And]{\begin{tikzpicture}
\draw[thick] (-0.3,0.05) -- (0.3,0.05);
\draw (-0.15,-0.05) -- (0.15,-0.05);
\draw (0,0.05) -- (0,0.5) -- (0.5,0.5);
\fill (0.5,0.5) circle (0.05);
\fill (1,0.5) circle (0.05);
\draw (1,0.5) -- (1.5,0.5);
\fill (1.5,0.5) circle (0.05);
\fill (2,0.5) circle (0.05);
\draw (2,0.5) -- (2.5,0.5) -- (2.5,0.25);
\draw[thick] (2.5,0) circle (0.25);
\draw[thick] (2.677,0.177) -- (2.323,-0.177);
\draw[thick] (2.677,-0.177) -- (2.323,0.177);
\draw (0,-0.05) -- (0,-0.5) -- (2.5,-0.5) -- (2.5,-0.25);
\begin{scope}[yshift=0.25 cm]
\draw (0.4,0.45) -- (1.1,0.45);
\draw (0.7,0.45) -- (0.7,0.65);
\draw (0.8,0.45) -- (0.8,0.65);
\draw (0.6,0.75) -- (0.75,0.6) -- (0.9,0.75);
\draw (0.75,0.75) node[anchor=south]{$x$};
\end{scope}
\begin{scope}[xshift=1 cm,yshift=0.25 cm]
\draw (0.4,0.45) -- (1.1,0.45);
\draw (0.7,0.45) -- (0.7,0.65);
\draw (0.8,0.45) -- (0.8,0.65);
\draw (0.6,0.75) -- (0.75,0.6) -- (0.9,0.75);
\draw (0.75,0.75) node[anchor=south]{$y$};
\end{scope}
\begin{scope}[xshift=4 cm]
\draw[thick] (-0.3,0.05) -- (0.3,0.05);
\draw (-0.15,-0.05) -- (0.15,-0.05);
\draw (0,0.05) -- (0,0.5) -- (0.5,0.5);
\fill (0.5,0.5) circle (0.05);
\fill (1,0.5) circle (0.05);
\draw (1,0.5) -- (1.5,0.5);
\fill (1.5,0.5) circle (0.05);
\fill (2,0.5) circle (0.05);
\draw (2,0.5) -- (2.5,0.5) -- (2.5,0.25);
\draw[thick] (2.5,0) circle (0.25);
\draw[thick] (2.677,0.177) -- (2.323,-0.177);
\draw[thick] (2.677,-0.177) -- (2.323,0.177);
\draw (0,-0.05) -- (0,-0.5) -- (2.5,-0.5) -- (2.5,-0.25);
\begin{scope}[yshift=0.25 cm]
\draw (0.4,0.45) -- (1.1,0.45);
\draw (0.7,0.45) -- (0.7,0.65);
\draw (0.8,0.45) -- (0.8,0.65);
\draw (0.6,0.75) -- (0.75,0.6) -- (0.9,0.75);
\draw (0.75,0.75) node[anchor=south]{$x$};
\end{scope}
\begin{scope}[xshift=1 cm,yshift=0.1 cm]
\draw (0.4,0.45) -- (1.1,0.45);
\draw (0.7,0.45) -- (0.7,0.65);
\draw (0.8,0.45) -- (0.8,0.65);
\draw (0.6,0.75) -- (0.75,0.6) -- (0.9,0.75);
\draw (0.75,0.75) node[anchor=south]{$y$};
\end{scope}
\end{scope}
\begin{scope}[xshift=8 cm]
\draw[thick] (-0.3,0.05) -- (0.3,0.05);
\draw (-0.15,-0.05) -- (0.15,-0.05);
\draw (0,0.05) -- (0,0.5) -- (0.5,0.5);
\fill (0.5,0.5) circle (0.05);
\fill (1,0.5) circle (0.05);
\draw (1,0.5) -- (1.5,0.5);
\fill (1.5,0.5) circle (0.05);
\fill (2,0.5) circle (0.05);
\draw (2,0.5) -- (2.5,0.5) -- (2.5,0.25);
\draw[thick] (2.5,0) circle (0.25);
\draw[thick] (2.677,0.177) -- (2.323,-0.177);
\draw[thick] (2.677,-0.177) -- (2.323,0.177);
\draw (0,-0.05) -- (0,-0.5) -- (2.5,-0.5) -- (2.5,-0.25);
\begin{scope}[yshift=0.1 cm]
\draw (0.4,0.45) -- (1.1,0.45);
\draw (0.7,0.45) -- (0.7,0.65);
\draw (0.8,0.45) -- (0.8,0.65);
\draw (0.6,0.75) -- (0.75,0.6) -- (0.9,0.75);
\draw (0.75,0.75) node[anchor=south]{$x$};
\end{scope}
\begin{scope}[xshift=1 cm,yshift=0.25 cm]
\draw (0.4,0.45) -- (1.1,0.45);
\draw (0.7,0.45) -- (0.7,0.65);
\draw (0.8,0.45) -- (0.8,0.65);
\draw (0.6,0.75) -- (0.75,0.6) -- (0.9,0.75);
\draw (0.75,0.75) node[anchor=south]{$y$};
\end{scope}
\end{scope}
\begin{scope}[xshift=12 cm]
\draw[thick] (-0.3,0.05) -- (0.3,0.05);
\draw (-0.15,-0.05) -- (0.15,-0.05);
\draw (0,0.05) -- (0,0.5) -- (0.5,0.5);
\fill (0.5,0.5) circle (0.05);
\fill (1,0.5) circle (0.05);
\draw (1,0.5) -- (1.5,0.5);
\fill (1.5,0.5) circle (0.05);
\fill (2,0.5) circle (0.05);
\draw (2,0.5) -- (2.5,0.5) -- (2.5,0.25);
\fill[black!20] (2.5,0) circle (0.25);
\draw[thick] (2.5,0) circle (0.25);
\draw[thick] (2.677,0.177) -- (2.323,-0.177);
\draw[thick] (2.677,-0.177) -- (2.323,0.177);
\draw (0,-0.05) -- (0,-0.5) -- (2.5,-0.5) -- (2.5,-0.25);
\begin{scope}[yshift=0.1 cm]
\draw (0.4,0.45) -- (1.1,0.45);
\draw (0.7,0.45) -- (0.7,0.65);
\draw (0.8,0.45) -- (0.8,0.65);
\draw (0.6,0.75) -- (0.75,0.6) -- (0.9,0.75);
\draw (0.75,0.75) node[anchor=south]{$x$};
\end{scope}
\begin{scope}[xshift=1 cm,yshift=0.1 cm]
\draw (0.4,0.45) -- (1.1,0.45);
\draw (0.7,0.45) -- (0.7,0.65);
\draw (0.8,0.45) -- (0.8,0.65);
\draw (0.6,0.75) -- (0.75,0.6) -- (0.9,0.75);
\draw (0.75,0.75) node[anchor=south]{$y$};
\end{scope}
\end{scope}
\end{tikzpicture}
\figlab{switchLightBasicGatesAnd}
}\\
\subfigure[Or]{\begin{tikzpicture}
\draw[thick] (-0.3,0.05) -- (0.3,0.05);
\draw (-0.15,-0.05) -- (0.15,-0.05);
\draw (0,0.05) -- (0,0.5) -- (0.5,0.5);
\draw (0,0.5) -- (0,1.5) -- (0.5,1.5);
\fill (0.5,0.5) circle (0.05);
\fill (1,0.5) circle (0.05);
\draw (1,1.5) -- (1.5,1.5) -- (1.5,0.5);
\draw (1,0.5) -- (1.5,0.5) -- (1.5,0.25);
\fill (0.5,1.5) circle (0.05);
\fill (1,1.5) circle (0.05);
\draw[thick] (1.5,0) circle (0.25);
\draw[thick] (1.677,0.177) -- (1.323,-0.177);
\draw[thick] (1.677,-0.177) -- (1.323,0.177);
\draw (0,-0.05) -- (0,-0.5) -- (1.5,-0.5) -- (1.5,-0.25);
\begin{scope}[yshift=0.25 cm]
\draw (0.4,0.45) -- (1.1,0.45);
\draw (0.7,0.45) -- (0.7,0.65);
\draw (0.8,0.45) -- (0.8,0.65);
\draw (0.6,0.75) -- (0.75,0.6) -- (0.9,0.75);
\draw (0.75,0.75) node[anchor=south]{$y$};
\end{scope}
\begin{scope}[yshift=1.25 cm]
\draw (0.4,0.45) -- (1.1,0.45);
\draw (0.7,0.45) -- (0.7,0.65);
\draw (0.8,0.45) -- (0.8,0.65);
\draw (0.6,0.75) -- (0.75,0.6) -- (0.9,0.75);
\draw (0.75,0.75) node[anchor=south]{$x$};
\end{scope}
\begin{scope}[xshift=2.5 cm]
\draw[thick] (-0.3,0.05) -- (0.3,0.05);
\draw (-0.15,-0.05) -- (0.15,-0.05);
\draw (0,0.05) -- (0,0.5) -- (0.5,0.5);
\draw (0,0.5) -- (0,1.5) -- (0.5,1.5);
\fill (0.5,0.5) circle (0.05);
\fill (1,0.5) circle (0.05);
\draw (1,1.5) -- (1.5,1.5) -- (1.5,0.5);
\draw (1,0.5) -- (1.5,0.5) -- (1.5,0.25);
\fill (0.5,1.5) circle (0.05);
\fill (1,1.5) circle (0.05);
\fill[black!20] (1.5,0) circle (0.25);
\draw[thick] (1.5,0) circle (0.25);
\draw[thick] (1.677,0.177) -- (1.323,-0.177);
\draw[thick] (1.677,-0.177) -- (1.323,0.177);
\draw (0,-0.05) -- (0,-0.5) -- (1.5,-0.5) -- (1.5,-0.25);
\begin{scope}[yshift=0.25 cm]
\draw (0.4,0.45) -- (1.1,0.45);
\draw (0.7,0.45) -- (0.7,0.65);
\draw (0.8,0.45) -- (0.8,0.65);
\draw (0.6,0.75) -- (0.75,0.6) -- (0.9,0.75);
\draw (0.75,0.75) node[anchor=south]{$y$};
\end{scope}
\begin{scope}[yshift=1.1 cm]
\draw (0.4,0.45) -- (1.1,0.45);
\draw (0.7,0.45) -- (0.7,0.65);
\draw (0.8,0.45) -- (0.8,0.65);
\draw (0.6,0.75) -- (0.75,0.6) -- (0.9,0.75);
\draw (0.75,0.75) node[anchor=south]{$x$};
\end{scope}
\end{scope}
\begin{scope}[xshift=5 cm]
\draw[thick] (-0.3,0.05) -- (0.3,0.05);
\draw (-0.15,-0.05) -- (0.15,-0.05);
\draw (0,0.05) -- (0,0.5) -- (0.5,0.5);
\draw (0,0.5) -- (0,1.5) -- (0.5,1.5);
\fill (0.5,0.5) circle (0.05);
\fill (1,0.5) circle (0.05);
\draw (1,1.5) -- (1.5,1.5) -- (1.5,0.5);
\draw (1,0.5) -- (1.5,0.5) -- (1.5,0.25);
\fill (0.5,1.5) circle (0.05);
\fill (1,1.5) circle (0.05);
\fill[black!20] (1.5,0) circle (0.25);
\draw[thick] (1.5,0) circle (0.25);
\draw[thick] (1.677,0.177) -- (1.323,-0.177);
\draw[thick] (1.677,-0.177) -- (1.323,0.177);
\draw (0,-0.05) -- (0,-0.5) -- (1.5,-0.5) -- (1.5,-0.25);
\begin{scope}[yshift=0.1 cm]
\draw (0.4,0.45) -- (1.1,0.45);
\draw (0.7,0.45) -- (0.7,0.65);
\draw (0.8,0.45) -- (0.8,0.65);
\draw (0.6,0.75) -- (0.75,0.6) -- (0.9,0.75);
\draw (0.75,0.75) node[anchor=south]{$y$};
\end{scope}
\begin{scope}[yshift=1.25 cm]
\draw (0.4,0.45) -- (1.1,0.45);
\draw (0.7,0.45) -- (0.7,0.65);
\draw (0.8,0.45) -- (0.8,0.65);
\draw (0.6,0.75) -- (0.75,0.6) -- (0.9,0.75);
\draw (0.75,0.75) node[anchor=south]{$x$};
\end{scope}
\end{scope}
\begin{scope}[xshift=7.5 cm]
\draw[thick] (-0.3,0.05) -- (0.3,0.05);
\draw (-0.15,-0.05) -- (0.15,-0.05);
\draw (0,0.05) -- (0,0.5) -- (0.5,0.5);
\draw (0,0.5) -- (0,1.5) -- (0.5,1.5);
\fill (0.5,0.5) circle (0.05);
\fill (1,0.5) circle (0.05);
\draw (1,1.5) -- (1.5,1.5) -- (1.5,0.5);
\draw (1,0.5) -- (1.5,0.5) -- (1.5,0.25);
\fill (0.5,1.5) circle (0.05);
\fill (1,1.5) circle (0.05);
\fill[black!20] (1.5,0) circle (0.25);
\draw[thick] (1.5,0) circle (0.25);
\draw[thick] (1.677,0.177) -- (1.323,-0.177);
\draw[thick] (1.677,-0.177) -- (1.323,0.177);
\draw (0,-0.05) -- (0,-0.5) -- (1.5,-0.5) -- (1.5,-0.25);
\begin{scope}[yshift=0.1 cm]
\draw (0.4,0.45) -- (1.1,0.45);
\draw (0.7,0.45) -- (0.7,0.65);
\draw (0.8,0.45) -- (0.8,0.65);
\draw (0.6,0.75) -- (0.75,0.6) -- (0.9,0.75);
\draw (0.75,0.75) node[anchor=south]{$y$};
\end{scope}
\begin{scope}[yshift=1.1 cm]
\draw (0.4,0.45) -- (1.1,0.45);
\draw (0.7,0.45) -- (0.7,0.65);
\draw (0.8,0.45) -- (0.8,0.65);
\draw (0.6,0.75) -- (0.75,0.6) -- (0.9,0.75);
\draw (0.75,0.75) node[anchor=south]{$x$};
\end{scope}
\end{scope}
\end{tikzpicture}
\figlab{switchLightBasicGatesOr}
}
\caption{Implementatie van de basispoorten volgens het lamp-model.}
\figlab{switchLightBasicGates}
\end{figure}
\paragraph{}
Door deze drie basisbewerkingen met elkaar te combineren kunnen we eindeloos veel nieuwe bewerkingen bouwen. Zoals bijvoorbeeld de exclusieve OR, ook wel de \gtermen{XOR}{Een populaire basisschakeling met twee ingangen. Hierbij worden $\tupl{0,0}$ en $\tupl{1,1}$ afgebeeld op $0$; en $\tupl{0,1}$ en $\tupl{1,0}$ afgebeeld op $1$.} genoemd. De XOR is een bewerking waarbij het lampje gaat branden indien juist \'e\'en van de twee schakelaars ingedrukt is. Deze bewerking kunnen we realiseren door NOT, AND en OR bewerkingen te combineren als volgt:
\begin{equation}
L=x \oplus y=x\mbox{ XOR }y=\left(x\cdot y'\right)+\left(x'\cdot y\right)
\end{equation}
Deze schakelingen kunnen we dan vervolgens in ons lamp-model omzetten zoals op \figref{switchLightXor}.
\begin{figure}[htb]
\centering
\begin{tikzpicture}
\draw[thick] (-0.3,0.05) -- (0.3,0.05);
\draw (-0.15,-0.05) -- (0.15,-0.05);
\draw (0,0.05) -- (0,0.5) -- (0.5,0.5);
\draw (0,0.5) -- (0,1.5) -- (0.5,1.5);
\fill (0.5,0.5) circle (0.05);
\fill (1,0.5) circle (0.05);
\fill (1.5,0.5) circle (0.05);
\fill (2,0.5) circle (0.05);
\draw (1,1.5) -- (1.5,1.5);
\draw (2,1.5) -- (2.5,1.5) -- (2.5,0.5);
\draw (1,0.5) -- (1.5,0.5);
\draw (2,0.5) -- (2.5,0.5) -- (2.5,0.25);
\fill (0.5,1.5) circle (0.05);
\fill (1,1.5) circle (0.05);
\fill (1.5,1.5) circle (0.05);
\fill (2,1.5) circle (0.05);
\draw[thick] (2.5,0) circle (0.25);
\draw[thick] (2.677,0.177) -- (2.323,-0.177);
\draw[thick] (2.677,-0.177) -- (2.323,0.177);
\draw (0,-0.05) -- (0,-0.5) -- (2.5,-0.5) -- (2.5,-0.25);
\begin{scope}[yshift=0 cm]
\draw (0.4,0.45) -- (1.1,0.45);
\draw (0.7,0.45) -- (0.7,0.65);
\draw (0.8,0.45) -- (0.8,0.65);
\draw (0.6,0.75) -- (0.75,0.6) -- (0.9,0.75);
\draw (0.75,0.75) node[anchor=south]{$x$};
\end{scope}
\begin{scope}[xshift=1cm,yshift=0.25 cm]
\draw (0.4,0.45) -- (1.1,0.45);
\draw (0.7,0.45) -- (0.7,0.65);
\draw (0.8,0.45) -- (0.8,0.65);
\draw (0.6,0.75) -- (0.75,0.6) -- (0.9,0.75);
\draw (0.75,0.75) node[anchor=south]{$y$};
\end{scope}
\begin{scope}[yshift=1.25 cm]
\draw (0.4,0.45) -- (1.1,0.45);
\draw (0.7,0.45) -- (0.7,0.65);
\draw (0.8,0.45) -- (0.8,0.65);
\draw (0.6,0.75) -- (0.75,0.6) -- (0.9,0.75);
\draw (0.75,0.75) node[anchor=south]{$x$};
\end{scope}
\begin{scope}[xshift=1cm,yshift=1 cm]
\draw (0.4,0.45) -- (1.1,0.45);
\draw (0.7,0.45) -- (0.7,0.65);
\draw (0.8,0.45) -- (0.8,0.65);
\draw (0.6,0.75) -- (0.75,0.6) -- (0.9,0.75);
\draw (0.75,0.75) node[anchor=south]{$y$};
\end{scope}
\end{tikzpicture}
\caption{XOR-poort in het lamp-model.}
\figlab{switchLightXor}
\end{figure}
\subsection{Waarheidstabellen}
We kunnen alle schakelingen voorstellen met het lamp-model. Toch is het niet echt praktisch, we gaan dus op zoek naar andere manieren om de logische formules te berekenen, en ook eenvoudig voor te stellen. Een makkelijke manier om logische schakelingen te berekenen is met behulp van \termen{waarheidstabellen}. Een waarheidstabel is een tabel waarbij we alle variabelen voorstellen, en vervolgens met eventuele tussenstappen de uiteindelijke bewerking berekenen. Hierbij maken we gebruik van waarheidstabellen die we reeds kennen: de waarheidstabellen van de basisfuncties.

\paragraph{}
Indien we $n$ variabelen beschouwen betekent dit dat onze waarheidstabel $2^n$ rijen telt. Immers kan elke variabele ofwel $0$ ofwel $1$ zijn. Bovendien is het aantal functies met $n$ variabelen beperkt tot $2^{2^n}$. Dit zegt echter niet over het aantal mogelijke implementaties: deze is onbegrensd. In \tblref{truthTablesBasicGates} geven we de waarheidstabellen van de basisoperaties weer.
\begin{table}[htb]
\centering
\subtable[Not]{
\begin{tabular}{c|c}
$x$&$x'$\\\hline
0&1\\
1&0
\end{tabular}}
\subtable[And]{
\begin{tabular}{cc|c}
$x$&$y$&$x\cdot y$\\\hline
0&0&0\\
0&1&0\\
1&0&0\\
1&1&1
\end{tabular}}
\subtable[Or]{
\begin{tabular}{cc|c}
$x$&$y$&$x+y$\\\hline
0&0&0\\
0&1&1\\
1&0&1\\
1&1&1
\end{tabular}}
\caption{Waarheidstabellen van de basisoperaties.}
\label{tbl:truthTablesBasicGates}
\end{table}
\paragraph{}
We kunnen vervolgens aan de hand van deze waarheidstabellen de werking van een XOR-bewerking bestuderen. We dienen eenvoudigweg basisoperaties toe te passen op deelresultaten om zo uiteindelijk het finale gedrag van de bewerking te kennen zoals in \tblref{truthTableXOR}. Indien twee implementaties voor iedere rij dezelfde uitvoer genereren, zijn de implementaties equivalent, en beschrijven ze dezelfde functie. Equivalente implementaties zijn nuttig om een schakeling effici\"enter te maken. We gaan immers altijd op zoek naar equivalente implementaties die minder kosten of sneller werken.
\begin{table}[htb]
\centering
\begin{tabular}{cc|ccccc|c}
$x$&$y$&$x'$&$y'$&$x'\cdot y$&$x\cdot y'$&$x'\cdot y+x\cdot y'$&$x\oplus y$\\\hline
0&0&1&1&0&0&0&0\\
0&1&1&0&1&0&1&1\\
1&0&0&1&0&1&1&1\\
1&1&0&0&0&0&0&0\\
\end{tabular}
\caption{Waarheidstabel voor de implementatie van een XOR.}
\label{tbl:truthTableXOR}
\end{table}
\subsection{Logische poorten}
\label{ss:logischePoorten}
\ssclab{logischePoorten}
Naast het uitrekenen van operaties heeft ons lamp-model nog een nadeel. Het is erg onpraktisch om grote en complexe schakelingen voor te stellen. Een algemeen geaccepteerde notatie is deze met behulp van \termen{logische poorten}. Poorten zijn kleine componenten die enkele ingangen bevatten, en \'e\'en uitgang. Op figuren \ref{fig:basicGatesNot} tot en met \ref{fig:basicGatesOr} geven we de poorten van de basisbewerkingen weer. We kunnen alle mogelijke schakelingen bouwen met deze poorten. Toch worden vaak ook alternatieve poorten gedefinieerd om veelgebruikte bewerkingen mee toe te passen.

\paragraph{}
Bovendien kunnen we de werking van de basis poorten ook veralgemenen naar meer ingangen. Zo defini\"eren we een \gtermen{$n$-and}{Een basisschakeling met $n$ ingangen. $\tupl{1,1,\ldots,1}$ wordt afgebeeld op $1$; en alle overige configuraties op $0$.} als een poort waar enkel een $1$ op de uitgang verschijnt indien op alle $n$ ingangen een $1$ staat. \figref{basicGatesAndExtended} toont een $3$-and. Een \gtermen{$n$-or}{Een basisschakeling met $n$ ingangen. $\tupl{0,0,\ldots,0}$ wordt afgebeeld op $0$; alle overige configuraties op $1$.} is een poort waar enkel een $0$ op de uitgang verschijnt indien op alle $n$ ingangen een $0$ staat. Zo staat op \figref{basicGatesOrExtended} een $5$-or.
\begin{figure}[htb]
\centering
\subfigure[Not]{\begin{tikzpicture}[circuit logic US]
  \node[not gate] (A) {};
  \draw (A.input -| -1,0) node[anchor=east]{$x$} -- (A.input);
  \draw (A.output) -- ++(1,0) node[anchor=west]{$L$};
\end{tikzpicture}
\figlab{basicGatesNot}
}
\subfigure[And]{\begin{tikzpicture}[circuit logic US]
  \node[and gate] (A) {};
  \draw (A.input 1 -| -1,0) node[anchor=east]{$x$} -- (A.input 1)
        (A.input 2 -| -1,0) node[anchor=east]{$y$} -- (A.input 2);
  \draw (A.output) -- ++(1,0) node[anchor=west]{$L$};
\end{tikzpicture}
\figlab{basicGatesAnd}
}
\subfigure[Or]{\begin{tikzpicture}[circuit logic US]
  \node[or gate] (A) {};
  \draw (A.input 1 -| -1,0) node[anchor=east]{$x$} -- (A.input 1)
        (A.input 2 -| -1,0) node[anchor=east]{$y$} -- (A.input 2);
  \draw (A.output) -- ++(1,0) node[anchor=west]{$L$};
\end{tikzpicture}
\figlab{basicGatesOr}
}
\subfigure[3-and]{\begin{tikzpicture}[circuit logic US]
  \node[and gate,inputs={normal,normal,normal}] (A) {};
  \draw (A.input 1 -| -0.75,0) -- (A.input 1)
        (A.input 2 -| -0.75,0) -- (A.input 2)
	(A.input 3 -| -0.75,0) -- (A.input 3);
  \draw (A.output) -- ++(0.5,0);
\end{tikzpicture}
\figlab{basicGatesAndExtended}
}
\subfigure[5-or]{\begin{tikzpicture}[circuit logic US]
  \node[or gate,inputs={normal,normal,normal,normal,normal}] (A) {};
  \draw (A.input 1 -| -0.75,0) -- (A.input 1)
        (A.input 2 -| -0.75,0) -- (A.input 2)
	(A.input 3 -| -0.75,0) -- (A.input 3)
	(A.input 4 -| -0.75,0) -- (A.input 4)
	(A.input 5 -| -0.75,0) -- (A.input 5);
  \draw (A.output) -- ++(0.5,0);
\end{tikzpicture}
\figlab{basicGatesOrExtended}
}
\caption{Basispoorten en uitbreidingen.}
\figlab{basicGates}
\end{figure}
\paragraph{Complexe poorten}
Complexe poorten die in de loop der tijd een eigen symbool kregen zijn onder meer de \gtermen{NOR}{Een basisschakeling met $2$ ingangen. De configuratie $\tupl{0,0}$ wordt afgebeeld op $1$; de overige configuraties op $0$}, \gtermen{NAND}{Een basisschakeling met $2$ ingangen. De configuratie $\tupl{1,1}$ wordt afgebeeld op $0$; de overige configuraties worden afgebeeld op $1$.} en XOR, deze staan afgebeeld op \figref{complexGates}, samen met een equivalent schema dat uitsluitend uit NOT, AND en OR poorten.
\begin{figure}[htb]
\centering
\subfigure[NAND]{\begin{tikzpicture}[circuit logic US]
  \node[nand gate] (A) at (0.5,0) {};
  \draw (A.input 1 -| -1,0) node[anchor=east]{$x$} -- (A.input 1)
        (A.input 2 -| -1,0) node[anchor=east]{$y$} -- (A.input 2);
  \draw (A.output) -- (2.5,0) node[anchor=west]{$L$};

  \node[and gate] (A2) at (0,1) {};
  \node[not gate] (A3) at (1,1) {};
  \draw (A2.input 1 -| -1,0) node[anchor=east]{$x$} -- (A2.input 1)
        (A2.input 2 -| -1,0) node[anchor=east]{$y$} -- (A2.input 2);
  \draw (A2.output) -- (A3.input);
  \draw (A3.output) -- ++(1,0) node[anchor=west]{$L$};
\end{tikzpicture}
\figlab{complexGatesNand}
}
\subfigure[NOR]{\begin{tikzpicture}[circuit logic US]
  \node[nor gate] (A) at (0.5,0) {};
  \draw (A.input 1 -| -1,0) node[anchor=east]{$x$} -- (A.input 1)
        (A.input 2 -| -1,0) node[anchor=east]{$y$} -- (A.input 2);
  \draw (A.output) -- (2.5,0) node[anchor=west]{$L$};
  \node[or gate] (A2) at (0,1) {};
  \node[not gate] (A3) at (1,1) {};
  \draw (A2.input 1 -| -1,0) node[anchor=east]{$x$} -- (A2.input 1)
        (A2.input 2 -| -1,0) node[anchor=east]{$y$} -- (A2.input 2);
  \draw (A2.output) -- (A3.input);
  \draw (A3.output) -- ++(1,0) node[anchor=west]{$L$};
\end{tikzpicture}
\figlab{complexGatesNor}
}
\subfigure[XOR]{\begin{tikzpicture}[circuit logic US]
  \node[xor gate] (A) at (0,-0.5) {};
  \draw (A.input 1 -| -2.5,0) node[anchor=east]{$x$} -- (A.input 1)
        (A.input 2 -| -2.5,0) node[anchor=east]{$y$} -- (A.input 2);
  \draw (A.output) -- ++(2,0) node[anchor=west]{$L$};
  \node[not gate] (A2) at (-1.5,0.5) {};
  \draw (A2.input -| -2.5,0) node[anchor=east]{$y$} -- (A2.input);
  \node[not gate] (A3) at (-1.5,2.5) {};
  \draw (A3.input -| -2.5,0) node[anchor=east]{$x$} -- (A3.input);
  \node[and gate] (A4) at (0,1) {};
  \draw (A2.output) -- ++(0.375,0) |- (A4.input 2);
  \draw (A3.input -| -1.875,0) |- (A4.input 1);
  \node[and gate] (A5) at (0,2) {};
  \draw (A3.output) -- ++(0.375,0) |- (A5.input 1);
  \draw (A2.input -| -2.125,0) |- (A5.input 2);
  \node[or gate] (A6) at (1.5,1.5) {};
  \draw (A4.output) -- ++(0.5,0) |- (A6.input 2);
  \draw (A5.output) -- ++(0.5,0) |- (A6.input 1);
  \draw (A6.output) -- ++(0.5,0) node[anchor=west]{$L$};
\end{tikzpicture}
\figlab{complexGatesXor}
}
\caption{Complexe poorten.}
\figlab{complexGates}
\end{figure}
De waarheidstabellen van deze complexe poorten staan in subsectie \sscref{appendixComplexePoorten}.
\paragraph{Universi\"ele poorten}
De reden dat NAND en NOR poorten populair zijn komt hoofdzakelijk omdat het \gtermen[universiele poorten]{universi\"ele poorten}{Een poort waarmee men elke functie kan bouwen. In de booleaanse logica komt dit er op neer dat men een NOT en AND moet kunnen bouwen. Voorbeelden van universi\"ele poorten zijn de NAND en NOR.} zijn. Dat betekent dat iedere basispoort kan ge\"implementeerd worden met behulp van NAND of NOR poorten. In \tblref{nandNorUniversal} staan deze implementaties. Bovendien is het realiseren van NAND en NOR poorten in de meeste technologie\"en goedkoper dan het bouwen van AND en OR poorten.
\begin{table}[htb]
\centering
\begin{tabular}{c|c|c|c}
&NOT&AND&OR\\\hline
met NAND&
\begin{tikzpicture}[circuit logic US]
  \node[anchor=east] (I) at (-1,0) {$x$};
  \node[nand gate] (A) at (0,0) {};
  \draw (-1,0) -- (-0.75,0)  |- (A.input 1);
  \draw (-0.75,0)  |- (A.input 2);
  \draw (A.output) -- (0.75,0) node[anchor=west]{$L$};
\end{tikzpicture}
&
\begin{tikzpicture}[circuit logic US]
  %\node[anchor=east] (I) at (-1,0) {$x$};
  \node[nand gate] (A1) at (-1.5,0) {};
  \node[nand gate] (A2) at (0,0) {};
  \draw (A1.output) -- (-0.75,0)  |- (A2.input 1);
  \draw (-0.75,0)  |- (A2.input 2);
  \draw (A1.input 1 -| -2.25,0) node[anchor=east]{$x$} -- (A1.input 1)
        (A1.input 2 -| -2.25,0) node[anchor=east]{$y$} -- (A1.input 2);
  \draw (A2.output) -- (0.75,0) node[anchor=west]{$L$};
\end{tikzpicture}
&
\begin{tikzpicture}[circuit logic US]
  \node[anchor=east] (I) at (-1,0.5) {$x$};
  \node[nand gate] (A) at (0,0.5) {};
  \draw (-1,0.5) -- (-0.75,0.5)  |- (A.input 1);
  \draw (-0.75,0.5)  |- (A.input 2);
  \node[anchor=east] (I2) at (-1,-0.5) {$y$};
  \node[nand gate] (A2) at (0,-0.5) {};
  \draw (-1,-0.5) -- (-0.75,-0.5)  |- (A2.input 1);
  \draw (-0.75,-0.5)  |- (A2.input 2);
  \node[nand gate] (A3) at (1.5,0) {};
  \draw (A3.output) -- (2.25,0) node[anchor=west]{$L$};
  \draw (A.output) -- ++(0.25,0) |- (A3.input 1);
  \draw (A2.output) -- ++(0.25,0) |- (A3.input 2);
\end{tikzpicture}
\\\hline
met NOR&
\begin{tikzpicture}[circuit logic US]
  \node[anchor=east] (I) at (-1,0) {$x$};
  \node[nor gate] (A) at (0,0) {};
  \draw (-1,0) -- (-0.75,0)  |- (A.input 1);
  \draw (-0.75,0)  |- (A.input 2);
  \draw (A.output) -- (0.75,0) node[anchor=west]{$L$};
\end{tikzpicture}
&
\begin{tikzpicture}[circuit logic US]
  \node[anchor=east] (I) at (-1,0.5) {$x$};
  \node[nor gate] (A) at (0,0.5) {};
  \draw (-1,0.5) -- (-0.75,0.5)  |- (A.input 1);
  \draw (-0.75,0.5)  |- (A.input 2);
  \node[anchor=east] (I2) at (-1,-0.5) {$y$};
  \node[nor gate] (A2) at (0,-0.5) {};
  \draw (-1,-0.5) -- (-0.75,-0.5)  |- (A2.input 1);
  \draw (-0.75,-0.5)  |- (A2.input 2);
  \node[nor gate] (A3) at (1.5,0) {};
  \draw (A3.output) -- (2.25,0) node[anchor=west]{$L$};
  \draw (A.output) -- ++(0.25,0) |- (A3.input 1);
  \draw (A2.output) -- ++(0.25,0) |- (A3.input 2);
\end{tikzpicture}
&
\begin{tikzpicture}[circuit logic US]
  %\node[anchor=east] (I) at (-1,0) {$x$};
  \node[nor gate] (A1) at (-1.5,0) {};
  \node[nor gate] (A2) at (0,0) {};
  \draw (A1.output) -- (-0.75,0)  |- (A2.input 1);
  \draw (-0.75,0)  |- (A2.input 2);
  \draw (A1.input 1 -| -2.25,0) node[anchor=east]{$x$} -- (A1.input 1)
        (A1.input 2 -| -2.25,0) node[anchor=east]{$y$} -- (A1.input 2);
  \draw (A2.output) -- (0.75,0) node[anchor=west]{$L$};
\end{tikzpicture}
\end{tabular}
\caption{Implementatie van de basispoorten met behulp van NAND en NOR poorten.}
\label{tbl:nandNorUniversal}
\end{table}
\paragraph{Ge\"inverteerde ingangen}
Tot slot introduceren we nog een andere conventie die vaak gebruikt wordt: \gtermen[geinverteerde ingangen]{ge\"inverteerde ingangen}{Een conventie die wordt gebruikt om het aantal getekende NOT poorten in een schema drastisch te verminderen. Men zet hierbij cirkels  bij de ingangen van een poort. Deze cirkels duiden dan op een invertering (en dus eventueel een NOT poort) voor deze ingang.}. NOT poorten worden heel vaak gebruikt, ook bij ingangen van andere poorten. Het symbool van de NOT poort neemt nogal wat plaats in op een schema. Daardoor is het de gewoonte om soms cirkels te tekenen aan de ingangen van een bepaalde poort: ge\"inverteerde ingangen. Deze cirkels stellen een NOT poort voor. Concrete voorbeelden staan op \figref{basicGatesInvertedInput}. In het algemeen kunnen we dus zeggen dat een cirkel duidt op het inverteren. Inverteren in het schema betekent echter niet noodzakelijk dat we bij de fysische implementatie gebruik moeten maken van een inverter (meestal is het zelfs omgekeerd): men kan vaak de implementatie van de poort zo aanpassen dat er geen extra transistoren nodig zijn voor deze transformatie.
\begin{figure}[htb]
\centering
\subfigure{\begin{tikzpicture}[circuit logic US]
  \node[and gate,inputs={inverted,inverted}] (A) {};
  \draw (A.input 1 -| -0.75,0) -- (A.input 1)
        (A.input 2 -| -0.75,0) -- (A.input 2);
  \draw (A.output) -- ++(0.5,0);
\end{tikzpicture}
}
\subfigure{\begin{tikzpicture}[circuit logic US]
  \node[or gate, inputs={normal,inverted}] (A) {};
  \draw (A.input 1 -| -0.75,0) -- (A.input 1)
        (A.input 2 -| -0.75,0) -- (A.input 2);
  \draw (A.output) -- ++(0.5,0);
\end{tikzpicture}
}
\subfigure{\begin{tikzpicture}[circuit logic US]
  \node[and gate,inputs={normal,normal,inverted}] (A) {};
  \draw (A.input 1 -| -0.75,0) -- (A.input 1)
        (A.input 2 -| -0.75,0) -- (A.input 2)
	(A.input 3 -| -0.75,0) -- (A.input 3);
  \draw (A.output) -- ++(0.5,0);
\end{tikzpicture}
}
\subfigure{\begin{tikzpicture}[circuit logic US]
  \node[or gate,inputs={normal,inverted,inverted,normal,normal}] (A) {};
  \draw (A.input 1 -| -0.75,0) -- (A.input 1)
        (A.input 2 -| -0.75,0) -- (A.input 2)
	(A.input 3 -| -0.75,0) -- (A.input 3)
	(A.input 4 -| -0.75,0) -- (A.input 4)
	(A.input 5 -| -0.75,0) -- (A.input 5);
  \draw (A.output) -- ++(0.5,0);
\end{tikzpicture}
}
\caption{Poorten met ge\"inverteerde ingangen.}
\figlab{basicGatesInvertedInput}
\end{figure}
\subsection{Logische schakelingen}
De vorige subsectie toonde al dat we met deze poorten netwerken kunnen bouwen. Deze netwerken worden \gtermen{logische schakelingen}{Een netwerk van logische poorten.} genoemd. Deze schakelingen implementeren dan uiteindelijk de functionaliteiten waarvoor we een digitaal circuit ontwerpen. We kunnen logische schakelingen beschrijven door middel van een schema zoals we dat tot nu toe altijd gedaan hebben. Een andere techniek is echter op basis van een taal: VHDL\footnote{VHDL: VHSIC Hardware Design Language; VHSIC: Very High Speed Integrated Circuit.}. Deze taal komt onder meer aan bod in sectie \secref{vhdl}, en verder in de verdere hoofdstukken van deze cursus.
\paragraph{Optimaliseren}
Een belangrijk probleem met schakelingen is het vinden van een optimale implementatie voor een bepaalde functie. Sommige schakelingen zijn immers equivalent. Bijgevolg zoeken we voor een probleem uit de set van equivalente schakelingen naar de schakeling die ons het meeste voordeel oplevert. Hiervoor zijn enkele parameters belangrijk: kostprijs en verwerkingskracht. We proberen de kostprijs immers te minimaliseren. De kostprijs is eenvoudig te berekenen met volgende formule:
\begin{equation}
\mbox{Kostprijs}=\mbox{\#poortingangen}+\mbox{\#poortuitgangen}-\mbox{\#inverters}
\label{eqn:kosten}
\end{equation}
Merk op dat deze formule geen eenheid heeft. Het is dan ook niet de bedoeling de kostprijs in bijvoorbeeld euro te berekenen. Het is eerder bedoeld als een ruwe metriek om verschillende schakelingen met elkaar te kunnen vergelijken.
\subparagraph{}
De verwerkingskracht wordt bepaald door het concept van de zwakste schakel. Ketens waarbij de uitgangen van poorten weer nieuwe ingangen aansturen zijn immers nefast. We streven dus naar circuits waarbij een signaal slechts door een beperkt aantal poorten heen moet. De lengte van het langste pad is echter niet makkelijk te berekenen. Immers hangt de vertraging van een poort af van onder meer het type en het aantal ingangen. Het totaal aantal ingangen daarentegen geeft in de meeste gevallen een goede ruwe schatting. We formaliseren tot:
\begin{equation}
\mbox{Minimale vertraging}\propto\mbox{\#poortingangen}%=?
\end{equation}
\paragraph{Tijdsgedrag}In de vorige paragraaf hadden we het reeds over performantie. Snelle systemen worden echter vaak beperkt door het tijdsgedrag van logische schakelingen. Wiskundig gezien levert een verandering aan de ingang immers altijd onmiddellijk een correct signaal aan de uitgang. Maar zoals we reeds gesteld hebben, heeft een poort een zekere tijd nodig om de verandering aan de ingang door te rekenen. Dit lijkt slechts een detail. Een nadelig effect hiervan is echter dat er overgangsverschijnselen kunnen ontstaan. In de eerste plaats omdat een waarde afhangt van twee ketens van poorten. En het veranderde ingangssignaal propageert zich sneller door de ene dan door de andere keten. Aan de andere kant ook omdat men nu eenmaal niet iedere poort uniform kan aanmaken. Tussen twee (dezelfde) poorten kunnen toch kleine tijdsverschillen optreden. In dat geval spreken we van een \termen{glitch}. \figref{timeBehaviorGlitchExample} toont een scenario van veranderende signalen in een logische schakeling met een glitch.
\begin{figure}[htb]
\centering
\begin{tikzpicture}[circuit logic US]
  \node[not gate] (N) at (0,0.5) {};
  \node[and gate] (A) at (0,-0.5) {};
  \node[or gate] (O) at (2,0) {};
  \draw (-1.5,0.5) node[anchor=east]{$x$} -- (N.input);
  \draw (A.input 2 -| -1.5,0) node[anchor=east]{$y$} -- (A.input 2);
  \draw (-1,0.5) |- (A.input 1);
  \draw (N.output) -- (1,0.5) node[anchor=south east]{$a$} |- (O.input 1);
  \draw (A.output) -- (1,-0.5) node[anchor=north east]{$b$} |- (O.input 2);
  \draw (O.output) -- ++(0.5,0) node[anchor=west]{$f$};
  \begin{scope}[xshift=5 cm]
  \draw[thick,->] (0,2) -- (0,-2) -- (6,-2) node[anchor=south east]{$t$};
  \foreach\y in {0,1,...,3} {
    \draw[gray] (-1,-1.2+0.8*\y) -- (6,-1.2+0.8*\y);
    \draw[dashed] (1+\y,-2) -- (1+\y,2);
  }
  \foreach\y/\ty in {0/f,1/b,2/a,3/y,4/x} {
    \draw (-0.5,-1.6+0.8*\y) node[anchor=east]{$\ty$};
    \draw (0,-1.8+0.8*\y) node[anchor=east]{0};
    \draw (0,-1.4+0.8*\y) node[anchor=east]{1};
  }
  \begin{scope}[yshift=1.4 cm,yscale=0.4]
    \draw (0,0) -- (1,0) -- (1,1) -- (3,1) -- (3,0) -- (6,0);
  \end{scope}
  \begin{scope}[yshift=0.6 cm,yscale=0.4]
    \draw (0,0) -- (2,0) -- (2,1) -- (4,1) -- (4,0) -- (6,0);
  \end{scope}
  \begin{scope}[yshift=-0.2 cm,yscale=0.4]%not is lagging 0.2
    \draw (0,1) -- (1.2,1) -- (1.2,0) -- (3.2,0) -- (3.2,1) -- (6,1);
  \end{scope}
  \begin{scope}[yshift=-1 cm,yscale=0.4]%and is lagging 0.1
    \draw (0,0) -- (2.1,0) -- (2.1,1) -- (3.1,1) -- (3.1,0) -- (6,0);
  \end{scope}
  \begin{scope}[yshift=-1.8 cm,yscale=0.4]%or is lagging 0.15
    \draw (0,1) -- (1.35,1) -- (1.35,0) -- (2.25,0) -- (2.25,1) -- (3.25,1) -- (3.25,0) -- (3.35,0) node[anchor=south west,yshift=-0.1 cm]{``glitch''} -- (3.35,1) -- (6,1);
  \end{scope}
  \end{scope}
\end{tikzpicture}
\caption{Voorbeeld van het tijdsgedrag van een logische schakeling met een ``glitch''.}
\figlab{timeBehaviorGlitchExample}
\end{figure}
\section{Booleaanse algebra}
\label{s:booleaanseAlgebra}
De tak van de wiskunde die zich bezighoudt met logische bewerkingen is de \termen{booleaanse algebra}. Deze algebra rekent uitsluitend met de twee logische waarden $\left\{0,1\right\}$ en drie logische operatoren: de ons reeds bekende NOT ($'$), AND ($\cdot$) en OR ($+$). Deze operaties hebben elk een specifieke prioriteit zodat men met een minimum aan haakjes toch een complexe expressie kan beschrijven. Zo heeft de NOT prioriteit op de AND die prioriteit heeft op de OR.\footnote{Een ezelsbruggetje om dit te onthouden is het woord NANO: \underline{N}ot \underline{AN}d \underline{O}r.}
\subsection{Theorema's en eigenschappen}
\label{ss:theoremasPropertiesBooleanAlgebra}
De booleaanse algebra maakt gebruik van theorema's om de operatoren te evalueren. Zo worden de resultaten van logische operatoren zonder variabelen gedefinieerd zoals we dat ook met een waarheidstabel kunnen doen. Deze definities staan in \tblref{booleanAlgebraNone}.
\begin{table}[htb]
\centering
\begin{tabular}{ll}
$0\cdot0=0$&$1+1=1$\\
$1\cdot1=1$&$0+0=0$\\
$0\cdot1=1\cdot0=0$&$1+0=0+1=1$\\
$0'=1$&$1'=0$
\end{tabular}
\caption{Booleaanse algebra zonder variabelen.}
\label{tbl:booleanAlgebraNone}
\end{table}
In de booleaanse algebra wordt ook met variabelen gerekend. Op die manier kan men vaak expressies manipuleren en optimaliseren. Indien we slechts \'e\'en variabele beschouwen gelden regels weergegeven in \tblref{booleanAlgebraSingle}.
\begin{table}[htb]
\centering
\begin{tabular}{ll}
$x\cdot 0=0$&$x+1=1$\\
$x\cdot 1=x$&$x+0=x$\\
$x\cdot x=x$&$x+x=x$\\
$x\cdot x'=0$&$x+x'=1$\\
$(x')'=x$&\\
\end{tabular}
\caption{Booleaanse algebra met \'e\'en variabele.}
\label{tbl:booleanAlgebraSingle}
\end{table}
Tot slot werden ook regels gedefinieerd voor meerdere variabelen. Deze regels kunnen onder meer het aantal variabelen reduceren evenals het aantal operatoren, of tot alternatieve implementaties leiden die mogelijk sneller werken. Een opsomming van deze wetten staan in \tblref{booleanAlgebraMultiple}.
\begin{table}[htb]
\centering
\begin{tabular}{ll}
\multicolumn{2}{c}{\termen{Commutativiteit}}\\
$x\cdot y=y\cdot x$&$x+y=y+x$\\
\multicolumn{2}{c}{\termen{Associativiteit}}\\
$x\cdot (y\cdot z)=(x\cdot y)\cdot z$&$x+(y+z)=(x+y)+z$\\
\multicolumn{2}{c}{\termen{Distributiviteit}}\\
$x\cdot (y+z)=x\cdot y+x\cdot z$&$x+(y\cdot z)=(x+y)\cdot (x+z)$\\
\multicolumn{2}{c}{\termen{Absorptie}}\\
$x+x\cdot y=x$&$x\cdot(x+y)=x$\\
\multicolumn{2}{c}{\termen{Wet van De Morgan}}\\
$(x\cdot y)'=x'+y'$&$(x+y)'=x'\cdot y'$
\end{tabular}
\caption{Booleaanse algebra met meerdere variabelen.}
\label{tbl:booleanAlgebraMultiple}
\end{table}
\paragraph{Dualiteit}Een opmerkelijke eigenschap bij booleaanse algebra is de dualiteit. Indien we bij een van de wetten uit \tblref{booleanAlgebraNone}, \tblref{booleanAlgebraSingle} of \tblref{booleanAlgebraMultiple} de OR operaties door AND operaties vervangen en vice-versa, en de 1 door 0 vervangen en vice-versa, bekomen we eenvoudigweg een andere wet uit deze tabellen. Deze eigenschap noemen we de \termen{dualiteit} van de booleaanse algebra. Bovendien geldt ook dat slechts de helft van de gestelde wetten vereist is. Het andere deel kan met de wetten van De Morgan afgeleid worden.
\paragraph{Verschil met de gewone algebra}Omdat ook de booleaanse algebra een optelling en vermenigvuldiging lijkt te defini\"eren en er bovendien ook analogie\"en te trekken zijn, lijkt het soms dat booleaanse algebra niet verschilt van de standaard algebra. Toch zijn er enkele opmerkelijke verschillen. Zo bestaat er geen verschil of deling in de booleaanse algebra. Verder voldoet in de booleaanse algebra volgend stelsel voor iedere $y$:
\begin{equation}
\left\{\begin{array}{l}
y+y'=1\\
y\cdot y=y
\end{array}\right.
\end{equation}
Terwijl in de standaard algebra er geen enkele $y$ is waarvoor dit geldt. Tot slot is bij standaard algebra de optelling ook niet distributief tegenover de vermenigvuldiging, zo is $5+2\cdot4\neq(5+2)\cdot(5+4)$.
\section{Synthese met logische poorten}
\label{s:synthese}
Nu we weten hoe logische schakelingen werken, en hoe we deze door middel van de booleaanse algebra kunnen uitwerken, wordt het tijd om zelf schakelingen te bouwen. Dit doen we vertrekkend vanuit een waarheidstabel. Soms is het makkelijk om een implementatie af te leiden uit deze waarheidstabellen. Maar indien het aantal variabelen groot wordt of de functie complex is, wordt het vinden van een implementatie moeilijk. In deze cursus stellen we dan ook enkele algemene technieken voor om uit een waarheidstabel een logische functie te genereren. In sectie \ref{s:minimalisatie} zullen we bovendien enkele methodes beschrijven om deze logische implementaties te optimaliseren. Doorheen deze sectie zullen we een logische functie genereren voor de waarheidstabel in \tblref{truthTableExample}.
\begin{table}[htb]
\centering
\begin{tabular}{ccc|c}
$x$&$y$&$z$&$f$\\\hline
0&0&0&0\\
0&0&1&1\\
0&1&0&0\\
0&1&1&0\\
1&0&0&1\\
1&0&1&1\\
1&1&0&1\\
1&1&1&0\\
\end{tabular}
\caption{Waarheidstabel voor het synthese-voorbeeld.}
\label{tbl:truthTableExample}
\end{table}
In de waarheidstabel zien we uitsluitend 0 en 1 staan. In sectie \ref{par:dontcare} zullen we echter kennis maken met een derde toestand: de \termen{don't care}.
\subsection{SOP \& POS}
\termen{Sum-of-Products (SOP)} en \termen{Product-of-Sums (POS)} zijn twee technieken die op een machinale manier uit een waarheidstabel een logische functie genereren. Ze maken respectievelijk gebruik van \termen{mintermen} en \termen{maxtermen}.
\paragraph{Sum-of-Products} De Sum-of-Products maakt gebruik van mintermen. Een minterm is een logische functie die slechts waar is voor \'e\'en gegeven rij in een waarheidstabel. Dit bekomen we door het booleaans product te nemen van alle variabelen. Indien de variabelen negatief zijn wordt de negatie in het product verwerkt. De minterm van rij $i$ noteren we als $m_i$. De Sum-of-Products methode neemt de booleaanse som van \termen{1-mintermen}. Een 1-minterm is een minterm die waar is bij een rij waarbij $f$ ook waar is. \tblref{truthTableExampleMinTerms} toont voor iedere rij de minterm. Wanneer $f$ waar is, wordt ook de 1-minterm ingevuld. Vervolgens bepalen we de som van alle 1-mintermen.
\begin{table}[htb]
\begin{center}
\begin{tabular}{m{0.65\textwidth}|m{0.35\textwidth}}
\begin{center}
\begin{tabular}{ccc|l|c|l}
$x$&$y$&$z$&minterm&$f$&1-minterm\\\hline
0&0&0&$m_0=x'\cdot y'\cdot z'$&0&\\
0&0&1&$m_1=x'\cdot y'\cdot z$&1&$m_1=x'\cdot y'\cdot z$\\
0&1&0&$m_2=x'\cdot y\cdot z'$&0&\\
0&1&1&$m_3=x'\cdot y\cdot z$&0&\\
1&0&0&$m_4=x\cdot y'\cdot z'$&1&$m_4=x\cdot y'\cdot z'$\\
1&0&1&$m_5=x\cdot y'\cdot z$&1&$m_5=x\cdot y'\cdot z$\\
1&1&0&$m_6=x\cdot y\cdot z'$&1&$m_6=x\cdot y\cdot z'$\\
1&1&1&$m_7=x\cdot y\cdot z$&0&\\
\end{tabular}
\end{center}
&
\begin{center}
\begin{tikzpicture}[circuit logic US]
\draw (-2.7,1.7) node[anchor=south]{$x$} -- (-2.7,-1.7);
\draw (-2.5,1.7) node[anchor=south]{$y$} -- (-2.5,-1.7);
\draw (-2.3,1.7) node[anchor=south]{$z$} -- (-2.3,-1.7);

\node[or gate,inputs={normal,normal,normal,normal}] (O) at (0,0) {};
\draw (O.output) -- ++(0.25,0) node[anchor=west]{$f$};

\node[and gate,inputs={inverted,inverted,normal}] (A1) at (-1.5,1.2) {$m_1$};
\draw (A1.output) -- ++(0.35,0) |- (O.input 1);
\draw (A1.input 1 -| -2.7,0) -- (A1.input 1);
\draw (A1.input 2 -| -2.5,0) -- (A1.input 2);
\draw (A1.input 3 -| -2.3,0) -- (A1.input 3);
\node[and gate,inputs={normal,inverted,inverted}] (A4) at (-1.5,0.4) {$m_4$};
\draw (A4.output) -- ++(0.2,0) |- (O.input 2);
\draw (A4.input 1 -| -2.7,0) -- (A4.input 1);
\draw (A4.input 2 -| -2.5,0) -- (A4.input 2);
\draw (A4.input 3 -| -2.3,0) -- (A4.input 3);
\node[and gate,inputs={normal,inverted,normal}] (A5) at (-1.5,-0.4) {$m_5$};
\draw (A5.output) -- ++(0.2,0) |- (O.input 3);
\draw (A5.input 1 -| -2.7,0) -- (A5.input 1);
\draw (A5.input 2 -| -2.5,0) -- (A5.input 2);
\draw (A5.input 3 -| -2.3,0) -- (A5.input 3);
\node[and gate,inputs={normal,normal,inverted}] (A6) at (-1.5,-1.2) {$m_6$};
\draw (A6.output) -- ++(0.35,0) |- (O.input 4);
\draw (A6.input 1 -| -2.7,0) -- (A6.input 1);
\draw (A6.input 2 -| -2.5,0) -- (A6.input 2);
\draw (A6.input 3 -| -2.3,0) -- (A6.input 3);
\end{tikzpicture}
\end{center}
\end{tabular}\\
SOP: $f=m_1+m_4+m_5+m_6=x'\cdot y'\cdot z+x\cdot y'\cdot z'+x\cdot y'\cdot z+x\cdot y\cdot z'$
\end{center}
\caption{Sum-of-Products methode toegepast op het voorbeeld.}
\label{tbl:truthTableExampleMinTerms}
\end{table}
\paragraph{Product-of-Sums} Analoog maakt de Product-of-Sums gebruik van maxtermen. Zoals de naam reeds doet vermoeden is een maxterm een logische functie die altijd waar is behalve voor \'e\'en rij in de waarheidstabel. Indien we de maxterm voor een bepaalde term willen genereren dienen we de booleaanse optelling van de negatie van de variabelen te nemen. We noteren een maxterm voor rij $i$ met $M_i$. De Product-of-Sums methode neemt analoog het booleaanse product van de \termen{0-maxtermen}. Een 0-maxterm is analoog een maxterm wanneer $f$ onwaar is. \tblref{truthTableExampleMaxTerms} toont voor iedere rij de maxterm. Wanneer $f$ onwaar is, wordt ook de 0-maxterm ingevuld. Vervolgens bepalen we het product van alle 0-maxtermen.
\begin{table}[htb]
\begin{center}
\begin{tabular}{m{0.65\textwidth}|m{0.35\textwidth}}
\begin{center}

\begin{tabular}{ccc|l|c|l}
$x$&$y$&$z$&maxterm&$f$&0-maxterm\\\hline
0&0&0&$M_0=x+y+z$&0&$M_0=x+y+z$\\
0&0&1&$M_1=x+y+z'$&1&\\
0&1&0&$M_2=x+y'+z$&0&$M_2=x+y'+z$\\
0&1&1&$M_3=x+y'+z'$&0&$M_3=x+y'+z'$\\
1&0&0&$M_4=x'+y+z$&1&\\
1&0&1&$M_5=x'+y+z'$&1&\\
1&1&0&$M_6=x'+y'+z$&1&\\
1&1&1&$M_7=x'+y'+z'$&0&$M_7=x'+y'+z'$\\
\end{tabular}
\end{center}
&
\begin{center}
\begin{tikzpicture}[circuit logic US]
\draw (-2.7,1.7) node[anchor=south]{$x$} -- (-2.7,-1.7);
\draw (-2.5,1.7) node[anchor=south]{$y$} -- (-2.5,-1.7);
\draw (-2.3,1.7) node[anchor=south]{$z$} -- (-2.3,-1.7);

\node[and gate,inputs={normal,normal,normal,normal}] (O) at (0,0) {};
\draw (O.output) -- ++(0.25,0) node[anchor=west]{$f$};

\node[or gate,inputs={normal,normal,normal}] (A1) at (-1.5,1.2) {$M_0$};
\draw (A1.output) -- ++(0.35,0) |- (O.input 1);
\draw (A1.input 1 -| -2.7,0) -- (A1.input 1);
\draw (A1.input 2 -| -2.5,0) -- (A1.input 2);
\draw (A1.input 3 -| -2.3,0) -- (A1.input 3);
\node[or gate,inputs={normal,inverted,normal}] (A4) at (-1.5,0.4) {$M_2$};
\draw (A4.output) -- ++(0.2,0) |- (O.input 2);
\draw (A4.input 1 -| -2.7,0) -- (A4.input 1);
\draw (A4.input 2 -| -2.5,0) -- (A4.input 2);
\draw (A4.input 3 -| -2.3,0) -- (A4.input 3);
\node[or gate,inputs={normal,inverted,inverted}] (A5) at (-1.5,-0.4) {$M_3$};
\draw (A5.output) -- ++(0.2,0) |- (O.input 3);
\draw (A5.input 1 -| -2.7,0) -- (A5.input 1);
\draw (A5.input 2 -| -2.5,0) -- (A5.input 2);
\draw (A5.input 3 -| -2.3,0) -- (A5.input 3);
\node[or gate,inputs={inverted,inverted,inverted}] (A6) at (-1.5,-1.2) {$M_7$};
\draw (A6.output) -- ++(0.35,0) |- (O.input 4);
\draw (A6.input 1 -| -2.7,0) -- (A6.input 1);
\draw (A6.input 2 -| -2.5,0) -- (A6.input 2);
\draw (A6.input 3 -| -2.3,0) -- (A6.input 3);
\end{tikzpicture}
\end{center}
\end{tabular}\\
POS: $f=M_0\cdot M_2\cdot M_3\cdot M_7=\left(x+y+z\right)\cdot\left(x+y'+z\right)\cdot\left(x+y'+z'\right)\cdot\left(x'+y'+z'\right)$
\end{center}
\caption{Product-of-Sums methode toegepast op het voorbeeld.}
\label{tbl:truthTableExampleMaxTerms}
\end{table}
\subsection{Canonieke versus standaard realisatie}
\label{ss:canoniekestandaardrealisatie}
Wanneer we de Sum-of-Products of Product-of-Sums methodes toepassen, zullen alle min- of maxtermen alle variabelen bevatten. Zo'n implementatie wordt de \termen{Canonieke Vorm} genoemd. Dit is bijgevolg een relatief dure implementatie. Nochtans bestaat er een eenvoudige methode om een groot deel van de poorten te elimineren of te reduceren in aantal ingangen. Deze methode zet de canonieke vorm om in de \termen{Standaard Vorm}. De methode komt er op neer verschillende min- of maxtermen samen te nemen door er de wetten van De Morgan op toe te passen. Op die manier kunnen we dan deelexpressies van de vorm $x+x'$ uitkomen. Omdat deze deelexpressies altijd waar zijn onafhankelijk van $x$ kan bijgevolg $x$ geëlimineerd worden uit de nieuwe min- of maxterm. Op \figref{standardizationMinMaxTerms} herleiden we de canonieke vormen van het voorbeeld naar hun standaard equivalent.
\begin{figure}[htb]
\centering
\begin{tikzpicture}
\def\dx{8 cm};
\def\dxh{0.5*\dx};
\def\dy{8 cm};
\draw[thick] (\dx,0) -- (\dx,\dy+0.25 cm);
\draw (\dxh,\dy) node {Sum-of-Products};
\draw (\dxh,\dy-0.5 cm) node {\small $x'y'z+xy'z'+xy'z+xyz'$};
\draw (\dxh,\dy-2.75 cm) node{\begin{tikzpicture}[circuit logic US]
\draw (-2.7,1.7) node[anchor=south]{$x$} -- (-2.7,-1.7);
\draw (-2.5,1.7) node[anchor=south]{$y$} -- (-2.5,-1.7);
\draw (-2.3,1.7) node[anchor=south]{$z$} -- (-2.3,-1.7);

\node[or gate,inputs={normal,normal,normal,normal}] (O) at (0,0) {};
\draw (O.output) -- ++(0.25,0) node[anchor=west]{$f$};

\node[and gate,inputs={inverted,inverted,normal}] (A1) at (-1.5,1.2) {};
\draw (A1.output) -- ++(0.35,0) |- (O.input 1);
\draw (A1.input 1 -| -2.7,0) -- (A1.input 1);
\draw (A1.input 2 -| -2.5,0) -- (A1.input 2);
\draw (A1.input 3 -| -2.3,0) -- (A1.input 3);
\node[and gate,inputs={normal,inverted,inverted}] (A4) at (-1.5,0.4) {};
\draw (A4.output) -- ++(0.2,0) |- (O.input 2);
\draw (A4.input 1 -| -2.7,0) -- (A4.input 1);
\draw (A4.input 2 -| -2.5,0) -- (A4.input 2);
\draw (A4.input 3 -| -2.3,0) -- (A4.input 3);
\node[and gate,inputs={normal,inverted,normal}] (A5) at (-1.5,-0.4) {};
\draw (A5.output) -- ++(0.2,0) |- (O.input 3);
\draw (A5.input 1 -| -2.7,0) -- (A5.input 1);
\draw (A5.input 2 -| -2.5,0) -- (A5.input 2);
\draw (A5.input 3 -| -2.3,0) -- (A5.input 3);
\node[and gate,inputs={normal,normal,inverted}] (A6) at (-1.5,-1.2) {};
\draw (A6.output) -- ++(0.35,0) |- (O.input 4);
\draw (A6.input 1 -| -2.7,0) -- (A6.input 1);
\draw (A6.input 2 -| -2.5,0) -- (A6.input 2);
\draw (A6.input 3 -| -2.3,0) -- (A6.input 3);
\end{tikzpicture}};
\draw (\dxh,\dy-5 cm) node {\small $=\left(x+x'\right)y'z+\left(y+y'\right)xz'=y'z+xz'$};
\draw (\dxh,\dy-6.5 cm) node{\begin{tikzpicture}[circuit logic US]
\draw (-2.7,0.9) node[anchor=south]{$x$} -- (-2.7,-0.9);
\draw (-2.5,0.9) node[anchor=south]{$y$} -- (-2.5,-0.9);
\draw (-2.3,0.9) node[anchor=south]{$z$} -- (-2.3,-0.9);

\node[or gate] (O) at (0,0) {};
\draw (O.output) -- ++(0.25,0) node[anchor=west]{$f$};
\node[and gate,inputs={inverted,normal}] (A1) at (-1.5,0.4) {};
\draw (A1.output) -- ++(0.2,0) |- (O.input 1);
\draw (A1.input 1 -| -2.5,0) -- (A1.input 1);
\draw (A1.input 2 -| -2.3,0) -- (A1.input 2);
\node[and gate,inputs={normal,inverted}] (A2) at (-1.5,-0.4) {};
\draw (A2.output) -- ++(0.2,0) |- (O.input 2);
\draw (A2.input 1 -| -2.7,0) -- (A2.input 1);
\draw (A2.input 2 -| -2.3,0) -- (A2.input 2);
\end{tikzpicture}};
\begin{scope}[xshift=\dx]
\draw (\dxh,\dy) node {Product-of-Sums};
\draw (\dxh,\dy-0.5 cm) node {\small $\left(x+y+z\right)\left(x+y'+z\right)\left(x+y'+z'\right)\left(x'+y'+z'\right)$};
\draw (\dxh,\dy-2.75 cm) node{\begin{tikzpicture}[circuit logic US]
\draw (-2.7,1.7) node[anchor=south]{$x$} -- (-2.7,-1.7);
\draw (-2.5,1.7) node[anchor=south]{$y$} -- (-2.5,-1.7);
\draw (-2.3,1.7) node[anchor=south]{$z$} -- (-2.3,-1.7);

\node[and gate,inputs={normal,normal,normal,normal}] (O) at (0,0) {};
\draw (O.output) -- ++(0.25,0) node[anchor=west]{$f$};

\node[or gate,inputs={normal,normal,normal}] (A1) at (-1.5,1.2) {};
\draw (A1.output) -- ++(0.35,0) |- (O.input 1);
\draw (A1.input 1 -| -2.7,0) -- (A1.input 1);
\draw (A1.input 2 -| -2.5,0) -- (A1.input 2);
\draw (A1.input 3 -| -2.3,0) -- (A1.input 3);
\node[or gate,inputs={normal,inverted,normal}] (A4) at (-1.5,0.4) {};
\draw (A4.output) -- ++(0.2,0) |- (O.input 2);
\draw (A4.input 1 -| -2.7,0) -- (A4.input 1);
\draw (A4.input 2 -| -2.5,0) -- (A4.input 2);
\draw (A4.input 3 -| -2.3,0) -- (A4.input 3);
\node[or gate,inputs={normal,inverted,inverted}] (A5) at (-1.5,-0.4) {};
\draw (A5.output) -- ++(0.2,0) |- (O.input 3);
\draw (A5.input 1 -| -2.7,0) -- (A5.input 1);
\draw (A5.input 2 -| -2.5,0) -- (A5.input 2);
\draw (A5.input 3 -| -2.3,0) -- (A5.input 3);
\node[or gate,inputs={inverted,inverted,inverted}] (A6) at (-1.5,-1.2) {};
\draw (A6.output) -- ++(0.35,0) |- (O.input 4);
\draw (A6.input 1 -| -2.7,0) -- (A6.input 1);
\draw (A6.input 2 -| -2.5,0) -- (A6.input 2);
\draw (A6.input 3 -| -2.3,0) -- (A6.input 3);
\end{tikzpicture}};
\draw (\dxh,\dy-5 cm) node {\small $=\left(y+y'\right)\left(x+z\right)\left(x+x'\right)\left(y'+z'\right)=\left(x+z\right)\left(y'+z'\right)$};
\draw (\dxh,\dy-6.5 cm) node{\begin{tikzpicture}[circuit logic US]
\draw (-2.7,0.9) node[anchor=south]{$x$} -- (-2.7,-0.9);
\draw (-2.5,0.9) node[anchor=south]{$y$} -- (-2.5,-0.9);
\draw (-2.3,0.9) node[anchor=south]{$z$} -- (-2.3,-0.9);

\node[and gate] (O) at (0,0) {};
\draw (O.output) -- ++(0.25,0) node[anchor=west]{$f$};
\node[or gate] (A1) at (-1.5,0.4) {};
\draw (A1.output) -- ++(0.2,0) |- (O.input 1);
\draw (A1.input 1 -| -2.7,0) -- (A1.input 1);
\draw (A1.input 2 -| -2.3,0) -- (A1.input 2);
\node[or gate,inputs={inverted,inverted}] (A2) at (-1.5,-0.4) {};
\draw (A2.output) -- ++(0.2,0) |- (O.input 2);
\draw (A2.input 1 -| -2.5,0) -- (A2.input 1);
\draw (A2.input 2 -| -2.3,0) -- (A2.input 2);
\end{tikzpicture}};
\end{scope}
\end{tikzpicture}
\caption{Herleiden naar Standaard Vorm van voorbeeld.}
\figlab{standardizationMinMaxTerms}
\end{figure}
\paragraph{}
De Standaard Vorm is gegarandeerd de meest minimale vorm voor schakelingen met twee lagen. Het is echter wel mogelijk goedkopere schakelingen te ontwerpen die uit meerdere lagen bestaat. In dat geval dient er echter een trade-off gemaakt te worden tussen kosten enerzijds en performantie anderzijds. Zo toont \figref{alternativeStandardForm} een alternatieve implementatie van een standaardvorm. Deze is 8\% goedkoper\footnote{De relatieve kost van een circuit kan berekend worden met vergelijking \ref{eqn:kosten}.}, maar zal slechts aan 66\% van de snelheid werken. In dat geval zal de toepassing vaak uitmaken wat het meest opportuun is.
\begin{figure}[htb]
\centering
\begin{tikzpicture}[circuit logic US]
\node[or gate,inputs={normal,normal,normal}] (O) at (0,0) {};
\draw (O.output) -- ++(0.25,0) node[anchor=west]{$f$};
\node[and gate] (A1) at (-1.5,0.8) {};
\draw (A1.input 2 -| -2.5,0) node[anchor=east]{$x$} -- (A1.input 2);
\draw (A1.output) -- ++(0.2,0) |- (O.input 1);
\node[and gate] (A2) at (-1.5,0) {};
\draw (A1.input 2 -| -2,0) |- (A2.input 1);
\draw (A2.input 2 -| -2.5,0) node[anchor=east]{$y$} -- (A2.input 2);
\draw (A2.output) -- ++(0.2,0) |- (O.input 2);
\node[and gate] (A3) at (-1.5,-0.8) {};
\draw (A2.input 2 -| -2,0) |- (A3.input 1);
\draw (A3.input 2 -| -2.5,0) node[anchor=east]{$z$} -- (A3.input 2);
\draw (A3.output) -- ++(0.2,0) |- (O.input 3);
\draw (A3.input 2 -| -2.25,0) |- (A1.input 1);
\draw (-0.75,-1.5) node{$f=xy+yz+xz$};
\begin{scope}[xshift=7.5 cm]
\node[or gate] (O1) at (0,0) {};
\draw (O1.output) -- ++(0.25,0) node[anchor=west]{$f$};
\node[and gate] (A1) at (-1.5,0.6) {};
\draw (A1.input 1 -| -4,0) node[anchor=east]{$x$} -- (A1.input 1);
\draw (A1.output) -- ++(0.2,0) |- (O1.input 1);
\node[and gate] (A2) at (-1.5,-0.6) {};
\draw (A2.input 2 -| -4,0) node[anchor=east]{$z$} -- (A2.input 2);
\draw (A2.output) -- ++(0.2,0) |- (O1.input 2);
\node[or gate] (O2) at (-3,0) {};
\draw (O2.input 1 -| -4,0) node[anchor=east]{$y$} -- (O2.input 1);
\draw (A2.input 2 -| -3.5,0) |- (O2.input 2);
\draw (O2.output) -- ++(0.2,0) |- (A1.input 2);
\draw (O2.input 1 -| -3.75,0) |- (A2.input 1);
\draw (-1.5,-1.5) node{$f=x(y+z)+yz$};
\end{scope}
\end{tikzpicture}
\caption{Standaard vorm en alternatief.}
\figlab{alternativeStandardForm}
\end{figure}
\subsection{Realisaties met NAND en NOR}
Zoals reeds kort vermeld in subsectie \ref{ss:logischePoorten} zijn NAND en NOR poorten erg populair bij heel wat implementaties. Dit komt omdat ze alle basispoorten kunnen emuleren zoals blijkt uit \tblref{nandNorUniversal} op pagina \pageref{tbl:nandNorUniversal}. Bovendien is hun kostprijs lager dan een AND of OR poort\footnote{Denk aan de formule van de kostprijs: het aantal inverters wordt afgetrokken van de kostprijs.}. NAND en NOR poorten blijken bovendien ook makkelijk implementeerbaar als substituut voor AND en OR poorten in standaard vorm. \figref{standardizationNandNor} toont hoe we ons voorbeeld in standaardvorm verder kunnen optimaliseren. In stap 1 inverteren we de min- en max-termen door van een AND en OR respectievelijk een NAND en NOR te maken. We behouden echter de functie door inverters aan de ingangen van het tweede niveau toe te voegen. Immers is tweemaal inverteren niets anders dan dezelfde waarde behouden. Door vervolgens de wetten van De Morgan toe te passen in stap 2, resulteert dit in een circuit die alleen gebruik maakt van NAND en NOR poorten. Bovendien geldt dat een circuit ge\"implementeerd met NAND en NOR poorten altijd zowel goedkoper en sneller is dan zijn canonisch equivalent.
\begin{figure}[htb]
\centering
\begin{tikzpicture}[circuit logic US]
\draw (-2.7,0.9) node[anchor=south]{$x$} -- (-2.7,-4.9);
\draw (-2.5,0.9) node[anchor=south]{$y$} -- (-2.5,-4.9);
\draw (-2.3,0.9) node[anchor=south]{$z$} -- (-2.3,-4.9);

\draw (1.5,2) -- (1.5,-5);
\draw (-3.5,1.5) -- (6.5,1.5);
\draw (-1,1.5) node[anchor=south]{Sum-of-Products};
\draw (4,1.5) node[anchor=south]{Product-of-Sums};
\draw[->] (-0.5,-0.5) -- (-0.5,-1.5);
\draw (-0.3,-1) -- (0.1,-1);
\draw[->] (0.2,-1) -- (0.3,-1);
\draw (0.4,-1) -- (0.45,-1);
\draw (0.5,-1) circle (0.05);
\draw (0.55,-1) -- (0.65,-1);
\draw (0.7,-1) circle (0.05);
\draw (0.75,-1) -- (0.8,-1);
\draw[->] (-0.5,-2.5) to node[midway,anchor=west]{De Morgan} (-0.5,-3.5);
\draw[->] (4.5,-0.5) -- (4.5,-1.5);
\draw[->] (4.5,-2.5) to node[midway,anchor=west]{De Morgan} (4.5,-3.5);

\node[or gate] (O) at (0,0) {};
\draw (O.output) -- ++(0.25,0) node[anchor=west]{$f$};
\node[and gate,inputs={inverted,normal}] (A1) at (-1.5,0.4) {};
\draw (A1.output) -- ++(0.2,0) |- (O.input 1);
\draw (A1.input 1 -| -2.5,0) -- (A1.input 1);
\draw (A1.input 2 -| -2.3,0) -- (A1.input 2);
\node[and gate,inputs={normal,inverted}] (A2) at (-1.5,-0.4) {};
\draw (A2.output) -- ++(0.2,0) |- (O.input 2);
\draw (A2.input 1 -| -2.7,0) -- (A2.input 1);
\draw (A2.input 2 -| -2.3,0) -- (A2.input 2);
\begin{scope}[yshift=-2 cm]
\node[or gate,inputs={inverted,inverted}] (O) at (0,0) {};
\draw (O.output) -- ++(0.25,0) node[anchor=west]{$f$};
\node[nand gate,inputs={inverted,normal}] (A1) at (-1.5,0.4) {};
\draw (A1.output) -- ++(0.2,0) |- (O.input 1);
\draw (A1.input 1 -| -2.5,0) -- (A1.input 1);
\draw (A1.input 2 -| -2.3,0) -- (A1.input 2);
\node[nand gate,inputs={normal,inverted}] (A2) at (-1.5,-0.4) {};
\draw (A2.output) -- ++(0.2,0) |- (O.input 2);
\draw (A2.input 1 -| -2.7,0) -- (A2.input 1);
\draw (A2.input 2 -| -2.3,0) -- (A2.input 2);
\end{scope}
\begin{scope}[yshift=-4 cm]
\node[nand gate] (O) at (0,0) {};
\draw (O.output) -- ++(0.25,0) node[anchor=west]{$f$};
\node[nand gate,inputs={inverted,normal}] (A1) at (-1.5,0.4) {};
\draw (A1.output) -- ++(0.2,0) |- (O.input 1);
\draw (A1.input 1 -| -2.5,0) -- (A1.input 1);
\draw (A1.input 2 -| -2.3,0) -- (A1.input 2);
\node[nand gate,inputs={normal,inverted}] (A2) at (-1.5,-0.4) {};
\draw (A2.output) -- ++(0.2,0) |- (O.input 2);
\draw (A2.input 1 -| -2.7,0) -- (A2.input 1);
\draw (A2.input 2 -| -2.3,0) -- (A2.input 2);
\end{scope}
\begin{scope}[xshift=5 cm]
\draw (-2.7,0.9) node[anchor=south]{$x$} -- (-2.7,-4.9);
\draw (-2.5,0.9) node[anchor=south]{$y$} -- (-2.5,-4.9);
\draw (-2.3,0.9) node[anchor=south]{$z$} -- (-2.3,-4.9);

\draw (-0.3,-1) -- (0.1,-1);
\draw[->] (0.2,-1) -- (0.3,-1);
\draw (0.4,-1) -- (0.45,-1);
\draw (0.5,-1) circle (0.05);
\draw (0.55,-1) -- (0.65,-1);
\draw (0.7,-1) circle (0.05);
\draw (0.75,-1) -- (0.8,-1);

\node[and gate] (O) at (0,0) {};
\draw (O.output) -- ++(0.25,0) node[anchor=west]{$f$};
\node[or gate] (A1) at (-1.5,0.4) {};
\draw (A1.output) -- ++(0.2,0) |- (O.input 1);
\draw (A1.input 1 -| -2.7,0) -- (A1.input 1);
\draw (A1.input 2 -| -2.3,0) -- (A1.input 2);
\node[or gate,inputs={inverted,inverted}] (A2) at (-1.5,-0.4) {};
\draw (A2.output) -- ++(0.2,0) |- (O.input 2);
\draw (A2.input 1 -| -2.5,0) -- (A2.input 1);
\draw (A2.input 2 -| -2.3,0) -- (A2.input 2);
\begin{scope}[yshift=-2 cm]
\node[and gate,inputs={inverted,inverted}] (O) at (0,0) {};
\draw (O.output) -- ++(0.25,0) node[anchor=west]{$f$};
\node[nor gate] (A1) at (-1.5,0.4) {};
\draw (A1.output) -- ++(0.2,0) |- (O.input 1);
\draw (A1.input 1 -| -2.7,0) -- (A1.input 1);
\draw (A1.input 2 -| -2.3,0) -- (A1.input 2);
\node[nor gate,inputs={inverted,inverted}] (A2) at (-1.5,-0.4) {};
\draw (A2.output) -- ++(0.2,0) |- (O.input 2);
\draw (A2.input 1 -| -2.5,0) -- (A2.input 1);
\draw (A2.input 2 -| -2.3,0) -- (A2.input 2);
\end{scope}
\begin{scope}[yshift=-4 cm]
\node[nor gate] (O) at (0,0) {};
\draw (O.output) -- ++(0.25,0) node[anchor=west]{$f$};
\node[nor gate] (A1) at (-1.5,0.4) {};
\draw (A1.output) -- ++(0.2,0) |- (O.input 1);
\draw (A1.input 1 -| -2.7,0) -- (A1.input 1);
\draw (A1.input 2 -| -2.3,0) -- (A1.input 2);
\node[nor gate,inputs={inverted,inverted}] (A2) at (-1.5,-0.4) {};
\draw (A2.output) -- ++(0.2,0) |- (O.input 2);
\draw (A2.input 1 -| -2.5,0) -- (A2.input 1);
\draw (A2.input 2 -| -2.3,0) -- (A2.input 2);
\end{scope}
\end{scope}
\end{tikzpicture}
\caption{Standaardvorm van het voorbeeld met NAND en NOR's.}
\figlab{standardizationNandNor}
\end{figure}
\section{Digitaal ontwerp in grote lijnen}
Nu we de concepten van een digitaal ontwerp in grote lijnen uitgetekend hebben, kunnen we het over de praktische kant van de zaak hebben. Immers komt bij een digitaal ontwerp veel meer kijken dan alleen het bouwen van de juiste schakeling. Een klant zal immers specificaties sturen waaraan de hardware moet voldoen, bovendien wil de klant naast de hardware vaak ook nog documentatie over hoe men het toestel dient te gebruiken. Bovendien willen we vermijden dat we bij het ontwerpen telkens opnieuw het wiel moeten uitvinden. Er bestaan immers al bibliotheken die allerhande schakelingen voor bijvoorbeeld optellingen, testen op gelijke waarde,... bevatten. Deze zijn meestal met de nodige zorg samengesteld zodat het vinden van een nog effici\"entere implementatie moeilijk wordt. \figref{digitalDesignManagement} vat het hele verloop goed samen. Verder zullen we in de volgende subsecties de verschillende stappen want meer in detail bekijken.
\begin{figure}
\centering
\begin{tikzpicture}[itm/.style={draw=black,rectangle,minimum width=3 cm,minimum height=0.5 cm}]
\node[itm] (Sp) at (0,0) {Specificatie \S\ref{ss:specificatie}};
\node[itm] (B) at (-3.5,-0.5) {Bibliotheek \S\ref{ss:bibliotheek}};
\node[itm] (Sy) at (0,-1) {Synthese \S\ref{ss:synthese}};
\node[itm] (A) at (0,-2) {Analyse \S\ref{ss:analyse}};
\node[itm] (D) at (-3.5,-2) {Documentatie \S\ref{ss:documentatie}};
\draw[->] (Sp) -- (Sy);
\draw[->] (B) |- (Sy);
\draw[->] (Sy) -- (A);
\draw (A) |- ++(2.5,-0.5) |- ++(-2.5,2);
\draw[<-] (Sp) |- ++(2.5,0.5) -- ++(0,-1);
\end{tikzpicture}
\caption{Typisch verloop van een digitaal ontwerp.}
\figlab{digitalDesignManagement}
\end{figure}
\subsection{Specificatie}
\label{ss:specificatie}
Een \termen{Specificatie} is een beschrijving van de functionaliteiten die van de hardware in kwestie gevraagd worden. In tegenstelling tot specificaties in bijvoorbeeld de informatica, is de \termen{interface} ook een onderdeel van de specificaties. De interface beschrijft hoe de hardware interageert met zijn omgeving. Deze omgeving is niet noodzakelijk de gebruiker die bijvoorbeeld toetsen indrukt en het scherm uitleest. Dit kan ook bijvoorbeeld de PCI\footnote{PCI: Peripheral Component Interconnect.} bus zijn waarmee met bijvoorbeeld een computer gecommuniceerd wordt. Dikwijls zijn initieel de specificaties onvolledig. De gaten in de specificaties worden dan ook opgevuld wanneer er zich problemen stellen in bijvoorbeeld de synthese. Specificaties zijn bijgevolg een iteratief proces. Dikwijls bevatten de specificaties ook reeds implementatiesbeslissingen die onnodige beperkingen opleggen aan het ontwerp. De beschrijving wordt ofwel in natuurlijke taal, wat soms dubbelzinnig is, of met behulp van een blokschema beschreven.
\subsection{Synthese}
\label{ss:synthese}
De \termen{Synthese} is niets anders dan de vertaling van de specificaties van een hoog en abstract naar een lager niveau. Hierbij dienen uiteraard concrete beslissingen genomen te worden over de implementatie. Zo kunnen de specificaties bijvoorbeeld vermelden dat $x$ en $y$ bij elkaar opgeteld moeten worden. De synthese moet dan een concrete implementatie voorstellen. Bijvoorbeeld een 16-bit ripple-carry adder met 2 registers. Synthese gebeurt meestal op verschillende niveaus. Zo worden op het laagste niveau componenten gebouwd. Dit zijn bijvoorbeeld de poorten maar ook bijvoorbeeld flipflops of multiplexers. Deze worden op een niveau hoger gebruikt om \termen{Register-Transfer-Level (RTL) Componenten} te bouwen. Deze RTL componenten zijn bijvoorbeeld optellers, schuifregisters en tellers. Schakelingen met deze componenten zijn dan \termen{Application Specific IC (ASIC) componenten}. Deze componenten vormen dan uiteindelijk de bouwblokken voor het systeem. Bij de systeemsynthese worden dan processoren, geheugens en de ASIC componenten gecombineerd. \figref{synthesisPyramid} toont de pyramide van synthese. Elk niveau combineert hierbij de componenten ge\"introduceerd op een niveau lager.
\begin{figure}[htb]
\centering
\begin{tikzpicture}[itm/.style={draw=black,thick,rectangle,fill=white,minimum width=10 cm,minimum height=1 cm}]
\draw[thick] (-3.75,0) -- (0,7.5) -- (3.75,0) -- cycle;
\node[itm] (LA) at (0,1.5) {Ontwerp van componenten};
\node[itm] (LB) at (0,3) {\begin{tabular}{c}Ontwerp op componentniveau\\(met basiscomponenten: poorten, flipflops)\end{tabular}};
\node[itm] (LC) at (0,4.5) {\begin{tabular}{c}Architectuursynthese (op RTL-niveau)\\(met RTL-componenten: optellers, tellers, schuifregisters)\end{tabular}};
\node[itm] (LD) at (0,6) {\begin{tabular}{c}Systeemsynthese\\(bouwblokken: processoren, geheugen, ASIC)\end{tabular}};
\end{tikzpicture}
\caption{De verschillende lagen bij de synthese.}
\figlab{synthesisPyramid}
\end{figure}
\subsection{Bibliotheek}
\label{ss:bibliotheek}
We dienen niet het volledige stuk hardware vanaf transistor of poortniveau te ontwerpen. Heel wat werk is dan ook al in het verleden door andere ontwerpers gedaan. Het hergebruiken van deze ontwerpen heeft dan ook heel wat voordelen:
\begin{itemize}
 \item De ontwerpen zijn meestal al door verschillende personen geoptimaliseerd. Het vinden van een nog optimalere implementatie is daardoor quasi onbestaand.
 \item Men streeft naar telkens hogere integratieniveaus waarbij soms volledige systemen op chip te verkrijgen zijn. Bovendien worden deze systemen telkens geavanceerder. Dit komt door de \termen{Wet van Moore}. Deze stelt dat elke 24 maanden het aantal transistoren op een chip verdubbelt.
 \item Vaak kan men componenten kopen die een bepaalde functie vervullen. Dit bespaart veel ontwerpwerk. Bovendien zijn deze componenten vaak veel goedkoper dan ze zelf te ontwerpen en te produceren. Bijvoorbeeld een 16 kB geheugen.
\end{itemize}
Er bestaan bibliotheken op elk syntheseniveau. Dus bijgevolg kan men tot op het niveau dat men zelf kiest putten uit de bibliotheken.
\subsection{Analyse}
\label{ss:analyse}
Na elke synthesestap is het belangrijk om te testen of aan de vereiste specificaties voldaan is. Dit wordt gedaan in de \termen{Analyse}. Niet alleen wordt hier getest of de implementatie de functionaliteit levert die gevraagd is, ook allerhande andere parameters worden in rekening gebracht:
\begin{itemize}
 \item Kostprijs: vaak wordt hier als metriek het aantal pinnen en de oppervlakte van de printplaat (PCB\footnote{PCB: Printed Circuit Board.}) gebruikt.
 \item Vermogengebruik: het vermogenverbruik wordt meestal berekend met de formule $C\cdot f\cdot V^2$ met als parameters:
 \begin{itemize}
  \item $C$ De oppervlakte van de chip. De oppervlakte is in de tijd toegenomen. In 1983 was een gemiddelde chip immers $0.25\mbox{ cm}^2$, in 2000 was dat ongeveer $4\mbox{ cm}^2$.
  \item $f$ De \termen{klokfrequentie}. De klokfrequentie is in de loop der tijd exponentieel gestegen. Met $1\mbox{ MHz}$ in 1983, en $1\mbox{ GHz}$ in 2000.
  \item $V$ De spanning die aan de chip geleverd wordt. De spanning is gedaald: van $5\mbox{ V}$ in 1983 tot $1.5\mbox{ V}$ ongeveer 17 jaar later.
 \end{itemize}
 \item Snelheid: de vertraging of \termen{doorvoer} (``\termen{throughput}''). Dit is het aantal resultaten per seconde.
 \item Testbaarheid: kunnen we alle fouten ontdekken met behulp van testvectoren?
\end{itemize}
\subsection{Documentatie}
\label{ss:documentatie}
Naast de hardware verlangt de klant meestal ook \termen{Documentatie}. Afhankelijk van het soort klant zal er andere documentatie vereist zijn. Indien we een volledig afgewerkt consumentenproduct maken, dienen we een handleiding voor de consument en de hersteller samen te stellen. Hierin beschrijven we in natuurlijke taal hoe de component aangestuurd kan worden. Indien we echter een component leveren, bijvoorbeeld een geheugenmodule zal men een handleiding maken met daarin de specifieke ontwerpdetails. Meestal zal ook intern binnen het bedrijf documentatie een vereiste zijn. Dit om de eventueel verdere ontwikkelingen te ondersteunen.
\subsection{Ontwerpen met CAD}
\termen{Computer Aided Design (CAD)} wordt vaak toegepast om een chip te ontwikkelen. Hierbij wordt met behulp van een computer een een speciaal softwarepakket meestal het volledige ontwerp begeleid. Een concreet voorbeeld hiervan is bijvoorbeeld \emph{KiCad} voor \emph{Linux}. Zoals andere CAD tools bevat \emph{KiCad} een project manager die de gebruiker door de verschillende stappen begeleidt, en een set tools die de gebruiker bij iedere stap de nodige ondersteuning bieden. Uiteraard ziet het ontwerp met een CAD-tool er gelijkaardig uit aan het ontwikkelingsproces op \figref{digitalDesignManagement}. De CAD-tools bieden echter meer mogelijkheden om ontwerpen al in een vroeg stadium te testen en te simuleren. \figref{digitalDesignManagementCad} toont het ontwikkelingsproces met behulp van een CAD-tool.
\begin{figure}[htb]
\centering
\begin{tikzpicture}[itm/.style={draw=black,rectangle,minimum width=4 cm}]
\node (Sp) at (0,0) {Specificatie};
\draw (Sp.north west) -- (Sp.north east) -- ++(0.15,-0.275) -- (Sp.south east) -- (Sp.south west) -- ++(-0.15,0.275) -- cycle;
\node (Sc) at (-1,-2.25) {Schema};
\draw (Sc.north west) -- (Sc.north east) -- ++(0.15,-0.275) -- (Sc.south east) -- (Sc.south west) -- ++(-0.15,0.275) -- cycle;
\node (V) at (1,-2.25) {VHDL};
\draw (V.north west) -- (V.north east) -- ++(0.15,-0.275) -- (V.south east) -- (V.south west) -- ++(-0.15,0.275) -- cycle;
\draw (-2,-1.25) rectangle ++(4,-1.5);
\draw (0,-1.25) node[anchor=north]{Ingave ontwerp};
\draw[->] (Sp) -- (0,-1.25);
\node[itm] (Sy) at (0,-4) {Synthese};
\draw[->] (0,-2.75) -- (Sy);
\node[itm] (Fs) at (0,-6) {Functionele Simulatie};
\draw[->] (Sy) -- (Fs);
\node[draw,shape aspect=2,diamond] (O) at (0,-8) {Ontwerp OK?};
\draw (O.south) node[anchor=north west]{Ja};
\draw (O.west) node[anchor=south east]{Nee};
\draw[->] (Fs) -- (O);
\node[itm] (Fo) at (6,-2) {Fysisch ontwerp};
\draw[->] (O.south) |- ++(3,-0.5) |- (Fo.west);
\node[itm] (St) at (6,-4) {Simulatie tijdsgedrag};
\draw[->] (Fo) -- (St);
\node[draw,shape aspect=2,diamond] (T) at (6,-6) {Tijdsgedrag OK?};
\draw (T.south) node[anchor=north west]{Ja};
\draw (T.west) node[anchor=south east]{Nee};
\draw[->] (St) -- (T);
\draw (T) -- ++(-3,0);
\draw (3.5,-6) |- (-3,-10) -- (-3,-8);
\draw (O) -| (-3,-0.625) -- ++(3,0);
\node[itm] (C) at (6,-8) {Chipconfiguratie};
\draw[->] (T) -- (C);
\end{tikzpicture}
\caption{Digitaal ontwerpen met CAD.}
\figlab{digitalDesignManagementCad}
\end{figure}
\section{Taalgebaseerd hardware ontwerp: VHDL}
\seclab{vhdl}
Zoals reeds vermeld werd op \figref{digitalDesignManagementCad}, wordt \termen{VHDL} veel gebruikt om schakelingen in te voeren in een computersysteem. VHDL is de afkorting van \termen{VHSIC Hardware Description Language}. Hierbij staat VHSIC voor \termen{Very High Speed Integrated Circuit}. VHDL is een programmeertaal waarmee men het gedrag van digitale circuits probeert te beschrijven. De taal biedt mogelijkheden aan om op een eenduidige manier het gedrag van een circuit te specifi\"eren op RTL niveau. Daarnaast is de taal erg nuttig om simulaties te draaien, synthese uit te voeren (VHDL software is meestal in staat om zelf effici\"ente implementaties voor te stellen) en documentatie te genereren. VHDL is gestandaardiseerd bij de IEEE\footnote{IEEE: Institute for Electrical and Electronics Engineers.}. De eerste versie van VHDL is VHDL-87, gestandaardiseerd onder IEEE 1076. In 1993 werd de tweede versie, VHDL-93 uitgebracht in IEEE 1164. Sinds 2002 bestaat er ook een derde versie die nog niet gestandaardiseerd is. Deze standaarden omvatten echter uitsluitend de syntax. De implementatie van de VHDL compiler is volledig vrij. Er is dan ook concurrentie tussen VHDL compilers in features om de meest effici\"ente implementatie te kunnen voorstellen.
\subsection{Alternatieven en uitbreidingen}
\paragraph{VHDL-AMS}Een uitbreiding op VHDL is \termen{VHDL Analog and Mixed Signals (VHDL-AMS)}. Hierbij worden niet alleen digitale maar ook analoge signalen beschouwd. VHDL-AMS kan dus als een superset van de orginele VHDL beschouwd worden. Bovendien wordt met continue tijd gerekend in plaats van de door VHDL gebruikte discrete tijdstippen. Dit werd ge\"implementeerd door een set algebra\"ische en differenti\"ele vergelijkingen. Hoewel het in 1999 door de IEEE gestandaardiseerd werd, is VHDL-AMS sinds zijn oorsprong in 1993 nooit echt doorgebroken.
\paragraph{Verilog}De concurrent van VHDL is \termen{Verilog} (IEEE 1364). Verilog is erg populair in de Verenigde Staten maar is nooit echt doorgebroken in Europa. Beide talen stammen ook uit andere taalfamilies met andere paradigma's. Terwijl \verb+VHDL+ eerder lijkt op \verb+Ada+, is \verb+Verilog+ meer verwant met \verb+C+. Ondanks de concurrentie lijkt geen van beide talen het pleit te kunnen beslechten. Of zoals D. Pellerin \& D. Taylor het verwoorden in ``VHDL Made Easy!''\cite{pellerin1997vhdl}:
\begin{quote}
Both languages are easy to learn and hard to master. And once you have learned one of these languages, you will have no trouble transitioning to the other.
\end{quote}
\paragraph{PLD talen}
Talen zoals Abel en Palasm zijn zogenaamde \termen{PLD talen}\footnote{PLD: programmable Logic Device, zie \ref{sss:pld}.}. Deze talen specifi\"eren schakelingen op het niveau van de poorten en dit slechts voor een speciale technologie. Deze talen hebben bijgevolg ook een ander objectief. Waar VHDL net bedoeld is om zich met de details bezig te houden, en de gebruiker met de grote lijnen, zijn PDL talen bedoelt om te implementeren op deze lagere niveaus.
\subsection{Voordelen}
VHDL wordt hoofdzakelijk in de industrie gebruikt omwille van zijn overdraagbaarheid. VHDL is immers een standaard die door allerhande programma's gehanteerd wordt. Elk van deze programma's kunnen heel diverse toepassingen hebben. Op die manier kan een stuk VHDL eerst gesimuleerd worden met het ene programma, daarna gesynthetiseerd met een ander, om bijvoorbeeld geanalyseerd te worden met een derde programma. Deze programma's kunnen bovendien afkomstig zijn van verschillende fabrikanten. Hierdoor kunnen fabrikanten zich ook toespitsen op \'e\'en kant van het ontwerp zonder dat er compatibiliteitsproblemen ontstaan.
\paragraph{}
Daarnaast is VHDL ook interessant om complexe schakelingen op een hogere abstractieniveau te beschrijven. Zo kan men een schakeling die vaak terugkomt groeperen in een bepaalde module. Repetitieve structuren dienen slechts \'e\'enmaal beschreven te worden, maar kan men eindeloos blijven gebruiken.
\paragraph{}
VHDL maakt het ook mogelijk om de gebruiker te laten ontwerpen los van de eigenlijke implementatie. Zo dient de gebruiker alleen te specificeren dat twee getallen opgeteld moeten worden. Het is dan aan het programma om de componenten te selecteren die dat op de beste manier doen (qua kosten, snelheid,...).
\paragraph{}
Tot slot zorgt VHDL er ook voor dat een ontwerp makkelijk te parametriseren valt. Indien de ontwerper niet zeker is van de woordlengte van zijn processor kan hij deze in een parameter onderbrengen. Als later blijkt dat de woordlengte groter moet zijn, kan met een eenvoudige verandering van de parameterwaarde het volledige model aangepast worden.
\subsection{Nadelen}
Naast de reeks opgesomde voordelen heeft VHDL ook enkele belangrijke nadelen. Zo is het een eenvoudig te leren taal, maar de taal echt beheersen vraagt heel wat geduld. Dit komt door een deel door de niet eenvoudige syntax. Een ander probleem is dat heel wat software afwijkt van de gestandaardiseerde versies. Er bestaan dan ook onnoemelijk veel VHDL ``dialecten'' waardoor sommige features die aan de taal werden toegevoegd slechts op bepaalde softwarepakketten werken.
\paragraph{}
Een ander nadeel is dat de taal nogal langdradig is. Bij complexe circuits zullen de groeperingen zeker hun effect hebben, maar om kleine schakelingen te realiseren is nogal veel code nodig.
\paragraph{}
Een probleem met code in het algemeen is dat het erg onoverzichtelijk is. Een eenvoudig blokschema is nog altijd overzichtelijker omdat mensen nu eenmaal grafisch sterker zijn. Pas bij grote complexe schakelingen verliest het visuele zijn kracht en zal een stuk code als doeltreffender worden aanzien.
\paragraph{}
Het feit dat VHDL redelijk uitgebreid is, brengt bovendien allerhande nadelen met zich mee. Alle extra features voor bijvoorbeeld tijdsgedrag simulaties dienen immers in de taal beschreven te kunnen worden. Bijgevolg worden sommige taalconcepten hierdoor hopeloos moeilijker gemaakt.
\subsection{Beperkingen}
Naast de voor- en nadelen heeft VHDL ook enkele beperkingen. Zo is VHDL slechts tot op zeker niveau automatisch synthetiseerbaar. Hierbij ondersteunt elke fabrikant van VHDL een verschillende subset. Het tweede grote nadeel is dat slechts de syntax en de semantiek van VHDL gestandaardiseerd is. Niet hoe de code geschreven moet worden. Dit houdt in dat eenzelfde gedrag op tientallen verschillende manieren beschreven kan worden. Dit zorgt er ook voor dat elk programma dat VHDL leest de code op een andere manier kan implementeren. Het gevolg is dat de codestijl, die in grote mate bepaalt hoe de code ge\"implementeerd zal worden, variabel is. In het ene programma kan een bepaalde code tot de meest optimale implementatie leiden, terwijl een ander programma met dezelfde code slechts een standaardimplementatie kiest. Men moet dus eerst heel wat ervaring opdoen met een programma alvorens men weet hoe men de code moet schrijven zodat deze een effici\"ente implementatie kiest.
\subsection{Concepten (entiteiten en architectuur)}
Na de voor en nadelen besproken te hebben, is het tijd voor een voorbeeld waarmee we de verschillende concepten zullen duidelijk maken. We ontwerpen een schakeling zoals op \figref{vhdl-example}. Hierbij willen we een schakeling ``test'' ontwerpen. Deze schakeling krijgt drie 8-bit ingangen (\texttt{In1}, \texttt{In2} en \texttt{In3}). Als uitgangen zijn er twee bits (\texttt{Out1} en \texttt{Out2}). \texttt{Out1} geeft 1 terug indien \texttt{In1} en \texttt{In2} aan elkaar gelijk zijn. Analoog geeft \texttt{Out2} 1 terug indien \texttt{In2} en \texttt{In3} gelijk zijn. Om deze vergelijker te bouwen werken we bijgevolg met een hi\"erarchisch schema. Waarbij we \texttt{Comp} op een andere plaats implementeren.
\importtikzfigure{vhdl-example}{Voorbeeldcircuit voor VHDL code.}
\paragraph{}We zullen eerst \texttt{Comp} implementeren in VHDL. De code hiervoor staat in \vhdlref{comp}. Als we de code goed bekijken onderscheiden we twee belangrijke delen \vhdltermen{entity} en \vhdltermen{architecture}.
\paragraph{Entity}
Het \texttt{entity} gedeelte beschrijft de zogenaamde ``\termen{blackbox}'' ofwel de interface. Hierdoor is VHDL in staat een blokje te tonen met de juiste in en uitgangen. Zoals we zien bevat de beschrijving het \vhdltermen{port} commando. Het \texttt{port} commando specificeert dan de in- en uitgangen. Dit doet men door eerst de namen van de in- of uitgangen te vermelden. Vervolgens plaatst men het token \vhdltermen{in} of \vhdltermen{out} om te specificeren of het om een in- of uitgang gaat. Merk dus op dat elke geleider naar de component een expliciete richting heeft. Vervolgens duidt men het type van de in- of uitgang aan. Logischerwijs bevat dit het token \vhdltermen{bit}. Een bit is niets anders dan \'e\'en lijn. Meestal echter willen we verschillende lijnen samennemen. In dat geval duiden we dit aan met een \vhdltermen{bit\_vector}, een bitvector is dus niets anders dan een lijst van geleiders naar het blok. Het is handig verschillende geleiders samen te nemen, dit maakt de code overzichtelijker. Indien bijvoorbeeld de lengte van A gewijzigd moet worden kost dit slechts een minimale ingreep.
\paragraph{Architecture}
De concrete werking wordt beschreven in het \texttt{architecture} gedeelte. Dit beschrijft het gedrag van de component op RTL-niveau. VHDL is in staat met de gegeven beschrijving in de \texttt{architecture} omgeving een implementatie op poortniveau te bouwen. Verder dient ook opgemerkt te worden dat \'e\'en \texttt{entity} verschillende \texttt{architecture}s kan hebben.\footnote{Vandaar dat we onze architectuur de naam \texttt{Behav1} noemen.}. Uiteraard moeten al deze \texttt{architecture}s met dezelfde interface werken.
\begin{vhdlcode}
\centering
\begin{lstlisting}
-- 8-bit comparator
--
entity Comp is
  port(	A,B: in bit_vector(0 to 7);
	EQ: out bit);
end entity Comp;

architecture Behav1 of Comp is
begin
  EQ <= '1' when (A=B) else '0';
end architecture Behav1;
\end{lstlisting}
\caption{8-bit comparator.}
\label{vhdl:comp}
\end{vhdlcode}
\paragraph{Component} Nu we een comparator component gebouwd hebben, zullen we deze component importeren in ons testcircuit. Hiervoor gebruiken we code beschreven in \vhdlref{test}. Dit bestand deelt dezelfde structuur als de structuur in \vhdlref{comp}: een \texttt{entity} en \texttt{architecture}. Dit wijst er dus op dat we componenten hi\"erarchisch kunnen gebruiken. We bemerken echter een nieuw token: \vhdltermen{component}. \texttt{component} is een virtuele verwijzing naar een andere entiteit. Wat die entiteit is laten we in de code nog in het midden. Het enige wat we moeten doen is de interface van deze component specificeren. Later zullen we dan in de configuratie (\vhdlref{testConfig}) een binding voorzien tussen ons virtuele component \texttt{comparator} en onze gedefinieerde entiteit \texttt{comp}. Verder merken we ook op dat in het \texttt{architecture} gedeelte eenmaal we de \texttt{component} interface gedefinieerd hebben, we er instanties van kunnen aanmaken. Zo maken we twee instanties aan: \texttt{Comp1} en \texttt{Comp2}. Vervolgens dienen we nog de verbindingen tussen de entiteit \texttt{test} en deze instanties te leggen. Dit doen we met het token \vhdltermen{map}. Hierbij mappen we de variabelen uit \texttt{entity Test} (\texttt{In1}, \texttt{In2}, \texttt{In3}, \texttt{Out1} en \texttt{Out2}) op de in- en uitgangen van het \texttt{component Comparator} (\texttt{X}, \texttt{Y} en \texttt{Z}). Hierbij dient dus opgemerkt te worden dat we het gedrag van \texttt{Test} specificeren aan de hand van reeds gedefinieerde entiteiten. Tot slot dient ook opgemerkt te worden dat in tegenstelling tot klassieke programmeertalen alle componenten tegelijk werken. Het is dus niet zo dat bij het uitrekenen van \texttt{Test} eerst \texttt{Comp1} en dan \texttt{Comp2} uitgerekend zal worden.
\begin{vhdlcode}
\centering
\begin{lstlisting}
-- Component Test met 2 comparatoren
--
entity Test is
  port(	In1,In2,In3: in bit_vector(0 to 7);
	Out1,Out2: out bit);
end entity Test;

architecture Struct1 of Test is
component Comparator is
  port(	X,Y: in bit_vector(0 to 7);
	Z: out bit);
end component Comparator;
begin
  Comp1: component Comparator port map (In1,In2,Out1);
  Comp2: component Comparator port map (In2,In3,Out2);
end architecture Struct1;
\end{lstlisting}
\caption{Voorbeeldcode.}
\label{vhdl:test}
\end{vhdlcode}
\paragraph{Configuration} We moeten nu \texttt{Comparator} met de entiteit \texttt{Comp} binden. Dit doen we in een \vhdltermen{con\-fig\-ura\-tion} omgeving. Deze configuratie wordt beschreven in VHDL-code \ref{vhdl:testConfig}. Zoals we zien kunnen we opnieuw verschillende \texttt{configuration}s aanmaken en deze een aparte naam geven. Op die manier kunnen we dus de feitelijke implementatie snel wijzigen. Verder dienen we ook te vermelden welke entiteit we zullen configureren (\texttt{Test}) en om welke architectuur het gaat (\texttt{Struct1}). Vervolgens kunnen we per instantie aangeven welke entiteit we gebruiken. Dit doen we door het token \vhdltermen{use entity}. Ook specifi\"eren we de \texttt{architecture} die we zullen gebruiken van deze \texttt{entity}. Vervolgens mappen we opnieuw met behulp van \texttt{map} de in- en uitgangen van de entiteit op het virtuele \texttt{component}.
\begin{vhdlcode}
\begin{lstlisting}
-- Configuratie: definieer koppeling component met een
-- bepaalde architectuur van een entiteit
--
configuration Build1 of Test is
  for Struct1
    for Comp1: Comparator use entity Comp(Behav1)
      port map (A => X, B => Y, EQ => Z);
    end for;
    for Comp2: Comparator use entity Comp(Behav1)
      port map (A => X, B => Y, EQ => Z);
    end for;
  end for;
end configuration Build1;
\end{lstlisting}
\caption{Configuratie van de voorbeeldcode.}
\label{vhdl:testConfig}
\end{vhdlcode}
\subsection{Gelijkenissen en verschillen met klassieke programmeertalen}
VHDL lijkt in sommige opzichten een beetje op programmeren in klassieke programmeertalen zoals \texttt{Java} en \texttt{C++}. Immers kunnen we de ingangen als parameters bij een methode zien. De methode zelf fungeert ook als een interface die duidelijk maakt welke types er ingevoerd moeten worden, en welke uitvoer verwacht mag worden, zonder de feitelijke implementatie te tonen. We zouden bovendien elk component die we in een circuit gebruiken kunnen zien als een functie-oproep naar de functie van het bijbehorende component. Deze vergelijking heeft echter enkele anomalie\"en.
\begin{itemize}
 \item \termen{Gelijktijdigheid} (``\termen{Concurrency}''): Alle hardwarecomponenten werken in parallel dit in tegenstelling tot klassieke talen waarin alles sequentieel wordt uitgevoerd.
 \item \termen{Tijdsconcept}: Alle hardware werkt continu en houdt nooit op met werken. Bovendien zal bij een simulatie de tijd uiteraard niet de re\"ele tijd zijn.
 \item \termen{Datatypes}: VHDL heeft nood aan typische hardware-types zoals bitvectoren, getallen met geparametriseerde grootte,... Dit terwijl klassieke talen meestal proberen abstractie te maken van dit hardwareniveau.
\end{itemize}

%%REVIEW
\chapter{Technologische Randvoorwaarden}
\chplab{technology}
\chapterquote{De enige limiet aan de realisatie van de toekomst is ons twijfelen van vandaag.}{Franklin D. Roosevelt, Amerikaans staatsman en president (26e) (1882-1945)}
\begin{chapterintro}
Tot nu toe hebben we altijd abstractie gemaakt van de werkelijkheid. We hebben in hoofdstuk \ref{ch:basis} poorten ge\"introduceerd, maar hebben altijd abstractie gemaakt van de concrete werking van deze poorten. Aan de hand van het lichtmodel konden we \'e\'en en ander verklaren, maar deze schakelaars moesten manueel geschakeld worden. In dit hoofdstuk zullen we een manier zien hoe we poorten kunnen implementeren, en een 0 en 1 kunnen voorstellen die gebruik maakt van elektronica. Dit is uiteraard slechts \'e\'en manier. Naast het implementeren van poorten stuiten we vaak op allerhande fysische problemen. Dit hoofdstuk geeft een overzicht van de verschillende aspecten die we in de gaten moeten houden bij het ontwerpen van een elektronische schakeling. Verder biedt het een overzicht van manieren om een digitale schakeling te realiseren. Tot slot bekijken we de ontwikkeling van digitale schakeling met een CAD tool.
\end{chapterintro}
\minitoc[n]
\section{Logische waarden voorstellen}
\label{s:logischeWaarden}
Alvorens we enige schakeling kunnen implementeren moeten we een conventie afspreken hoe we een 0 en 1 voorstellen. Deze logische waarden moeten we voorstellen door twee fysische waarden. Deze fysische waarden noemen we ``\termen{High}'' en ``\termen{Low}''. In de elektronica kiezen we meestal als fysische grootheid de spanning. De referentie-spanningen noteren we dan respectievelijk als $V_{\mbox{\tiny H}}$ en $V_{\mbox{\tiny L}}$. Meestal houden we echter niet vast aan \'e\'en spanning, maar defini\"eren we een bereik rond $V_{\mbox{\tiny H}}$ en $V_{\mbox{\tiny L}}$. Immers is het systeem niet volledig deterministisch, en er kunnen bijgevolg kleine variaties in de spanningen optreden. Daarnaast hebben we ook nog allerhande omgevingsparameters zoals de temperatuur, waardoor deze spanning sowieso enigszins zal afwijken. Tussen het bereik van $V_{\mbox{\tiny H}}$ en $V_{\mbox{\tiny L}}$ defini\"eren we ook nog een zone waar het niet duidelijk is of er een 1 of 0 op staat. Dit laat ons toe om fouten te detecteren. Immers indien de spanning dicht bij het midden ligt, is de kans op een foute interpretatie groot. Met een detectiesysteem zouden we dan bijvoorbeeld kunnen vragen om de bit opnieuw uit te sturen. We defini\"eren het bereik van $V_{\mbox{\tiny H}}$ als $\left[V_{\mbox{\tiny IH}},V_{\mbox{\tiny DD}}\right]$, dat van $V_{\mbox{\tiny L}}$ als $\left[V_{\mbox{\tiny SS}},V_{\mbox{\tiny IL}}\right]$. \figref{potentialRange} toont schematisch het bereik van deze waarden. Overigens hebben $V_{\mbox{\tiny DD}}$ en $V_{\mbox{\tiny SS}}$ nog een andere betekenis bij een circuit: het zijn de spanningen die op de voeding van het elektronische apparaat geplaatst worden. Hierbij staat de S voor ``\termen{Source}'' en D voor ``\termen{Drain}''.
\begin{figure}[hbt]
\centering
\begin{tikzpicture}
\fill[black!20] (-0.1,0) rectangle (0.1,1);
\fill[black!20] (-0.1,2) rectangle (0.1,3);
\draw[thick,->] (0,-0.1) -- (0,3.2);
\draw (-0.1,0) node[anchor=east,scale=0.75]{$V_{\mbox{\tiny SS}}$} -- (0.1,0);
\draw (-0.1,1) node[anchor=east,scale=0.75]{$V_{\mbox{\tiny IL}}$} -- (0.1,1);
\draw (-0.1,2) node[anchor=east,scale=0.75]{$V_{\mbox{\tiny IH}}$} -- (0.1,2);
\draw (-0.1,3) node[anchor=east,scale=0.75]{$V_{\mbox{\tiny DD}}$} -- (0.1,3);
\draw (1,3) node[scale=0.75,anchor=south]{\bf Fysisch};
\draw (3.5,3) node[scale=0.75,anchor=south]{\bf Positieve Logica};
\draw (6,3) node[scale=0.75,anchor=south]{\bf Negatieve Logica};
\draw (1,2.5) node[scale=0.75]{High};
\draw (1,1.5) node[scale=0.75]{Ongedefinieerd};
\draw (1,0.5) node[scale=0.75]{Low};
\draw (3.5,2.5) node[scale=0.75]{1};
\draw (3.5,1.5) node[scale=0.75]{-};
\draw (3.5,0.5) node[scale=0.75]{0};
\draw (6,2.5) node[scale=0.75]{0};
\draw (6,1.5) node[scale=0.75]{-};
\draw (6,0.5) node[scale=0.75]{1};
\draw[dotted] (2.25,3.25) -- (2.25,0);
\draw[dotted] (4.75,3.25) -- (4.75,0);
\draw[dotted] (0.1,3) -- (7.25,3);
\draw[dotted] (0.1,2) -- (7.25,2);
\draw[dotted] (0.1,1) -- (7.25,1);
\draw[dotted] (0.1,0) -- (7.25,0);
\end{tikzpicture}
\caption{Schematisch bereik van ``High'' en ``Low'' spanning.}
\figlab{potentialRange}
\end{figure}
\paragraph{Negatieve logica}Het is niet per definitie zo dat $V_{\mbox{\tiny H}}$ geassocieerd wordt met 1, en $V_{\mbox{\tiny L}}$ met 0. Dit hangt af van het type logica dat gehanteerd wordt. In geval van \termen{Positieve Logica} is dit inderdaad het geval. Soms komt het echter voordeliger uit om deze orde om te draaien, in dat geval spreekt men van \termen{Negatieve Logica}. Merk op dat bij negatieve logica de poorten fysisch anders ge\"implementeerd moeten worden. Immers kent de fysische poort alleen maar spanningen. De belangrijkste reden om negatieve logica te gebruiken, is dan ook omdat sommige implementaties hierbij goedkoper kunnen zijn. Bovendien kan men in een circuit op verschillende plaatsen een andere logica gebruiken. Een concreet voorbeeld is de reset-module in veel elektronicasystemen.
\section{Implementatie van poorten}
\subsection{Schakelaars}
In hoofdstuk \ref{ch:basis} maakten we gebruik van het lichtmodel om poorten te implementeren. Daarbij werd gebruik gemaakt van schakelaars. Ook in de echte implementatie van poorten maakt men gebruik van \termen{schakelaars}. Deze schakelaars kunnen door een derde ingang automatisch open en dicht geschakeld worden, deze ingang wordt ook het ``\termen{stuursignaal}'' genoemd. Net als de schakelaars in het lichtmodel hebben deze schakelaars twee toestanden: open en gesloten. We kunnen dit interpreteren als een elektronische weerstand die respectievelijk een weerstand van $\infty\ \Omega$ en $0\ \Omega$ heeft. \figref{switchNotationSwitch} toont hoe dergelijke schakelaars genoteerd worden. Afhankelijk van hun toestand worden ze bovendien anders genoteerd. Zoals eerder gezegd dienen schakelaars een stuursignaal te hebben. Op \figref{switchNotationControlledNmos} en \ref{fig:switchNotationControlledPmos} staan de schakelaars met stuursignaal. We merken op dat er twee varianten van schakelaars zijn. De ene variant, \termen{NMOS}, is gesloten wanneer het stuursignaal een hoog fysische waarde heeft, en open bij een lage waarde. De tweede soort, \termen{PMOS}, inverteert dit principe en is gesloten bij een laag fysische waarde, en open bij een hoge waarde.
\begin{figure}[hbt]
\centering
\subfigure[Schakelaars]{\begin{tikzpicture}
\draw[thick] (0,0) circle(0.625);
\draw (0,-0.625) node[anchor=north]{Open} -- ++(0,0.375) -- ++(-0.2,0) -- ++(-0.2,0.5);
\draw (0,0.625) -- ++(0,-0.375) -- ++(-0.2,0);

\draw[thick] (1.5,0) circle(0.625);
\draw (1.5,-0.625) node[anchor=north]{Gesloten} -- ++(0,0.375) -- ++(-0.2,0) -- ++(0,0.5) -- ++(0.2,0) -- ++(0,0.375);
\end{tikzpicture}
\figlab{switchNotationSwitch}
}
\subfigure[NMOS Schakelaars met stuursignaal] {\begin{tikzpicture}
\draw[thick] (0,0) circle(0.625);
\draw (0,-0.625) -- ++(0,0.375) -- ++(-0.2,0) -- ++(-0.2,0.5);
\draw (0,0.625) -- ++(0,-0.375) -- ++(-0.2,0);
\draw[thick] (0,-0.625) -- ++(0,-0.5);
\draw[thick] (0,0.625) -- ++(0,0.5);
\draw[thick] (-0.625,0) -- ++(-0.75,0) node[anchor=east]{L};

\draw[thick] (3,0) circle(0.625);
\draw (3,-0.625) -- ++(0,0.375) -- ++(-0.2,0) -- ++(0,0.5) -- ++(0.2,0) -- ++(0,0.375);
\draw[thick] (3,-0.625) -- ++(0,-0.5);
\draw[thick] (3,0.625) -- ++(0,0.5);
\draw[thick] (2.375,0) -- ++(-0.75,0) node[anchor=east]{H};
\end{tikzpicture}
\figlab{switchNotationControlledNmos}
}
\subfigure[PMOS Schakelaars met stuursignaal] {\begin{tikzpicture}
\draw[thick] (6,0) circle(0.625);
\draw[thick] (5.25,0) circle(0.125);
\draw (6,-0.625) -- ++(0,0.375) -- ++(-0.2,0) -- ++(0,0.5) -- ++(0.2,0) -- ++(0,0.375);
\draw[thick] (6,-0.625) -- ++(0,-0.5);
\draw[thick] (6,0.625) -- ++(0,0.5);
\draw[thick] (5.125,0) -- ++(-0.5,0) node[anchor=east]{L};

\draw[thick] (9,0) circle(0.625);
\draw[thick] (8.25,0) circle(0.125);
\draw (9,-0.625) -- ++(0,0.375) -- ++(-0.2,0) -- ++(-0.2,0.5);
\draw (9,0.625) -- ++(0,-0.375) -- ++(-0.2,0);
\draw[thick] (9,-0.625) -- ++(0,-0.5);
\draw[thick] (9,0.625) -- ++(0,0.5);
\draw[thick] (8.125,0) -- ++(-0.75,0) node[anchor=east]{H};
\end{tikzpicture}
\figlab{switchNotationControlledPmos}
}
\caption{Notatie van een schakelaar (met stuursignaal).}
\figlab{switchNotation}
\end{figure}
\subsubsection{Werking van NMOS en PMOS}
\label{ss:nmosPmosWork}
In de vorige subsectie werden twee types schakelaars ge\"introduceerd: NMOS en PMOS. Dit zijn \termen{transistoren} die in bijna elk elektronisch apparaat gebruikt worden. In deze subsectie zullen we de terminologie van een transistor bespreken, en de werking van deze componenten verklaren. Zoals de meeste transistoren hebben een NMOS en PMOS 3 aansluitingen: de \termen{Source}, \termen{Gate} en \termen{Drain}, in het Nederlands worden deze aansluitingen ook respectievelijk \termen{Collector}, \termen{Basis} en \termen{Emitter} genoemd. Een transistor is eigenlijk niets anders dan een schakelaar tussen de source en drain. De weerstand tussen de source en de drain wordt geregeld volgens de spanning die op de gate staat. \figref{mosWork} toont de concrete werking van NMOS en PMOS transistoren. Voor beide typen beschouwen we een substraat van siliciumdioxide (SiO$_2$). We kunnen dit substraat zowel positief als negatief \termen{doperen}\footnote{Het introduceren van alternatieve atomen.}. Bij NMOS doperen we het substraat hoofdzakelijk positief (p). Bij de ingangen van de source en de drain doperen we negatief (n). Er kan nauwelijks stroom vloeien tussen het negatief en positief gedopeerde substraat. Bijgevolg fungeert de p-laag als een barri\`ere tussen de source en de drain. Indien we echter een positieve spanning aanbrengen op de Gate, zullen de negatieve deeltjes in de p-laag zich aangetrokken voelen tot de gate. Er ontstaat een reorganisatie van de p-laag waardoor er een n-laagje gevormd wordt ter hoogte van de gate. Hierdoor ontstaat er een kanaal tussen de source en de drain, waardoor de schakelaar zich sluit. De spanning die tussen de gate en de source moet staan om dit te bereiken wordt de \termen{threshold spanning $V_{\mbox{\tiny T}}$} genoemd. Indien de spanning tussen de gate en de source $V_{\mbox{\tiny GS}}$ dus kleiner is dan $V_{\mbox{\tiny T}}$ is de NMOS schakelaar open, anders is de schakelaar gesloten. Een PMOS transistor werkt op analoge manier, maar dan omgekeerd.
\begin{figure}[hbt]
\centering
\subfigure[NMOS bij $V_{\mbox{\tiny GS}}<V_{\mbox{\tiny T}}$.]{\begin{tikzpicture}
\fill[black!20] (0,0) rectangle (3,0.75);
\fill[black!80] (0,0.75) node[anchor=north west,scale=0.75,white]{n+} -- (0,0.45) .. controls (0.4,0.35) and (1.15,0.45) .. (1.15,0.75) -- cycle;
\fill[black!80] (3,0.75) node[anchor=north east,scale=0.75,white]{n+} -- (3,0.45) .. controls (2.6,0.35) and (1.85,0.45) .. (1.85,0.75) -- cycle;
\draw (1,0.75) rectangle ++(1,0.3);
\draw (1.5,0.9) node[scale=0.75]{isolator};
\filldraw[fill=gray] (1,1.05) rectangle ++(1,0.3);
\draw (1.5,1.2) node[scale=0.75]{metaal};
\draw[thick] (1.5,1.35) -- ++(0,0.5) node[anchor=south,scale=0.75]{Gate};
\draw[thick] (0.75,0.75) .. controls (0.75,1.05) and (0.75,1.05) .. (0.5,1.05) node[anchor=east,scale=0.75]{Source};
\draw[thick] (2.25,0.75) .. controls (2.25,1.05) and (2.25,1.05) .. (2.5,1.05) node[anchor=west,scale=0.75]{Drain};
\draw (0,0) to node[above,midway,scale=0.75]{p} (3,0);
\draw (0,0.75) -- (3,0.75);
\end{tikzpicture}}
\subfigure[NMOS bij $V_{\mbox{\tiny GS}}\geq V_{\mbox{\tiny T}}$.]{\begin{tikzpicture}
\fill[black!20] (0,0) rectangle (3,0.75);
\fill[black!80] (0,0.75) node[anchor=north west,scale=0.75,white]{n+} -- (0,0.45) .. controls (0.4,0.35) and (1.15,0.45) .. (1.25,0.55)  .. controls (1.5,0.675) and (1.5,0.675) .. (1.75,0.55) .. controls (1.85,0.45) and (2.6,0.35) .. (3,0.45) -- (3,0.75) node[anchor=north east,scale=0.75,white]{n+} -- cycle;
\draw (1,0.75) rectangle ++(1,0.3);
\draw (1.5,0.9) node[scale=0.75]{isolator};
\filldraw[fill=gray] (1,1.05) rectangle ++(1,0.3);
\draw (1.5,1.2) node[scale=0.75]{metaal};
\draw[thick] (1.5,1.35) -- ++(0,0.5) node[anchor=south,scale=0.75]{Gate};
\draw[thick] (0.75,0.75) .. controls (0.75,1.05) and (0.75,1.05) .. (0.5,1.05) node[anchor=east,scale=0.75]{Source};
\draw[thick] (2.25,0.75) .. controls (2.25,1.05) and (2.25,1.05) .. (2.5,1.05) node[anchor=west,scale=0.75]{Drain};
\draw (0,0) to node[above,midway,scale=0.75]{p} (3,0);
\draw (0,0.75) -- (3,0.75);
\end{tikzpicture}}
\subfigure[PMOS bij $V_{\mbox{\tiny GD}}<V_{\mbox{\tiny T}}$.]{\begin{tikzpicture}
\fill[black!80] (0,0) rectangle (3,0.75);
\fill[black!20] (0,0.75) node[anchor=north west,scale=0.75,black]{p+} -- (0,0.45) .. controls (0.4,0.35) and (1.15,0.45) .. (1.25,0.55)  .. controls (1.5,0.675) and (1.5,0.675) .. (1.75,0.55) .. controls (1.85,0.45) and (2.6,0.35) .. (3,0.45) -- (3,0.75) node[anchor=north east,scale=0.75,black]{p+} -- cycle;
\draw (1,0.75) rectangle ++(1,0.3);
\draw (1.5,0.9) node[scale=0.75]{isolator};
\filldraw[fill=gray] (1,1.05) rectangle ++(1,0.3);
\draw (1.5,1.2) node[scale=0.75]{metaal};
\draw[thick] (1.5,1.35) -- ++(0,0.5) node[anchor=south,scale=0.75]{Gate};
\draw[thick] (0.75,0.75) .. controls (0.75,1.05) and (0.75,1.05) .. (0.5,1.05) node[anchor=east,scale=0.75]{Drain};
\draw[thick] (2.25,0.75) .. controls (2.25,1.05) and (2.25,1.05) .. (2.5,1.05) node[anchor=west,scale=0.75]{Source};
\draw (0,0) to node[above,midway,scale=0.75,white]{n} (3,0);
\draw (0,0.75) -- (3,0.75);
\end{tikzpicture}}
\subfigure[PMOS bij $V_{\mbox{\tiny GD}}\geq V_{\mbox{\tiny T}}$.]{\begin{tikzpicture}
\fill[black!80] (0,0) rectangle (3,0.75);
\fill[black!20] (0,0.75) node[anchor=north west,scale=0.75,black]{p+} -- (0,0.45) .. controls (0.4,0.35) and (1.15,0.45) .. (1.15,0.75) -- cycle;
\fill[black!20] (3,0.75) node[anchor=north east,scale=0.75,black]{p+} -- (3,0.45) .. controls (2.6,0.35) and (1.85,0.45) .. (1.85,0.75) -- cycle;
\draw (1,0.75) rectangle ++(1,0.3);
\draw (1.5,0.9) node[scale=0.75]{isolator};
\filldraw[fill=gray] (1,1.05) rectangle ++(1,0.3);
\draw (1.5,1.2) node[scale=0.75]{metaal};
\draw[thick] (1.5,1.35) -- ++(0,0.5) node[anchor=south,scale=0.75]{Gate};
\draw[thick] (0.75,0.75) .. controls (0.75,1.05) and (0.75,1.05) .. (0.5,1.05) node[anchor=east,scale=0.75]{Drain};
\draw[thick] (2.25,0.75) .. controls (2.25,1.05) and (2.25,1.05) .. (2.5,1.05) node[anchor=west,scale=0.75]{Source};
\draw (0,0) to node[above,midway,scale=0.75,white]{n} (3,0);
\draw (0,0.75) -- (3,0.75);
\end{tikzpicture}}
\caption{Werking van NMOS en PMOS.}
\figlab{mosWork}
\end{figure}
\subsection{Basispoorten}
Nu we automatische schakelaars kunnen gebruiken, wordt het tijd om met behulp van NMOS en PMOS de poorten te implementeren. Eerst zullen we de basispoorten met behulp van NMOS implementeren. Vervolgens zal blijken dat deze implementaties enkele nadelen hebben. Hierdoor zullen we opteren voor implementatie met \termen{CMOS}. CMOS is in feite een combinatie van NMOS en PMOS. We zullen alle schakelingen implementeren in positieve logica. In sectie \ref{s:negativeLogic} behandelen we nog enkele zaken in verband met negatieve logica.
\subsubsection{Implementatie in NMOS}
\paragraph{NOT} \figref{notNmos} toont de implementatie van een NOT-poort in NMOS. Als ingang beschouwen we $x$, als uitgang $f$ de $V_{\mbox{\tiny DD}}$ en $V_{\mbox{\tiny SS}}$ zijn lijnen voor de voeding van de poort. Indien we een laag voltage aanbrengen op $x$ is de NMOS transistor open. Hierdoor staat er een hoge spanning op $f$. Indien we echter een hoge spanning op $x$ aanbrengen, zal de transistor zich sluiten. Hierdoor vloeit er stroom tussen $V_{\mbox{\tiny DD}}$ en $V_{\mbox{\tiny SS}}$. Omdat de stroom door $f$ nog andere componenten zal aansturen, zal de stroom bijgevolg verkiezen om door de transistor te stromen, en krijgt $f$ dus een laag potentiaal. Het principe van het verlagen van de spanning door stroom door te laten wordt \termen{Pull-Down Network (PDN)} genoemd. Theoretisch gezien mag dit model mooi lijken, de werkelijkheid verschilt echter. Wanneer de NMOS transistor immers gesloten is, zal er immers nog steeds een kleine weerstand over staan. Deze weerstand noteren we als $R_{\mbox{\tiny on}}$. Dit betekent dat $f$ niet volledig gelijk zal zijn aan $L$. We berekenen de spanning op $f$ dan ook met volgende formule:
\begin{equation}
V_{\mbox{\tiny out}}\left(x=1\right)=\displaystyle\frac{R_{\mbox{\tiny on}}}{R+R_{\mbox{\tiny on}}}V_{\mbox{\tiny DD}}
\end{equation}
Een tweede groot nadeel is dat we bij $x=H$ statisch vermogen verbruiken:
\begin{equation}
P\left(x=1\right)=\displaystyle\frac{V_{\mbox{\tiny DD}}^2}{R+R_{\mbox{\tiny on}}}
\end{equation}
Dit is op energieniveau een grote kost, vooral omdat er in een gemiddelde processor makkelijk miljoenen transistoren zitten. Bovendien zou dit hoge temperaturen genereren. Dit is dan ook de hoofdreden waarom er nauwelijks elektronica ge\"implementeerd wordt met NMOS.
\begin{figure}[hbt]
\centering
\subfigure[x=L, f=H]{\begin{circuitikz}[american resistors]
\node [nmoso] (T) at (0,3) {};
\draw[<-] (0,4.75) node[anchor=west,scale=0.75]{$V_{\mbox{\tiny DD}}\ (H)$} -- (0,4.5);
\draw (0,4.5) to [R,size=0.5,l={\small $R$}] (0,3.5) -- (T.drain);
\draw (0,3.5) -- ++(1,0) node[anchor=west,scale=0.75]{$f=H$};
\draw (T.gate) -- ++(-0.5,0) node[anchor=east,scale=0.75]{$x=L$};
\draw (T.source) -- (0,2.65) node[ground]{};
\draw (0.4,2.3) node[anchor=west,scale=0.75]{$V_{\mbox{\tiny SS}}\ (L)$};
\end{circuitikz}}
\subfigure[x=H, f=L]{\begin{circuitikz}[american resistors]
\node [nmosc] (T) at (0,3) {};
\draw[<-] (0,4.75) node[anchor=west,scale=0.75]{$V_{\mbox{\tiny DD}}\ (H)$} -- (0,4.5);
\draw (0,4.5) to [R,size=0.5,l={\small $R$}] (0,3.5) -- (T.drain);
\draw (0,3.5) -- ++(1,0) node[anchor=west,scale=0.75]{$f=L$};
\draw (T.gate) -- ++(-0.5,0) node[anchor=east,scale=0.75]{$x=H$};
\draw (T.source) -- (0,2.65) node[ground]{};
\draw (0.4,2.3) node[anchor=west,scale=0.75]{$V_{\mbox{\tiny SS}}\ (L)$};
\end{circuitikz}}
\caption{NOT poort ge\"implementeerd in NMOS.}
\figlab{notNmos}
\end{figure}
\paragraph{``Open-drain poort''}Indien we de weerstand als een extern component beschouwen, en meer NMOS transistoren in parallel aan de draad hangen, kunnen we het gedrag van een AND poort nabootsen, zoals op \figref{openDrainNmos}. Vanaf het moment dat \'e\'en van de transistoren gesloten wordt, lekt de stroom door die transistor, en krijgt $f$ dus een laag niveau. Alle ingangen moeten dus een lage spanning hebben, om een hoge spanning aan de uitgang te verkrijgen. We kunnen dit principe dus zien als een AND waarbij aan alle ingangen een inverter staat. Dit systeem wordt ``\termen{Open-Drain Poort}'' genoemd. En heeft het voordeel dat we een AND kunnen bouwen aan de hand van draden. Deze implementatie is dus goedkoop indien we een AND-poort willen maken met een zeer groot aantal ingangen. Standaard wordt dit soort implementatie dan ook gebruikt in de reset-functionaliteit van elektronica. Meestal wordt er immers een reset procedure aangeroepen indien \'e\'en van de detectoren een fout registreert. In dat geval zal de detector een hoge spanning aan zijn ingang genereren. Dit resulteert een lage uitgang, bij een lage uitgang treed de reset-procedure dan in werking.
\begin{figure}[hbt]
\centering
\begin{circuitikz}[american resistors]
\node [nmosc] (T1) at (-1,2.75) {};
\node [nmosc] (T2) at (-1,0.75) {};
\draw[<-] (0,4.75) node[anchor=west,scale=0.75]{$V_{\mbox{\tiny DD}}\ (H)$} -- (0,4.5);
\draw (0,4.5) to [R,size=0.5,l={\small $R$}] (0,3.5) -- (0,-0.25);
\draw (0,3.5) -| (T1.drain);
\draw (0,1.5) -| (T2.drain);
\draw (0,2.5) -- ++(1,0) node[anchor=west,scale=0.75]{$f$};
\draw (T1.gate) -- ++(-0.25,0) node[anchor=east,scale=0.75]{$x$};
\draw (T1.source) -- ++(0,0) node[ground]{};
\draw (T2.gate) -- ++(-0.25,0) node[anchor=east,scale=0.75]{$y$};
\draw (T2.source) -- ++(0,0) node[ground]{};
\end{circuitikz}
\caption{Open-Drain Poort in NMOS.}
\figlab{openDrainNmos}
\end{figure}
\paragraph{NAND en NOR}
De vorige paragraaf gaf reeds een opzet hoe we een NOR en NAND poort kunnen bouwen. We moeten er eenvoudig weg voor zorgen dat er stroom kan vloeien tussen $V_{\mbox{\tiny DD}}$ en $V_{\mbox{\tiny SS}}$ indien respectievelijk minstens \'e\'en of alle transistoren gesloten worden. \figref{nandNorNmos} toont dan ook de implementaties voor een NAND en NOR poort. Een AND en OR poort kunnen we dan vervolgens synthetiseren door een inverter achter de poort te plaatsen. Deze implementaties maken ook meteen duidelijk waarom een NAND goedkoper is dan een AND.
\begin{figure}
\centering
\subfigure[NAND poort.]{\begin{circuitikz}[american resistors]
\node [nmosc] (T1) at (0,3) {};
\node [nmosc] (T2) at (0,2) {};
\draw[<-] (0,4.75) node[anchor=west,scale=0.75]{$V_{\mbox{\tiny DD}}\ (H)$} -- (0,4.5);
\draw (0,4.5) to [R,size=0.5,l={\small $R$}] (0,3.5) -- (T1.drain);
\draw (0,3.5) -- ++(1,0) node[anchor=west,scale=0.75]{$f$};
\draw (T1.gate) -- ++(-0.5,0) node[anchor=east,scale=0.75]{$x$};
\draw (T1.source) -- (T2.drain);
\draw (T2.gate) -- ++(-0.5,0) node[anchor=east,scale=0.75]{$y$};
\draw (T2.source) -- ++(0,0) node[ground]{};
\end{circuitikz}}
\subfigure[NOR poort.]{\begin{circuitikz}[american resistors]
\node [nmosc] (T1) at (-1,2.5) {};
\node [nmosc] (T2) at (1,2.5) {};
\draw[<-] (0,4.75) node[anchor=west,scale=0.75]{$V_{\mbox{\tiny DD}}\ (H)$} -- (0,4.5);
\draw (0,4.5) to [R,size=0.5,l={\small $R$}] (0,3.5);
\draw (0,3.5) -- ++(1,0) node[anchor=west,scale=0.75]{$f$};
\draw (T1.gate) -- ++(-0.5,0) node[anchor=east,scale=0.75]{$x$};
\draw (T1.drain) |- ++(1,0.25);
\draw (T1.source) |- ++(1,-0.25);
\draw (T2.gate) -- ++(-0.5,0) node[anchor=east,scale=0.75]{$y$};
\draw (T2.drain) |- ++(-1,0.25) -- (0,3.5);
\draw (T2.source) |- ++(-1,-0.25) -- ++(0,-0.25) node[ground]{};
\end{circuitikz}}
\caption{NAND en NOR poort ge\"implementeerd met NMOS.}
\figlab{nandNorNmos}
\end{figure}
\subsubsection{Implementatie in CMOS}
Het grote argument tegen het gebruik van NMOS is dat het veel vermogen nutteloos verbruikt. CMOS biedt hiervoor een oplossing. CMOS is eigenlijk een techniek waarbij we zowel een NMOS als een PMOS transistor gebruiken. Analoog aan het ``Pull-Down Network'' fenomeen van de NMOS, spreken we dan over het ``\termen{Pull-Up Network (PUN)}'' fenomeen bij PMOS.
\paragraph{NOT}Indien we de uitgang $f$ plaatsen tussen een NMOS en PMOS transistor, die we allebei met dezelfde ingang $x$ verbinden, kunnen we een NOT poort implementeren. De uitwerking hiervan staat op \figref{notCmos}. Op het moment dat we een hoge spanning aanleggen op de ingang sluit de NMOS transistor zich, en opent de PMOS transistor zich, hierdoor krijgt de uitgang dezelfde spanning als de ground. Indien we een lage spanning aan de ingang aanleggen is de configuratie van de transistoren omgekeerd, en wordt komt aan de uitgang een spanning $V_{\mbox{\tiny DD}}$ te liggen.
\begin{figure}[hbt]
\centering
\begin{circuitikz}[american resistors]
\node [pmosc] (T1) at (0,3.75) {};
\node [nmoso] (T2) at (0,2.75) {};
\draw[<-] (0,4.5) node[anchor=west,scale=0.75]{$V_{\mbox{\tiny DD}}\ (H)$} -- (0,4);
\draw (0,3.25) -- ++(1,0) node[anchor=west,scale=0.75]{$f=H$};
\draw (T1.source) -- (0,4.5);
\draw (T1.drain) -- (T2.drain);
\draw (T1.gate) -- (T1.gate -| -0.75,0) |- (T2.gate);
\draw (-0.75,3.25) -- ++(-0.5,0) node[anchor=east,scale=0.75]{$x=L$};
\draw (T2.source) -- ++(0,0) node[ground]{};
\begin{scope}[xshift=6 cm]
\node [pmoso] (T1) at (0,3.75) {};
\node [nmosc] (T2) at (0,2.75) {};
\draw[<-] (0,4.5) node[anchor=west,scale=0.75]{$V_{\mbox{\tiny DD}}\ (H)$} -- (0,4);
\draw (0,3.25) -- ++(1,0) node[anchor=west,scale=0.75]{$f=L$};
\draw (T1.source) -- (0,4.5);
\draw (T1.drain) -- (T2.drain);
\draw (T1.gate) -- (T1.gate -| -0.75,0) |- (T2.gate);
\draw (-0.75,3.25) -- ++(-0.5,0) node[anchor=east,scale=0.75]{$x=H$};
\draw (T2.source) -- ++(0,0) node[ground]{};
\end{scope}
\end{circuitikz}
\caption{NOT poort ge\"implementeerd met CMOS.}
\figlab{notCmos}
\end{figure}
\paragraph{NAND en NOR} Ook NAND en NOR poorten hebben hun equivalent in CMOS. \figref{nandNorCmos} toont hun implementatie. Wat opvalt is dat we redelijk eenvoudig het CMOS equivalent kunnen halen uit een NMOS implementatie. Immers is het onderste gedeelte van de NMOS-schakelaars volledig equivalent met de NMOS-implementatie. We plaatsen eenvoudigweg een PMOS circuit boven de uitgang. Dit circuit werkt met duale logica: indien de NMOS verbindingen parallel zijn, zullen we de PMOS transistoren in serie plaatsen, indien de NMOS transistoren in serie stonden, zetten we de PMOS transistoren in parallel. Verder kunnen we de weerstand ook weglaten. Deze weerstand stond er immers alleen om verschillende poorten aan eenzelfde voeding te kunnen hangen. Nu er echter geen stroom vloeit in om het even welke configuratie van de transistoren, is de weerstand dus nagenoeg nutteloos geworden, en brengt deze hoge kosten met zich mee.
\begin{figure}[hbt]
\centering
\subfigure[NAND]{\begin{circuitikz}[american resistors]
\node [pmoso] (T1) at (-1,1) {};
\node [pmoso] (T2) at (1,1) {};
\node [nmosc] (T3) at (0,-0.5) {};
\node [nmosc] (T4) at (0,-1.5) {};
\draw[->] (T1.source) |- ++(1,0.25) -- ++(0,0.5) node[anchor=west,scale=0.75]{$V_{\mbox{\tiny DD}}\ (H)$};
\draw (T2.source) |- ++(-1,0.25);
\draw (T1.drain) |- ++(1,-0.25) -- (0,0);
\draw (T1.gate) -- (T1.gate -| -2,0) node[anchor=east,scale=0.75]{$x$};
\draw (T2.drain) |- ++(-1,-0.25);
\draw (T2.gate) -- ++(-0.5,0) -- (T3.gate -| -0.75,0);
\draw (T4.gate) -- ++(-0.5,0) -- (T1.gate -| -1.75,0);
\draw (T3.gate) -- (T3.gate -| -2,0) node[anchor=east,scale=0.75]{$y$};
\draw (0,0) -- ++(2,0) node[anchor=west,scale=0.75]{$f$};
\draw (0,0) -- (T3.drain);
\draw (T3.source) -- (T4.drain);
\draw (T4.source) -- ++(0,0) node[ground]{};
\end{circuitikz}
\figlab{nandCmos}}
\subfigure[NOR]{\begin{circuitikz}[american resistors]
\node [nmosc] (T1) at (-1,-1) {};
\node [nmosc] (T2) at (1,-1) {};
\node [pmoso] (T3) at (0,0.5) {};
\node [pmoso] (T4) at (0,1.5) {};
%\draw[<-] (0,4.5) node[anchor=west,scale=0.75]{$V_{\mbox{\tiny DD}}\ (H)$} -- (0,4);
\draw[->] (T4.source) -- ++(0,0.25) node[anchor=west,scale=0.75]{$V_{\mbox{\tiny DD}}\ (H)$};
\draw (T2.drain) |- ++(-1,0.25) -- (0,0);
\draw (T1.drain) |- ++(1,0.25);
\draw (T1.source) |- ++(1,-0.25) -- ++(0,-0.25) node[ground]{};
\draw (T2.source) |- ++(-1,-0.25);
\draw (T1.gate) -- (T1.gate -| -2,0) node[anchor=east,scale=0.75]{$y$};
\draw (T2.source) |- ++(-1,-0.25);
\draw (T2.gate) -- ++(-0.5,0) -- (T3.gate -| -0.75,0);
\draw (T4.gate) -- ++(-0.5,0) -- (T1.gate -| -1.75,0);
\draw (T3.gate) -- (T3.gate -| -2,0) node[anchor=east,scale=0.75]{$x$};
\draw (0,0) -- ++(2,0) node[anchor=west,scale=0.75]{$f$};
\draw (0,0) -- (T3.drain);
\draw (T3.source) -- (T4.drain);
\end{circuitikz}
\figlab{norCmos}}
\caption{NAND en NOR poorten ge\"implementeerd met CMOS.}
\figlab{nandNorCmos}
\end{figure}
\paragraph{Uitgangen verbinden}
\label{term:kortsluiting}
Bij NMOS konden we uitgangen met elkaar verbinden, zonder dat dit problemen met zich meebracht, sterker nog, we konden sommige poorten implementeren aan de hand van verbindingen. Een groot nadeel van CMOS is dat dit niet langer mogelijk is. \figref{cmosFail} toont de reden hiervoor. Op het moment dat de ene ingang een hoog potentiaal heeft, en de andere een laag potentiaal ontstaat er immers een kortsluiting. De stroom is in staat om vanaf de $V_{\mbox{\tiny DD}}$ rechtstreeks de aarding te bereiken. Hoewel de transistoren een minimale weerstand hebben, volstaat deze meestal niet. Het gevolg zijn zeer hoge stromen die het elektronische circuit zouden kunnen beschadigen.
\begin{figure}[hbt]
\centering
\begin{circuitikz}[american resistors]
\def\dx{3};
\node [pmosc] (T1) at (0,3.75) {};
\node [nmoso] (T2) at (0,2.75) {};
\draw[<-] (0,4.5) node[anchor=west,scale=0.75]{$V_{\mbox{\tiny DD}}\ (H)$} -- (0,4);
\draw (0,3.25) -| ++(1,-2) -| (\dx+1,3.25);
\draw (T1.source) -- (0,4);
\draw (T1.drain) -- (T2.drain);
\draw (T1.gate) -- (T1.gate -| -0.75,0) |- (T2.gate);
\draw (-0.75,3.25) -- ++(-0.5,0);
\draw (T2.source) -- ++(0,0) node[ground]{};
\node [pmoso] (T3) at (\dx,3.75) {};
\node [nmosc] (T4) at (\dx,2.75) {};
\draw[<-] (\dx,4.5) node[anchor=west,scale=0.75]{$V_{\mbox{\tiny DD}}\ (H)$} -- (\dx,4);
\draw (\dx,3.25) -- ++(1,0);
\draw (T3.source) -- (\dx,4);
\draw (T3.drain) -- (T4.drain);
\draw (T3.gate) -- (T3.gate -| \dx-0.75,0) |- (T4.gate);
\draw (\dx-0.75,3.25) -- ++(-0.5,0);
\draw (T4.source) -- ++(0,0) node[ground]{};
\draw[dotted,thick,rounded corners,->] (0.35,4.1) -- (0.35,3.6) -| (1.35,1.6) -| (\dx+0.65,2.9) -| (\dx+0.35,2);
\end{circuitikz}
\caption{Kortsluiting bij wired poort implementaties met CMOS.}
\figlab{cmosFail}
\end{figure}
\subsection{Complexe poorten}
\begin{figure}[hbt]
\centering
\subfigure[AOI op poortniveau.]{\begin{tikzpicture}[circuit logic US]
\node[and gate,rotate=90] (A1) at (-0.85,0) {};
\draw (A1.input 1) -- (A1.input 1 |- 0,-1.75) node[anchor=north]{$a$};
\draw (A1.input 2) -- (A1.input 2 |- 0,-2.25) node[anchor=north]{$b$};
\node[and gate,inputs={normal,normal,normal},rotate=90] (A2) at (0.85,0) {};
\draw (A2.input 1) -- (A2.input 1 |- 0,-1.75) node[anchor=north]{$x$};
\draw (A2.input 2) -- (A2.input 2 |- 0,-2) node[anchor=north]{$y$};
\draw (A2.input 3) -- (A2.input 3 |- 0,-2.25) node[anchor=north]{$z$};
\node[nor gate,rotate=90] (NO) at (0,3) {};
\draw (NO.output) -- ++(0,1.5) node[anchor=south]{$f$};
\draw (A1.output) -- ++(0,1.4) -| (NO.input 1);
\draw (A2.output) -- ++(0,1.4) -| (NO.input 2);
\end{tikzpicture}
\figlab{aoiExample}}
\subfigure[AOI op CMOS-niveau.]{
\begin{circuitikz}
\def\dxy{1.5};
\def\dg{0.2};
\def\ds{0.1};
\node[pmoso] (PB1) at (-0.5*\dxy,2*\dxy) {};
\draw (PB1.gate) -- ++(-\dg,0) node[anchor=east,scale=0.75]{$a$};
\node[pmoso] (PB2) at (0.5*\dxy,2*\dxy) {};
\draw (PB2.gate) -- ++(-\dg,0) node[anchor=east,scale=0.75]{$b$};

\coordinate (SPBb) at (0,2.5*\dxy-\ds);
\draw (PB1.source) |- (SPBb);
\draw (PB2.source) |- (SPBb);
\draw[->] (SPBb) -- ++(0,0.5) node[anchor=west,scale=0.75]{$V_{\mbox{\tiny DD}}\ (H)$};
\coordinate (SPBe) at (0,1.5*\dxy+\ds);
\draw (PB1.drain) |- (SPBe);
\draw (PB2.drain) |- (SPBe);

\node[pmoso] (PA1) at (-\dxy,1*\dxy) {};
\draw (PA1.gate) -- ++(-\dg,0) node[anchor=east,scale=0.75]{$x$};
\node[pmoso] (PA2) at (0,1*\dxy) {};
\draw (PA2.gate) -- ++(-\dg,0) node[anchor=east,scale=0.75]{$y$};
\node[pmoso] (PA3) at (\dxy,1*\dxy) {};
\draw (PA3.gate) -- ++(-\dg,0) node[anchor=east,scale=0.75]{$z$};

\coordinate (SPAb) at (0,1.5*\dxy-\ds);
\draw (PA1.source) |- (SPAb);
\draw (PA2.source) |- (SPAb);
\draw (PA3.source) |- (SPAb);
\draw (SPBe) -- (SPAb);

\coordinate (SPAe) at (0,0.5*\dxy+\ds);
\draw (PA1.drain) |- (SPAe);
\draw (PA2.drain) |- (SPAe);
\draw (PA3.drain) |- (SPAe);

\fill (SPAe) circle (0.05 cm);
\fill (SPAb) circle (0.05 cm);

\draw (0,0.25) -- ++(2,0) node[anchor=west,scale=0.75]{$f$};

\node[nmosc] (NA1) at (0.5*\dxy,-0.5*\dxy) {};
\draw (NA1.gate) -- ++(-\dg,0) node[anchor=east,scale=0.75]{$x$};
\node[nmosc] (NA2) at (0.5*\dxy,-1.5*\dxy) {};
\draw (NA2.gate) -- ++(-\dg,0) node[anchor=east,scale=0.75]{$y$};
\node[nmosc] (NA3) at (0.5*\dxy,-2.5*\dxy) {};
\draw (NA3.gate) -- ++(-\dg,0) node[anchor=east,scale=0.75]{$z$};
\node[nmosc] (NB1) at (-0.5*\dxy,-\dxy) {};
\draw (NB1.gate) -- ++(-\dg,0) node[anchor=east,scale=0.75]{$a$};
\node[nmosc] (NB2) at (-0.5*\dxy,-2*\dxy) {};
\draw (NB2.gate) -- ++(-\dg,0) node[anchor=east,scale=0.75]{$b$};

\coordinate (SNAb) at (0,-\ds);

\draw (SPAe) -- (SNAb);

\draw (SNAb) -| (NA1.drain);
\draw (SNAb) -| (NB1.drain);
\draw (NA1.source) -- (NA2.drain);
\draw (NA2.source) -- (NA3.drain);
\draw (NB1.source) -- (NB2.drain);

\coordinate (SNAe) at (0,-3*\dxy+\ds);

\draw (NB2.source) |- (SNAe);
\draw (NA3.source) |- (SNAe);
\draw (SNAe) node[ground]{};
\end{circuitikz}
\figlab{aoiCmos}
}
\subfigure[OAI op poortniveau.]{\begin{tikzpicture}[circuit logic US]
\node[or gate,rotate=90] (A1) at (-0.85,0) {};
\draw (A1.input 1) -- (A1.input 1 |- 0,-1.75) node[anchor=north]{$a$};
\draw (A1.input 2) -- (A1.input 2 |- 0,-2.25) node[anchor=north]{$b$};
\node[or gate,inputs={normal,normal,normal},rotate=90] (A2) at (0.85,0) {};
\draw (A2.input 1) -- (A2.input 1 |- 0,-1.75) node[anchor=north]{$x$};
\draw (A2.input 2) -- (A2.input 2 |- 0,-2) node[anchor=north]{$y$};
\draw (A2.input 3) -- (A2.input 3 |- 0,-2.25) node[anchor=north]{$z$};
\node[nand gate,rotate=90] (NO) at (0,3) {};
\draw (NO.output) -- ++(0,1.5) node[anchor=south]{$f$};
\draw (A1.output) -- ++(0,1.4) -| (NO.input 1);
\draw (A2.output) -- ++(0,1.4) -| (NO.input 2);
\end{tikzpicture}
\figlab{oaiExample}}
\subfigure[OAI op CMOS-niveau.]{
\begin{circuitikz}
\def\dxy{1.5};
\def\dg{0.2};
\def\ds{0.1};
\node[nmosc] (PB1) at (-0.5*\dxy,-2*\dxy) {};
\draw (PB1.gate) -- ++(-\dg,0) node[anchor=east,scale=0.75]{$a$};
\node[nmosc] (PB2) at (0.5*\dxy,-2*\dxy) {};
\draw (PB2.gate) -- ++(-\dg,0) node[anchor=east,scale=0.75]{$b$};

\coordinate (SPBb) at (0,-2.5*\dxy+\ds);
\draw (PB1.source) |- (SPBb);
\draw (PB2.source) |- (SPBb);
\draw (SPBb) node[ground] {};
\coordinate (SPBe) at (0,-1.5*\dxy-\ds);
\draw (PB1.drain) |- (SPBe);
\draw (PB2.drain) |- (SPBe);

\node[nmosc] (PA1) at (-\dxy,-1*\dxy) {};
\draw (PA1.gate) -- ++(-\dg,0) node[anchor=east,scale=0.75]{$x$};
\node[nmosc] (PA2) at (0,-1*\dxy) {};
\draw (PA2.gate) -- ++(-\dg,0) node[anchor=east,scale=0.75]{$y$};
\node[nmosc] (PA3) at (\dxy,-1*\dxy) {};
\draw (PA3.gate) -- ++(-\dg,0) node[anchor=east,scale=0.75]{$z$};

\coordinate (SPAb) at (0,-1.5*\dxy+\ds);
\draw (PA1.source) |- (SPAb);
\draw (PA2.source) |- (SPAb);
\draw (PA3.source) |- (SPAb);
\draw (SPBe) -- (SPAb);

\coordinate (SPAe) at (0,-0.5*\dxy-\ds);
\draw (PA1.drain) |- (SPAe);
\draw (PA2.drain) |- (SPAe);
\draw (PA3.drain) |- (SPAe);

\fill (SPAe) circle (0.05 cm);
\fill (SPAb) circle (0.05 cm);

\draw (0,-0.25) -- ++(2,0) node[anchor=west,scale=0.75]{$f$};

\node[pmoso] (NA1) at (0.5*\dxy,0.5*\dxy) {};
\draw (NA1.gate) -- ++(-\dg,0) node[anchor=east,scale=0.75]{$x$};
\node[pmoso] (NA2) at (0.5*\dxy,1.5*\dxy) {};
\draw (NA2.gate) -- ++(-\dg,0) node[anchor=east,scale=0.75]{$y$};
\node[pmoso] (NA3) at (0.5*\dxy,2.5*\dxy) {};
\draw (NA3.gate) -- ++(-\dg,0) node[anchor=east,scale=0.75]{$z$};
\node[pmoso] (NB1) at (-0.5*\dxy,\dxy) {};
\draw (NB1.gate) -- ++(-\dg,0) node[anchor=east,scale=0.75]{$a$};
\node[pmoso] (NB2) at (-0.5*\dxy,2*\dxy) {};
\draw (NB2.gate) -- ++(-\dg,0) node[anchor=east,scale=0.75]{$b$};

\coordinate (SNAb) at (0,\ds);

\draw (SPAe) -- (SNAb);

\draw (SNAb) -| (NA1.drain);
\draw (SNAb) -| (NB1.drain);
\draw (NA1.source) -- (NA2.drain);
\draw (NA2.source) -- (NA3.drain);
\draw (NB1.source) -- (NB2.drain);

\coordinate (SNAe) at (0,3*\dxy-\ds);

\draw (NB2.source) |- (SNAe);
\draw (NA3.source) |- (SNAe);
\draw[->] (SNAe) -- ++(0,0.5) node[anchor=west,scale=0.75]{$V_{\mbox{\tiny DD}}\ (H)$};
\end{circuitikz}
\figlab{oaiCmos}
}
\caption{AND-OR-Inverter (AOI) en OR-AND-Inverter (OAI) in CMOS.}
\figlab{aoiOai}
\end{figure}
In hoofdstuk \ref{ch:basis} hadden we het reeds over de XOR-poort. Een poort die 1 teruggeeft als de twee ingangen niet gelijk zijn aan elkaar. Deze poort hadden we toen ge\"implementeerd met een ingewikkeld schema van poorten. Omdat we echter op transistor niveau werken, kunnen we verschillende complexe poorten toch relatief simpel implementeren. In de volgende subsecties zullen we eerst de \termen{AND-OR-Invert (AOI)} en \termen{OR-AND-Invert (OAI)} implementeren. Deze schakelingen worden relatief vaak gebruikt, om bijvoorbeeld XOR en \termen{XNOR} poorten te implementeren.
\subsubsection{AND-OR-Invert (AOI)}
Een veelgebruikt component is een AND-OR-Inverter. Dit is een component die we op poort-niveau kunnen beschrijven als een tweelagige structuur waarbij de ingangen eerst door een reeks AND-poorten gaan. De uitgangen van de AND-poorten vormen op hun beurt de ingangen van een NOR-poort. Een concreet voorbeeld hiervan staat op \figref{aoiExample}. Indien we echter de schakeling zouden bouwen zoals we dit voorstellen op poortniveau\footnote{We substitueren dus iedere poort door zijn equivalent in CMOS.}, hebben we 18 transistoren nodig. Een effici\"entere manier is het implementeren van een nieuw component de AOI zoals op \figref{aoiCmos}. Hierbij hebben we slechts 10 transistoren nodig. Bovendien hebben we de implementatie gereduceerd tot \'e\'en laag. Hierdoor wordt de vertraging van de component ook teruggedrongen.
\subsubsection{OR-AND-Invert (OAI)}
De tegenhanger van de AND-OR-Inverter is de OR-AND-Inverter. \figref{oaiExample} toont een voorbeeld van dit type component. Naar analogie met de AOI bestaat deze component op poortniveau opnieuw uit twee lagen, ditmaal gaan de ingangen door een reeks OR poorten waarbij hun uitgangen dan weer de invoer van een NAND poort vormen. Opnieuw kunnen we door een implementatie in CMOS het aantal transistoren van 18 naar 10 reduceren.
\subsubsection{Andere veelgebruikte poorten}
Naast de NOT, NAND, NOR, AOI en OAI poorten, worden er nog enkele andere poorten frequent gebruikt. De implementatie van deze poorten wordt weergegeven in \figref{alternativeGatesCmos}. In de volgende paragrafen wordt hun nut en werking kort toegelicht.
\begin{figure}[hbt]
\centering
\subfigure[Buffer]{\begin{tikzpicture}[circuit logic US]
\node [buffer gate,scale=0.75] (B) at (-1,0) {};
\node [not gate,scale=0.75] (N1) at (1,0) {};
\node [not gate,scale=0.75] (N2) at (2,0) {};
\draw (N2.output) -- ++(0.25,0);
\draw (N1.output) -- (N2.input);
\draw (N1.input) -- ++(-0.25,0);
\draw (B.output) -- ++(0.25,0);
\draw (B.input) -- ++(-0.25,0);
\draw (0,0) node{$\equiv$};
\begin{scope}[xshift=1 cm, yshift=-3 cm,xscale=0.75]
\coordinate (F0) at (-3,0);
\coordinate (I) at (-2,0);
\draw (F0) node[anchor=east,scale=0.75]{$x$} -- (I);
\node [pmoso] (P1) at (-1,0.75) {};
\draw (I) |- (P1.gate);
\draw[->] (P1.source) -- ++(0,0.5) node[anchor=west,scale=0.75]{$V_{\mbox{\tiny DD}}$};
\node [nmosc] (N1) at (-1,-0.75) {};
\draw (I) |- (N1.gate);
\draw (N1.source) node [ground] {};
\draw (N1.drain) -- (P1.drain);
\node [pmosc] (P2) at (1,0.75) {};
\coordinate (F1) at (-1,0);
\coordinate (O) at (0,0);
\draw (F1) -- (O);
\draw[->] (P2.source) -- ++(0,0.5) node[anchor=west,scale=0.75]{$V_{\mbox{\tiny DD}}$};
\draw (O) |- (P2.gate);
\node [nmoso] (N2) at (1,-0.75) {};
\draw (O) |- (N2.gate);
\draw (N2.source) node [ground] {};
\draw (N2.drain) -- (P2.drain);
\coordinate (F2) at (1,0);
\draw (F2) -- ++(1,0) node[anchor=west,scale=0.75]{$f$};
\end{scope}
\end{tikzpicture}
\figlab{bufferCmos}
}
\subfigure[AND-poort]{\begin{tikzpicture}[circuit logic US]
\node [and gate,scale=0.75] (B) at (-1,0) {};
\node [nand gate,scale=0.75] (N1) at (1,0) {};
\node [not gate,scale=0.75] (N2) at (2,0) {};
\draw (N2.output) -- ++(0.25,0);
\draw (N1.output) -- (N2.input);
\draw (N1.input 1) -- ++(-0.25,0);
\draw (N1.input 2) -- ++(-0.25,0);
\draw (B.output) -- ++(0.25,0);
\draw (B.input 1) -- ++(-0.25,0);
\draw (B.input 2) -- ++(-0.25,0);
\draw (0,0) node{$\equiv$};
\begin{scope}[xshift=0.25 cm, yshift=-3 cm,xscale=0.75]
\node [pmoso] (P1) at (-1,0.75) {};
\draw (P1.gate) -- ++(-0.5,0) node[anchor=east,scale=0.75]{$x$};
\node [pmoso] (P1b) at (-1,1.75) {};
\draw (P1b.gate) -- ++(-0.5,0) node[anchor=east,scale=0.75]{$y$};
\draw (P1b.drain) -- (P1.source);
\draw[->] (P1b.source) -- ++(0,0.5) node[anchor=west,scale=0.75]{$V_{\mbox{\tiny DD}}$};
\node [nmosc] (N1) at (-2,-0.65) {};
\draw (N1.gate) -- ++(-0.5,0) node[anchor=east,scale=0.75]{$x$};
\draw (N1.source) |- ++(1,-0.1) node [ground] {};
\coordinate (O) at (1,0);
\coordinate (Fb) at (-1,-0.2);
\coordinate (F1) at (-1,0);
\draw (P1.drain) -- (F1) -- (Fb) -| (N1.drain);
\node [nmosc] (N1b) at (0,-0.65) {};
\draw (N1b.gate) -- ++(-0.5,0) node[anchor=east,scale=0.75]{$y$};
\draw (Fb) -| (N1b.drain);
\draw (N1b.source) |- ++(-1,-0.1);
\node [pmosc] (P2) at (2,0.75) {};
\draw (F1) -- (O);
\draw[->] (P2.source) -- ++(0,0.5) node[anchor=west,scale=0.75]{$V_{\mbox{\tiny DD}}$};
\draw (O) |- (P2.gate);
\node [nmoso] (N2) at (2,-0.75) {};
\draw (O) |- (N2.gate);
\draw (N2.source) node [ground] {};
\draw (N2.drain) -- (P2.drain);
\coordinate (F2) at (2,0);
\draw (F2) -- ++(1,0) node[anchor=west,scale=0.75]{$f$};
\end{scope}
\end{tikzpicture}
\figlab{andCmos}
}
\subfigure[OR-poort]{\begin{tikzpicture}[circuit logic US]
\node [or gate,scale=0.75] (B) at (-1,0) {};
\node [nor gate,scale=0.75] (N1) at (1,0) {};
\node [not gate,scale=0.75] (N2) at (2,0) {};
\draw (N2.output) -- ++(0.25,0);
\draw (N1.output) -- (N2.input);
\draw (N1.input 1) -- ++(-0.25,0);
\draw (N1.input 2) -- ++(-0.25,0);
\draw (B.output) -- ++(0.25,0);
\draw (B.input 1) -- ++(-0.25,0);
\draw (B.input 2) -- ++(-0.25,0);
\draw (0,0) node{$\equiv$};
\begin{scope}[xshift=0.25 cm, yshift=-3 cm,xscale=0.75]
\node [pmoso] (P1) at (-2,1.65) {};
\draw (P1.gate) -- ++(-0.5,0) node[anchor=east,scale=0.75]{$x$};
\node [pmoso] (P1b) at (0,1.65) {};
\draw (P1b.gate) -- ++(-0.4,0) node[anchor=east,scale=0.75]{$y$};
\draw (P1.source) |- ++(1,0.1);
\draw[->] (P1b.source) |- ++(-1,0.1) -- ++(0,0.5) node[anchor=west,scale=0.75]{$V_{\mbox{\tiny DD}}$};
\node [nmosc] (N1) at (-1,0.25) {};
\draw (N1.gate) -- ++(-0.5,0) node[anchor=east,scale=0.75]{$x$};
\coordinate (O) at (1,0);
\coordinate (Fb) at (-1,1.2);
\draw (P1b.drain) |- (Fb);
\coordinate (F1) at (-1,1);
\draw (P1.drain) |- (Fb) -- (F1) -- (N1.drain);
\node [nmosc] (N1b) at (-1,-0.75) {};
\draw (N1.source) -- (N1b.drain);
\draw (N1b.source) node [ground] {};
\draw (N1b.gate) -- ++(-0.5,0) node[anchor=east,scale=0.75]{$y$};
\node [pmosc] (P2) at (2,0.75) {};
\draw (F1) -- ++(1,0) |- (O);
\draw[->] (P2.source) -- ++(0,0.5) node[anchor=west,scale=0.75]{$V_{\mbox{\tiny DD}}$};
\draw (O) |- (P2.gate);
\node [nmoso] (N2) at (2,-0.75) {};
\draw (O) |- (N2.gate);
\draw (N2.source) node [ground] {};
\draw (N2.drain) -- (P2.drain);
\coordinate (F2) at (2,0);
\draw (F2) -- ++(1,0) node[anchor=west,scale=0.75]{$f$};
\end{scope}
\end{tikzpicture}
\figlab{orCmos}
}
\subfigure[XOR-poort]{\begin{tikzpicture}[circuit logic US]
\node [xor gate,scale=0.75] (X) at (-1,0) {};
\draw (X.output) -- ++(0.25,0);
\draw (X.input 1) -- ++(-0.25,0);
\draw (X.input 2) -- ++(-0.25,0);
\draw (0,0) node{$\equiv$};
\node [nand gate,scale=0.75] (NA) at (3,0) {};
\draw (NA.output) -- ++(0.5,0);
\node [or gate,scale=0.75] (O1) at (2,0.5) {};
\node [or gate,scale=0.75] (O2) at (2,-0.5) {};
\draw (O1.output) -- ++(0.1,0) |- (NA.input 1);
\draw (O2.output) -- ++(0.1,0) |- (NA.input 2);
\node [not gate,scale=0.35] (N1) at (O1.input 1 -| 1,0) {};
\node [not gate,scale=0.35] (N2) at (O2.input 1 -| 1,0) {};
\draw (N1.output) -- (O1.input 1);
\draw (N2.output) -- (O2.input 1);
\draw (N1.input) -- ++(-0.5,0);
\draw (N2.input) -- ++(-0.5,0);
\draw (N2.input -| 0.8,0) |- (O1.input 2);
\draw (N1.input -| 0.65,0) |- (O2.input 2);
\begin{scope}[xshift=0.25 cm, yshift=-5.5 cm,xscale=0.75,yscale=0.8]
\def\dxs{4};
\foreach\l/\lt/\x in {A/x/0,B/y/\dxs} {
  \coordinate (F0\l) at (-3,\x+0.1);
  \coordinate (I) at (-2,\x+0.1);
  \fill (I) circle (0.06 cm);
  \coordinate (O\l0) at (1,\x+0.1);
  \draw (F0\l) node[anchor=east,scale=0.75]{$\lt$} -- (I) -- (O\l0);
  \node [pmoso] (P1\l) at (-1,0.75+\x) {};
  \draw (I) |- (P1\l.gate);
  \draw[->] (P1\l.source) -- ++(0,0.5) node[anchor=west,scale=0.75]{$V_{\mbox{\tiny DD}}$};
  \node [nmosc] (N1\l) at (-1,\x-0.75) {};
  \draw (I) |- (N1\l.gate);
  \draw (N1\l.source) node [ground] {};
  \draw (N1\l.drain) -- (P1\l.drain);
  \coordinate (F1\l) at (-1,\x-0.1);
  \coordinate (O\l1) at (0,\x-0.1);
  \draw (F1\l) -- (O\l1);
}
\begin{scope}[xshift=3 cm]
\node[pmoso] (PaAA) at (-1,\dxs+0.65) {};
\node[pmoso] (PaBA) at (1,\dxs+0.65) {};
\node[pmosc] (PaAB) at (-1,\dxs-0.95) {};
\node[pmosc] (PaBB) at (1,\dxs-0.95) {};
\draw[->] (PaAA.source) |- ++(1,0.1) -- ++(0,0.5) node[anchor=west,scale=0.75]{$V_{\mbox{\tiny DD}}$};
\draw (PaBA.source) |- ++(-1,0.1);
\draw (PaAA.drain) -- (PaAB.source);
\draw (PaBA.drain) -- (PaBB.source);

\coordinate (Mid) at (0,0.5*\dxs);
\coordinate (MidT) at (0,0.5*\dxs+0.4);
\draw (PaAB.drain) |- (MidT);
\draw (PaBB.drain) |- (MidT);
\coordinate (MidB) at (0,0.5*\dxs-0.4);
\draw (MidB) -- (MidT);
\draw (Mid) -- ++(2,0) node[anchor=west,scale=0.75]{$f$};

\node[nmosc] (NaAA) at (-1,0.95) {};
\node[nmoso] (NaBA) at (1,0.95) {};
\draw (NaAA.drain) |- (MidB);
\draw (NaBA.drain) |- (MidB);
\node[nmoso] (NaAB) at (-1,-0.65) {};
\node[nmosc] (NaBB) at (1,-0.65) {};
\draw (NaAB.source) |- ++(1,-0.1) node[ground]{};
\draw (NaBB.source) |- ++(-1,-0.1);
\coordinate (Nmt) at (0,0.25);
\coordinate (Nmb) at (0,0.05);
\draw (Nmt) -- (Nmb);
\draw (Nmb) -| (NaAB.drain);
\draw (Nmb) -| (NaBB.drain);
\draw (Nmt) -| (NaAA.source);
\draw (Nmt) -| (NaBA.source);
\draw (OB0) |- (PaAA.gate);
\draw (OB1) -- ++(2.75,0) |- (PaBB.gate);
\coordinate (yn) at (-2.25,0 |- OB0);
\coordinate (yi) at (-0.25,0 |- PaBB.gate);
\coordinate (xn) at (OA0 |- NaAA.gate);
\coordinate (xnc) at (xn -| -1.85,0);
\draw (xnc) |- ++(1.5,3.1) |- (PaBA.gate);
\draw (OA0) |- (NaAA.gate);
\draw (OA1) |- (NaAB.gate);
\draw (OA1) |- (PaAB.gate);
\draw (yi) |- (NaBA.gate);
\draw (yn) |- (0.25,-1.5) |- (NaBB.gate);
\end{scope}
\end{scope}
\end{tikzpicture}
\figlab{xorCmos}
}
\subfigure[XNOR-poort]{\begin{tikzpicture}[circuit logic US]
\node [xnor gate,scale=0.75] (X) at (-1,0) {};
\draw (X.output) -- ++(0.25,0);
\draw (X.input 1) -- ++(-0.25,0);
\draw (X.input 2) -- ++(-0.25,0);
\draw (0,0) node{$\equiv$};
\node [nor gate,scale=0.75] (NA) at (3,0) {};
\draw (NA.output) -- ++(0.5,0);
\node [and gate,scale=0.75] (O1) at (2,0.5) {};
\node [and gate,scale=0.75] (O2) at (2,-0.5) {};
\draw (O1.output) -- ++(0.1,0) |- (NA.input 1);
\draw (O2.output) -- ++(0.1,0) |- (NA.input 2);
\node [not gate,scale=0.35] (N1) at (O1.input 1 -| 1,0) {};
\node [not gate,scale=0.35] (N2) at (O2.input 1 -| 1,0) {};
\draw (N1.output) -- (O1.input 1);
\draw (N2.output) -- (O2.input 1);
\draw (N1.input) -- ++(-0.5,0);
\draw (N2.input) -- ++(-0.5,0);
\draw (N2.input -| 0.8,0) |- (O1.input 2);
\draw (N1.input -| 0.65,0) |- (O2.input 2);

\begin{scope}[xshift=0.25 cm, yshift=-5.5 cm,xscale=0.75,yscale=0.8]
\def\dxs{4};
\foreach\l/\lt/\x in {A/x/0,B/y/\dxs} {
  \coordinate (F0\l) at (-3,\x+0.1);
  \coordinate (I) at (-2,\x+0.1);
  \fill (I) circle (0.06 cm);
  \coordinate (O\l0) at (1,\x+0.1);
  \draw (F0\l) node[anchor=east,scale=0.75]{$\lt$} -- (I) -- (O\l0);
  \node [pmoso] (P1\l) at (-1,0.75+\x) {};
  \draw (I) |- (P1\l.gate);
  \draw[->] (P1\l.source) -- ++(0,0.5) node[anchor=west,scale=0.75]{$V_{\mbox{\tiny DD}}$};
  \node [nmosc] (N1\l) at (-1,\x-0.75) {};
  \draw (I) |- (N1\l.gate);
  \draw (N1\l.source) node [ground] {};
  \draw (N1\l.drain) -- (P1\l.drain);
  \coordinate (F1\l) at (-1,\x-0.1);
  \coordinate (O\l1) at (0,\x-0.1);
  \draw (F1\l) -- (O\l1);
}
\begin{scope}[xshift=3 cm]
\node[pmoso] (PaAA) at (-1,\dxs+0.65) {};
\node[pmosc] (PaBA) at (1,\dxs+0.65) {};
\node[pmosc] (PaAB) at (-1,\dxs-0.95) {};
\node[pmoso] (PaBB) at (1,\dxs-0.95) {};
\draw[->] (PaAA.source) |- ++(1,0.1) -- ++(0,0.5) node[anchor=west,scale=0.75]{$V_{\mbox{\tiny DD}}$};
\draw (PaBA.source) |- ++(-1,0.1);

\coordinate (Pmt) at (0,\dxs-0.25);
\coordinate (Pmb) at (0,\dxs-0.05);
\draw (Pmt) -- (Pmb);
\draw (Pmb) -| (PaAA.drain);
\draw (Pmb) -| (PaBA.drain);
\draw (Pmt) -| (PaAB.source);
\draw (Pmt) -| (PaBB.source);

\coordinate (Mid) at (0,0.5*\dxs);
\coordinate (MidT) at (0,0.5*\dxs+0.4);
\draw (PaAB.drain) |- (MidT);
\draw (PaBB.drain) |- (MidT);
\coordinate (MidB) at (0,0.5*\dxs-0.4);
\draw (MidB) -- (MidT);
\draw (Mid) -- ++(2,0) node[anchor=west,scale=0.75]{$f$};

\node[nmosc] (NaAA) at (-1,0.95) {};
\node[nmoso] (NaBA) at (1,0.95) {};
\draw (NaAA.drain) |- (MidB);
\draw (NaBA.drain) |- (MidB);
\node[nmoso] (NaAB) at (-1,-0.65) {};
\node[nmosc] (NaBB) at (1,-0.65) {};
\draw (NaAA.source) -- (NaAB.drain);
\draw (NaBA.source) -- (NaBB.drain);

\draw (NaAB.source) |- ++(1,-0.1) node[ground]{};
\draw (NaBB.source) |- ++(-1,-0.1);

\coordinate (MidOB1) at (OB1 -| -2.5,0);
\draw (OB1) -- (MidOB1);
\draw (PaAB.gate -| MidOB1) |- (PaAB.gate);
\draw (MidOB1) |- (Mid -| -0.25,0) |- (NaBA.gate);
\draw (OB0) |- (PaAA.gate);
\draw (OB0) |- (NaAA.gate);
\draw (OA1) |- (NaAB.gate);
\draw (OA0) -- (0.25,0 |- OA0) |- (PaBB.gate);
\draw (0.25,0 |- OA0) |- (NaBB.gate);
\draw (OA1) |- (0.25,\dxs+1.5) |- (PaBA.gate);

%\draw (OB0) |- (PaAA.gate);
%\draw (OB1) -- ++(2.75,0) |- (PaBB.gate);
%\coordinate (yn) at (-2.25,0 |- OB0);
%\coordinate (yi) at (-0.25,0 |- PaBB.gate);
%\coordinate (xn) at (OA0 |- NaAA.gate);
%\coordinate (xnc) at (xn -| -1.85,0);
%\draw (xnc) |- ++(1.5,3.1) |- (PaBA.gate);
%\draw (OA0) |- (NaAA.gate);
%\draw (OA1) |- (NaAB.gate);
%\draw (OA1) |- (PaAB.gate);
%\draw (yi) |- (NaBA.gate);
%\draw (yn) |- (0.25,-1.5) |- (NaBB.gate);
\end{scope}
\end{scope}

\end{tikzpicture}
\figlab{xnorCmos}
}
\caption{Implementatie van populaire alternatieve poorten in CMOS.}
\figlab{alternativeGatesCmos}
\end{figure}
\paragraph{Buffer}
Een \termen{buffer} of \termen{driver} is een speciaal type poort met \'e\'en ingang waarbij de uitgang dezelfde waarde heeft als de ingang. Men zou dus in principe een buffer kunnen vervangen door een draad. Toch wordt een buffer frequent gebruikt, indien een bepaalde draad verbonden wordt met vele ingangen van andere poorten. In dat geval immers, zou de spanning op deze draad verlaagd worden. Een buffer dient dus om bij complexe circuits een signaal over een groot aantal componenten te kunnen verspreiden, waar een eenvoudige vertakking zou falen. \figref{bufferCmos} toont de notatie en de implementatie van een buffer. We implementeren een buffer meestal met twee opeenvolgende inverters.
\paragraph{AND-poort}
We hebben een AND-poort reeds voldoende ge\"introduceerd om te weten wat deze component doet. Als we naar \figref{andCmos} kijken, zien we dat we deze poort implementeren door een NOT-poort aan de uitgang van een NAND-poort te plaatsen. Dit verklaart ook meteen waarom we de voorkeur zullen geven aan inverterende poorten: een NAND werkt sneller en is bovendien twee transistoren goedkoper.
\paragraph{OR-poort}
Analoog aan een AND-poort implementeren we een OR-poort ook met een inverter na een NOR-poort. \figref{orCmos} toont hierbij de implementatie. De conclusies de we trokken voor AND-poorten gelden uiteraard ook voor een OR-poort.
\paragraph{XOR-poort}
Ook de XOR-poort werd reeds ge\"implementeerd op poortniveau. Om dit te implementeren werd toen een complex netwerk van poorten opgezet. Hierbij werd echter impliciet een AND-OR-Inverter gebruikt. Dit houdt in dat we een XOR poort relatief goedkoop kunnen implementeren. \figref{xorCmos} toont dat deze implementatie een OR-AND-Inverter vraagt en twee invertoren. In totaal hebben we dus 12 transistoren nodig. Indien we deze schakeling zouden implementeren zoals op \figref{complexGatesXor} op pagina \pageref{fig:complexGatesXor} zouden we 22 transistoren gebruiken.
\paragraph{XNOR-poort}
Een laatste populaire poort is de XNOR-poort. Deze poort is niets anders dan de ge\"inverteerde van de XOR-poort. Door eenvoudigweg de AND-OR inverter te vervangen door een OR-AND-Inverter kunnen we deze poort implementeren zoals op \figref{xnorCmos}. De XNOR poort is dan ook de enige poort die even goedkoop is als zijn invers.
\section{Negatieve logica}
\label{s:negativeLogic}
We hebben reeds kort negatieve logica behandeld. Nu we echter poorten ge\"implementeerd hebben, zijn we beter in staat om te vatten wat negatieve logica doet. Indien we de NAND-poort beschouwen op \figref{nandCmos}, kunnen we rekenen met high en low. Tabel \ref{tbl:positiveNegativeLogicHighLow}
\begin{table}[hbt]
\centering
\subtable[High en Low]{
\begin{tabular}{cc|c}
$x$&$y$&$f$\\\hline
L&L&H\\
L&H&H\\
H&L&H\\
H&H&L
\end{tabular}
\label{tbl:positiveNegativeLogicHighLow}
}
\subtable[Positief]{
\begin{tabular}{cc|c}
$x$&$y$&$f$\\\hline
0&0&1\\
0&1&1\\
1&0&1\\
1&1&0
\end{tabular}
\label{tbl:positiveNegativeLogicPositive}
}
\subtable[Negatief]{
\begin{tabular}{cc|c}
$x$&$y$&$f$\\\hline
1&1&0\\
1&0&0\\
0&1&0\\
0&0&1
\end{tabular}
\label{tbl:positiveNegativeLogicNegative}
}
\caption{Verschil tussen positieve en negatieve logica.}
\label{tbl:positiveNegativeLogic}
\end{table}
toont de functie met high en low signalen. Indien we deze signalen interpreteren met positieve logica zoals in tabel \ref{tbl:positiveNegativeLogicPositive} bekomen we zoals verwacht de NAND-poort. Indien we negatieve logica toepassen zoals in tabel \ref{tbl:positiveNegativeLogicNegative} bekomen we een NOR-poort.Bijgevolg kunnen we dus stellen dat positieve en negatieve logica elkaars duale vorm zijn (zie \ref{ss:theoremasPropertiesBooleanAlgebra}).
\section{Technologie\"en}
Niet elk toestel wordt volledig ontworpen en ontwikkeld. Voor goedkope toestellen in beperkte oplages zal men vaak geen nieuwe specifieke chips ontwikkelen. Meestal maakt men gebruik van reeds bestaande componenten, die men vervolgens op een printplaat combineert. Nog een alternatief zijn programmeerbare chips. Hierbij wordt de logica in de chip geprogrammeerd. Dit laat toe programmeerbare chips in grote oplages te produceren die dan vervolgens voor allerhande verschillende toepassingen gebruikt worden. Tot slot worden sommige chips ook volledig zelf geassembleerd. We overlopen eerst kort de drie vormen, waarna we ze in detail bespreken in de volgende subsecties.
\begin{itemize}
 \item \termen{Specifieke Chips (ASIC)} (\S\ref{ss:specifiekeChips}): Hierbij maken we de chips volledig zelf. Deze techniek is echter duur, omdat er bijvoorbeeld een masker moet aangemaakt worden. Bijgevolg is deze techniek enkel winstgevend bij grote volumes.
 \item \termen{Programmeerbare Chips} (\S\ref{ss:programmeerbareChips}): Dit zijn chips waarbij de logica geprogrammeerd kan worden. Meestal is dit echter volgens het ``\termen{Write Once Read Many (WORM)}'' principe. Deze chips kunnen een redelijke complexiteit aan en bevatten anno 2010 2 miljoen logische cellen. Meestal worden deze chips dan ook gebruikt voor prototypes en voor apparaten met kleine tot middelgrote oplages\footnote{minder dan 100 000 stukken per jaar.}.
 \item \termen{Standaard Chips}: Hierbij worden simpele chips gekocht zoals poorten (SSI) en chips die eenvoudige functies vervullen (MSI/LSI). Het enige wat men nog moet doen is deze componenten met elkaar verbinden. De zogenaamde ``\termen{Glue Logic}''. Het gevolg is echter dat we enkel circuits kunnen bouwen met een beperkte complexiteit. Een typevoorbeeld van zo'n chips is bijvoorbeeld de 744 chip. Deze bevat 6 NOT poorten.
\end{itemize}
\subsection{Specifieke chips}
\label{ss:specifiekeChips}
Bij specifieke chips ontwerpen we zelf het volledige ge\"integreerde circuit. Deze techniek is dan ook zeer arbeidsintensief. Er bestaan drie verschillende technieken om specifieke chips te maken: maatwerk, standaard cellen en gate-arrays. Deze technieken worden in de volgende subsubsecties besproken.
\subsubsection{Maatwerk}
Bij \termen{maatwerk} zullen we elke transistor en verbinding zelf ontwerpen. Deze componenten stellen we dan voor als een set rechthoeken, die we op de chipoppervlakte plaatsen. Deze techniek zal in het algemeen tot het meest optimale ontwerp leiden qua snelheid, vermogenverbruik en afmetingen. Toch is deze techniek niet haalbaar voor complexe systemen. Deze techniek wordt wel vaak toegepast bij het ontwerpen van componenten voor een bibliotheek. In dat geval zal men bijvoorbeeld een opteller zo effici\"ent mogelijk implementeren, zodat complexere systemen die een opteller nodig hebben, een zeer effici\"ente implementatie kunnen gebruiken. Nog een groot nadeel van deze techniek is, dat de technologie verder evolueert, de transistoren worden telkens kleiner, en om de 18 maanden is een volledige herimplementatie nodig. Deze manier van werken wordt ook wel ``\termen{Custom Design}'' genoemd.
\subsubsection{Standaard cellen}
Een effici\"entere manier van werken is met de zogenoemde ``\termen{standaard cellen}''. Deze techniek is min of meer analoog met het maatwerk. Alleen gebruiken we hier componenten uit een bibliotheek als cellen, in plaats van transistoren. Elk van deze cellen heeft een vaste hoogte, en een variabele breedte. Bovendien wordt er in de hoogte ruimte voorzien voor bedrading. Elke logische cel heeft enkele ingangen aan de bovenkant, en enkele uitgangen aan de onderkant van de cel. Het komt er dus alleen nog op aan de cellen op een interessante manier te plaatsen, de zogenoemde ``\termen{placement}''. En het leggen van bedrading, wat ``\termen{routing}'' genoemd wordt. Deze taken dienen met de nodige zorg te gebeuren. Componenten die veel met elkaar interageren worden beter dicht bij elkaar gezet, om de snelheid op te drijven, en bovendien is het aantal lagen voor bekabeling heel beperkt. Deze techniek laat echter wel toe om snel complexe bouwblokken te ontwerpen, en bovendien laat deze methode toe dat fabrikanten cellen kunnen optimaliseren. \figref{standardcells} toont hoe het ontwerp van deze standaard cellen er ongeveer uitziet.
\begin{figure}[hbt]
\centering
\begin{tikzpicture}
\foreach \y in {0,1,2} {
  \draw[thick] (0,1.5*\y) -- (0,0.5+1.5*\y);
}
\foreach \y/\xa/\xb in {0/0/5,0/5/8,0/8/10,0/10/13,0/13/16,1/0/1,1/1/3,1/3/9,1/9/11,1/11/13,1/13/14,1/14/16,2/0/2,2/2/4,2/4/5,2/5/7,2/7/9,2/9/11,2/11/16} {
  \draw[thick] (0.3*\xa,1.5*\y) -- (0.3*\xb,1.5*\y) -- (0.3*\xb,0.5+1.5*\y) -- (0.3*\xa,0.5+1.5*\y);
}
\foreach \ya/\xa/\ym/\yb/\xb in {2/1/1.5/1/4,2/3/1.25/0/2,2/6/0.5/0/14.5,2/8/1.5/1/8,2/13.5/1.5/1/10,1/4/0.5/0/3,1/10/0.25/0/6.5} {
  \draw (0.3*\xa,1.5*\ya) -- (0.3*\xa,0.25+1.5*\ym) -| (0.3*\xb,0.5+1.5*\yb);
}
\end{tikzpicture}
\caption{Ontwerp met standaard cellen.}
\figlab{standardcells}
\end{figure}
\subsubsection{Gate-array}
Een techniek die nog minder ontwerp-kosten met zich meebrengt is de \termen{Gate Array}. Hierbij beschouwen we een tweedimensionaal rooster van identieke poorten\footnote{Meestal worden hiervoor NAND-poorten gebruikt. Bijvoorbeeld een 3-input NAND-poort.}, de zogenoemde ``\termen{Sea of Gates}''. In dit rooster heeft elke poort identieke afmetingen, en wordt er ruimte gelaten voor bedrading. Opnieuw bevinden zich alle ingangen bovenaan en de uitgangen onderaan. Enkel de bedrading is vervolgens nog uniek aan het product. Dit heeft niet alleen het voordeel van goedkoop ontwerp, de rooster kunnen in massa geproduceerd worden, om daarna elk tot een specifieke circuit uit te groeien. Enkel de metallisatielaag\footnote{De laag waar de verbindingen gelegd worden.} is dus uniek.
\subsection{Programmeerbare chips}
\label{ss:programmeerbareChips}
Prototypes waar nog fouten uitgehaald moeten worden, of elektronica in bijvoorbeeld computers waar de firmware kan veranderen, worden meestal ge\"implementeerd met \termen{programmeerbare chips}. Maar ook onder de programmeerbare chips bestaan nog verschillende technologie\"en. Vooral de programmeertechnieken zijn zeer divers. Hieronder worden de meest courante technieken opgesomd. Vervolgens worden deze technieken verder toegelicht in de subsubsecties.
\begin{itemize}
 \item \termen{Zekeringen}: Hierbij bevat de chip een groot aantal zekeringen. We kunnen dan vervolgens een deel van deze zekeringen doorbranden, om de logica te implementeren. Deze techniek is irreversibel. Immers kan een doorgebrande zekeringen niet hersteld worden. Weliswaar blijft bijprogrammeren\footnote{In een tweede iteratie andere zekeringen doorbranden.} wel mogelijk.%TODO: doorgebrande weglaten?
 \item \termen{Flash-programmeerbaar}: Hierbij kunnen we verbindingen openen en sluiten met transistoren waarbij de gate opgeladen kan worden. Hierdoor wordt herprogrammeren mogelijk, dit is echter traag, en bovendien zal na enige tijd het flash geheugen niet meer herprogrammeerbaar zijn.
 \item \termen{Geheugen-programmeerbaar}: Hierbij bevat de chip een geheugen component, meestal in \termen{SRAM}. De transistoren die de verbindingen bepalen worden dan gekoppeld aan dit geheugen. Het voordeel hierbij is dat we makkelijk en snel de chip kunnen programmeren, en dit kan zelfs dynamisch\footnote{De chip kan tijdens uitvoering zijn gedrag aanpassen.}. Het nadeel is dat dit geheugen telkens bij het aanleggen van voedingsspanning weer opnieuw ingeladen moet worden. Het typevoorbeeld van deze techniek is een \termen{Field Programmable Gate Array (FPGA)}.
\end{itemize}
\subsubsection{Programmable Logic Array (PLA)}
In sectie \ref{s:synthese} zagen we reeds dat we alle logische functies kunnen maken met een \termen{Sum-of-Products (SOP)}. Een \termen{Programmable Logic Array (PLA)} is gebaseerd op dit idee. Deze chip bevat twee rasters van poorten: een \termen{AND-matrix} en een \termen{OR-matrix}. Als ingangen kunnen we vervolgens zowel een variable als zijn inverse aanleggen. \figref{plaSchema}
\begin{figure}[hbt]
\centering
\begin{tikzpicture}[circuit logic US]
\foreach \y in {0,1,...,7} {
  \node[and gate, inputs={normal,normal,normal,normal,normal,normal,normal,normal},scale=0.5] (A\y) at (-1,0.8*\y) {$A\y$};
  \node[or gate, inputs={normal,normal,normal,normal,normal,normal,normal,normal},rotate=-90,scale=0.5] (O\y) at (0.8*\y,-1) {$O\y$};
  \draw (O\y.output) -- ++(0,-0.5) node[scale=0.75,anchor=north]{$f_\y$};
}
\foreach \t/\xa/\xb/\xc/\xd in {0/0/1/8/7,1/2/3/6/5,2/4/5/4/3,3/6/7/2/1} {
  \node[not gate,scale=0.5,rotate=-90] (N\t) at (-6.8+1.6*\t,6.6) {$N\t$};
  \draw[very thin] (N\t.input) |- ++(-0.8,0.5);
  \draw[very thin] (-7.6+1.6*\t,7.6) node[anchor=south,scale=0.75] {$x_\t$} -- (-7.6+1.6*\t,0 |- N\t.output);
  \coordinate (I\xa) at (-7.6+1.6*\t,0 |- N\t.output);
  \coordinate (I\xb) at (N\t.output);
  \draw[very thin] (I\xa) -- (I\xa |- A0.input \xc);
  \draw[very thin] (I\xb) -- (I\xb |- A0.input \xd);
}
\draw[dashed] (-8,-0.9) node[anchor=south west,scale=0.75]{AND-matrix} rectangle (-1.8,6);
\draw[dashed] (-0.4,-0.4) rectangle (6,6.5) node[anchor=north east,scale=0.75]{OR-matrix};
\foreach \y/\yf in {0/8,1/7,2/6,3/5,4/4,5/3,6/2,7/1} {
  \draw[very thin] (A\y.output) -- (O7.input \yf |- A\y.output);
  \foreach \yi/\yid in {1/7,2/6,3/5,4/4,5/3,6/2,7/1,8/0} {
    \draw[very thin] (A\y.input \yi -| I\yid) -- (A\y.input \yi);
    \draw[very thin] (O\y.input \yi) -- (O\y.input \yi |- A\yid.output);
    \fill (O\y.input \yi |- A\yid.output) circle (0.035 cm);
    \fill (A\y.input \yi -| I\yid) circle (0.035 cm);
  }
}
\foreach \y/\x/\xi in {0/0/8,0/2/6,0/4/4,0/7/1,1/0/8,1/2/6,1/4/4,1/6/2,2/1/7,2/2/6,2/5/3,2/6/2,3/1/7,3/2/6,3/5/3,3/6/2,4/0/8,4/3/5,4/5/3,4/7/1,5/0/8,5/2/6,5/4/4,5/7/1,6/0/8,6/2/6,6/5/3,6/7/1,7/1/7,7/3/5,7/4/4,7/6/2,2/3/5,2/4/4,0/6/2} {
  \draw[thick] (A\y.input \xi -| I\x) -- ++(0.05,0.05);
  \draw[thick] (A\y.input \xi -| I\x) -- ++(-0.05,0.05);
  \draw[thick] (A\y.input \xi -| I\x) -- ++(0.05,-0.05);
  \draw[thick] (A\y.input \xi -| I\x) -- ++(-0.05,-0.05);
}

\foreach \y/\x/\xi in {0/1/7,0/2/6,0/3/5,0/5/3,0/7/1,1/1/7,1/2/6,1/3/5,2/0/8,2/2/6,2/3/5,2/5/3,2/7/1,3/0/8,3/5/3,3/6/2,4/1/7,4/2/6,4/4/4,4/7/1,5/1/7,5/2/6,6/2/6,6/4/4,6/5/3,6/6/2,7/0/8,7/5/3,7/7/1} {
  \draw[thick] (O\y.input \xi |- A\x.output) -- ++(0.05,0.05);
  \draw[thick] (O\y.input \xi |- A\x.output) -- ++(-0.05,0.05);
  \draw[thick] (O\y.input \xi |- A\x.output) -- ++(0.05,-0.05);
  \draw[thick] (O\y.input \xi |- A\x.output) -- ++(-0.05,-0.05);
}
\end{tikzpicture}
\caption{Schematische voorstelling van een Programmable Logic Array (PLA).}
\figlab{plaSchema}
\end{figure}
toont een schematische voorstelling van zo'n PLA. De punten tussen twee verbindingen stellen een zekering door. Door zekeringen door te branden kunnen we dus het model aanpassen. Punten waar een kruisje bij staat zijn in dit voorbeeld doorgebrand. Bij een doorgebrande zekering in de AND-matrix, komt er een 1 op de lijn, bij een doorgebrande zekering op de OR-matrix een 0. Er bestaan nog twee varianten op een PLA:
\begin{itemize}
 \item een \termen{Programmable Array Logic (PAL)} heeft een vaste OR-matrix.
 \item een \termen{Programmable Read Only Memory (PROM)} heeft een vaste AND-matrix. Hierbij fungeert deze matrix als een adresdecoder (zie \ref{ss:decoder}).
\end{itemize}
\subsubsection{Programmable Logic Device (PLD)}
\label{sss:pld}
Een PLA is niet in staat complexe functies te beschrijven zonder de AND- en OR-matrix zeer groot te maken. Omdat immers voor iedere verbinding ook nog lijnen moeten voorzien worden, zou de component al snel te groot worden. Een \termen{Programmable Logic Device (PLD)} probeert hierop een antwoord te bieden. Hierbij wordt meer lagen logica gehanteerd. \figref{pldSchema}
\begin{figure}[hbt]
\centering
\begin{tikzpicture}[circuit logic US] %yscale=0.8
\foreach\y/\ia/\ib in {0/1/2,1/3/4,2/5/6,3/7/8,4/9/10} {
  \node[not gate,scale=0.5] (N\y) at (0,\y) {$N\y$};
  \coordinate (I\y) at (-1-0.1*\y,0.5+\y);
  \draw[very thin] (I\y) -- (-0.5,0.5+\y -| N\y.output);
  \draw[very thin] (N\y) -| ++(-0.5,0.5);
  \coordinate (B\ia) at (N\y.output);
  \coordinate (B\ib) at (N\y.output |- -0.5,0.5+\y);
}
\draw (I3) node[anchor=east,scale=0.75]{$x_1$};
\draw (I4) node[anchor=east,scale=0.75]{$x_2$};
\foreach\s in {0,1,...,3} {
  \node[or gate,inputs={normal,normal,normal,normal,normal},scale=0.5,rotate=-90] (O\s) at (3*\s+2.2,-2) {};
  \foreach\x/\xi/\iy in {0/5/0.7,1/4/0.6,2/3/0.5,3/2/0.6,4/1/0.7} {
    \node[and gate,inputs={normal,normal,normal,normal,normal,normal,normal,normal,normal,normal},scale=0.333,rotate=-90] (A\s\x) at (3*\s+0.6*\x+1,-1) {};
    \draw[very thin] (A\s\x.output) -- ++(0,-\iy+0.5) -| (O\s.input \xi);
    \foreach \g in {1,2,...,10} {
      \fill (B\g -| A\s\x.input \g) circle (0.035 cm);
      \draw[very thin] (B\g -| A\s\x.input \g) -- (A\s\x.input \g);
    }
  }
}
\foreach\s in {0,3} {
  \coordinate (OO\s) at (O\s.output |- 0,-3-0.1*\s);
  \coordinate (OOO\s) at (OO\s);
  \draw (O\s.output) -- (OO\s);
}
\foreach\s in {1,2} {
  \node[draw=black,rectangle] (D\s) at (3*\s+2.8,-3.125) {D};
  \draw[very tiny] (D\s.north) |- ++(-0.6,0.25);
  \coordinate (OO\s) at (3*\s+2.5,-4.5-0.1*\s);
  \draw[very tiny] (D\s.south) -- (D\s.south |- 0,-3.75);
  \draw[very thin] (O\s.output) -- (O\s.output |- 0,-3.75);
  \draw[very thin] (3*\s+2.5,-4.25) -- (OO\s);
  \coordinate (OOO\s) at (OO\s |- 0,-5.5);
  \draw[very thin] (OO\s) -- (OOO\s);
  \draw (3*\s+2,-3.75) -- (3*\s+3,-3.75) -- (3*\s+2.8,-4.25) -- (3*\s+2.2,-4.25) -- cycle;
}
\foreach\s in {0,1,...,2} {
  \draw[very thin] (OO\s) -| (I\s);
}
\foreach\s in {1,2,3} {
  \draw(OOO\s) node[anchor=north,scale=0.75] {$f_\s$};
}
\foreach\y in {1,2,...,10} {
  \draw[very thin] (B\y) -- (B\y -| A34.input \y);
}
\foreach \s/\x/\g in {0/0/1,0/0/2,0/0/3,0/0/4,0/0/6,0/0/7,0/0/10,0/1/1,0/1/3,0/1/4,0/1/5,0/1/6,0/1/8,0/1/10,0/2/1,0/2/2,0/2/3,0/2/4,0/2/6,0/2/8,0/2/10,0/3/1,0/3/2,0/3/3,0/3/6,0/3/7,0/3/9,0/3/10,0/4/1,0/4/3,0/4/5,0/4/6,0/4/7,0/4/10,1/0/1,1/0/2,1/0/4,1/0/6,1/0/7,1/0/9,1/0/10,1/1/2,1/1/4,1/1/6,1/1/8,1/1/9,1/1/10,1/2/1,1/2/2,1/2/3,1/2/6,1/2/7,1/2/8,1/2/9,1/3/1,1/3/3,1/3/4,1/3/6,1/3/7,1/3/8,1/3/10,1/4/1,1/4/2,1/4/3,1/4/4,1/4/5,1/4/7,1/4/9,1/4/10,2/0/2,2/0/3,2/0/4,2/0/6,2/0/7,2/0/8,2/0/10,2/1/2,2/1/4,2/1/5,2/1/6,2/1/8,2/1/9,2/1/10,2/2/1,2/2/2,2/2/3,2/2/5,2/2/6,2/2/7,2/2/8,2/2/10,2/3/1,2/3/2,2/3/3,2/3/5,2/3/7,2/3/8,2/3/9,2/3/10,3/4/1,3/4/3,3/4/4,3/4/6,3/4/8,3/4/10,3/0/2,3/0/3,3/0/4,3/0/6,3/0/7,3/0/8,3/0/9,3/1/2,3/1/3,3/1/6,3/1/7,3/1/9,3/2/1,3/2/2,3/2/4,3/2/5,3/2/7,3/2/10,3/3/1,3/3/2,3/3/3,3/3/4,3/3/6,3/3/8,3/3/9,3/4/2,3/4/4,3/4/6,3/4/7,3/4/9} {
  \draw[thick] (B\g -| A\s\x.input \g) -- ++(0.05,0.05);
  \draw[thick] (B\g -| A\s\x.input \g) -- ++(-0.05,0.05);
  \draw[thick] (B\g -| A\s\x.input \g) -- ++(0.05,-0.05);
  \draw[thick] (B\g -| A\s\x.input \g) -- ++(-0.05,-0.05);
}
\end{tikzpicture}%TODO: make more compact + multiplexer input
\caption{Schematische voorstelling van een Programmable Logic Device (PLD).??}
\figlab{pldSchema}
\end{figure}
toont het idee achter deze techniek. We beschouwen nog steeds een matrix, alleen is een groot deel van de invoer eigenlijk de uitvoer van andere geprogrammeerde functies. Zo zijn $x_1$ en $x_2$ de enig invoerlijnen, de andere lijnen zijn het resultaat van geprogrammeerde logica. Op de figuur bemerken we verder nog dat er componenten onder de uitgangen staan. Deze componenten komen later in de cursus aan bod. Het vierkant stelt een 1-bit geheugen voor (zie \ref{s:memory}), meestal een D-Flip Flop. De trapezium een multiplexer (zie \ref{ss:multiplexer}). Dit component laat ons dus toe complexere schakelingen te maken, en tegelijk het aantal zekeringen onder controle te houden.
\subsubsection{Complex Programmable Logic Device (CPLD)}
\begin{figure}[H]%TODO: check alignment, fixed floating environment used [H]
\centering
\begin{tikzpicture}[scale=0.75]
\draw (-3.375,-2.35) node[scale=0.75]{O};
\draw (-4,-2.6) rectangle (-2.75,-2.1);
\draw (-1.625,-2.35) node[scale=0.75]{I/O};
\draw (-1,-2.6) rectangle (-2.25,-2.1);
\draw (-2.5,-1.2) node[scale=0.75]{PLD};
\draw (-4,-1.7) rectangle (-1,-0.7);

\draw (3.375,-2.35) node[scale=0.75]{O};
\draw (4,-2.6) rectangle (2.75,-2.1);
\draw (1.625,-2.35) node[scale=0.75]{I/O};
\draw (1,-2.6) rectangle (2.25,-2.1);
\draw (2.5,-1.2) node[scale=0.75]{PLD};
\draw (4,-1.7) rectangle (1,-0.7);

\draw (-3.375,2.35) node[scale=0.75]{O};
\draw (-4,2.6) rectangle (-2.75,2.1);
\draw (-1.625,2.35) node[scale=0.75]{I/O};
\draw (-1,2.6) rectangle (-2.25,2.1);
\draw (-2.5,1.2) node[scale=0.75]{PLD};
\draw (-4,1.7) rectangle (-1,0.7);

\draw (3.375,2.35) node[scale=0.75]{O};
\draw (4,2.6) rectangle (2.75,2.1);
\draw (1.625,2.35) node[scale=0.75]{I/O};
\draw (1,2.6) rectangle (2.25,2.1);
\draw (2.5,1.2) node[scale=0.75]{PLD};
\draw (4,1.7) rectangle (1,0.7);
\draw (0,0) node[scale=0.75]{Schakelmatrix};
\draw (-4,-0.3) rectangle (4,0.3);
\foreach \x in {-3,-2,...,3} {
  \draw (-3.375+0.05*\x,-2.1) -- (-3.375+0.05*\x,-1.7);
  \draw (-1.625+0.05*\x,-2.1) -- (-1.625+0.05*\x,-1.7);
  \draw (-1.625+0.05*\x,-2.6) -- (-1.625+0.05*\x,-3.1);

  \draw (3.375+0.05*\x,-2.1) -- (3.375+0.05*\x,-1.7);
  \draw (1.625+0.05*\x,-2.1) -- (1.625+0.05*\x,-1.7);
  \draw (1.625+0.05*\x,-2.6) -- (1.625+0.05*\x,-3.1);

  \draw (-3.375+0.05*\x,2.1) -- (-3.375+0.05*\x,1.7);
  \draw (-1.625+0.05*\x,2.1) -- (-1.625+0.05*\x,1.7);
  \draw (-1.625+0.05*\x,2.6) -- (-1.625+0.05*\x,3.1);

  \draw (3.375+0.05*\x,2.1) -- (3.375+0.05*\x,1.7);
  \draw (1.625+0.05*\x,2.1) -- (1.625+0.05*\x,1.7);
  \draw (1.625+0.05*\x,2.6) -- (1.625+0.05*\x,3.1);
  
  \draw (-0.5+0.05*\x,0.3) |- (-1.625+0.05*\x,2.9+0.05*\x);
  \draw (0.5-0.05*\x,0.3) |- (1.625-0.05*\x,2.9+0.05*\x);
  \draw (-0.5+0.05*\x,-0.3) |- (-1.625+0.05*\x,-2.9-0.05*\x);
  \draw (0.5-0.05*\x,-0.3) |- (1.625-0.05*\x,-2.9-0.05*\x);
}
\foreach \x in {-12,-11,...,12} {
  \draw (-2.5-0.05*\x,-0.7) -- (-2.5-0.05*\x,-0.3);
  \draw (-2.5-0.05*\x,0.7) -- (-2.5-0.05*\x,0.3);
  \draw (2.5-0.05*\x,-0.7) -- (2.5-0.05*\x,-0.3);
  \draw (2.5-0.05*\x,0.7) -- (2.5-0.05*\x,0.3);
}
\end{tikzpicture}
\caption{Schematische voorstelling van een Complex Programmable Logic Device (CPLD).}
\figlab{cpldSchema}
\end{figure}
Een aantal PLD componenten samen op \'e\'en chip samen met de zogenaamde \termen{Glue Logic} om ze samen te laten werken, noemen we een \termen{Complex Programmable Logic Device (CPLD)}. \figref{cpldSchema} toont dat deze glue eigenlijk neerkomt op een \termen{schakelmatrix} om PLDs te laten samenwerken en \termen{in- en uitvoer modules} om te intrageren met de buitenwereld. Naast de PLDs zelf dient dus ook de schakelmatrix geprogrammeerd te worden. Een ander belangrijk verschil is dat CPLDs niet geprogrammeerd worden met behup van zekeringen. De chip wordt geprogrammeerd door ladingen op transistoren te plaatsen, bijgevolg kunnen de we deze chip enkele keren volledig herprogrammeren.
\subsubsection{Field Programmable Gate Array (FPGA)}
Een nog meer geavanceerde programmeerbare chip is de \termen{Field Programmable Gate Array (FPGA)}. Net zoals bij een CPLD bevat deze chip enkele logische blokken met daartussen ``glue''. Deze ``glue'' uit zich opnieuw in een schakelmatrix, maar ook in korte en lange verbindingen. Deze laatsten worden vaak ook \termen{lange lijnen} genoemd. Daarnaast bevat de chip uiteraard ook opnieuw en- en uitvoer modulen. \figref{fpgaSchemaFull}
\begin{figure}[hbt]
\centering
\subfigure[Volledige FPGA]{\begin{tikzpicture}[scale=1.15]
\filldraw[fill=black!35,draw=black] (0,0) rectangle (4.5,4.5);
\foreach \y in {1,2,...,16} {
  \fill[fill=white,draw=black] (0.25*\y+0.025,0) rectangle ++(0.20,0.225);
  \fill[fill=white,draw=black] (0.25*\y+0.025,4.275) rectangle ++(0.20,0.225);
  \fill[fill=white,draw=black] (0,0.25*\y+0.025) rectangle ++(0.225,0.20);
  \fill[fill=white,draw=black] (4.275,0.25*\y+0.025) rectangle ++(0.225,0.20);
  \foreach \xs in {1,2,3} {
    \draw (0.25*\y+0.05*\xs+0.025,0) -- ++(0,0.225);
    \draw (0.25*\y+0.05*\xs+0.025,4.275) -- ++(0,0.225);
  \draw (0,0.25*\y+0.05*\xs+0.025) -- ++(0.225,0);
    \draw (4.275,0.25*\y+0.05*\xs+0.025) -- ++(0.225,0);
  }
  \foreach \x in {1,2,4,5,...,13,15,16} {
    \filldraw[fill=white,draw=black] (0.25*\x+0.025,0.25*\y+0.025) rectangle ++(0.20,0.20);
  }
}
\foreach \x in {3,14} {
  \fill (0.25*\x+0.025,0) rectangle ++(0.20,0.225);
  \fill (0.25*\x+0.025,4.275) rectangle ++(0.20,0.225);
  \foreach \y in {0,1,...,3} {
    \filldraw[fill=white,draw=black] (0.25*\x+0.025,\y+0.275) rectangle ++(0.075,0.95);
    \filldraw[fill=white,draw=black] (0.25*\x+0.15,\y+0.275) rectangle ++(0.075,0.95);
  }
}
\draw[thick] (1.5,4.525) rectangle ++(1.25,-1.275);
\begin{scope}[xshift=5 cm]
\draw[thick] (-3.5,3.25) -- (0,0);
\draw[thick] (-2.25,3.25) -- (4.5,0);
\draw[thick] (-2.25,4.525) -- (4.5,4.5);
\fill[fill=black!35] (0,0) rectangle (4.5,4.5);
\foreach \x in {0,1,...,4} {
  \filldraw[fill=black!60,draw=black] (0.9*\x+0.15,3.75) rectangle ++(0.6,0.75);
  \draw (0.9*\x+0.45,4.125) node[white] {I/O};
}
\foreach \x in {0,2,4} {
  \foreach \y in {1,3} {
    \filldraw[fill=white,draw=black] (0.9*\x+0.15,0.9*\y+0.15) rectangle ++(0.6,0.6);
    \draw (0.9*\x+0.45,0.9*\y+0.45) node{SM};
  }
  \foreach \y in {0,2} {
    \filldraw[fill=white,draw=black] (0.9*\x+0.15,0.9*\y+0.15) rectangle ++(0.6,0.6);
    \draw (0.9*\x+0.45,0.9*\y+0.45) node{SMc};
  }
}
\foreach \x in {1,3} {
  \foreach \y in {0,2} {
    \filldraw[fill=black!70,draw=black] (0.9*\x+0.15,0.9*\y+0.15) rectangle ++(0.6,0.6);
    \draw (0.9*\x+0.45,0.9*\y+0.45) node[white]{LB};
  }
  \foreach \y in {1,3} {
    \filldraw[fill=white,draw=black] (0.9*\x+0.15,0.9*\y+0.15) rectangle ++(0.6,0.6);
    \draw (0.9*\x+0.45,0.9*\y+0.45) node{SMc};
  }
}
\foreach \y in {0,1,...,3} {
  \foreach \s in {-4,-3,...,4} {
    \draw (0,0.9*\y+0.45+0.04*\s) -- ++(0.15,0);
    \draw (4.35,0.9*\y+0.45+0.04*\s) -- ++(0.15,0);
    \foreach \x in {0,1,...,4} {
      \draw (0.9*\x+0.45+0.04*\s,0.9*\y+0.75) -- ++(0,0.3);
    }
    \foreach \x in {0,1,...,3} {
      \draw (0.9*\x+0.75,0.9*\y+0.45+0.04*\s) -- ++(0.3,0);
    }
  }
}
\foreach \x in {0,1,...,4} {
  \foreach \s in {-4,-3,...,4} {
    \draw (0.9*\x+0.45+0.04*\s,0) -- ++(0,0.15);
  }
}
\foreach \s in {-1,0,1} {
  \foreach \y in {0,2} {
    \draw (0,0.9*\y+0.04*\s+0.9) -- ++(4.5,0);
    \draw (0.9*\y+0.04*\s+0.9,0) -- ++(0,0.04*\s+2.7);
  }
}
\end{scope}
\end{tikzpicture}
\figlab{fpgaSchemaFull}}
\subfigure[Logical Block (LB)] {
\begin{tikzpicture}
\filldraw[fill=black!35,draw=black] (-0.5,-0.5) rectangle (3.5,2.5);
\filldraw[fill=white,draw=black] (0.1,0.1) rectangle ++(1.8,1.8);
\filldraw[fill=white,draw=black] (2.9,0.8) rectangle ++(-0.7,0.4);
\draw (1.9,1) -- (2.2,1);
\draw (2.05,1) |- (4,1.5) node[anchor=west,scale=0.7]{$y$};
\draw (2.9,1) -- (4,1) node[anchor=west,scale=0.7]{$y_Q$};
\draw (2.55,1) node[scale=0.7]{D-FF};
\draw (1,1) node[alignment=center,text width=2 cm,scale=0.7] {$16\times1$ LUT: logische functie van 4 variabelen};
\foreach \y/\yt in {-2/4,-1/3,0/2,1/1} {
  \draw (-1,1.2+0.4*\y) node[anchor=east,scale=0.7]{$x_\yt$} -- ++(1.1,0);
}
\end{tikzpicture}
\figlab{fpgaSchemaLB}}
\subfigure[Schakelmatrix]{
\begin{tikzpicture}
\filldraw[fill=black!35,draw=black] (0,0) rectangle (3,3);
\node[nmoso,rotate=-90,scale=0.8] (N1) at (0.75,2.5) {};
\node[nmosc,rotate=-90,scale=0.8] (N2) at (2.25,2.5) {};
\node[nmosc,scale=0.8] (N3) at (1.5,2) {};
\node[nmoso,rotate=-90,scale=0.8] (N4) at (2,1.5) {};
\node[nmoso,rotate=-90,scale=0.8] (N5) at (0.75,0.5) {};
\node[nmoso,rotate=-90,scale=0.8] (N6) at (2.25,0.5) {};
\draw (N2.drain) -- (N2.drain -| 2.75,0) |- (N6.drain);
\draw (N1.drain) -- (N2.source);
\draw (N1.source) -- (N1.source -| 0.25,0) |- (N5.source);
\draw (N5.drain) -- (N6.source);
\draw (N4.source) -- (-0.5,1.5);
\draw (N4.drain) -- (3.5,1.5);
\draw (N3.source) -- (1.5,-0.5);
\draw (N3.drain) -- (1.5,3.5);
\fill (N4.source -| 0.25,0) circle (0.035 cm);
\fill (N4.drain -| 2.75,0) circle (0.035 cm);
\fill (N3.source |- N1.source) circle (0.035 cm);
\fill (N3.drain |- N6.source) circle (0.035 cm);
\end{tikzpicture}
\figlab{fpgaSchemaSchakelMatrix}
}
\caption{Schematische voorstelling van een Field Programmable Gate Array (FPGA).}
\figlab{fpgaSchema}
\end{figure}
schematiseert dit concept. In tegenstelling tot een CPLD heeft een FPGA andere logische bouwblokken, namelijk \termen{Logical Blocks (LB)}. Dit logische blok staat weergegeven op \figref{fpgaSchemaLB}. Het bestaat typisch uit 4 ingangen ($x_1$, $x_2$, $x_3$ en $x_4$) en twee uitgangen ($y$ en $y_Q$). Inwendig bestaat het uit een 16 bit \termen{Look-Up Table (LUT)} en een \termen{Data Flip-Flop (D-FF)}. Deze Look-Up Table is in feite niets anders dan een klein geheugen. Met de 4 ingangen zijn we in staat om 16 verschillende invoer-waarden te genereren. Voor elk van deze waarden programmeren we een bit als uitvoer. Deze wordt uitgevoerd langs de $y$. De Data Flip-Flop houdt deze bit 1 klokcyclus bij, en zal hem de volgende klokcyclus op $y_Q$ zetten. Het gevolg is dus dat $y_Q$ een klokflank na-ijlt op $y$. Tot slot beschouwen we op \figref{fpgaSchemaSchakelMatrix} de implementatie van een \termen{schakelmatrix}. Hierbij verbinden we 4 lijnen met elkaar. Dit levert dus 6 mogelijke verbindingen. Door het programmeren van de chip, kunnen we de transistoren openen of sluiten, wat resulteert in het al dan niet verbinden van twee verbindingen. De transistoren die men hierbij gebruikt zijn de zogenaamde ``\termen{Pass-transistoren}''.
\paragraph{Spartan-3}
Een beroemde FPGA-chip is de \termen{Spartan-3} van \emph{Xilinx}. Deze FPGA maakt gebruik van \termen{Configurable Logic Blocks (CLB)} in plaats van LBs. Een CLB wordt onderverdeeld in 4 ``\termen{slices}''. Twee van deze slices kunnen geconfigureerd worden als schuifregisters (zie \ref{s:schuifregisters}), geheugencellen, of logische blokken, de overige twee alleen als logische blokken. Deze complexere slices bevatten twee verschillende componenten voor logica: 2 functies met 4 variabelen, en 1 functie met 5 variabelen, meestal aangevuld met flipflops. Verder bevatten deze vaak een multiplexer en een schuifregister. De specificaties van deze slices zijn verder ook terug te vinden in \cite[p.~22-23]{xilinxFpgaDs099}. Verder zien we op \figref{fpgaSchemaFull} ook dat er naast logische en in- en uitvoer blokken ook nog andere componenten op een FPGA zitten. Meestal gaat het dan om RAM en een multiplexer. Bovendien bevat de chip ook enkele klokken die het systeem aansturen. Deze zijn als een zwart blokje weergegeven. Typische frequenties bevinden zich rond de 50 MHz. Modernere FPGA's uiten zich in grotere geheugens, soms zelfs met een Stack (LIFO, zie \sscref{stack}), specifieke in- en uitvoer blokken zoals Ethernet, PCIe,... tot zelfs microprocessoren.
\section{Praktische aspecten}
Tot nu toe hebben we ons bezig gehouden met de implementatie van chips. Deze implementatie hebben we afgeleid uit de circuits. Er zijn echter ook aspecten die een invloed hebben op de werking die niet altijd van het circuit kunnen afgelezen worden. Zoals bijvoorbeeld de geometrie van de chip. De volgende subsecties beschrijven met welke praktische aspecten rekening gehouden moeten worden bij het ontwerpen.
\subsection{Spanningsniveaus en ruismarge}
\subsubsection{Ruismarge}
In sectie \ref{s:logischeWaarden} hebben we reeds kort beschreven dat we een bereik specificeren voor een \termen{High} en \termen{Low} signaal.  We definieerden ook een ongedefinieerde zone. Deze zone wordt gebruikt indien het onduidelijk wordt wat er precies aan de ingang staat. Uiteraard willen we vermijden dat we ooit in deze ruismarge terecht komen. Daarom introduceren we een \termen{Ruismarge}. Dit is een spanningsmarge die we beschikbaar stellen door een kleiner gebied van uitgangsspanningen te gebruiken dan toegelaten. We zullen dus de poorten spanningen laten genereren die zich in het hoogste gedeelte van High bevinden, of in het laagste gedeelte van Low. De ruis zal het signaal uiteraard kunnen afzwakken, maar we bijven in de acceptabele zone van High en Low. Het resultaat is dus dat bij de marges van de ingangen ruimer is dan de marges van de uitgangen. \figref{noiseMargin} geeft dit concept schematisch weer. Ook vermeldt de figuur de typische spanningsniveaus bij CMOS en TTL\footnote{TTL: Transistor-Transistor Logic.} implementatie.
\begin{figure}[hbt]
\centering
\begin{tikzpicture}
\fill[black!20] (-0.1,0) rectangle (0,1.25);
\fill[black!20] (-0.1,1.75) rectangle (0,3);
\fill[black!20] (0,0.625) rectangle (0.1,0);
\fill[black!20] (0,2.375) rectangle (0.1,3);
\draw[thick,->] (0,-0.1) -- (0,3.2);
\draw (-0.1,0) node[anchor=east,scale=0.75]{$V_{\mbox{\tiny SS}}$} -- (0.1,0);
\draw (-0.1,0.625) node[anchor=east,scale=0.75]{$V_{\mbox{\tiny OL}}$} -- (0.1,0.625);
\draw (-0.1,1.25) node[anchor=east,scale=0.75]{$V_{\mbox{\tiny IL}}$} -- (0.1,1.25);
\draw (-0.1,1.75) node[anchor=east,scale=0.75]{$V_{\mbox{\tiny IH}}$} -- (0.1,1.75);
\draw (-0.1,2.375) node[anchor=east,scale=0.75]{$V_{\mbox{\tiny OH}}$} -- (0.1,2.375);
\draw (-0.1,3) node[anchor=east,scale=0.75]{$V_{\mbox{\tiny DD}}$} -- (0.1,3);
\draw[<->] (-7,3.5) to node[midway,above,scale=0.75]{Ingang} (0,3.5);
\draw[<->] (7,3.5)  to node[midway,above,scale=0.75]{Uitgang} (0,3.5);

\draw (4,3.5) node[anchor=north,scale=0.75]{CMOS};
\draw (6,3.5) node[anchor=north,scale=0.75]{TTL};
\draw (-4,3.5) node[anchor=north,scale=0.75]{CMOS};
\draw (-6,3.5) node[anchor=north,scale=0.75]{TTL};
\draw (2.25,1.5) node[scale=0.75]{ongedefinieerd};
\draw (-2.25,1.5) node[scale=0.75]{ongedefinieerd};
\draw (2.25,2.6875) node[scale=0.75]{High (H)};
\draw (2.25,0.3125) node[scale=0.75]{Low (L)};
\draw (2.25,0.9375) node[scale=0.75]{ruismarge (NM$_L$)};
\draw (2.25,2.0625) node[scale=0.75]{ruismarge (NM$_L$)};
\draw (-2.25,2.375) node[scale=0.75]{High (H)};
\draw (-2.25,0.625) node[scale=0.75]{Low (L)};

\draw[dotted] (0.1,3) -- ++(6.9,0);
\draw[dotted] (0.1,2.375) -- ++(6.9,0);
\draw[dotted] (0.1,1.75) -- ++(6.9,0);
\draw[dotted] (0.1,1.25) -- ++(6.9,0);
\draw[dotted] (0.1,0.625) -- ++(6.9,0);
\draw[dotted] (0.1,0) -- ++(6.9,0);

\draw[dotted] (-0.8,3) -- ++(-6.2,0);
\draw[dotted] (-0.8,1.75) -- ++(-6.2,0);
\draw[dotted] (-0.8,1.25) -- ++(-6.2,0);
\draw[dotted] (-0.8,0) -- ++(-6.2,0);

\filldraw[fill=white,draw=black] (4,3) node[scale=0.75,fill=white]{5.0 V};
\filldraw[fill=white,draw=black] (4,2.375) node[scale=0.75,fill=white]{4.4 V};
\filldraw[fill=white,draw=black] (4,1.75) node[scale=0.75,fill=white]{3.5 V};
\filldraw[fill=white,draw=black] (4,1.25) node[scale=0.75,fill=white]{1.5 V};
\filldraw[fill=white,draw=black] (4,0.625) node[scale=0.75,fill=white]{0.5 V};
\filldraw[fill=white,draw=black] (4,0) node[scale=0.75,fill=white]{0.0 V};
\filldraw[fill=white,draw=black] (6,3) node[scale=0.75,fill=white]{5.0 V};
\filldraw[fill=white,draw=black] (6,2.375) node[scale=0.75,fill=white]{2.4 V};
\filldraw[fill=white,draw=black] (6,1.75) node[scale=0.75,fill=white]{2.0 V};
\filldraw[fill=white,draw=black] (6,1.25) node[scale=0.75,fill=white]{0.8 V};
\filldraw[fill=white,draw=black] (6,0.625) node[scale=0.75,fill=white]{0.4 V};
\filldraw[fill=white,draw=black] (6,0) node[scale=0.75,fill=white]{0.0 V};

\filldraw[fill=white,draw=black] (-4,3) node[scale=0.75,fill=white]{5.0 V};
\filldraw[fill=white,draw=black] (-4,1.75) node[scale=0.75,fill=white]{3.5 V};
\filldraw[fill=white,draw=black] (-4,1.25) node[scale=0.75,fill=white]{1.5 V};
\filldraw[fill=white,draw=black] (-4,0) node[scale=0.75,fill=white]{0.0 V};
\filldraw[fill=white,draw=black] (-6,3) node[scale=0.75,fill=white]{5.0 V};
\filldraw[fill=white,draw=black] (-6,1.75) node[scale=0.75,fill=white]{2.0 V};
\filldraw[fill=white,draw=black] (-6,1.25) node[scale=0.75,fill=white]{0.8 V};
\filldraw[fill=white,draw=black] (-6,0) node[scale=0.75,fill=white]{0.0 V};
\end{tikzpicture}
\caption{Werking van ruismarge bij CMOS en TTL.}
\figlab{noiseMargin}
\end{figure}
\paragraph{Onlogische Spanningen}
Wanneer een transistor omschakelt, betekent dit niet dat van het ene moment op het andere er een andere spanning op de uitgang komt te staan, Deze overgang is een continue proces. Dit betekent dus dat op een zeker moment aan de uitgang de spanning in het ongedefinieerde gebied komt te liggen. Ideaal zou uiteraard zijn dat bij een continue verandering aan de ingang er toch een discrete verandering aan de uitgang plaatsvindt. Dit is echter fysisch onmogelijk. De ontwerper dient dus rekening te houden met dit overgangsverschijnsel. Dit vormt bovendien een groot probleem omdat op dat moment beide transistoren\footnote{PMOS en NMOS.} halfopen zijn, en er dus stroom door de componenten vloeit. Verder weet men ook niet altijd welke spanning er aan de uitgang zal staan. De \termen{Transferfunctie} is immers afhankelijk van zowel het productieproces als omgevingsfactoren zoals warmte,... \figref{transferFunctionMOS} toont verschillende realistische transferfuncties samen met de ideale transferfunctie.
\begin{figure}[hbt]
\centering
\subfigure[Transferfuncties bij een inverter]{
\begin{tikzpicture}
\def\sxy{0.625};
\fill[gray!50] (0,\sxy*2) rectangle ++(1,\sxy*1.5);
\fill[gray!50] (2,0) rectangle ++(1.5,\sxy);
\draw (0,0) node[anchor=north east,scale=0.66]{$V_{\mbox{\small{SS}}}$};
\foreach\x/\t in {1/IL,2/IH,3.5/DD} {
  \draw (\x,0) node[anchor=north,scale=0.66]{$V_{\mbox{\small{\t}}}$};
  \draw (0,\sxy*\x) node[anchor=east,scale=0.66]{$V_{\mbox{\small{\t}}}$};
}
\draw (1.5,0) node[anchor=north,scale=0.66]{$V_{\mbox{\small{T}}}$};
\draw[thick,->] (-0.1,0) -- (4,0);
\draw[thick,->] (0,-0.1) -- (0,2.5);
\draw[thick,dashed,black!80] (2,2) -- (2.25,2) node[anchor=west,scale=0.75]{Ideaal};
\draw[thick,black!80] (2,1.65) -- (2.25,1.65) node[anchor=west,scale=0.75]{Realistisch};
\draw[thick,dashed,black!80] (0,\sxy*3.4) -| (1.5,0.1*\sxy) -- (3.5,0.1*\sxy);
\draw[thick,black!80,variable=\x,domain=0:3.5,smooth] plot (\x,{2.7*\sxy/(1.0+exp(4*\x-6.5))+0.4*\sxy});
\draw[thick,black!80,variable=\x,domain=0:3.5,smooth] plot (\x,{2.7*\sxy/(1.0+exp(4*\x-6))+0.4*\sxy});
\draw[thick,black!80,variable=\x,domain=0:3.5,smooth] plot (\x,{2.7*\sxy/(1.0+exp(4*\x-5.5))+0.4*\sxy});
\end{tikzpicture}
\figlab{transferFunctionMOS}
}
\subfigure[Schmitt-Trigger transferfunctie]{
\begin{tikzpicture}[circuit logic US]
\def\sxy{0.625};
\fill[gray!50] (0,\sxy*2) rectangle ++(1,\sxy*1.5);
\fill[gray!50] (2,0) rectangle ++(1.5,\sxy);
\draw (0,0) node[anchor=north east,scale=0.66]{$V_{\mbox{\small{SS}}}$};
\foreach\x/\t in {1/IL,2/IH,3.5/DD} {
  \draw (\x,0) node[anchor=north,scale=0.66]{$V_{\mbox{\small{\t}}}$};
  \draw (0,\sxy*\x) node[anchor=east,scale=0.66]{$V_{\mbox{\small{\t}}}$};
}
\draw[thick,->] (-0.1,0) -- (4,0);
\draw[thick,->] (0,-0.1) -- (0,2.5);
\draw[thick,black!80,variable=\x,domain=0:3.5,smooth] plot (\x,{2.7*\sxy/(1.0+exp(4*\x-6.5))+0.4*\sxy});
\draw[thick,black!80,variable=\x,domain=0:3.5,smooth] plot (\x,{2.7*\sxy/(1.0+exp(4*\x-4.5))+0.4*\sxy});
\draw[thick,black!80,->] (1.625-0.05,1.75*\sxy+2.7*\sxy*0.05) -- (1.625,1.75*\sxy);
\draw[thick,black!80,->] (1.125+0.05,1.75*\sxy-2.7*\sxy*0.05) -- (1.125,1.75*\sxy);
\node[not gate,scale=0.75] (N) at (2.5,1.75) {};
\draw (N.input) -- ++(-0.25,0);
\draw (N.output) -- ++(0.25,0);
\begin{scope}[xshift=2.5 cm,yshift=1.75 cm,scale=0.6]
\draw (-0.1875,-0.125) -| (0.0625,0.125) -- (0.1875,0.125);
\draw (-0.0625,-0.125) |- (0.0625,0.125);
\end{scope}
\end{tikzpicture}
\figlab{transferFunctionSchmittTrigger}
}
\caption{Transferfuncties en Schmitt-Trigger ingangen.}
\figlab{transferSchmittTrigger}
\end{figure}
\paragraph{Schmitt-trigger ingangen} Dit effect is vooral nadelig voor bijvoorbeeld in- en uitvoer. Deze signalen veranderen doorgaans traag, en bovendien niet strikt stijgend of dalend. Dit heeft tot gevolg dat indien het ingangssignaal van een 0 naar een 1 moet gaan, het meestal verschillende malen van signaal verandert. Een oplossing hiervoor zijn \termen{Schmitt-trigger ingangen}, dit zijn componenten met een \termen{hysteresis}. Dit betekent dat de overgangsspanning bij een overgang van laag naar hoog, hoger is dan de overgangsspanning van hoog naar laag. Deze transferfunctie staat op \figref{transferFunctionSchmittTrigger}, samen met het symbool voor een Schmitt-Trigger ingang: een NOT poort, met een miniatuur van de inverse transferfunctie.
\subsection{Dynamisch gedrag}
\begin{figure}[hbt]
\centering
\begin{tikzpicture}[circuit logic US,american resistors]
\node[not gate] (N1) at (-1,0) {};
\node[not gate] (N2) at (2,0) {};
\draw (N1.output) -- (N2.input);
\draw (N1.input) -- ++(-0.5,0);
\draw (N2.output) -- ++(0.5,0);
\draw (-4,0) node[anchor=west]{Schakeling};
\draw (-4,-2) node[anchor=west]{Laag$\Rightarrow$Hoog};
\draw (-4,-4) node[anchor=west]{Hoog$\Rightarrow$Laag};
\draw (0,-2) -- (2,-2) to [resistor,label=\small{$R_{IH}$}] (2,-3) -- (0.5,-3) to [capacitor,label=\small{$C$}] (0.5,-2) -- (-1,-2) to [resistor,label=\small{$R_{OH}$}] (-1,-1) node[anchor=south,scale=0.75]{$V_{\mbox{\small{DD}}}$};
\begin{scope}[xshift=4.5 cm, yshift=-3 cm]
\draw[thick,->] (0,-0.1) -- (0,1.75) node[anchor=north west,scale=0.75]{$V$};
\draw[thick,->] (-0.1,0) -- (2,0) node[anchor=south east,scale=0.75]{$t$};
\draw[thick,dashed,black!80] (0,0.1) -- (0.5,0.1) |- (2,1.4);
\draw[thick,black!80] (0,0.1) -- (0.5,0.1);
\draw[variable=\x,domain=0.5:2,smooth,black!80,thick] plot (\x,{1.4-1.3*exp(-2*\x+1)});
\end{scope}
\draw (0.5,-4) -- (2,-4) to [resistor,label=\small{$R_{IL}$}] (2,-5) -- (0.5,-5) to [capacitor,label=\small{$C$}] (0.5,-4);
\draw (0.5,-5) -- (-1,-5) to [resistor,label=\small{$R_{OL}$}] (-1,-4) -- (0.5,-4);
\begin{scope}[xshift=4.5 cm, yshift=-5 cm]
\draw[thick,->] (0,-0.1) -- (0,1.75) node[anchor=north west,scale=0.75]{$V$};
\draw[thick,->] (-0.1,0) -- (2,0) node[anchor=south east,scale=0.75]{$t$};
\draw[thick,dashed,black!80] (0,1.4) -- (0.5,1.4) |- (2,0.1);
\draw[thick,black!80] (0,1.4) -- (0.5,1.4);
\draw[variable=\x,domain=0.5:2,smooth,black!80,thick] plot (\x,{1.3*exp(-2*\x+1)+0.1});
\end{scope}
\end{tikzpicture}
\caption{Dynamisch gedrag bij twee sequenti\"ele NOT poorten.}
\figlab{dynamicBehavior}
\end{figure}
Op simulaties zullen we meestal uitgaan van dynamisch gedrag. Dit betekent dus dat we veronderstellen dat indien we bijvoorbeeld de spanning aan de ingang van een poort naar een andere spanning brengen, de uitgang na enige vertraging van de oude naar de nieuwe spanning omschakelt. Dit zullen we illustreren met een voorbeeld zoals op \figref{dynamicBehavior} waar we twee NOT-poorten na elkaar plaatsen. We kunnen het gedrag van dit circuit nabootsen bij overgangen, door een lijn als een capaciteit te zien, verder zien we een gesloten transistor als een weerstand, open transistoren zijn open lijnen en worden dus niet beschouwd. We zullen eerst uit de beschreven modellen de spanning in functie van de tijd plaatsen.
\paragraph{Hoog naar laag}
Bij een overgang van hoog naar laag aan de ingang, zal de spanning op de kabel tussen de twee poorten stijgen. Zoals we op de grafiek zien, zal dit exponentieel gebeuren. We zullen eerst de doelspanning $V_{\infty}$ bepalen, en vervolgens de \termen{tijdsconstante $\tau$}. De doelspanning kunnen we eenvoudig afleiden uit de stroom, indien de condensator volledig opgeladen is, stroomt alle stroom uitsluitend door de twee weerstanden. Hierdoor kunnen we de stroomsterkte afleiden. De spanning over de condensator kunnen we dan berekenen, omdat deze gelijk is aan de spanning op de tweede weerstand $R_{IH}$:
\begin{equation}
I_{\infty}=\displaystyle\frac{V_{DD}}{R_{IH}+R_{OH}}\Rightarrow V_{\infty}=R_{IH}\cdot I_{\infty}=\displaystyle\frac{R_{IH}\cdot V_{DD}}{R_{IH}+R_{OH}}
\label{eqn:vInfty}
\end{equation}
We leiden vervolgens de tijdsconstante af door de het systeem als een RC-keten te modelleren. In dat geval moeten we de twee weerstanden als parallel beschouwen. De tijdsconstante is dan de vervangweerstand vermenigvuldigd met de capaciteit van de condensator:
\begin{equation}
R_{\mbox{subs.}}=\displaystyle\frac{R_{IH}\cdot R_{OH}}{R_{IH}+R_{OH}}\Rightarrow \tau=R_{\mbox{subs.}}\cdot C=\displaystyle\frac{R_{IH}\cdot R_{OH}\cdot C}{R_{IH}+R_{OH}}
\label{eqn:tauHL}
\end{equation}
Voor het opladen van een condensator geldt volgende exponenti\"ele functie:
\begin{equation}
V\left(t\right)=V_{\infty}\cdot\left(1-e^{-t/\tau}\right)
\end{equation}
\paragraph{Laag naar hoog}
We berekenen de vergelijking van laag naar hoog volledig analoog. Aangezien er geen bron meer op de schakeling staat, is de schakeling een zuivere RC-keten. De condensator zal dus ontladen, de eindspanning is dus $V_{\infty}=0\mbox{ V}$. We berekenen de tijdsconstante dan ook opnieuw met behulp van de vervangweerstand:
\begin{equation}
R_{\mbox{subs.}}=\displaystyle\frac{R_{IL}\cdot R_{OL}}{R_{IL}+R_{OL}}\Rightarrow \tau=R_{\mbox{subs.}}\cdot C=\displaystyle\frac{R_{IL}\cdot R_{OL}\cdot C}{R_{IL}+R_{OL}}
\label{eqn:tauLH}
\end{equation}
Bij een RC-keten geldt voor het ontladen van een condensator vervolgens deze exponenti\"ele functie:
\begin{equation}
V\left(t\right)=V_0\cdot e^{-t/\tau}
\end{equation}
\paragraph{Tijdsconstante minimaliseren}
We zien dus dat bij overgangen van laag naar hoog en hoog naar laag, we te maken hebben met exponentieel tijdsgedrag. Deze constante wordt bepaald door de capaciteit, en door de in- en uitgangsimpedanties. Deze parameters stellen we in met de volgende vier doelen:
\begin{equation}
\left\{\begin{array}{l|ll}
\tau\ssearrow&\tau=R_{\mbox{\small{subs.}}}\cdot C&\mbox{lage tijdsconstante}\\
P_{\mbox{\small{stat.}}}??\ssearrow&P_{\mbox{\small{stat.}}}=V^2/\left(R_I+R_O\right)=V^2/R_{\mbox{\small{tot.}}}&\mbox{minimaal statisch vermogenverbruik}\\
P_{\mbox{\small{dyn.}}}??\ssearrow&P_{\mbox{\small{dyn.}}}=V^2/R_O&\mbox{minimaal dynamisch vermogenverbruik}\\
V_{\infty}\approx V_{DD}&V_{\infty}=R_I\cdot V_{DD}/\left(R_I+R_O\right)&\mbox{uitgangsspanning dicht bij de voedingsspanning}\\
\end{array}\right.
\label{eqn:PVTsummary}
\end{equation}
Hiervoor streven we naar bepaalde waardes voor de verschillende parameters:
\begin{itemize}
 \item \termen{Ingangsimpedantie $R_I$}: We streven naar een zo hoog mogelijke ingangsimpedantie, immers stelt vergelijking (\ref{eqn:vInfty}) dat indien $R_I\gg R_O$, $V_{\infty}\approx V_{DD}$. Bovendien is het statisch vermogenverbruik lager (zie vergelijking~\ref{eqn:PVTsummary}).
 \item \termen{Uitgangsimpedantie $R_O$}: We proberen de uitgangsimpedantie zo laag mogelijk te houden. Dit haalt de tijdsconstante naar beneden, waardoor de chip sneller schakelt (zie vergelijking (\ref{eqn:tauHL}) en (\ref{eqn:tauLH})). Een nadelig bijeffect is een hoger dynamisch vermogenverbruik (zie vergelijking \ref{eqn:PVTsummary}).
 \item \termen{Capaciteit $C$}: De capaciteit probeert men zo laag mogelijk te houden. Immers verhoogt een hoge capaciteit rechtstreeks de vertraging (zie vergelijking (\ref{eqn:PVTsummary})). Hoge capaciteiten zijn kenmerkend voor lange verbindingen, vandaar dat men deze meestal op chips tot een minimum beperkt.
\end{itemize}
Al deze effecten zorgen ervoor dat een CMOS-implementatie betere karakteristieken heeft dan een NMOS-poort: NMOS heeft een groot statisch vermogenverbruik. Om dit tegen te gaan, kunnen we een sterke weerstand $R$ in de schakeling plaatsen. Dit leidt echter tot een grote uitgangsimpedantie vermits $R\approx R_O$, hierdoor wordt de stijgtijd van de schakeling groter, en dus bijgevolg de algemene vertraging.??%TODO: recheck
\paragraph{Stijg- en daaltijd}
\begin{figure}[hbt]
\centering
\begin{tikzpicture}
\def \yt{2.5};
\def \yts{2.325};
\def \yb{-3.5};
\def \ybs{-3.325};
\begin{scope}[yscale=2]
\draw[dotted,thick] (0,0.9) node[anchor=east]{$90\%$} -- ++(10,0) node[anchor=west]{$V_{IH}$};
\draw[dotted,thick] (0,0.5) node[anchor=east]{$50\%$} -- ++(10,0) node[anchor=west]{$V_T$};
\draw[dotted,thick] (0,0.1) node[anchor=east]{$10\%$} -- ++(10,0) node[anchor=west]{$V_{IL}$};
\draw plot[thick,samples=7,black!80,variable=\x,domain=0:1,smooth] (\x,{0.05+0.025*rand-0.0125}) -- plot[thick,samples=14,black!80,variable=\x,domain=1:3,smooth] (\x,{0.05+0.45*\x-0.45+0.025*rand-0.0125}) -- plot[thick,samples=21,black!80,variable=\x,domain=3:6,smooth] (\x,{0.95+0.025*rand-0.0125}) -- plot[thick,samples=7,black!80,variable=\x,domain=6:7,smooth] (\x,{0.95-0.9*\x+5.4+0.025*rand-0.0125}) -- plot[thick,samples=21,black!80,variable=\x,domain=7:10,smooth] (\x,{0.05+0.025*rand-0.0125});
\coordinate (IA) at (1.1111,0.1);
\coordinate (IMA) at (2,0.5);
\coordinate (IB) at (2.8889,0.9);
\coordinate (IC) at (6.0556,0.9);
\coordinate (IMB) at (6.5,0.5);
\coordinate (ID) at (6.9444,0.1);
\end{scope}
\draw (IA) -- (IA |- 0,\yt);
\draw (IB) -- (IB |- 0,\yt);
\draw (IC) -- (IC |- 0,\yt);
\draw (ID) -- (ID |- 0,\yt);
\draw[<->] (IA |- 0,\yts) to node[midway,above]{Stijgtijd $t_r$} (IB |- 0,\yts);
\draw[<->] (IC |- 0,\yts) to node[midway,above]{Daaltijd $t_f$} (ID |- 0,\yts);
\begin{scope}[yscale=2,yshift=-1.5cm]
\draw[dotted,thick] (0,0.9) node[anchor=east]{$90\%$} -- ++(10,0) node[anchor=west]{$V_{OH}$};
\draw[dotted,thick] (0,0.5) node[anchor=east]{$50\%$} -- ++(10,0) node[anchor=west]{$V_T$};
\draw[dotted,thick] (0,0.1) node[anchor=east]{$10\%$} -- ++(10,0) node[anchor=west]{$V_{OL}$};
\draw plot[thick,samples=7,black!80,variable=\x,domain=0:1,smooth] (\x,{0.95+0.025*rand-0.0125}) -- plot[thick,samples=14,black!80,variable=\x,domain=1.5:3.5,smooth] (\x,{0.95-0.45*\x+1.5*0.45+0.025*rand-0.0125}) -- plot[thick,samples=21,black!80,variable=\x,domain=3.5:6.35,smooth] (\x,{0.05+0.025*rand-0.0125}) -- plot[thick,samples=14,black!80,variable=\x,domain=6.35:8.35,smooth] (\x,{0.05+0.45*\x-0.45*6.35+0.025*rand-0.0125}) -- plot[thick,samples=11,black!80,variable=\x,domain=8.35:10,smooth] (\x,{0.95+0.025*rand-0.0125});
\coordinate (OMA) at (2.5,0.5);
\coordinate (OMB) at (7.35,0.5);
\end{scope}
\draw (IMA) -- (IMA |- 0,\yb);
\draw (OMA) -- (OMA |- 0,\yb);
\draw (IMB) -- (IMB |- 0,\yb);
\draw (OMB) -- (OMB |- 0,\yb);
\draw[<->] (IMA |- 0,\ybs) to node[midway,below]{$t_{pHL}$} (OMA |- 0,\ybs);
\draw[<->] (IMB |- 0,\ybs) to node[midway,below]{$t_{pLH}$} (OMB |- 0,\ybs);
\end{tikzpicture}
\caption{Het dynamisch gedrag van een NOT-poort.}
\figlab{dynamicBehaviorNotGate}
\end{figure}
We hebben het reeds uitvoerig gehad over het dynamisch karakter van een elektronische schakeling. Naast het feit dat het enige tijd duurt alvorens een spanning die op een draad wordt aangelegd daadwerkelijk dit spanningsniveau bereikt, heeft ook een poort een dynamisch karakter. Een interessante eigenschap is dat de stijg- en daaltijd invloed hebben op de vertraging van de poort. We zullen deze drie grootheden nu formaliseren:
\begin{itemize}
 \item \termen{Stijgtijd $t_r$} ofwel \termen{rise time}: de tijd die de spanning nodig heeft om van een lage spanning naar een hoge spanning te stijgen. Men neemt hiervoor de 10\% en 90\% grenzen tussen de basisspanning en de topspanning\footnote{In sommige publicaties wordt 80\% en 20\% gebruikt.}.
 \item \termen{Daaltijd $t_f$} ofwel \termen{fall time}: de tijd die de spanning nodig heeft om van een hoge spanning naar een lage spanning te dalen. Men neemt hiervoor de 90\% en 10\% grenzen tussen de basisspanning en de topspanning.
 \item \termen{Vertragingstijd $t_p$} ofwel \termen{propagation delay}: De tijd die de poort nodig heeft tussen de ingangspanning die het midden bereikt tussen de basisspanning en de topspanning, en de uitgang die deze 50\% grens bereikt. Vermits de stijg- en daaltijd ook bepalen hoe snel een signaal naar 50\% stijgt of daalt, hebben deze parameters ook invloed op de vertragingstijd. We berekenen de vertraging door het gemiddelde te nemen tussen de vertraging tijdens het dalen $t_{pHL}$ en het stijgen $t_{pLH}$:
\begin{equation}
\mbox{Vertragingstijd $t_p$}=\displaystyle\frac{t_{pHL}+t_{pLH}}{2}
\end{equation}
\end{itemize}
Dit principe wordt ge\"illustreerd op \figref{dynamicBehaviorNotGate}. We zien hier twee situaties: in de eerste situatie is de stijgtijd van de ingang gelijk aan de daaltijd van de uitgang. De poort heeft weliswaar een vertraging, maar deze is vrij beperkt. In het tweede geval is de stijgtijd van de uitgang dubbel zolang als de daaltijd van de ingang. Hoewel de poort sneller reageert op het veranderende signaal, is de vertraging groter. Vermits de stijg- en daaltijden be\"invloed worden door de parasitaire capaciteit\footnote{De capaciteit die wordt veroorzaakt door de draad zelf.}, wordt dus ook de vertraging be\"invloed door de aard van de draden. De vertraging bij korte draden is aanzienlijk kleiner dan deze van lange draden.
%??%TODO
%De ingangsimpedantie is hierbij omgekeerd evenredig met de \termen{Stijgtijd}. De uitgangsimpedantie is evenredig met de \termen{Daaltijd}. Deze 
\subsection{Vermogenverbruik}
Een van de belangrijkste ontwerpbeperkingen bij grote schakelingen is het \termen{vermogenverbruik}. Het vermogenverbruik is dan ook een parameter die zich negatief uit op twee verschillende domeinen:
\begin{itemize}
 \item Levering: De verbruikte energie heeft een kostprijs, bovendien uit het vermogenverbruik zich ook in een kortere levensduur van eventuele batterijen.
 \item Gedissipeerd: het vermogenverbruik leidt tot warmteontwikkeling. Deze warmte moet afgevoerd worden, wat ook vermogenverbruik met zich meebrengt.
\end{itemize}
We beschouwen twee vormen van vermogenverbruik:
\begin{itemize}
 \item \termen{Statisch vermogenverbruik}: Dit is het vermogen die continu verbruikt wordt, ook indien de schakeling niets doet. We zagen reeds dat dit bij NMOS implementaties het geval is, omdat we bij het sluiten van een NMOS-transistor, een lekstroom genereren van de source naar de drain. Bij NMOS-poorten mogen we dan ook uitgaan van een vermogenverbruik van $1\mbox{ mW}$. Dit betekent dus voor een gemiddelde chip dat dit makkelijk oploopt in tot $1\mbox{ kW}$ wat totaal onaanvaardbaar is. Bij CMOS hebben we eenvoudigweg geen lekstroom en dus ook geen statisch vermogenverbruik.
 \item \termen{Dynamisch vermogenverbruik}: Dit is het vermogen die verbruikt wordt op het moment dat een poort omschakelt. Hierdoor wordt de parasitaire capaciteit $C$ op- of ontladen. In \cite[3.12]{brown2004fundamentals} wordt hiervoor volgende formule gegeven:
\begin{equation}
P_{\mbox{\small{dyn.}}}=C\times f\times V^2
\end{equation}
Met chipoppervlakte $C$, een klokfrequentie $f$ en voedingsspanning $V$. We kunnen deze formule als volgt verklaren: Vermits het chipoppervlakte hoofdzakelijk draden bevat is dit een goede schatting voor de totale lengte van de draden, verder bepaalt de klokfrequentie hoeveel maal we per seconde deze draden moeten op- en ontladen. Het vermogen ten slotte om een capaciteit $C$ op te laden is evenredig met het kwadraat van de spanning. In de praktijk werken de meeste schakelingen met een klokfrequentie in de grootorde van $100\mbox{ MHz}$. Verder schakelen uiteraard niet alle draden telkens om, een realistische schatting ligt bij de 20\%. Indien we ook de oppervlakte van een poort in rekening brengen komen we uit op een ordegrootte van $35\mbox{ nW}$ per inverter. We kunnen dus in de grootorde van $300\ 000\ 000$ invertoren plaatsen voor een vermogenverbruik van $1\mbox{ W}$. In de praktijk is vermogenverbruik dan ook een van de belangrijkste beperkingen om rekening mee te houden bij het ontwerpen van digitale schakelingen, vandaar dat er in deze cursus ook een grote nadruk ligt op het minimaliseren van digitale schakelingen.
??%TODO%In de praktijk komt dit ongeveer overeen met $35\mbox{ nW}$ per inverter in een poort. Hierbij maken we een assumptie dat per klokcyclus, 20\% van de poorten omschakelt, en we een klokfrequentie rond de $100\mbox{ MHz}$ gebruiken.
\end{itemize}
\subsection{``1'' en ``0'' doorgeven}
Een oplettend lezer zal zich misschien de vraag gesteld hebben, waarom we een buffer zoals op \figref{bufferCmos} op pagina \pageref{fig:bufferCmos} niet implementeren door de NMOS en PMOS transistor om te wisselen. Dit zou ons immers twee transistoren besparen, en bovendien zou de poort effici\"enter werken, zoals op \figref{badBufferCmos}. Deze methode kunnen we echter niet consistent hanteren. Voor de verklaring moeten we terug naar een belangrijk detail bij de werking van transistoren in sectie \ref{ss:nmosPmosWork}. Een NMOS transistor geleidt immers alleen maar indien de spanning tussen de basis en de collector $V_{GS}$, groter is dan de transferspanning $V_{T}$. Indien de NMOS transistor dus gesloten is, zal de uitgangsspanning $V_{T}$ lager zijn dan de ingangsspanning. Dit geeft misschien bij \'e\'en transistor geen noemenswaardige problemen, maar bij een sequentie van transistoren, zal de spanning uiteindelijk onder het grensniveau vallen. Samenvattend kunnen we zeggen dat NMOS transistoren hoge spanningen niet goed doorgeven, en dus een slechte pull-up zijn. Analoog voor PMOS geldt dezelfde redenering, maar dan toegepast op lage spanningen. Een PMOS transistor is dus een slechte pull-down. Deze spanningsverschillen zijn ook gevisualiseerd op \figref{badBufferCmos}.
\begin{figure}[hbt]
\centering
\subfigure[Buffer met inverterende poorten]{\begin{tikzpicture}[circuit logic US]
\coordinate (F0) at (-3,0);
\coordinate (I) at (-2,0);
\draw (F0) node[anchor=east,scale=0.75]{$x$} -- (I);
\node [pmoso] (P1) at (-1,0.75) {};
\draw (I) |- (P1.gate);
\draw[->] (P1.source) -- ++(0,0.5) node[anchor=west,scale=0.75]{$V_{\mbox{\tiny DD}}$};
\node [nmosc] (N1) at (-1,-0.75) {};
\draw (I) |- (N1.gate);
\draw (N1.source) node [ground] {};
\draw (N1.drain) -- (P1.drain);
\node [pmosc] (P2) at (1,0.75) {};
\coordinate (F1) at (-1,0);
\coordinate (O) at (0,0);
\draw (F1) -- (O);
\draw[->] (P2.source) -- ++(0,0.5) node[anchor=west,scale=0.75]{$V_{\mbox{\tiny DD}}$};
\draw (O) |- (P2.gate);
\node [nmoso] (N2) at (1,-0.75) {};
\draw (O) |- (N2.gate);
\draw (N2.source) node [ground] {};
\draw (N2.drain) -- (P2.drain);
\coordinate (F2) at (1,0);
\draw (F2) -- ++(1,0) node[anchor=west,scale=0.75]{$f$};
\end{tikzpicture}
\figlab{goodBufferCmos}
}
\subfigure[Fout buffer]{\begin{tikzpicture}[circuit logic US]
\coordinate (F0) at (-3,0);
\coordinate (I) at (-2,0);
\draw (F0) node[anchor=east,scale=0.75]{$x$} -- (I);
\node [nmosc] (N1) at (-1,0.75) {};
\draw (I) |- (N1.gate);
\draw[->] (N1.drain) -- ++(0,0.5) node[anchor=west,scale=0.75]{$V_{\mbox{\tiny DD}}$};
\node [pmoso] (P1) at (-1,-0.75) {};
\draw (I) |- (P1.gate);
\draw (P1.drain) node [ground] {};
\draw (P1.source) -- (N1.source);
\coordinate (F1) at (-1,0);
\coordinate (O) at (0,0);
\draw[->,black!80,dashed] (-1.625,0 |- N1.gate) arc (180:270:0.625);
\draw[black!80] (-0.625*0.607-1,0.75-0.625*0.607) node[scale=0.75,anchor=north east]{$V_T$};
\draw (F1) -- (O) node[anchor=west,scale=0.75]{$f$};
\end{tikzpicture}
\figlab{badBufferCmos}
}
\caption{Buffer ge\"implementeerd met omgekeerde transistoren: NMOS is een slechte pull-up.??}%TODO: betere naam zoeken.
\figlab{badNmosPmos}
\end{figure}
\subsection{Fan-in en fan-out}
Enkele belangrijke eigenschappen van een poort zijn de \termen{fan-in} en \termen{fan-out}. De fan-in is het aantal ingangen die de poort in kwestie heeft. Deze eigenschap is eigen aan het type poort\footnote{Bijvoorbeeld een 3-nand heeft een fan-in van 3.}. De fan-out is het aantal ingangen van andere poorten die de poort kan aansturen, het aantal draden die uit de poort in kwestie komt zeg maar. Deze eigenschappen bepalen in grote mate de vertraging van een circuit. Indien we immers een groot aantal ingangen hebben, kan het een tijdje duren vooraleer de stroom doorheen de transistoren van de andere ingangen stroomt, als \'e\'en van de transistoren omschakelt. Verder zullen we ook de spanning moeten opdrijven bij een groot aantal transistoren die we in serie schakelen. Elke transistor heeft immers indien gesloten nog steeds een kleine weerstand. Indien de stroom door een serie transistoren moet vloeien, betekent dit dat de uitgangsspanning van de poort teveel gereduceerd zou zijn. Een hogere spanning leidt weer tot hogere vermogens wat we juist willen tegengaan. Bijgevolg willen we de fan-in zo laag mogelijk houden. Daarom komen in de realiteit poorten met een groot\footnote{Met groot bedoelen we meer dan 5 ingangen.} aantal ingangen nooit voor, indien men een poort met een groot aantal ingangen wil realiseren zal men deze meestal met een sequentie van eenvoudige poorten realiseren. Ook de fan-out moeten we onder controle houden. Elke poort heeft immers maar \'e\'en uitgang. Door deze uitgang moet alle stroom naar de ingangen van andere poorten stromen. Vermits de stroom over een draad beperkt is, kunnen we spreken over een \termen{maximale stroomsterkte $I_{\mbox{\small{Omax}}}$}. Deze stroomsterkte bepaalt dan weer hoe snel we de parasitaire capaciteit kunnen op- en ontladen:
\begin{equation}
I_{\mbox{\small{Omax}}}=C\cdot\displaystyle\frac{\partial V}{\partial t}
\end{equation}
Vermits de snelheid waarmee deze capaciteit op- en ontlaadt op zijn beurt weer invloed heeft op de vertraging, heeft het ook invloed op de \termen{maximale frequentie $f_{\mbox{\small{max}}}$} van de elektronische schakeling:
\begin{equation}
f_{\mbox{\small{max}}}=\displaystyle\frac{1}{\Delta V}\cdot\displaystyle\frac{\partial V}{\partial t}=\displaystyle\frac{I_{\mbox{\small{Omax}}}}{C\cdot \Delta V}
\end{equation}
Verder zal ook de parasitaire capaciteit zelf toenemen: de uitgang die met alle ingangen verbonden is vormt \'e\'en draad. Vermits de grootte van deze draad zal afhangen van de fan-out zal dus ook de parasitaire capaciteit toenemen. We kunnen de fan-out beperken door gebruik te maken van buffers: we verdelen de uitgang van de poort onder een paar buffers die op hun beurt de ingangen van de andere poorten bevoorraden. Het nadeel van het gebruik van een buffer is dat dit buffer ook moet omschakelen, wat extra vermogenverbruik en vertragingen teweeg brengt.
\subsection{Tri-state buffer}
Een uitbreiding van een buffer is een \termen{Tri-state buffer} ofwel \termen{3-state buffer}. \figref{triStateSymbol} toont het symbool die men voor deze component gebruikt: een buffer met een derde ingang aan de zijkant van de driehoek. Een tri-state buffer heeft drie mogelijke uitgangswaarden: 0, 1 en de zogenaamde \termen{zwevende modus $Z$}. Deze laatste toestand wordt ook wel \termen{hoog impedant} genoemd. Het betekent dat de lijn losgekoppeld is van enige bron. Er staat dus zogezegd niets op de lijn. Hiertoe wordt een buffer uitgebreid met een tweede ingang: \termen{Enable $E$}. Hierdoor kan de component in drie toestanden worden gebracht: wanneer enable op 0 staat, staat de tri-state buffer in de $Z$ stand. In het geval er een hoog signaal op de enable-ingang wordt aangelegd, laat het de ingang door, deze kan uiteraard in twee standen staan. Op \figref{triStateTruth} wordt dit concept met behulp van een waarheidstabel weergegeven. Er bestaan verschillende manieren om deze component te implementeren. Een goedkope manier wordt voorgesteld op \figref{triStateImplementation} en maakt gebruik van een \termen{transmission gate}, een component die functioneert als een schakelaar.
\begin{figure}[hbt]
\centering
\subfigure[Symbool]{
\begin{tikzpicture}
\node[tris] (TS) at (0,0) {};
\draw (TS.c) -- ++(0,-0.5) node[anchor=north]{enable $e$};
\draw (TS.x) -- ++(-0.5,0) node[anchor=east]{input $x$};
\draw (TS.z) -- ++(0.5,0) node[anchor=west]{output $f$};
\end{tikzpicture}
\figlab{triStateSymbol}}
\subfigure[Tabel]{
\begin{tikzpicture}
\node (A) at (0,0) {\begin{tabular}{cc|c}
$e$&$x$&$f$\\\hline
$0$&$0$&$Z$\\
$0$&$1$&$Z$\\
$1$&$0$&$0$\\
$1$&$1$&$1$
\end{tabular}};
\end{tikzpicture}
\figlab{triStateTruth}}
\subfigure[Implementatie]{
\begin{tikzpicture}[circuit logic US]
\node[transgate] (TG) at (0,0) {};
\node[not gate,scale=0.8] (N1) at (-1,0) {};
\node[not gate,scale=0.8] (N2) at (-2,0) {};
\node[not gate,scale=0.8] (N3) at (-1,1) {};
\draw (N1.output) -- (TG.x);
\draw (N2.output) -- (N1.input);
\draw (N2.input) -- (-3,0 |- N2.input) node[anchor=east]{$x$};
\draw (N3.input) -- (-3,0 |- N3.input) node[anchor=east]{$e$};
\draw (N3.output) -| (TG.cn);
\draw (TG.z) -- ++(0.5,0) node[anchor=west]{$f$};
\pdot{-2.6,1};
\draw (-2.6,1) |- (0,-1) -- (TG.c);
\end{tikzpicture}
\figlab{triStateImplementation}}
\subfigure[Transmission Gate]{
\begin{tikzpicture}
\node[transgate] (TG) at (0,0) {};
\draw (TG.x) -- ++(-0.5,0) node[anchor=east]{$x$};
\draw (TG.z) -- ++(0.5,0) node[anchor=west]{$f$};
\draw (TG.cn) -- ++(0,0.5) node[anchor=south]{$\bar{s}$};
\draw (TG.c) -- ++(0,-0.5) node[anchor=north]{$s$};
\draw (1.75,0) node{$\equiv$};
\begin{scope}[xshift=4 cm]
\node[pmosc,rotate=-90] (P) at (0,0.5) {};
\node[nmosc,rotate=90] (N) at (0,-0.5) {};
\coordinate (X) at (-0.75,0);
\coordinate (F) at (0.75,0);
\draw (P.source) -| (F) |- (N.source);
\draw (P.drain) -| (X) |- (N.drain);
\draw (X) -- ++(-0.5,0) node[anchor=east]{$x$};
\draw (F) -- ++(0.5,0) node[anchor=west]{$f$};
\draw (P.gate) -- ++(0,0.5) node[anchor=south]{$\bar{s}$};
\draw (N.gate) -- ++(0,-0.5) node[anchor=north]{$s$};
\pdot{X};\pdot{F};
\end{scope}
\end{tikzpicture}
\figlab{triStateTransmission}}
\subfigure[Bus]{
\begin{tikzpicture}[circuit logic US]
\node[tris] (T1) at (-0.75,0.5) {};
\node[tris] (T2) at (-0.75,-0.5) {};
\node[buffer gate] (B1) at (0.75,1) {};
\node[buffer gate] (B2) at (0.75,0) {};
\node[buffer gate] (B3) at (0.75,-1) {};
\draw (B1.input) -| (0,0) |- (B3.input);
\draw (0,0) -- (B2.input);\draw (0,0.5) -- (T1.z);\draw (0,-0.5) -- (T2.z);
\draw (T1.x) -- (T1.x -| -1.5,0) node[anchor=east]{$x_1$};
\draw (T1.c) |- (-1.25,0) node[anchor=east]{$e_1$};
\draw (T2.x) -- (T2.x -| -1.5,0) node[anchor=east]{$x_2$};
\draw (T2.c) |- (-1.25,-1) node[anchor=east]{$e_2$};
\draw (B1.output) -- ++(0.5,0) node[anchor=west]{$f_1$};
\draw (B2.output) -- ++(0.5,0) node[anchor=west]{$f_2$};
\draw (B3.output) -- ++(0.5,0) node[anchor=west]{$f_3$};
\pdot{0,0};\pdot{0,-0.5};\pdot{0,0.5};
\end{tikzpicture}
\figlab{triStateBus}}
\caption{Tri-state buffer.}
\end{figure}
De transmission gate zelf bestaat zoals te zien op \figref{triStateTransmission} uit twee transistoren. Indien $s=0$ zijn beide schakelaars open, en staat er dus geen signaal op uitgang $f$, in dat geval is de uitgang dus hoog impedant. Bij $s=1$ zijn beide transistoren gesloten, en wordt het signaal die aangelegd wordt op $x$ verder gepropageerd. De schakeling op \ref{fig:triStateImplementation} bevat extra componenten om het binnenkomende signaal opnieuw voldoende sterk te maken, dezelfde argumenten als met fan-in en fan-out gelden hier immers.
\paragraph{Bus}Tri-state buffers worden vaak gebruikt om bijvoorbeeld over een lijn gegevens van verschillende bronnen over te brengen. We hebben reeds in \ref{term:kortsluiting} gezien dat het verbinden van uitgangen tot kortsluiting leidt. Hierdoor zouden we echter genoodzaakt zijn om voor elke uitgang een aparte draad te voorzien. Verbindingen kunnen een groot gedeelte van het chipoppervlak beslaan, vandaar dat we het aantal draden ook tot een minimum proberen te beperken. We kunnen dit probleem oplossen met een zogenaamde \termen{bus}. Bij een bus wordt een draad gedeeld tussen verschillende bronnen $x_i$. Elk van die bronnen stuurt een tri-state buffer aan. Verder zorgt men er voor dat er slechts \'e\'en tri-state buffer tegelijk actief is. Omdat de andere tri-state buffers op dat moment hoog-impedant zijn, is er geen gevaar voor kortsluiting. \figref{triStateBus} geeft een implementatie van dit principe. In subsectie \ref{ss:multiplexer} zullen we verder een component tegenkomen die dit principe met een algemenere techniek realiseert: de multiplexer. In dat geval moeten de andere bronnen een $Z$ op de lijn zetten.
% \section{CAD-ontwerp in de praktijk}
% \subsection{Ingave}
% \subsection{Synthese}
% \subsection{Fysisch}
% \subsection{Chip}
%TODO: cad

%\part{Combinatorische en Sequenti\"ele Schakelingen}
%\chapter{Combinatorische Schakelingen (Schakelingen zonder geheugen)}
\chapterquote{Zelfs al zou er niets nieuws geschapen worden, dan is er nog altijd een nieuwe combinatie.}{Henry Ford, Amerikaans automobielfabrikant (1863-1947)}
\begin{chapterintro}
Nu we de basis van het bouwen van digitale circuits onder de knie hebben, en de fysische beperkingen hiervan kennen, wordt het tijd om ook schakelingen te ontwikkelen. \termen{Combinatorische Schakelingen} zijn schakelingen waarbij een bitvector aan uitgangen uitsluitend bepaald wordt aan de hand van een bitvector aan ingangen. We kunnen dus stellen dat het circuit een functie $\vec{f}(\vec{x})$ berekent. In sectie \ref{s:synthese} werd reeds een manier voorgesteld om tot een canonieke vorm te komen. In sectie \ref{s:minimalisatie} zullen we technieken zien om deze implementaties verder te minimaliseren. Verder zullen we in secties \ref{s:rekenkundig} en \ref{s:andereBasis} de implementatie van enkele populaire combinatorische schakelingen zien.% Tot slot zullen we ook combinatorische schakelingen in VHDL bouwen in sectie \ref{s:combinatorischVHDL}.
\end{chapterintro}
\minitoc[n]
\section{Minimaliseren van logische functies}
\label{s:minimalisatie}
% \subsection{Leidend voorbeeld: Worteltrekking en 7 segment display}
% We zullen in deze sectie de minimalisatie-technieken uitleggen aan de hand van een worteltrekking en 7 Segment Display. Een 7 Segment display wordt vaak in kleine elektronica gebruikt zoals bijvoorbeeld horloges om getallen weer te geven. Op figuur \ref{fig:sevenSegmentDisplay}
% \begin{figure}[hbt]
% \centering
% \begin{tikzpicture}
% \filldraw[fill=gray,draw=black] (-0.6,-1.1) rectangle (0.6,1.1);
% \fill[black!80] (0.325,0.95) -- (0.375,0.9) -- (0.325,0.85) -- (-0.325,0.85) -- (-0.375,0.9) -- (-0.325,0.95) -- cycle;
% \draw (0,0.85) node[scale=0.75,anchor=north]{$c$};
% \fill[black!80] (0.325,-0.95) -- (0.375,-0.9) -- (0.325,-0.85) -- (-0.325,-0.85) -- (-0.375,-0.9) -- (-0.325,-0.95) -- cycle;
% \draw (0,-0.85) node[scale=0.75,anchor=south]{$f$};
% \fill[black!80] (0.325,0.05) -- (0.375,0) -- (0.325,-0.05) -- (-0.325,-0.05) -- (-0.375,0) -- (-0.325,0.05) -- cycle;
% \draw (0,0.05) node[scale=0.75,anchor=south]{$i$};
% \fill[black!80] (0.4,0.875) -- (0.45,0.825) -- (0.45,0.075) -- (0.4,0.025) -- (0.35,0.075) -- (0.35,0.825) -- cycle;
% \draw (0.35,0.5) node[scale=0.75,anchor=east]{$d$};
% \fill[black!80] (-0.4,0.875) -- (-0.45,0.825) -- (-0.45,0.075) -- (-0.4,0.025) -- (-0.35,0.075) -- (-0.35,0.825) -- cycle;
% \draw (0.35,-0.5) node[scale=0.75,anchor=east]{$e$};
% \fill[black!80] (0.4,-0.875) -- (0.45,-0.825) -- (0.45,-0.075) -- (0.4,-0.025) -- (0.35,-0.075) -- (0.35,-0.825) -- cycle;
% \draw (-0.35,-0.5) node[scale=0.75,anchor=west]{$g$};
% \fill[black!80] (-0.4,-0.875) -- (-0.45,-0.825) -- (-0.45,-0.075) -- (-0.4,-0.025) -- (-0.35,-0.075) -- (-0.35,-0.825) -- cycle;
% \draw (-0.35,0.5) node[scale=0.75,anchor=west]{$h$};
% \draw (-5,1.4) rectangle (-2,2.6);
% \draw (-5,2) node[anchor=west,text width=2.5 cm] {Combinatorische Schakeling 1};
% \draw (-5,-1.1) rectangle (-2,1.1);
% \draw (-5,0) node[anchor=west,text width=2.5 cm] {Combinatorische Schakeling 2};
% \foreach \x/\t in {1/a,-1/b} {
%   \draw (-2,0.2*\x+2) -- (-0.6,0.2*\x+2) node[scale=0.75,anchor=south east,yshift=-0.1 cm]{$\t$};
% }
% \foreach \x/\t in {3/c,2/d,1/e,0/f,-1/g,-2/h,-3/i} {
%   \draw (-2,0.3*\x) -- (-0.6,0.3*\x) node[scale=0.75,anchor=south east,yshift=-0.1 cm]{$\t$};
% }
% \foreach \x/\t in {1.5/w,0.5/x,-0.5/y,-1.5/z} {
%   \draw (-8,0.5*\x) -- (-5,0.5*\x) node[scale=0.75,anchor=south east,yshift=-0.1 cm]{$\t$};
%   \fill (-6.5-0.24*\x,0.5*\x) circle (0.4 mm);
%   \draw (-6.5-0.24*\x,0.5*\x) -- (-6.5-0.24*\x,0.24*\x+2) -- (-5,0.24*\x+2) node[scale=0.75,anchor=south east,yshift=-0.1 cm]{$\t$};
% }
% \draw (5,0) node[scale=0.75]{\begin{tabular}{cccc|cc|ccccccc}
% $w$&$x$&$y$&$z$&$a$&$b$&$c$&$d$&$e$&$f$&$g$&$h$&$i$\\\hline
% 0&0&0&0		&0&0	&1&1&1&1&1&1&0\\
% 0&0&0&1		&0&1	&0&1&1&0&0&0&0\\
% 0&0&1&0		&0&1	&1&1&0&1&1&0&1\\
% 0&0&1&1		&0&1	&1&1&1&1&0&0&1\\
% 0&1&0&0		&1&0	&0&1&1&0&0&1&1\\
% 0&1&0&1		&1&0	&1&0&1&1&0&1&1\\
% 0&1&1&0		&1&0	&1&0&1&1&1&1&1\\
% 0&1&1&1		&1&0	&1&1&1&0&0&0&0\\
% 1&0&0&0		&1&0	&1&1&1&1&1&1&1\\
% 1&0&0&1		&1&1	&1&1&1&1&0&1&1\\
% 1&0&1&0		&1&1	&-&-&-&-&-&-&-\\
% 1&0&1&1		&1&1	&-&-&-&-&-&-&-\\
% 1&1&0&0		&1&1	&-&-&-&-&-&-&-\\
% 1&1&0&1		&1&1	&-&-&-&-&-&-&-\\
% 1&1&1&0		&1&1	&-&-&-&-&-&-&-\\
% 1&1&1&1		&1&1	&-&-&-&-&-&-&-
% \end{tabular}};
% \begin{scope}[xshift=-5.1 cm, yshift=-2.05 cm,scale=0.5]
% \foreach\x/\a/\b/\c/\d/\e/\f/\g in {0/8/8/8/8/8/8/3,1.2/3/8/8/3/3/3/3,2.4/8/8/3/8/8/3/8,3.6/8/8/8/8/3/3/8,4.8/3/8/8/3/3/8/8,6/8/3/8/8/3/8/8,7.2/8/3/8/8/8/8/8,8.4/8/8/8/3/3/3/3,9.6/8/8/8/8/8/8/8,10.8/8/8/8/8/3/8/8} {
% \begin{scope}[xshift=\x cm]
% \filldraw[fill=gray,draw=black] (-0.6,-1.1) rectangle (0.6,1.1);
% \fill[black!\a0] (0.325,0.95) -- (0.375,0.9) -- (0.325,0.85) -- (-0.325,0.85) -- (-0.375,0.9) -- (-0.325,0.95) -- cycle;%a
% \fill[black!\b0] (0.4,0.875) -- (0.45,0.825) -- (0.45,0.075) -- (0.4,0.025) -- (0.35,0.075) -- (0.35,0.825) -- cycle;%b
% \fill[black!\c0] (0.4,-0.875) -- (0.45,-0.825) -- (0.45,-0.075) -- (0.4,-0.025) -- (0.35,-0.075) -- (0.35,-0.825) -- cycle;%c
% \fill[black!\d0] (0.325,-0.95) -- (0.375,-0.9) -- (0.325,-0.85) -- (-0.325,-0.85) -- (-0.375,-0.9) -- (-0.325,-0.95) -- cycle;%d
% \fill[black!\e0] (-0.4,-0.875) -- (-0.45,-0.825) -- (-0.45,-0.075) -- (-0.4,-0.025) -- (-0.35,-0.075) -- (-0.35,-0.825) -- cycle;%e
% \fill[black!\f0] (-0.4,0.875) -- (-0.45,0.825) -- (-0.45,0.075) -- (-0.4,0.025) -- (-0.35,0.075) -- (-0.35,0.825) -- cycle;%f
% \fill[black!\g0] (0.325,0.05) -- (0.375,0) -- (0.325,-0.05) -- (-0.325,-0.05) -- (-0.375,0) -- (-0.325,0.05) -- cycle;%g
% \end{scope}
% }
% \end{scope}
% \end{tikzpicture}
% \caption{Leidend voorbeeld: Worteltrekking en 7 segment display.}
% \label{fig:sevenSegmentDisplay}
% \end{figure}
% tekenen we de 7 segment display samen met de waarheidstabel. We zijn enkel ge\"interesseerd in het weergeven van de nummers 0 tot 9. Indien een ander nummer aan de ingang verschijnt mag om het even wat op de display verschijnen. We introduceren hier dan ook al het begrip \termen{don't care}. Een don't care betekent dat het niet uitmaakt over er een 0 of 1 op de uitgang verschijnt, en worden genoteerd met een horizontale streep (-). Het is de bedoeling dat we de schakeling ontwikkelen die de 4 lijnen ($w$, $x$, $y$ en $z$) omzet naar 9 lijnen ($a$, $b$, $c$, $d$, $e$, $f$, $g$, $h$ en $i$). We zullen echter niet voor iedere techniek elke functie volledig ontwikkelen.
\subsection{Waarom minimaliseren}
We willen schakelingen ontwerpen voor de worteltrekking en 7 segment display in een minimale implementatie. Streven naar een minimalisatie is een algemeen probleem en is vergelijkbaar met optimalisatie in de informatica. Minimalisatie levert niet enkel snellere doorvoer op. Hieronder sommen we de meest courante voordelen op, samen met hun metriek:
\begin{itemize}
\item Minimaliseren van de \termen{kostprijs}. Afhankelijk van de realisatie hanteren we hiervoor 2 metrieken: Voor CMOS verfijnen we hiervoor de oorspronkelijke formule van de kostprijs uit vergelijking \ref{eqn:kosten}. En bekomen:
\begin{equation}
\mbox{kostprijs}=\displaystyle\sum_{\tiny\begin{array}{c}\mbox{alle}\\\mbox{poorten}\end{array}}{\mbox{kostprijs}\left(\mbox{poort}\right)}
\label{eqn:kostenCmos}
\end{equation}
De kostprijs van een poort wordt dan bepaald met volgende formule:
\begin{equation}
\mbox{kostprijs}\left(\mbox{poort}\right)=\left\{\begin{array}{lcl}
\mbox{fan-in}&\mbox{if}&\mbox{poort}\in\left\{\mbox{INV},\mbox{NAND},\mbox{NOR},\mbox{AOI},\mbox{OAI}\right\}\\
\mbox{fan-in}+1&\mbox{if}&\mbox{poort}\in\left\{\mbox{AND},\mbox{OR}\right\}\\
\end{array}\right.
\label{eqn:kostenCmosPoort}
\end{equation}
We zien dus dat alle \termen{inverterende poorten} relatief 1 goedkoper zijn dat de \termen{niet-inverterende poorten}. Bij een FPGA bepalen we de kostprijs aan de hand van het aantal logische cellen:
\begin{equation}
\mbox{kostprijs}=\#\mbox{LB's}
\end{equation}
Uiteraard dient hierbij de kanttekening gemaakt te worden, dat niet voor elk aantal logische cellen, er een FPGA bestaat. Indien er voldoende logische cellen op de FPGA aanwezig zijn, is de kostprijs dan ook van minder belang. We zullen immers toch dezelfde FPGA gebruiken.
\item Snelheid: We maximaliseren de snelheid door de maximale vertraging te minimaliseren. Deze vertraging is afhankelijk van de poorten in het \termen{Kritische Pad}. Het kritische pad is een pad van de ingang naar de uitgang met de grootste vertraging. Deze vertraging is afhankelijk van twee parameters:
\begin{itemize}
\item Vertraging van de poorten: een poort heeft tijd nodig om bij verandering van de ingang ook de uitgang te veranderen. Deze vertraging is afhankelijk van de technologie en evenredig met de fan-in. We zullen deze benaderen met volgende formule:
\begin{equation}
\mbox{vertraging}\left(\mbox{poort}\right)=\left\{\begin{array}{lcl}
0.6+0.4\cdot\mbox{fan-in}&\mbox{if}&\mbox{poort}\in\left\{\mbox{INV},\mbox{NAND},\mbox{NOR},\mbox{AOI},\mbox{OAI}\right\}\\
1.6+0.4\cdot\mbox{fan-in}&\mbox{if}&\mbox{poort}\in\left\{\mbox{AND},\mbox{OR}\right\}\\
\end{array}\right.
\label{eqn:speedPoort}
\end{equation}
Deze formule toont dus dat opnieuw de fan-in een belangrijke factor is, en dat niet-inverterende poorten opnieuw een nadeel hebben tegenover inverterende poorten.
\item (capacitieve) belasting: dit hangt hoofdzakelijk af van de geometrische implementatie van de printplaat. Deze vertraging is dan ook zeer moeilijk te berekenen en wordt niet beschouwd.
\end{itemize}
Bijgevolg berekenen we de vertraging als volgt:
\begin{equation}
\mbox{vertraging}=\displaystyle\sum_{\tiny\begin{array}{c}\mbox{kritisch}\\\mbox{pad}\end{array}}{\mbox{vertraging}\left(\mbox{poort}\right)}
\label{eqn:speed}
\end{equation}
\end{itemize}
\paragraph{Hoe minimaliseren?}Om te minimaliseren hebben we een methode nodig, deze methode manipuleert de logische uitdrukking van $\vec{f}(\vec{x})$ tot een uitdrukking die equivalent is, maar rekening houdend met de metrieken van vergelijkingen \ref{eqn:kostenCmos} en \ref{eqn:speed} voordeliger. Het probleem is dat er geen methodes bestaan die ons een reeks manipulaties voorstellen, waardoor we altijd tot de meest optimale implementatie komen. We moeten dus bijgevolg methodes bedenken waardoor we in staat zijn tot een redelijk optimale oplossingen te komen. Hiervoor zullen we methodes gebruiken zoals Karnaugh-kaarten. Computers bieden bovendien de mogelijkheid een groot aantal implementaties te simuleren. Bij een groot probleem volstaat deze rekenkracht echter ook niet om tot de beste implementatie te komen\footnote{Wat is \"uberhaupt de beste oplossing, in veel gevallen zal de ene metriek verbeteren ten koste van de tweede.}.
\subsection{Karnaugh-kaarten}
\termen{Karnaugh-kaarten} of \termen{K-kaarten} proberen het aantal nutteloze ingangen tot een minimum te beperken. Het is een visueel hulpmiddel die gebruik maakt van een vermogen waar mensen goed in zijn: het herkennen en begrijpen van patronen. De basis van een Karnaugh-kaart is dan ook de waarheidstabel. In een waarheidstabel kunnen we vaak door bepaalde rijen te beschouwen verbanden zien.
%Zo zien we in de waarheidtabel op figuur \ref{fig:sevenSegmentDisplay} dat $e=0$ als $w=1$. Uiteraard blijft er onduidelijkheid waarom $e=0$ als $\left(w,x,y,z\right)=\left(0,1,0,0\right)$. Toch kunnen we met dergelijke patronen de expressie al behoorlijk optimaliseren. Zo kunnen we voor $e$ al een expressie maken van de vorm $e=(\mbox{NOT }w)\mbox{ AND }\varphi$ met $\varphi$ een nog onbekende logische expressie. We weten immers dat indien $w=1$ er een 0 aan de ingang van de AND poort verschijnt. Een AND met aan minstens \'e\'en ingang een 0 is per definitie 0. We hebben dus het probleem kunnen reduceren met het zoeken naar een specifiek patroon.
\paragraph{N-kubus}In feite zijn Karnaugh-kaarten niets anders dan waarheidstabellen waarbij we het aantal dimensies verhogen. Een zogenoemde \termen{$N$-kubus}. We stellen immers een bepaalde ingangstoestand voor als een knooppunt op een kubus. Figuur \ref{fig:nCube} toont een N-kubus voor de dimensies 1 tot 4.
\begin{figure}[htb]
\centering
\subfigure[1D]{\begin{tikzpicture}
\draw[thick] (0,0) node[scale=0.75,anchor=north]{0} -- (1,0) node[scale=0.75,anchor=north]{1};
\fill (0,0) circle(0.05 cm);
\fill (1,0) circle(0.05 cm);
\end{tikzpicture}}
\subfigure[2D]{\begin{tikzpicture}
\draw[thick] (0,0) node[scale=0.75,anchor=south]{00} -- (1,0) node[scale=0.75,anchor=south]{01} -- (1,-1) node[scale=0.75,anchor=north]{11} -- (0,-1) node[scale=0.75,anchor=north]{10} -- cycle;
\fill (0,0) circle(0.05 cm);
\fill (1,0) circle(0.05 cm);
\fill (1,-1) circle(0.05 cm);
\fill (0,-1) circle(0.05 cm);
\end{tikzpicture}}
\subfigure[3D]{\begin{tikzpicture}
\draw[thick] (0,0) node[scale=0.75,anchor=south]{000} -- (1,0) node[scale=0.75,anchor=south]{001} -- (1,-1) node[scale=0.75,anchor=north]{011} -- (0,-1) node[scale=0.75,anchor=north]{010} -- cycle;
\draw[thick] (0,0,-1) node[scale=0.75,anchor=south]{100} -- (1,0,-1) node[scale=0.75,anchor=south]{101} -- (1,-1,-1) node[scale=0.75,anchor=north]{111} -- (0,-1,-1) node[scale=0.75,anchor=north]{110} -- cycle;
\draw[thick] (0,0,0) -- (0,0,-1);
\draw[thick] (0,-1,0) -- (0,-1,-1);
\draw[thick] (1,0,0) -- (1,0,-1);
\draw[thick] (1,-1,0) -- (1,-1,-1);
\fill (0,0) circle(0.05 cm);
\fill (1,0) circle(0.05 cm);
\fill (1,-1) circle(0.05 cm);
\fill (0,-1) circle(0.05 cm);
\fill (0,0,-1) circle(0.05 cm);
\fill (1,0,-1) circle(0.05 cm);
\fill (1,-1,-1) circle(0.05 cm);
\fill (0,-1,-1) circle(0.05 cm);
\end{tikzpicture}}
\subfigure[4D]{\begin{tikzpicture}
\draw[thick] (0,0) node[scale=0.75,anchor=south]{0000} -- (1,0) node[scale=0.75,anchor=south]{0001} -- (1,-1) node[scale=0.75,anchor=north]{0011} -- (0,-1) node[scale=0.75,anchor=north]{0010} -- cycle;
\draw[thick] (0,0,-1) node[scale=0.75,anchor=south]{0100} -- (1,0,-1) node[scale=0.75,anchor=south]{0101} -- (1,-1,-1) node[scale=0.75,anchor=north]{0111} -- (0,-1,-1) node[scale=0.75,anchor=north]{0110} -- cycle;
\draw[thick] (0,0,0) -- (0,0,-1);
\draw[thick] (0,-1,0) -- (0,-1,-1);
\draw[thick] (1,0,0) -- (1,0,-1);
\draw[thick] (1,-1,0) -- (1,-1,-1);
\fill (0,0) circle(0.05 cm);
\fill (1,0) circle(0.05 cm);
\fill (1,-1) circle(0.05 cm);
\fill (0,-1) circle(0.05 cm);
\fill (0,0,-1) circle(0.05 cm);
\fill (1,0,-1) circle(0.05 cm);
\fill (1,-1,-1) circle(0.05 cm);
\fill (0,-1,-1) circle(0.05 cm);
\begin{scope}[xshift=2 cm, yshift=1 cm]
\draw[thick] (0,0) node[scale=0.75,anchor=south]{1000} -- (1,0) node[scale=0.75,anchor=south]{1001} -- (1,-1) node[scale=0.75,anchor=north]{1011} -- (0,-1) node[scale=0.75,anchor=north]{1010} -- cycle;
\draw[thick] (0,0,-1) node[scale=0.75,anchor=south]{1100} -- (1,0,-1) node[scale=0.75,anchor=south]{1101} -- (1,-1,-1) node[scale=0.75,anchor=north]{1111} -- (0,-1,-1) node[scale=0.75,anchor=north]{1110} -- cycle;
\draw[thick] (0,0,0) -- (0,0,-1);
\draw[thick] (0,-1,0) -- (0,-1,-1);
\draw[thick] (1,0,0) -- (1,0,-1);
\draw[thick] (1,-1,0) -- (1,-1,-1);
\fill (0,0) circle(0.05 cm);
\fill (1,0) circle(0.05 cm);
\fill (1,-1) circle(0.05 cm);
\fill (0,-1) circle(0.05 cm);
\fill (0,0,-1) circle(0.05 cm);
\fill (1,0,-1) circle(0.05 cm);
\fill (1,-1,-1) circle(0.05 cm);
\fill (0,-1,-1) circle(0.05 cm);
\end{scope}
\foreach\x in {0,1} {
 \foreach\y in {0,-1} {
  \foreach\z in {0,-1} {
   \draw[thick] (\x,\y,\z) -- (\x+2,\y+1,\z);
  }
 }
}
\end{tikzpicture}}
\caption{N-cubus voor dimensies 1 tot 4.}
\label{fig:nCube}
\end{figure}
Hierbij is elke knoop van de kubus een bitvector van waarden aan de ingang. We kunnen dan vervolgens op deze knooppunten de waarde die we aan de uitgang verwachten plaatsen. De $N$-kubus toont ook de buren van deze toestanden. Dit zijn toestanden waarbij exact \'e\'en bit aan de ingang verandert is. Buren zijn cruciaal, indien de uitgang niet verandert tussen twee buren, kunnen we zeggen dat de veranderde ingangsbit irrelevant is voor de uitgang.
\paragraph{Karnaugh-Kaarten}Karnaugh-kaarten zijn in feite niets anders dan 2D projecties van de $N$-kubus. Uiteraard is deze omvorming tot 2 dimensies helemaal niet moeilijk: het is gewoon de $N$-kubus zelf. Vanaf dimensies hoger dan 2 wordt het moeilijker. Figuur \ref{fig:nCubeKarnaugh}
\begin{figure}[htb]
\centering
\begin{tikzpicture}
\begin{scope}[yshift=-0.25 cm]
\draw[thick] (0,0) node[scale=0.75,anchor=south]{000} -- (1,0) node[scale=0.75,anchor=south]{001} -- (1,-1) node[scale=0.75,anchor=north]{011} -- (0,-1) node[scale=0.75,anchor=north]{010} -- cycle;
\draw[thick] (0,0,-1) node[scale=0.75,anchor=south]{100} -- (1,0,-1) node[scale=0.75,anchor=south]{101} -- (1,-1,-1) node[scale=0.75,anchor=north]{111} -- (0,-1,-1) node[scale=0.75,anchor=north]{110} -- cycle;
\draw[thick] (0,0,0) -- (0,0,-1);
\draw[thick] (0,-1,0) -- (0,-1,-1);
\draw[thick] (1,0,0) -- (1,0,-1);
\draw[thick] (1,-1,0) -- (1,-1,-1);
\fill (0,0) circle(0.05 cm);
\fill (1,0) circle(0.05 cm);
\fill (1,-1) circle(0.05 cm);
\fill (0,-1) circle(0.05 cm);
\fill (0,0,-1) circle(0.05 cm);
\fill (1,0,-1) circle(0.05 cm);
\fill (1,-1,-1) circle(0.05 cm);
\fill (0,-1,-1) circle(0.05 cm);
\end{scope}
\draw (2,-0.5) node{\Large $\Rightarrow$};
\begin{scope}[xshift=2.5 cm]
\foreach\y in {0,1} {
  \foreach\z in {0,1} {
    \coordinate (N0\y\z) at (\z,-\y);
    \fill (N0\y\z) circle(0.05 cm) node[scale=0.75,anchor=south west]{0\y\z};
  }
}
\foreach\y in {0,1} {
  \foreach\z in {0,1} {
    \coordinate (N1\y\z) at (3-\z,-\y);
    \fill (N1\y\z) circle(0.05 cm) node[scale=0.75,anchor=south west]{1\y\z};
  }
}
\draw[thick] (N000) -- (N001) -- (N101) -- (N100) -- (N110) -- (N111) -- (N011) -- (N010) -- cycle;
\draw[thick] (N001) -- (N011);
\draw[thick] (N101) -- (N111);
\draw[thick] (N000) .. controls (-0.75,0.75) and (3.75,0.75) .. (N100);
\draw[thick] (N010) .. controls (-0.75,-1.75) and (3.75,-1.75) .. (N110);
\end{scope}
\draw (6.5,-0.5) node{\Large $\Rightarrow$};
\begin{scope}[xshift=7.25 cm, yshift=-0.25 cm]
\foreach\y in {0,1} {
  \foreach\z in {0,1} {
    \coordinate (N0\y\z) at (\z,-\y);
    \fill[gray] (N0\y\z) circle(0.05 cm) node[scale=0.75,black,anchor=south west]{0\y\z};
  }
}
\foreach\y in {0,1} {
  \foreach\z in {0,1} {
    \coordinate (N1\y\z) at (3-\z,-\y);
    \fill[gray] (N1\y\z) circle(0.05 cm) node[scale=0.75,black,anchor=south west]{1\y\z};
  }
}
\draw[gray] (N000) -- (N001) -- (N101) -- (N100) -- (N110) -- (N111) -- (N011) -- (N010) -- cycle;
\draw[gray] (N001) -- (N011);
\draw[gray] (N101) -- (N111);
\draw[gray] (N000) .. controls (-0.75,0.75) and (3.75,0.75) .. (N100);
\draw[gray] (N010) .. controls (-0.75,-1.75) and (3.75,-1.75) .. (N110);
\foreach\x in {0,1,2,3} {
  \foreach\y in {0,1} {
    \draw[thick] (\x-0.25,-\y+0.75) -- ++(1,0);
    \draw[thick] (\x-0.25,-\y+0.75) -- ++(0,-1);
  }
}
\draw[thick] (-0.25,-1.25) -| ++(4,2);
\draw [thick] (1.75,-1.5) to node[midway,below]{$x$} (3.75,-1.5);
\draw [thick] (0.75,1) to node[midway,above]{$z$} (2.75,1);
\draw [thick] (4,-0.25) to node[midway,sloped,above]{$y$} (4,-1.25);
\end{scope}
\end{tikzpicture}
\caption{Van $N$-kubus naar Karnaugh-kaart.}
\label{fig:nCubeKarnaugh}
\end{figure}
toont de overgang van een 3-kubus naar de respectievelijke Karnaugh-kaart. Bij een $N$-kubus heeft elke toestand $N$ buren. Nochtans zien we op de figuur dat de linkse en rechtse toestanden slechts 2 buren hebben in plaats van 3. We moeten dan ook een Karnaugh-kaart modulo rekenen. De linkse buur van het meest linkse veld is het rechtse veld. Verder omvat elke cel in de Karnaugh-kaart een bepaalde configuratie aan de ingang. Aan de rand van de Karnaugh-kaart staan de variabelen, en een lijn. De cellen die deze lijn omvatten zijn de cellen waar deze specifieke variabele 1 is. Logischerwijs zijn de cellen die niet omvat worden, de cellen waar deze variabele 0 is. Figuur \ref{fig:karnaughKaarten}
\begin{figure}[hbt]
\centering
\subfigure[1D]{\begin{tikzpicture}[scale=0.75]
\def\sc{0.75};
\draw[thick] (0,0.5) -- ++(0,-0.5) -- ++(0.5,0) -- ++(0,0.5);
\draw[thick] (0,0.5) rectangle ++(0.5,0.5);
\draw[thick] (0,1) -- (-0.25,1.25);
\draw (-0.125,1.125) node[anchor=north east,scale=0.75]{$x$};
\draw[thick] (-0.125,0) to node[sloped,below,rotate=180]{$x$} (-0.125,0.5);
\draw (0.25,0.25) node[scale=\sc]{1};
\draw (0.25,0.75) node[scale=\sc]{0};
\end{tikzpicture}}
\subfigure[2D]{\begin{tikzpicture}[scale=0.75]
\def\sc{0.75};
\draw[thick] (0,0) -- ++(1,0) -- ++(0,1) -- ++(-1,0) -- cycle;
\draw[thick] (0,0.5) -- ++(1,0);
\draw[thick] (0.5,0) -- ++(0,1);
\draw[thick] (0,1) -- (-0.25,1.25);
\draw (-0.125,1.125) node[anchor=north east,scale=0.75]{$x$};
\draw (-0.125,1.125) node[anchor=south west,scale=0.75]{$y$};
\draw[thick] (-0.125,0) to node[sloped,below,rotate=180]{$x$} (-0.125,0.5);
\draw[thick] (0.5,1.125) to node[sloped,above]{$y$} (1,1.125);
\draw (0.25,0.25) node[scale=\sc]{2};
\draw (0.25,0.75) node[scale=\sc]{0};
\draw (0.75,0.25) node[scale=\sc]{3};
\draw (0.75,0.75) node[scale=\sc]{1};
\end{tikzpicture}}
\subfigure[3D]{\begin{tikzpicture}[scale=0.75]
\def\sc{0.75};
\draw[thick] (0,0) -- ++(1,0) -- ++(0,2) -- ++(-1,0) -- cycle;
\draw[thick] (0,0.5) -- ++(1,0);
\draw[thick] (0,1) -- ++(1,0);
\draw[thick] (0,1.5) -- ++(1,0);
\draw[thick] (0.5,0) -- ++(0,2);
\draw[thick] (0,2) -- (-0.25,2.25);
\draw (-0.125,2.125) node[anchor=north east,scale=0.75]{$xy$};
\draw (-0.125,2.125) node[anchor=south west,scale=0.75]{$z$};
\draw[thick] (-0.125,0) to node[sloped,below,rotate=180]{$x$} (-0.125,1);
\draw[thick] (1.125,0.5) to node[sloped,above,rotate=180]{$y$} (1.125,1.5);
\draw[thick] (0.5,2.125) to node[sloped,above]{$z$} (1,2.125);
\draw (0.25,1.25) node[scale=\sc]{2};
\draw (0.25,1.75) node[scale=\sc]{0};
\draw (0.75,1.25) node[scale=\sc]{3};
\draw (0.75,1.75) node[scale=\sc]{1};
\draw (0.25,0.25) node[scale=\sc]{4};
\draw (0.25,0.75) node[scale=\sc]{6};
\draw (0.75,0.25) node[scale=\sc]{5};
\draw (0.75,0.75) node[scale=\sc]{7};
\end{tikzpicture}}
\subfigure[4D]{\begin{tikzpicture}[scale=0.75]
\def\sc{0.75};
\draw[thick] (0,0) -- ++(2,0) -- ++(0,2) -- ++(-2,0) -- cycle;
\draw[thick] (0,0.5) -- ++(2,0);
\draw[thick] (0,1) -- ++(2,0);
\draw[thick] (0,1.5) -- ++(2,0);
\draw[thick] (0.5,0) -- ++(0,2);
\draw[thick] (1,0) -- ++(0,2);
\draw[thick] (1.5,0) -- ++(0,2);
\draw[thick] (0,2) -- (-0.25,2.25);
\draw (-0.125,2.125) node[anchor=north east,scale=0.75]{$wx$};
\draw (-0.125,2.125) node[anchor=south west,scale=0.75]{$yz$};
\draw[thick] (-0.125,0) to node[sloped,below,rotate=180]{$w$} (-0.125,1);
\draw[thick] (2.125,0.5) to node[sloped,above,rotate=180]{$x$} (2.125,1.5);
\draw[thick] (0.5,2.125) to node[sloped,above]{$y$} (1.5,2.125);
\draw[thick] (1,-0.125) to node[sloped,below]{$z$} (2,-0.125);
\draw (0.25,1.75) node[scale=\sc]{0};
\draw (1.75,1.75) node[scale=\sc]{1};
\draw (0.75,1.75) node[scale=\sc]{2};
\draw (1.25,1.75) node[scale=\sc]{3};
\draw (0.25,1.25) node[scale=\sc]{4};
\draw (1.75,1.25) node[scale=\sc]{5};
\draw (0.75,1.25) node[scale=\sc]{6};
\draw (1.25,1.25) node[scale=\sc]{7};
\draw (0.25,0.25) node[scale=\sc]{8};
\draw (1.75,0.25) node[scale=\sc]{9};
\draw (0.75,0.25) node[scale=\sc]{10};
\draw (1.25,0.25) node[scale=\sc]{11};
\draw (0.25,0.75) node[scale=\sc]{12};
\draw (1.75,0.75) node[scale=\sc]{13};
\draw (0.75,0.75) node[scale=\sc]{14};
\draw (1.25,0.75) node[scale=\sc]{15};
\end{tikzpicture}}
\subfigure[5D gespiegeld]{\begin{tikzpicture}[scale=0.75]
\def\sc{0.75};
\draw[thick] (0,0) -- ++(2,0) -- ++(0,2) -- ++(-2,0) -- cycle;
\draw[thick] (0,0.5) -- ++(2,0);
\draw[thick] (0,1) -- ++(2,0);
\draw[thick] (0,1.5) -- ++(2,0);
\draw[thick] (0.5,0) -- ++(0,2);
\draw[thick] (1,0) -- ++(0,2);
\draw[thick] (1.5,0) -- ++(0,2);
\draw[thick] (0,2) -- (-0.25,2.25);
\draw (-0.125,2.125) node[anchor=north east,scale=0.75]{$wx$};
\draw (-0.125,2.125) node[anchor=south west,scale=0.75]{$yz$};
\draw[thick] (3,2) -- (2.75,2.25);
\draw (2.875,2.125) node[anchor=north east,scale=0.75]{$wx$};
\draw (2.875,2.125) node[anchor=south west,scale=0.75]{$yz$};

\draw[thick] (-0.125,0) to node[sloped,below,rotate=180]{$w$} (-0.125,1);
\draw[thick] (5.125,0.5) to node[sloped,above,rotate=180]{$x$} (5.125,1.5);
\draw[thick] (0.5,2.125) to node[sloped,above]{$y$} (1.5,2.125);
\draw[thick] (3.5,2.125) to node[sloped,above]{$y$} (4.5,2.125);
\draw[thick] (1,-0.125) to node[sloped,below]{$z$} (2,-0.125);
\draw[thick] (3,-0.125) to node[sloped,below]{$z$} (4,-0.125);
\draw[thick] (3,2.75) to node[sloped,above]{$v$} (5,2.75);

\draw[thick] (3,0) -- ++(2,0) -- ++(0,2) -- ++(-2,0) -- cycle;
\draw[thick] (3,0.5) -- ++(2,0);
\draw[thick] (3,1) -- ++(2,0);
\draw[thick] (3,1.5) -- ++(2,0);
\draw[thick] (3.5,0) -- ++(0,2);
\draw[thick] (4,0) -- ++(0,2);
\draw[thick] (4.5,0) -- ++(0,2);

\draw (0.25,1.75) node[scale=\sc]{0};
\draw (1.75,1.75) node[scale=\sc]{1};
\draw (0.75,1.75) node[scale=\sc]{2};
\draw (1.25,1.75) node[scale=\sc]{3};
\draw (0.25,1.25) node[scale=\sc]{4};
\draw (1.75,1.25) node[scale=\sc]{5};
\draw (0.75,1.25) node[scale=\sc]{6};
\draw (1.25,1.25) node[scale=\sc]{7};
\draw (0.25,0.25) node[scale=\sc]{8};
\draw (1.75,0.25) node[scale=\sc]{9};
\draw (0.75,0.25) node[scale=\sc]{10};
\draw (1.25,0.25) node[scale=\sc]{11};
\draw (0.25,0.75) node[scale=\sc]{12};
\draw (1.75,0.75) node[scale=\sc]{13};
\draw (0.75,0.75) node[scale=\sc]{14};
\draw (1.25,0.75) node[scale=\sc]{15};

\draw (4.75,1.75) node[scale=\sc]{16};
\draw (3.25,1.75) node[scale=\sc]{17};
\draw (4.25,1.75) node[scale=\sc]{18};
\draw (3.75,1.75) node[scale=\sc]{19};
\draw (4.75,1.25) node[scale=\sc]{20};
\draw (3.25,1.25) node[scale=\sc]{21};
\draw (4.25,1.25) node[scale=\sc]{22};
\draw (3.75,1.25) node[scale=\sc]{23};
\draw (4.75,0.25) node[scale=\sc]{24};
\draw (3.25,0.25) node[scale=\sc]{25};
\draw (4.25,0.25) node[scale=\sc]{26};
\draw (3.75,0.25) node[scale=\sc]{27};
\draw (4.75,0.75) node[scale=\sc]{28};
\draw (3.25,0.75) node[scale=\sc]{29};
\draw (4.25,0.75) node[scale=\sc]{30};
\draw (3.75,0.75) node[scale=\sc]{31};
\label{fig:karnaughKaarten5DMirror}
\end{tikzpicture}}
\subfigure[5D gekopieerd]{\begin{tikzpicture}[scale=0.75]
\def\sc{0.75};
\draw[thick] (0,0) -- ++(2,0) -- ++(0,2) -- ++(-2,0) -- cycle;
\draw[thick] (0,0.5) -- ++(2,0);
\draw[thick] (0,1) -- ++(2,0);
\draw[thick] (0,1.5) -- ++(2,0);
\draw[thick] (0.5,0) -- ++(0,2);
\draw[thick] (1,0) -- ++(0,2);
\draw[thick] (1.5,0) -- ++(0,2);
\draw[thick] (-0.125,0) to node[sloped,below,rotate=180]{$w$} (-0.125,1);
\draw[thick] (4.625,0.5) to node[sloped,above,rotate=180]{$x$} (4.625,1.5);
\draw[thick] (0.5,2.125) to node[sloped,above]{$y$} (1.5,2.125);
\draw[thick] (3,2.125) to node[sloped,above]{$y$} (4,2.125);
\draw[thick] (1,-0.125) to node[sloped,below]{$z$} (2,-0.125);
\draw[thick] (3.5,-0.125) to node[sloped,below]{$z$} (4.5,-0.125);
\draw[thick] (2.5,2.75) to node[sloped,above]{$v$} (4.5,2.75);

\draw[thick] (2.5,0) -- ++(2,0) -- ++(0,2) -- ++(-2,0) -- cycle;
\draw[thick] (2.5,0.5) -- ++(2,0);
\draw[thick] (2.5,1) -- ++(2,0);
\draw[thick] (2.5,1.5) -- ++(2,0);
\draw[thick] (3,0) -- ++(0,2);
\draw[thick] (3.5,0) -- ++(0,2);
\draw[thick] (4,0) -- ++(0,2);

\draw (0.25,1.75) node[scale=\sc]{0};
\draw (1.75,1.75) node[scale=\sc]{1};
\draw (0.75,1.75) node[scale=\sc]{2};
\draw (1.25,1.75) node[scale=\sc]{3};
\draw (0.25,1.25) node[scale=\sc]{4};
\draw (1.75,1.25) node[scale=\sc]{5};
\draw (0.75,1.25) node[scale=\sc]{6};
\draw (1.25,1.25) node[scale=\sc]{7};
\draw (0.25,0.25) node[scale=\sc]{8};
\draw (1.75,0.25) node[scale=\sc]{9};
\draw (0.75,0.25) node[scale=\sc]{10};
\draw (1.25,0.25) node[scale=\sc]{11};
\draw (0.25,0.75) node[scale=\sc]{12};
\draw (1.75,0.75) node[scale=\sc]{13};
\draw (0.75,0.75) node[scale=\sc]{14};
\draw (1.25,0.75) node[scale=\sc]{15};

\draw (2.75,1.75) node[scale=\sc]{16};
\draw (4.25,1.75) node[scale=\sc]{17};
\draw (3.25,1.75) node[scale=\sc]{18};
\draw (3.75,1.75) node[scale=\sc]{19};
\draw (2.75,1.25) node[scale=\sc]{20};
\draw (4.25,1.25) node[scale=\sc]{21};
\draw (3.25,1.25) node[scale=\sc]{22};
\draw (3.75,1.25) node[scale=\sc]{23};
\draw (2.75,0.25) node[scale=\sc]{24};
\draw (4.25,0.25) node[scale=\sc]{25};
\draw (3.25,0.25) node[scale=\sc]{26};
\draw (3.75,0.25) node[scale=\sc]{27};
\draw (2.75,0.75) node[scale=\sc]{28};
\draw (4.25,0.75) node[scale=\sc]{29};
\draw (3.25,0.75) node[scale=\sc]{30};
\draw (3.75,0.75) node[scale=\sc]{31};
\label{fig:karnaughKaarten5DCopy}
\end{tikzpicture}}
\caption{Karnaugh-kaarten voor verschillende dimensies met binaire waarden.}
\label{fig:karnaughKaarten}
\end{figure}
toont Karnaugh-kaarten voor een verschillend aantal variabelen. In de cellen staat de decimale waarde van de ingang die deze cel vertegenwoordigt. Hierbij gebruiken we de variabelen in de volgende volgorde: $\left(v,w,x,y,z\right)$. We kunnen in principe blijven uitbreiden. Vanaf 6 dimensies wordt het echter moeilijk om nog patronen te herkennen. Karnaugh-kaarten hebben bijgevolg maar een beperkt vermogen. Vanaf 5 dimensies groepeert men meestal cellen in groepen van $4\times 4$. Men gebruikt hierbij 2 varianten: ofwel spiegelt men \'e\'en van de twee tabellen zoals op figuur \ref{fig:karnaughKaarten5DMirror}, ofwel zijn de tabellen exacte kopies (op enkele variabelen na) zoals op figuur \ref{fig:karnaughKaarten5DCopy}. Het spiegelen van variabelen is intu\"itiever omdat dit consistent is met lagere dimensies.
\subsubsection{Optimaliseren met behulp van Karnaugh-kaarten}
\paragraph{Terminologie} Alvorens we aan het optimalisatiewerk kunnen beginnen, hebben we nood aan enige terminologie. Een \termen{implicant} is een productterm waarvoor de functie 1 is. Deze definitie heeft nauwe banden met de 1-minterm, het verschil is echter dat bij een implicant niet alle variabelen aanwezig moeten zijn. Zo zien we op figuur \ref{fig:karnaughKaartenImplicanten} verschillende implicanten met een verschillend aantal variabelen. Een \termen{priemimplicant} is een implicant die geen onderdeel is van een andere implicant met strikt minder variabelen. De priemimplicanten van figuur \ref{fig:karnaughKaartenImplicanten} worden weergegeven op figuur \ref{fig:karnaughKaartenPriemimplicanten}. Een verdere uitbreiding is de \termen{essenti\"ele priemimplicant}, dit is een priemimplicant die minstens \'e\'en 1-minterm omvat die niet in een andere priemimplicant verweven zit. Tenslotte defini\"eren we de \termen{dekking} of \termen{cover} als een verzameling van implicanten die in alle mogelijkheden voorziet waar de functie 1 is.
\begin{figure}[hbt]
\centering
% \subfigure[Implicanten]{\begin{tikzpicture}[scale=1.5]
% \draw[thick] (-1,-1) rectangle ++(2,2);
% \draw[thick] (-0.5,-1) -- ++(0,2);
% \draw[thick] (0,-1) -- ++(0,2);
% \draw[thick] (0.5,-1) -- ++(0,2);
% \draw[thick] (-1,-0.5) -- ++(2,0);
% \draw[thick] (-1,0) -- ++(2,0);
% \draw[thick] (-1,0.5) -- ++(2,0);
% \draw[thick] (-1,1) -- ++(-0.25,0.25) node[anchor=south east]{$b$};
% \draw[thick] (-1.125,0) to node[midway,below,sloped]{$w$} (-1.125,-1);
% \draw[thick] (1.125,0.5) to node[midway,above,sloped]{$x$} (1.125,-0.5);
% \draw[thick] (-0.5,1.125) to node[midway,above,sloped]{$y$} (0.5,1.125);
% \draw[thick] (0,-1.125) to node[midway,below,sloped]{$z$} (1,-1.125);
% \draw (-0.75,0.75) node{0};
% \draw (-0.25,0.75) node{1};
% \draw (0.25,0.75) node{1};
% \draw (0.75,0.75) node{1};
% \draw (-0.75,0.25) node{0};
% \draw (-0.25,0.25) node{0};
% \draw (0.25,0.25) node{0};
% \draw (0.75,0.25) node{0};
% \draw (-0.75,-0.25) node{1};
% \draw (-0.25,-0.25) node{1};
% \draw (0.25,-0.25) node{1};
% \draw (0.75,-0.25) node{1};
% \draw (-0.75,-0.75) node{0};
% \draw (-0.25,-0.75) node{1};
% \draw (0.25,-0.75) node{1};
% \draw (0.75,-0.75) node{1};
% \foreach \x/\y/\w/\h/\r/\s in {1/0/1/1/0.375/dashed, 2/0/1/1/0.375/dashed, 3/0/1/1/0.375/dashed, 0/2/1/1/0.375/dashed, 1/2/1/1/0.375/dashed, 2/2/1/1/0.375/dashed, 3/2/1/1/0.375/dashed, 1/3/1/1/0.375/dashed, 2/3/1/1/0.375/dashed, 3/3/1/1/0.375/dashed, 1/0/2/1/0.4/thin, 2/0/2/1/0.4/thin, 0/2/2/1/0.4/thin, 1/2/2/1/0.4/thin, 2/2/2/1/0.4/thin, 1/3/2/1/0.4/thin, 2/3/2/1/0.4/thin,1/2/1/2/0.4/thin, 2/2/1/2/0.4/thin, 3/2/1/2/0.4/thin,0/2/4/1/0.425/thick,1/2/2/2/0.425/thick,2/2/2/2/0.425/thick} {
%   \draw[rounded corners,semitransparent,\s] (0.5*\x-0.5-\r,0.5+\r-0.5*\y) rectangle ++(0.5*\w-1+2*\r,-0.5*\h+1-2*\r);
% }
%\end{tikzpicture}
%\label{fig:karnaughKaartenImplicanten}}
% \subfigure[Priemimplicanten]{\begin{tikzpicture}[scale=1.2]
% \draw[thick] (-1,-1) rectangle ++(2,2);
% \draw[thick] (-0.5,-1) -- ++(0,2);
% \draw[thick] (0,-1) -- ++(0,2);
% \draw[thick] (0.5,-1) -- ++(0,2);
% \draw[thick] (-1,-0.5) -- ++(2,0);
% \draw[thick] (-1,0) -- ++(2,0);
% \draw[thick] (-1,0.5) -- ++(2,0);
% \draw[thick] (-1,1) -- ++(-0.25,0.25) node[anchor=south east]{$b$};
% \draw[thick] (-1.125,0) to node[midway,below,sloped]{$w$} (-1.125,-1);
% \draw[thick] (1.125,0.5) to node[midway,above,sloped]{$x$} (1.125,-0.5);
% \draw[thick] (-0.5,1.125) to node[midway,above,sloped]{$y$} (0.5,1.125);
% \draw[thick] (0,-1.125) to node[midway,below,sloped]{$z$} (1,-1.125);
% \end{tikzpicture}
% \label{fig:karnaughKaartenPriemimplicanten}}
\subfigure[Implicanten]{\begin{tikzpicture}
\def\ta{0.12}
\def\tb{0.09}
\def\tc{0.06}
\def\td{0.03}
\kkaartdmarks[1.6]{0}{0}{/0/1/0/1/\ta,/1/2/1/2/\ta,/2/1/2/1/\ta,/3/0/3/0/\ta,/3/1/3/1/\ta,/3/3/3/3/\ta,/3/0/3/1/\tc,/2/1/3/1/\td}{/1/1/1/\tb}{/3/3/1/\tb};
\kkaartd[1.6]{0}{0}{$f$}{$x_1$/$x_2$/$x_3$/$x_4$}{0/0/1/0/0/0/0/1/1/1/1/0/0/0/1/0};
\end{tikzpicture}
\label{fig:karnaughKaartenImplicanten}}
\subfigure[Priemimplicanten]{\begin{tikzpicture}
\def\ta{0.12}
\def\tb{0.09}
\def\tc{0.06}
\def\td{0.03}
\kkaartdmarks[1.6]{0}{0}{/1/2/1/2/\ta,/2/1/3/1/\tb,/3/0/3/1/\tc}{/1/1/1/\ta}{/3/3/1/\ta};
\kkaartd[1.6]{0}{0}{$f$}{$x_1$/$x_2$/$x_3$/$x_4$}{0/0/1/0/0/0/0/1/1/1/1/0/0/0/1/0};
\end{tikzpicture}
\label{fig:karnaughKaartenPriemimplicanten}}
\subfigure[Essenti\"ele priemimplicanten]{\begin{tikzpicture}
\def\ta{0.12}
\def\tb{0.09}
\def\tc{0.06}
\def\td{0.03}
\kkaartdmarks[1.6]{0}{0}{/1/2/1/2/\ta,/2/1/3/1/\tb}{/1/1/1/\ta}{/3/3/1/\ta};
\kkaartd[1.6]{0}{0}{$f$}{$x_1$/$x_2$/$x_3$/$x_4$}{0/0/1/0/0/0/0/1/1/1/1/0/0/0/1/0};
\end{tikzpicture}
\label{fig:karnaughKaartenEssentielePriemimplicanten}}
\caption{Terminologie van een Karnaugh-kaart.}
\label{fig:karnaughKaartTerminologie}
\end{figure}
\paragraph{Stappenplan}
We minimaliseren een functie met behulp van een Karnaugh-kaart in 4 stappen, deze stappen zullen we in de volgende paragrafen besproken:
\begin{enumerate}
 \item Maak de Karnaugh-kaart.
 \item Bepaal alle priemimplicanten.
 \item Bepaal alle essenti\"ele priemimplicanten.
 \item Zoek de minimale dekking.
\end{enumerate}
Daarna zullen we nog drie speciale gevallen bestuderen.
\paragraph{}
We bestuderen deze methode aan de hand van een voorbeeld. We zullen een schakeling synthetiseren die de functies $f$ en $g$ berekend. Functie $f$ geeft 1 terug bij de getallen 0, 1, 3, 7, 5, 8, 10, 11, 14 en 15, en 0 in de andere gevallen, $g$ is waar  als de afgeronde vierkantswortel van het getal even is.
\paragraph{Stap 1: maak de Karnaugh-kaart}
In de eerste stap bouwen we een Karnaugh-kaart op, op basis van de gegeven functie. We tekenen een Karnaugh-kaart met het juiste aantal ingangsvariabelen (zie figuur \ref{fig:karnaughKaarten}) en vullen vervolgens de uitgangswaarden voor \'e\'en bepaalde uitgang in op de respectievelijke plaatsen. Er dient dus per binaire uitgang zo'n kaart gemaakt te worden. Op figuur \ref{fig:karnaughKaartenVoorbeeldGetekend}
\begin{figure}[hbt]
\centering
\begin{tikzpicture}
% \foreach \x in {-1,0,1} {
%   \foreach \y in {1} {
%     \draw[thick] (-1+3*\x,-1+3*\y) rectangle ++(2,2);
%     \draw[thick] (3*\x-0.5,3*\y-1) -- ++(0,2);
%     \draw[thick] (3*\x,3*\y-1) -- ++(0,2);
%     \draw[thick] (3*\x+0.5,3*\y-1) -- ++(0,2);
%     \draw[thick] (3*\x-1,3*\y-0.5) -- ++(2,0);
%     \draw[thick] (3*\x-1,3*\y) -- ++(2,0);
%     \draw[thick] (3*\x-1,3*\y+0.5) -- ++(2,0);
%   }
%   \draw[thick] (3*\x-0.5,4.125) to node[midway,above,sloped]{$y$} (3*\x+0.5,4.125);
%   \draw[thick] (3*\x,1.875) to node[midway,below,sloped]{$z$} (3*\x+1,1.875);
% }
% \draw[thick] (-4.125,3) to node[midway,below,sloped]{$w$} (-4.125,2);
% \draw[thick] (4.125,3.5) to node[midway,above,sloped]{$x$} (4.125,2.5);
% \foreach \t/\x/\y/\va/\vb/\vc/\vd/\ve/\vf/\vg/\vh/\vi/\vj/\vk/\vl/\vm/\vn/\vo/\vp in {a/-1/-1/0/0/0/0/1/1/1/1/1/1/1/1/1/1/1/1, b/0/-1/0/1/1/1/0/0/0/0/1/1/1/1/0/1/1/1, c/1/-1/1/1/1/0/0/1/1/1/-/-/-/-/1/-/-/1} {
%   \draw[thick] (3*\x-1,-3*\y+1) -- ++(-0.25,0.25) node[anchor=south east]{$\t$};
%   \draw (3*\x-0.75,-3*\y+0.75) node{\va};
%   \draw (3*\x-0.25,-3*\y+0.75) node{\vb};
%   \draw (3*\x+0.25,-3*\y+0.75) node{\vc};
%   \draw (3*\x+0.75,-3*\y+0.75) node{\vd};
%   \draw (3*\x-0.75,-3*\y+0.25) node{\ve};
%   \draw (3*\x-0.25,-3*\y+0.25) node{\vf};
%   \draw (3*\x+0.25,-3*\y+0.25) node{\vg};
%   \draw (3*\x+0.75,-3*\y+0.25) node{\vh};
%   \draw (3*\x-0.75,-3*\y-0.25) node{\vi};
%   \draw (3*\x-0.25,-3*\y-0.25) node{\vj};
%   \draw (3*\x+0.25,-3*\y-0.25) node{\vk};
%   \draw (3*\x+0.75,-3*\y-0.25) node{\vl};
%   \draw (3*\x-0.75,-3*\y-0.75) node{\vm};
%   \draw (3*\x-0.25,-3*\y-0.75) node{\vn};
%   \draw (3*\x+0.25,-3*\y-0.75) node{\vo};
%   \draw (3*\x+0.75,-3*\y-0.75) node{\vp};
% }
\kkaartd{0}{0}{$f$}{$x_1$/$x_2$/$x_3$/$x_4$}{1/1/0/1/0/1/0/1/1/0/1/1/0/0/1/1};
\kkaartd{4.5}{0}{$g$}{$x_1$/$x_2$/$x_3$/$x_4$}{1/0/0/1/1/1/1/0/0/0/0/0/0/1/1/1};
\end{tikzpicture}
\caption{Ingevulde Karnaugh-kaarten voor de uitgangen van het leidend voorbeeld.}
\label{fig:karnaughKaartenVoorbeeldGetekend}
\end{figure}
%tonen we de Karnaugh-kaarten voor de eerste 3 uitgangen van het leidend voorbeeld ($a$, $b$ en $c$). De overige uitgangen worden als oefening aan de lezer overgelaten. De oplossing is te vinden op figuur \ref{fig:apxKKaartenFill} op pagina \pageref{fig:apxKKaartenFill}.
staan de Karnaugh-kaarten voor de functies $f$ en $g$. De variabelen $x_1$, $x_2$, $x_3$ en $x_4$ zijn de binaire voorstelling van het invoergetal.
\paragraph{Stap 2: bepaal alle priemimplicant} In de volgende stap bepalen we alle priemimplicanten van de Karnaugh-kaart. Visueel is een implicant niets anders dan een rechthoek waarbij zowel de lengte en breedte een lengte hebben van machten van twee. Deze rechthoeken vallen uiteraard ook onder de modulo-regel. op een Karnaugh-kaart. Deze priemimplicanten kunnen we dan ook vinden door vanuit een cel waar de uitgangswaarde 1 is, telkens ofwel de lengte of breedte te verdubbelen, uiteraard mogen er wel geen nullen onder de rechthoek vallen. Indien er verschillende uitbreidingen mogelijk zijn, dienen al de uitbreidingen gevolgd te worden. Figuur \label{fig:karnaughKaartenVoorbeeldPriemimplicanten} toont de priemimplicanten voor de twee uitgangen van het leidend voorbeeld.
\begin{figure}[hbt]
\centering
\begin{tikzpicture}
% \foreach \x in {-1,0} {
%   \foreach \y in {1} {
%     \draw[thick] (-1+3*\x,-1+3*\y) rectangle ++(2,2);
%     \draw[thick] (3*\x-0.5,3*\y-1) -- ++(0,2);
%     \draw[thick] (3*\x,3*\y-1) -- ++(0,2);
%     \draw[thick] (3*\x+0.5,3*\y-1) -- ++(0,2);
%     \draw[thick] (3*\x-1,3*\y-0.5) -- ++(2,0);
%     \draw[thick] (3*\x-1,3*\y) -- ++(2,0);
%     \draw[thick] (3*\x-1,3*\y+0.5) -- ++(2,0);
%   }
%   \draw[thick] (3*\x-0.5,4.125) to node[midway,above,sloped]{$y$} (3*\x+0.5,4.125);
%   \draw[thick] (3*\x,1.875) to node[midway,below,sloped]{$z$} (3*\x+1,1.875);
% }
% \draw[thick] (-4.125,3) to node[midway,below,sloped]{$w$} (-4.125,2);
% \draw[thick] (1.125,3.5) to node[midway,above,sloped]{$x$} (1.125,2.5);
% \foreach \t/\x/\y/\va/\vb/\vc/\vd/\ve/\vf/\vg/\vh/\vi/\vj/\vk/\vl/\vm/\vn/\vo/\vp in {a/-1/-1/0/0/0/0/1/1/1/1/1/1/1/1/1/1/1/1, b/0/-1/0/1/1/1/0/0/0/0/1/1/1/1/0/1/1/1} {
%   \draw[thick] (3*\x-1,-3*\y+1) -- ++(-0.25,0.25) node[anchor=south east]{$\t$};
%   \draw (3*\x-0.75,-3*\y+0.75) node{\va};
%   \draw (3*\x-0.25,-3*\y+0.75) node{\vb};
%   \draw (3*\x+0.25,-3*\y+0.75) node{\vc};
%   \draw (3*\x+0.75,-3*\y+0.75) node{\vd};
%   \draw (3*\x-0.75,-3*\y+0.25) node{\ve};
%   \draw (3*\x-0.25,-3*\y+0.25) node{\vf};
%   \draw (3*\x+0.25,-3*\y+0.25) node{\vg};
%   \draw (3*\x+0.75,-3*\y+0.25) node{\vh};
%   \draw (3*\x-0.75,-3*\y-0.25) node{\vi};
%   \draw (3*\x-0.25,-3*\y-0.25) node{\vj};
%   \draw (3*\x+0.25,-3*\y-0.25) node{\vk};
%   \draw (3*\x+0.75,-3*\y-0.25) node{\vl};
%   \draw (3*\x-0.75,-3*\y-0.75) node{\vm};
%   \draw (3*\x-0.25,-3*\y-0.75) node{\vn};
%   \draw (3*\x+0.25,-3*\y-0.75) node{\vo};
%   \draw (3*\x+0.75,-3*\y-0.75) node{\vp};
% }
% \foreach \i/\x/\y/\w/\h/\r in {0/0/1/4/2/0.15,0/0/2/4/2/0.20,1/0/2/4/1/0.11,1/1/2/2/2/0.16,1/2/2/2/2/0.21} {
%   \draw[rounded corners] (-3+3*\i-0.75+0.5*\x-\r,3.75+\r-0.5*\y) rectangle ++(2*\r+0.5*\w-0.5,-2*\r-0.5*\h+0.5);
% }
% \foreach \i/\x/\w/\r in {1/1/2/0.11,1/2/2/0.14} {
%   \draw[rounded corners] (-3+3*\i-0.75+0.5*\x-\r,4) -- ++(0,-0.25-\r) -- ++(2*\r+0.5*\w-0.5,0) -- ++(0,0.25+\r);
%   \draw[rounded corners] (-3+3*\i-0.75+0.5*\x-\r,2) -- ++(0,0.25+\r) -- ++(2*\r+0.5*\w-0.5,0) -- ++(0,-0.25-\r);
%   %\draw[rounded corners] (-3+3*\i-0.75+0.5*\x-\r,3.75+\r-0.5*\y) rectangle ++(2*\r+0.5*\w-0.5,-2*\r-0.5*\h+0.5);
% }
\def\ta{0.12}
\def\tb{0.09}
\def\tc{0.06}
\kkaartdmarks{0}{0}{/0/2/3/2/\ta,/0/2/1/3/\tc,/2/1/3/2/\tc,/3/0/3/1/\ta}{/0/0/1/\tc}{/0/0/1/\ta};
\kkaartd{0}{0}{$f$}{$x_1$/$x_2$/$x_3$/$x_4$}{1/1/0/1/0/1/0/1/1/0/1/1/0/0/1/1};
\kkaartdmarks{4.5}{0}{/0/0/1/0/\ta,/0/2/0/2/\tb,/1/0/1/1/\tc,/1/1/2/1/\ta,/2/1/2/2/\tc,/2/2/2/3/\ta,/1/3/2/3/\tc}{}{/1/1/1/\tb};
\kkaartd{4.5}{0}{$g$}{$x_1$/$x_2$/$x_3$/$x_4$}{1/0/0/1/1/1/1/0/0/0/0/0/0/1/1/1};
\end{tikzpicture}
\caption{Karnaugh-kaarten met priemimplicanten van het leidend voorbeeld.}
\label{fig:karnaughKaartenVoorbeeldPriemimplicanten}
\end{figure}
\paragraph{Stap 3: Bepaal alle essenti\"ele priemimplicanten} Nadat we de priemimplicanten bepaald hebben, zullen we uit deze verzameling de essenti\"ele priemimplicanten halen. Deze stap is dan ook heel eenvoudig: als een priemimplicant \'e\'en of meer cellen omvat die geen enkele andere priemimplicant omvat is het een essenti\"ele priemimplicant. Deze implicaten zullen sowieso al tot de resulterende functie behoren. Op figuur \ref{fig:karnaughKaartenVoorbeeldEssentielePriemimplicanten} staan de essenti\"ele priemimplicanten voor $f$ en $g$. We zien duidelijk dat in beide gevallen de priemimplicanten onvoldoende zijn om de volledige functie te beschrijven daar er nog enen niet niet gedekt worden.
\begin{figure}[hbt]
\centering
\begin{tikzpicture}
\def\ta{0.12}
\def\tb{0.09}
\def\tc{0.06}
\kkaartdmarks{0}{0}{/0/2/1/3/\tc,/2/1/3/2/\tc}{}{};
\kkaartd{0}{0}{$f$}{$x_1$/$x_2$/$x_3$/$x_4$}{1/1/0/1/0/1/0/1/1/0/1/1/0/0/1/1};
\kkaartdmarks{4.5}{0}{/0/0/1/0/\ta,/0/2/0/2/\tb}{}{};
\kkaartd{4.5}{0}{$g$}{$x_1$/$x_2$/$x_3$/$x_4$}{1/0/0/1/1/1/1/0/0/0/0/0/0/1/1/1};
\end{tikzpicture}
\caption{Karnaugh-kaarten met essenti\"ele priemimplicanten van het leidend voorbeeld.}
\label{fig:karnaughKaartenVoorbeeldEssentielePriemimplicanten}
\end{figure}
\paragraph{Stap 4: Zoek de minimale dekking}
De essenti\"ele priemimplicanten zijn de goedkoopste manier om de cellen die ze dekken te implementeren, we zien echter dat dit in de meeste gevallen onvoldoende is om de volledige functie te beschrijven. We dienen nog extra priemimplicanten toe te voegen om tot volledige dekking te komen. In het ideale geval doen we dit door alle mogelijke toevoegingen van priemimplicanten na te gaan. Dit is echter een erg arbeidsintensief proces. Men lost dit probleem dan ook meestal op met een ``\termen{gulzige strategie}'' ofwel ``\termen{greedy algorithm}''. Hierbij beschouwen we initieel de set van essenti\"ele priemgetallen, per iteratie voegen we de priemimplicant toe die het meeste aantal cellen dekt die tot dan toe ongedekt bleven. We stoppen op het moment dat de set van implicanten de volledige functie dekt. We illustreren dit proces op figuur \ref{fig:karnaughKaartenVoorbeeldGreedySearch} waarbij we elke iteratiestap tonen.
\begin{figure}[hbt]
\centering
\begin{tikzpicture}
\def\ta{0.12}
\def\tb{0.09}
\def\tc{0.06}
\kkaartdmarks{0}{0}{/0/2/1/3/\tc,/2/1/3/2/\tc}{}{};
\kkaartd{0}{0}{$f$}{$x_1$/$x_2$/$x_3$/$x_4$}{1/1/0/1/0/1/0/1/1/0/1/1/0/0/1/1};
\draw (3,1) node{$\Rightarrow$};
\kkaartdmarks{4}{0}{/0/2/1/3/\tc,/2/1/3/2/\tc}{/0/0/1/\tc}{};
\kkaartd{4}{0}{$f$}{$x_1$/$x_2$/$x_3$/$x_4$}{1/1/0/1/0/1/0/1/1/0/1/1/0/0/1/1};

\kkaartdmarks{0}{-3}{/0/0/1/0/\ta,/0/2/0/2/\tb}{}{};
\kkaartd{0}{-3}{$g$}{$x_1$/$x_2$/$x_3$/$x_4$}{1/0/0/1/1/1/1/0/0/0/0/0/0/1/1/1};
\draw (3,-2) node{$\Rightarrow$};
\kkaartdmarks{4}{-3}{/0/0/1/0/\ta,/0/2/0/2/\tb,/1/1/2/1/\ta}{}{};
\kkaartd{4}{-3}{$g$}{$x_1$/$x_2$/$x_3$/$x_4$}{1/0/0/1/1/1/1/0/0/0/0/0/0/1/1/1};
\draw (7,-2) node{$\Rightarrow$};
\kkaartdmarks{8}{-3}{/0/0/1/0/\ta,/0/2/0/2/\tb,/1/1/2/1/\ta,/1/3/2/3/\ta}{}{};
\kkaartd{8}{-3}{$g$}{$x_1$/$x_2$/$x_3$/$x_4$}{1/0/0/1/1/1/1/0/0/0/0/0/0/1/1/1};
\draw (11,-2) node{$\Rightarrow$};
\kkaartdmarks{12}{-3}{/0/0/1/0/\ta,/0/2/0/2/\tb,/1/1/2/1/\ta,/1/3/2/3/\ta,/2/1/2/2/\tc}{}{};
\kkaartd{12}{-3}{$g$}{$x_1$/$x_2$/$x_3$/$x_4$}{1/0/0/1/1/1/1/0/0/0/0/0/0/1/1/1};
\end{tikzpicture}
\caption{Werking van het greedy algoritme bij het leidend voorbeeld.}
\label{fig:karnaughKaartenVoorbeeldGreedySearch}
\end{figure}
\paragraph{Synthese}
Elk van de priemimplicanten die we geselecteerd hebben stelt het product voor van enkele variabelen. We implementeren de functie door de som te nemen van deze priemimplicanten. Voor het voorbeeld wordt dit dus:
\begin{equation}
\begin{array}{ll}
\left\{
\begin{array}{l}
f=x_2'x_3'x_4'+x_1'x_4+x_1x_3\\
g=x_1'x_3'x_4'+x_1'x_2'x_3x_4+x_2x_3x_4'+x_2x_3'x_4+x_1x_2x_3
\end{array}\right.&\mbox{(Leidend voorbeeld)}
\end{array}
\end{equation}
\paragraph{Uitbreiding: Dambordpatroon}
Een patroon die men vaak tegenkomt in Karnaugh-kaarten is het \termen{dambordpatroon}. Dit dambordpatroon hoeft niet noodzakelijk uit vierkanten te bestaan, rechthoek zijn ook mogelijk. Indien we dit patroon met de klassieke Karnaugh-methode implementeren bekomen we een groot aantal priemimplicanten wat leidt tot kostelijke implementaties, we kunnen in dat geval gebruik maken van XOR-operaties die we achter elkaar schakelen. Een aaneenschakeling van XOR-operaties heeft een grotere vertraging maar heeft een grote invloed op de kostprijs. Figuur \ref{fig:dambordpatronen} toont enkele dambordpatronen en hun implementatie met XOR-logica.
\begin{figure}[hbt]
\centering
\begin{tikzpicture}
\kkaartd{0}{0}{$f_1$}{$x_1$/$x_2$/$x_3$/$x_4$}{0/1/1/0/1/0/0/1/1/0/0/1/0/1/1/0}
\draw (1,-0.5) node[anchor=north]{\small{$f_1=\left(x_1 \oplus x_2\right) \oplus \left(x_3 \oplus x_4\right)$}};
\kkaartd{5}{0}{$f_2$}{$x_1$/$x_2$/$x_3$/$x_4$}{0/1/1/0/0/1/1/0/1/0/0/1/1/0/0/1}
\draw (6,-0.5) node[anchor=north]{\small{$f_2=\left(x_3 \oplus x_4\right) \oplus x_1$}};
\kkaartd{10}{0}{$f_3$}{$x_1$/$x_2$/$x_3$/$x_4$}{0/1/0/1/0/1/0/1/1/0/1/0/1/0/1/0}
\draw (11,-0.5) node[anchor=north]{\small{$f_3=x_1 \oplus x_4$}};
\end{tikzpicture}
\caption{Voorbeelden van dambordpatronen in Karnaugh-kaarten.}
\label{fig:dambordpatronen}
\end{figure}
\paragraph{Uitbreiding: Don't cares}
In sommige gevallen dienen we slechts een beperkte set van invoer-configuraties te beschouwen. Stel bijvoorbeeld dat we een digitale display implementeren die getallen van 0 tot en met 9 voorstelt. In dat geval hebben we 4 ingangen nodig. Maar we zullen bijvoorbeeld nooit de ingang $1011_2=11_{10}$ tegenkomen. Er is echter wel een plaats gereserveerd op de Karnaugh-kaart voor deze configuratie. In dat geval maken we gebruik van de zogenaamde \termen{don't care}. Dit wordt genoteerd met een horizontale streep\footnote{Engels: dash.}, een ``X'' of ``d''. Een don't care is geen speciale vorm van uitvoer. We kunnen enkel nullen of enen teruggeven. Een don't care wordt enkel gebruikt om aan te geven dat we vrij mogen kiezen of de uitvoer een 0 of 1 wordt. Uiteraard proberen we een keuze te maken die de implementatie goedkoper maakt. We kunnen tot betere implementaties komen door een don't care als een 1 te zien indien dit de priemimplicanten kan vergroten. Op die manier bereiken deze priemgetallen immers een groter gebied waardoor ze minder variabelen bevatten. Op figuur \ref{fig:sevenDigitDisplay} geven we de Karnaugh-kaart van led $A$ en $B$ bij een \termen{seven-segment display}, samen met de priemimplicanten die we bekomen na het toewijzen van de don't cares. In appendix ?? staan de Karnaugh-kaarten van de andere leds. Deze kaarten zijn een goede oefening om het volledige proces te leren.
\begin{figure}[hbt]
\centering
\begin{tikzpicture}
\def\ta{0.12}
\def\tb{0.09}
\def\tc{0.06}
\kkaartdmarks{0}{0}{/2/0/3/3/\tc,/1/2/2/3/\tb,/0/1/3/2/\ta}{/0/1/1/\tb}{};
\kkaartd{0}{0}{$A$}{$x_1$/$x_2$/$x_3$/$x_4$}{1/0/1/1/0/1/1/1/1/1/-/-/-/-/-/-}
\kkaartdmarks{4}{0}{/2/0/3/3/\tc,/0/0/3/0/\ta,/0/2/3/2/\ta}{/0/3/1/\tb}{};
\kkaartd{4}{0}{$B$}{$x_1$/$x_2$/$x_3$/$x_4$}{1/1/1/1/1/0/0/1/1/1/-/-/-/-/-/-}
\end{tikzpicture}
\caption{Karnaugh-kaart met don't cares van led $A$ en $B$ van een seven-segment display.}
\label{fig:sevenDigitDisplay}
\end{figure}
\paragraph{Uitbreiding: Meerdere uitgangen}
Tot dusver hebben we steeds aangenomen dat we de functie voor \'e\'en signaaluitgang optimaliseren, uit de voorbeelden die we beschouwd hebben blijkt echter dat een component verschillende uitgangen moet uitrekenen (de seven-segment display). Daar we de functies implementeren met AND-OR logica is het mogelijk dat de AND-poorten van \'e\'en uitgang ook nuttig kunnen zijn voor de uitgang van een andere uitgang. Dit leidt misschien niet tot de goedkoopste schakeling per uitgang maar globaal kunnen we eventueel kosten besparen. We zullen hieronder een procedure bespreken die gebruik maakt van de priemimplicanten, er is echter geen garantie dat deze de globale goedkoopste schakeling realiseert. Soms is het zelfs goedkoper om met niet-priemimplicanten te werken. Met trail-and-error kunnen we dus soms tot nog goedkopere implementaties komen. Volgende procedure bekomt echter meestal een goed resultaat:
\begin{enumerate}
 \item We realiseren eerst bij elke uitgang de essenti\"ele priemimplicanten.
 \item Selecteer vervolgens priemimplicanten die essenti\"ele priemimplicanten zijn bij een andere uitgang. Bij deze keuze is het ook belangrijk om de priemimplicant te selecteren die in het kleinste aantal functies voorkomt, dit doen we om de fan-out laag te houden waardoor we minder vertraging induceren. Merk op dat deze operatie ons niets kost: we hebben immers deze implicanten al gerealiseerd.
 \item De overige priemimplicaten realiseren we per uitvoer zoals op de klassieke manier. Indien een priemimplicant in verschillende uitgangen voorkomt, kunnen we deze eventueel bevoordelen.
\end{enumerate}
\paragraph{Duale vorm}
Tot dusver hebben we telkens met behulp van Karnaugh-kaarten een AND-OR implementatie gerealiseerd. Zoals we al vaak zijn tegengekomen hebben quasi alle logische methodes een duale vorm. Ook de Karnaugh-kaarten kunnen we gebruiken om een minimale OR-AND implementatie te realiseren. In tegenstelling tot de AND-OR vorm draait alles hier rond nullen en niet rond enen. De priemimplicaten zijn hierbij gerelateerd aan 0-maxtermen: functies die overal 1 teruggeven behalve op een bepaald patroon. Verder werkt deze methode volledig analoog: we bepalen eerst de essenti\"ele priemimplicanten en voegen vervolgens andere priemimplicanten toe. We synthetiseren vervolgens de schakeling door een AND tussen alle gekozen priemimplicanten te plaatsen. Op figuur \ref{fig:karnaughKaartenVoorbeeldDualeVorm} voeren we deze methode uit op de $f$-functie van het leidend voorbeeld.
\begin{figure}[hbt]
\centering
\begin{tikzpicture}
\def\ta{0.09}
\def\tb{0.06}
\draw (1,-0.5) node[anchor=north]{\small{Ingevulde Karnaugh-kaart}};
\draw (3,1) node[anchor=north]{$\Rightarrow$};
\kkaartd{0}{0}{$f$}{$x_1$/$x_2$/$x_3$/$x_4$}{1/1/0/1/0/1/0/1/1/0/1/1/0/0/1/1};
\draw (5,-0.5) node[anchor=north]{\small{Priemimplicanten}};
\draw (7,1) node[anchor=north]{$\Rightarrow$};
\kkaartdmarks{4}{0}{/0/1/1/1/\ta,/1/0/1/1/\tb,/1/0/2/0/\ta,/2/3/3/3/\ta}{}{/2/2/1/\tb};
\kkaartd{4}{0}{$f$}{$x_1$/$x_2$/$x_3$/$x_4$}{1/1/0/1/0/1/0/1/1/0/1/1/0/0/1/1};
\draw (9,-0.5) node[anchor=north]{\small{Essenti\"ele Priemimplicanten}};
\draw (11,1) node[anchor=north]{$\Rightarrow$};
\kkaartdmarks{8}{0}{/0/1/1/1/\ta,/2/3/3/3/\ta}{}{};
\kkaartd{8}{0}{$f$}{$x_1$/$x_2$/$x_3$/$x_4$}{1/1/0/1/0/1/0/1/1/0/1/1/0/0/1/1};
\kkaartdmarks{12}{0}{/0/1/1/1/\ta,/2/3/3/3/\ta,/1/0/2/0/\ta}{}{};
\draw (13,-0.5) node[anchor=north]{\small{Resultaat}};
\draw (7,-1) node[anchor=north]{\small{$f=\left(x_1+x_3'+x_4\right)\left(x_2'+x_3+x_4\right)\left(x_1'+x_3+x_4'\right)$}};
\kkaartd{12}{0}{$f$}{$x_1$/$x_2$/$x_3$/$x_4$}{1/1/0/1/0/1/0/1/1/0/1/1/0/0/1/1};
\end{tikzpicture}
\caption{Duale methode met Karnaugh-kaarten.}
\label{fig:karnaughKaartenVoorbeeldDualeVorm}
\end{figure}
% %Hoe gaat optimaliseren met Karnaugh-kaarten nu concreet in zijn werk? Om een bepaalde functie te optimaliseren genereren we een Karnaugh-kaart met als dimensie het aantal variabelen in de functie. Vervolgens vullen we de kaart in met de resultaten van de functie. We vullen dus 1 in in de cellen waarbij de functie voor die toestand een 1 genereert, dit doen we ook met 0 en don't cares. Vervolgens gaan we op zoek naar aaneengesloten\footnote{Aaneengesloten betekent niet dat ze op de Karnaugh-kaarten naast elkaar liggen. De modulo positionering telt ook mee.} gebieden met eenzelfde functiewaarde. De grootte van deze gebieden moet $2^i\times 2^j$ zijn. We selecteren dus rechthoeken waarbij zowel de lengte als breedte een macht van twee is. Deze rechthoeken vertegenwoordigen optimalisaties van bestaande logica. We proberen dus eerst grote rechthoeken te alloceren. Het doel is, elk gebied met hierop een 1 of eventueel een don't care, te selecteren. Hierbij mogen geen cellen met 0 geselecteerd worden. Bovendien moge cellen meerdere keren geselecteerd worden. Figuur \ref{fig:karnaughKaartExample} toont de Karnaugh-kaart van $e$ van de 7 segment display. Hierbij worden vervolgens de rechthoeken geselecteerd.
% \begin{figure}[hbt]
% \centering
% \begin{tikzpicture}
% \draw[thick] (0,0) -- ++(2,0) -- ++(0,2) -- ++(-2,0) -- cycle;
% \draw[thick] (0,0.5) -- ++(2,0);
% \draw[thick] (0,1) -- ++(2,0);
% \draw[thick] (0,1.5) -- ++(2,0);
% \draw[thick] (0.5,0) -- ++(0,2);
% \draw[thick] (1,0) -- ++(0,2);
% \draw[thick] (1.5,0) -- ++(0,2);
% \draw[thick] (-0.125,0) to node[sloped,below,rotate=180]{$x$} (-0.125,1);
% \draw[thick] (2.125,0.5) to node[sloped,above,rotate=180]{$y$} (2.125,1.5);
% \draw[thick] (0.5,2.125) to node[sloped,above]{$z$} (1.5,2.125);
% \draw[thick] (1,-0.125) to node[sloped,below]{$w$} (2,-0.125);
% \draw (0.25,1.75) node{1};
% \draw (1.75,1.75) node{0};
% \draw (0.75,1.75) node{1};
% \draw (1.25,1.75) node{0};
% \draw (0.25,1.25) node{0};
% \draw (1.75,1.25) node{0};
% \draw (0.75,1.25) node{1};
% \draw (1.25,1.25) node{0};
% \draw (0.25,0.25) node{1};
% \draw (1.75,0.25) node{0};
% \draw (0.75,0.25) node{-};
% \draw (1.25,0.25) node{-};
% \draw (0.25,0.75) node{-};
% \draw (1.75,0.75) node{-};
% \draw (0.75,0.75) node{-};
% \draw (1.25,0.75) node{-};
% \draw (2.875,1) node{\Large $\Rightarrow$};
% \begin{scope}[xshift=3.5 cm]
% \draw[gray,thick] (0,0) -- ++(2,0) -- ++(0,2) -- ++(-2,0) -- cycle;
% \draw[gray,thick] (0,0.5) -- ++(2,0);
% \draw[gray,thick] (0,1) -- ++(2,0);
% \draw[gray,thick] (0,1.5) -- ++(2,0);
% \draw[gray,thick] (0.5,0) -- ++(0,2);
% \draw[gray,thick] (1,0) -- ++(0,2);
% \draw[gray,thick] (1.5,0) -- ++(0,2);
% \draw[gray,thick] (-0.125,0) to node[sloped,below,rotate=180]{$x$} (-0.125,1);
% \draw[gray,thick] (2.125,0.5) to node[sloped,above,rotate=180]{$y$} (2.125,1.5);
% \draw[gray,thick] (0.5,2.125) to node[sloped,above]{$z$} (1.5,2.125);
% \draw[gray,thick] (1,-0.125) to node[sloped,below]{$w$} (2,-0.125);
% \draw (0.25,1.75) node{1};
% \draw (1.75,1.75) node{0};
% \draw (0.75,1.75) node{1};
% \draw (1.25,1.75) node{0};
% \draw (0.25,1.25) node{0};
% \draw (1.75,1.25) node{0};
% \draw (0.75,1.25) node{1};
% \draw (1.25,1.25) node{0};
% \draw (0.25,0.25) node{1};
% \draw (1.75,0.25) node{0};
% \draw (0.75,0.25) node{-};
% \draw (1.25,0.25) node{-};
% \draw (0.25,0.75) node{-};
% \draw (1.75,0.75) node{-};
% \draw (0.75,0.75) node{-};
% \draw (1.25,0.75) node{-};
% \draw[dashed,thick,rounded corners] (0.55,0.05) rectangle ++(0.4,1.9);
% \draw (1,-0.75) node{$z\cdot w'$};
% \draw (2.875,1) node{\Large $+$};
% \end{scope}
% \begin{scope}[xshift=7 cm]
% \draw[gray,thick] (0,0) -- ++(2,0) -- ++(0,2) -- ++(-2,0) -- cycle;
% \draw[gray,thick] (0,0.5) -- ++(2,0);
% \draw[gray,thick] (0,1) -- ++(2,0);
% \draw[gray,thick] (0,1.5) -- ++(2,0);
% \draw[gray,thick] (0.5,0) -- ++(0,2);
% \draw[gray,thick] (1,0) -- ++(0,2);
% \draw[gray,thick] (1.5,0) -- ++(0,2);
% \draw[gray,thick] (-0.125,0) to node[sloped,below,rotate=180]{$x$} (-0.125,1);
% \draw[gray,thick] (2.125,0.5) to node[sloped,above,rotate=180]{$y$} (2.125,1.5);
% \draw[gray,thick] (0.5,2.125) to node[sloped,above]{$z$} (1.5,2.125);
% \draw[gray,thick] (1,-0.125) to node[sloped,below]{$w$} (2,-0.125);
% \draw (0.25,1.75) node{1};
% \draw (1.75,1.75) node{0};
% \draw (0.75,1.75) node{1};
% \draw (1.25,1.75) node{0};
% \draw (0.25,1.25) node{0};
% \draw (1.75,1.25) node{0};
% \draw (0.75,1.25) node{1};
% \draw (1.25,1.25) node{0};
% \draw (0.25,0.25) node{1};
% \draw (1.75,0.25) node{0};
% \draw (0.75,0.25) node{-};
% \draw (1.25,0.25) node{-};
% \draw (0.25,0.75) node{-};
% \draw (1.75,0.75) node{-};
% \draw (0.75,0.75) node{-};
% \draw (1.25,0.75) node{-};
% \draw[dashed,thick,rounded corners] (0.05,-0.05) -- ++(0,0.5) -- ++(0.9,0) -- ++(0,-0.5);
% \draw[dashed,thick,rounded corners] (0.05,2.05) -- ++(0,-0.5) -- ++(0.9,0) -- ++(0,0.5);
% \draw (1,-0.75) node{$y'\cdot w'$};
% \draw (2.875,1) node{\Large $\Rightarrow$};
% \end{scope}
% \begin{scope}[xshift=10.5 cm]
% \draw[gray,thick] (0,0) -- ++(2,0) -- ++(0,2) -- ++(-2,0) -- cycle;
% \draw[gray,thick] (0,0.5) -- ++(2,0);
% \draw[gray,thick] (0,1) -- ++(2,0);
% \draw[gray,thick] (0,1.5) -- ++(2,0);
% \draw[gray,thick] (0.5,0) -- ++(0,2);
% \draw[gray,thick] (1,0) -- ++(0,2);
% \draw[gray,thick] (1.5,0) -- ++(0,2);
% \draw[gray,thick] (-0.125,0) to node[sloped,below,rotate=180]{$x$} (-0.125,1);
% \draw[gray,thick] (2.125,0.5) to node[sloped,above,rotate=180]{$y$} (2.125,1.5);
% \draw[gray,thick] (0.5,2.125) to node[sloped,above]{$z$} (1.5,2.125);
% \draw[gray,thick] (1,-0.125) to node[sloped,below]{$w$} (2,-0.125);
% \draw (0.25,1.75) node{1};
% \draw (1.75,1.75) node{0};
% \draw (0.75,1.75) node{1};
% \draw (1.25,1.75) node{0};
% \draw (0.25,1.25) node{0};
% \draw (1.75,1.25) node{0};
% \draw (0.75,1.25) node{1};
% \draw (1.25,1.25) node{0};
% \draw (0.25,0.25) node{1};
% \draw (1.75,0.25) node{0};
% \draw (0.75,0.25) node{-};
% \draw (1.25,0.25) node{-};
% \draw (0.25,0.75) node{-};
% \draw (1.75,0.75) node{-};
% \draw (0.75,0.75) node{-};
% \draw (1.25,0.75) node{-};
% \draw[dashed,thick,rounded corners] (0.55,0.05) rectangle ++(0.4,1.9);
% \draw[dashed,thick,rounded corners] (0.05,-0.05) -- ++(0,0.5) -- ++(0.9,0) -- ++(0,-0.5);
% \draw[dashed,thick,rounded corners] (0.05,2.05) -- ++(0,-0.5) -- ++(0.9,0) -- ++(0,0.5);
% \draw (1,-0.75) node{$e=z\cdot w'+y'\cdot w'$};
% \end{scope}
% \end{tikzpicture}
% \caption{Karnaugh-kaart voor $e$ van de 7 segment display.}
% \label{fig:karnaughKaartExample}
% \end{figure}
%Elke rechthoek vertegenwoordigt een geoptimaliseerde minterm. Hierbij worden factoren genegeerd waarbij zowel de 0 en 1 waarde tot de rechthoek behoren. Dit betekent bijgevolg dat hun waarde niet relevant is. Zo bevat de eerste rechthoek op figuur \ref{fig:karnaughKaarExample}.
\subsection{Quine-McCluskey}
Een alternatieve methode voor Karnaugh-kaarten is het \termen{Quine-McCluskey algoritme}. Dit algoritme wordt gebruikt in CAD-pakketten voor booleaanse optimalisatie en werkt op basis van tabellen. Het algoritme zoekt ook naar priemimplicanten en essenti\"ele priemimplicanten om een functie te optimaliseren en is dus het tabel-equivalent van de methode met de Karnaugh-kaarten. Het algoritme werkt in exponenti\"ele tijd, namelijk \bigoh{3^n} met $n$ het aantal variabelen. In de meeste gevallen is het aantal variabelen te groot om deze functie te optimaliseren, in dat geval wordt er gewerkt met de Espresso heuristic logic minimizer.
\subsection{Realisatie in meer dan 2 lagen}
Een Karnaugh-kaart laat toe tot een sterke implementatie te komen met twee lagen (een AND- en OR-laag). Zoals we echter in de volgende secties zullen zien, zullen complexe schakelingen bij twee lagen toch een hoge kost met zich meebrengen. Daarom is het soms aangewezen om de logica in meer lagen te implementeren. Dit veroorzaakt tragere schakelingen maar aan een goedkopere kostprijs. Hieronder geven we enkele technieken:
\begin{itemize}
 \item Specificatie: in heel wat gevallen gaat de specificatie van het component reeds gepaard met een expliciete implementatie. Bijvoorbeeld ``1 indien $x$ en ofwel $y$ ofwel $z$ en $t$'' kunnen we dan rechtstreeks implementeren als: $f=x\wedge\left(y\oplus\left(z\wedge t\right)\right)$.
 \item \termen{Factoranalyse}: We kunnen een expressie ook \termen{algebra\"isch manipuleren} met de wetten uit sectie \ref{s:booleaanseAlgebra}. Factoranalyse wordt ook gebruikt wanneer we een schakeling dienen te implementeren met beperkte fan-in: stel dat we enkel NAND-poorten met 2 ingangen ter beschikking hebben. Bij realisaties met beperkte fan-in moet men altijd proberen deze te implementeren in een boomstructuur. Indien we dus $f=x+y+z+t$ moeten implementeren converteren we dit naar $f=\left(x+y\right)+\left(z+t\right)$ en niet naar $f=x+\left(y+\left(z+t\right)\right)$. De boomstructuur laat toe schakelingen te realiseren die een vertraging van \bigoh{\log n} hebben tegenover de lineare implementatie met een vertraging van \bigoh{n}.
 \item \termen{Functionele ontbinding}: Soms zijn we ook in staat om een functie op te delen in verschillende deelfuncties. In subsectie \ref{sss:fulladder} zullen we bijvoorbeeld een volledige opteller beschouwen. In plaats van een optelling van drie bits rechtstreeks te implementeren kunnen we twee optellingen van twee bits realiseren. In het algemeen betekent dit dat we de functie $\vec{f}\left(\vec{x}\right)$ soms kunnen herschrijven als $\vec{h}\left(\vec{g}\left(\vec{x}\right),\vec{x}\right)$ waarbij $\vec{g}$ en $\vec{h}$ meestal eenvoudiger en goedkoper zijn.
\end{itemize}
Geen enkele van deze methodes levert altijd een betere resultaat het probleem moet dan ook opgelost worden in ``\termen{trail-and-error}'' stijl.
\subsection{Welke methode kiezen?}
Samen met de methodes uit subsectie \ref{ss:canoniekestandaardrealisatie} hebben we nu volgende methodes om een schakeling te synthetiseren:
\begin{itemize}
 \item Canonieke Sum-of-Products
 \item Canonieke Product-of-Sums
 \item Minimale Sum-of-Products (Karnaugh-kaarten)
 \item Minimale Product-of-Sums (Karnaugh-kaarten)
 \item Meerlagenlogica
\end{itemize}
\paragraph{}Verder kunnen we ook vrij kiezen tussen AND-OR en NAND-NAND in het geval van Sum-of-Products, en voor OR-AND en NOR-NOR bij Product-of-Sums. Het is altijd voordeliger om voor NAND-NAND en NOR-NOR te kiezen. Deze schakelingen zijn altijd goedkoper en sneller. Het aantal poorten en de structuur blijft immers gelijk en uit vergelijkingen (\ref{eqn:kostenCmosPoort}) en (\ref{eqn:speedPoort}) blijkt duidelijk dat dit een betere keuze is. Een andere mogelijkheid is om de Minimale AND-OR implementatie om te zetten naar een AND-OR-Invert en een OR-AND implementatie naar zijn OR-AND-Invert equivalent.
\paragraph{}Ook kunnen we in het algemeen bewijzen dat de implementatie met Karnaugh-kaarten altijd goedkoper en sneller is. Immers in het slechtste geval komt dit neer op dezelfde implementatie als de canonieke sum-of-products. In de meeste gevallen zal meerlagenlogica verder een grotere vertraging induceren dan de implementatie met Karnaugh-kaarten, dit is echter niet algemeen en bovendien kan meerlagenlogica gepaard gaan met grote kostenbesparingen. Veel rekenkundige schakelingen die we in de volgende sectie zullen tegenkomen maken dan ook gebruik van meerdere functionele lagen.
\paragraph{}Tot slot is het niet altijd belangrijk om tot de meest optimale implementatie te komen. Indien we bijvoorbeeld de logica op een FPGA programmeren hebben we per functie een logic block ter beschikking. Het aantal poorten in dit blok staat al vast. Indien we dus onder dat aantal blijven levert het ons niks op om de functie verder te minimaliseren. We hebben immers toch reeds voor deze poorten betaalt. Deze realisatie van de schakeling naar de beschikbare elementen (poorten, logic blocks,...) wordt dan ook de ``\termen{technology mapping}'' genoemd.
\paragraph{}Om de verschillende implementaties te illustreren zullen we tot slot een schakeling implementeren in de verschillende vormen van logica. De Karnaugh-kaart en de implementaties staan op figuur \ref{fig:differentImplementationsSSD}. Een samenvatting van deze implementaties in termen van kosten en vertraging staan in tabel \ref{tbl:differentImplementationsSSD}.
\begin{figure}[hbt]
\centering
\subfigure[Karnaugh-kaart]{
\begin{tikzpicture}
\kkaartd{0}{0}{$A$}{$x_1$/$x_2$/$x_3$/$x_4$}{0/0/0/0/1/1/1/1/0/0/1/0/0/0/1/1};
\end{tikzpicture}}
\subfigure[Canonieke SOP (AND-OR)]{
\begin{tikzpicture}[circuit logic US]
\def\dy{0.5};
\def\dx{0.125};
\def\dxp{0.75};
\def\xpo{0.25};
\def\wi{\xpo+\dxp*7.5};
\foreach \i/\ii/\j in {a/4/1,b/3/2,c/2/3,d/1/4} {
  \coordinate (x\i) at (0,\dy*\ii);
  \coordinate (xn\i) at (0,\dy*\ii-0.5*\dy);
  \node[anchor=east] (xt\i) at (x\i) {$x_\j$};
  \pdot{x\i};
  \draw (x\i) -- ++(\wi,0);
  \node[not gate,anchor=input,scale=\dy] (Nx\i) at (xn\i -| \dx,0) {};
  \draw (x\i) |- (Nx\i.input);
  \draw (Nx\i.output) -- (xn\i -| \wi,0);
}
\foreach \x/\ya/\yb/\yc/\yd in {1/1.5/2/3.5/4.5,2/1.5/2/3.5/4,3/1.5/2/3/4.5,4/1.5/2/3/4,5/1/2/3/4.5,6/1/2/3/4,7/1/2.5/3/4.5} {
  \node[and gate,scale=\dy,inputs={normal,normal,normal,normal},rotate=-90] (A\x) at (\xpo+\dxp*\x,-0.25) {};
  \draw (A\x.input 4) -- (A\x.input 4 |- 0,5*\dy-\dy*\ya);
  \pdot{A\x.input 4 |- 0,5*\dy-\dy*\ya};
  \draw (A\x.input 3) -- (A\x.input 3 |- 0,5*\dy-\dy*\yb);
  \pdot{A\x.input 3 |- 0,5*\dy-\dy*\yb};
  \draw (A\x.input 2) -- (A\x.input 2 |- 0,5*\dy-\dy*\yc);
  \pdot{A\x.input 2 |- 0,5*\dy-\dy*\yc};
  \draw (A\x.input 1) -- (A\x.input 1 |- 0,5*\dy-\dy*\yd);
  \pdot{A\x.input 1 |- 0,5*\dy-\dy*\yd};
}
\node[or gate,scale=\dy,inputs={normal,normal,normal,normal,normal,normal,normal},rotate=-90] (O) at (\xpo+\dxp*4,-1.25) {};
\foreach \x/\y/\z in {1/1/7,2/2/6,3/3/5,4/4/4,5/3/3,6/2/2,7/1/1} {
  \draw (O.input \x) -- (O.input \x |- 0,-0.85+0.065*\y) -| (A\z.output);
}
\draw (O.output) |- ++(\dy,-0.5*\dy) node[anchor=west]{$A$};
\end{tikzpicture}}
\subfigure[Minimale SOP]{
\begin{tikzpicture}[circuit logic US]
\def\dy{0.5};
\def\dx{0.125};
\def\dxp{0.75};
\def\xpo{0.25};
\def\wi{\xpo+\dxp*3.5};
\foreach \i/\ii/\j in {a/4/1,b/3/2,c/2/3,d/1/4} {
  \coordinate (x\i) at (0,\dy*\ii);
  \coordinate (xn\i) at (0,\dy*\ii-0.5*\dy);
  \node[anchor=east] (xt\i) at (x\i) {$x_\j$};
  \pdot{x\i};
  \draw (x\i) -- ++(\wi,0);
}
\foreach \i in {a,d} {
  \node[not gate,anchor=input,scale=\dy] (Nx\i) at (xn\i -| \dx,0) {};
  \draw (x\i) |- (Nx\i.input);
  \draw (Nx\i.output) -- (xn\i -| \wi,0);
}
\foreach \x/\ya/\yb in {1/1.5/2,2/2/3} {
  \node[and gate,scale=\dy,inputs={normal,normal},rotate=-90] (A\x) at (\xpo+\dxp*\x,-0.25) {};
  \draw (A\x.input 2) -- (A\x.input 2 |- 0,5*\dy-\dy*\ya);
  \pdot{A\x.input 2 |- 0,5*\dy-\dy*\ya};
  \draw (A\x.input 1) -- (A\x.input 1 |- 0,5*\dy-\dy*\yb);
  \pdot{A\x.input 1 |- 0,5*\dy-\dy*\yb};
}
\foreach \x/\ya/\yb/\yc/\yd in {3/1/3/4.5} {
  \node[and gate,scale=\dy,inputs={normal,normal,normal},rotate=-90] (A\x) at (\xpo+\dxp*\x,-0.25) {};
  \draw (A\x.input 3) -- (A\x.input 3 |- 0,5*\dy-\dy*\ya);
  \pdot{A\x.input 3 |- 0,5*\dy-\dy*\ya};
  \draw (A\x.input 2) -- (A\x.input 2 |- 0,5*\dy-\dy*\yb);
  \pdot{A\x.input 2 |- 0,5*\dy-\dy*\yb};
  \draw (A\x.input 1) -- (A\x.input 1 |- 0,5*\dy-\dy*\yc);
  \pdot{A\x.input 1 |- 0,5*\dy-\dy*\yc};
}
\node[or gate,scale=\dy,inputs={normal,normal,normal},rotate=-90] (O) at (\xpo+\dxp*2,-1.25) {};
\foreach \x/\y/\z in {1/1/3,2/2/2,3/1/1} {
  \draw (O.input \x) -- (O.input \x |- 0,-0.85+0.065*\y) -| (A\z.output);
}
\draw (O.output) |- ++(\dy,-0.5*\dy) node[anchor=west]{$A$};
\end{tikzpicture}}
\subfigure[Canonieke POS (OR-AND)]{
\begin{tikzpicture}[circuit logic US]
\def\dy{0.5};
\def\dx{0.125};
\def\dxp{0.75};
\def\xpo{0.25};
\def\wi{\xpo+\dxp*9.5};
\foreach \i/\ii/\j in {a/4/1,b/3/2,c/2/3,d/1/4} {
  \coordinate (x\i) at (0,\dy*\ii);
  \coordinate (xn\i) at (0,\dy*\ii-0.5*\dy);
  \node[anchor=east] (xt\i) at (x\i) {$x_\j$};
  \pdot{x\i};
  \draw (x\i) -- ++(\wi,0);
  \node[not gate,anchor=input,scale=\dy] (Nx\i) at (xn\i -| \dx,0) {};
  \draw (x\i) |- (Nx\i.input);
  \draw (Nx\i.output) -- (xn\i -| \wi,0);
}
\foreach \x/\ya/\yb/\yc/\yd in {1/1/2/3/4,2/1/2/3/4.5,3/1/2/3.5/4,4/1/2/3.5/4.5,5/1.5/2.5/3/4,6/1.5/2.5/3/4.5,7/1.5/2/3/4,8/1.5/2/3.5/4.5,9/1.5/2/3/4.5} {
  \node[or gate,scale=\dy,inputs={normal,normal,normal,normal},rotate=-90] (A\x) at (\xpo+\dxp*\x,-0.25) {};
  \draw (A\x.input 4) -- (A\x.input 4 |- 0,5*\dy-\dy*\ya);
  \pdot{A\x.input 4 |- 0,5*\dy-\dy*\ya};
  \draw (A\x.input 3) -- (A\x.input 3 |- 0,5*\dy-\dy*\yb);
  \pdot{A\x.input 3 |- 0,5*\dy-\dy*\yb};
  \draw (A\x.input 2) -- (A\x.input 2 |- 0,5*\dy-\dy*\yc);
  \pdot{A\x.input 2 |- 0,5*\dy-\dy*\yc};
  \draw (A\x.input 1) -- (A\x.input 1 |- 0,5*\dy-\dy*\yd);
  \pdot{A\x.input 1 |- 0,5*\dy-\dy*\yd};
}
\node[and gate,scale=\dy,inputs={normal,normal,normal,normal,normal,normal,normal,normal,normal},rotate=-90] (O) at (\xpo+\dxp*5,-1.25) {};
\foreach \x/\y/\z in {1/1/9,2/2/8,3/3/7,4/4/6,5/5/5,6/4/4,7/3/3,8/2/2,9/1/1} {
  \draw (O.input \x) -- (O.input \x |- 0,-0.85+0.065*\y) -| (A\z.output);
}
\draw (O.output) |- ++(\dy,-0.5*\dy) node[anchor=west]{$A$};
\end{tikzpicture}}
\subfigure[Minimale POS]{
\begin{tikzpicture}[circuit logic US]
\def\dy{0.5};
\def\dx{0.125};
\def\dxp{0.75};
\def\xpo{0.25};
\def\wi{\xpo+\dxp*3.5};
\foreach \i/\ii/\j in {a/4/1,b/3/2,c/2/3,d/1/4} {
  \coordinate (x\i) at (0,\dy*\ii);
  \coordinate (xn\i) at (0,\dy*\ii-0.5*\dy);
  \node[anchor=east] (xt\i) at (x\i) {$x_\j$};
  \pdot{x\i};
  \draw (x\i) -- ++(\wi,0);
}
\foreach \i in {a,b,c,d} {
  \node[not gate,anchor=input,scale=\dy] (Nx\i) at (xn\i -| \dx,0) {};
  \draw (x\i) |- (Nx\i.input);
  \draw (Nx\i.output) -- (xn\i -| \wi,0);
}
\foreach \x/\ya/\yb in {1/1/2,2/1.5/3.5,3/2.5/4.5} {
  \node[or gate,scale=\dy,inputs={normal,normal},rotate=-90] (A\x) at (\xpo+\dxp*\x,-0.25) {};
  \draw (A\x.input 2) -- (A\x.input 2 |- 0,5*\dy-\dy*\ya);
  \pdot{A\x.input 2 |- 0,5*\dy-\dy*\ya};
  \draw (A\x.input 1) -- (A\x.input 1 |- 0,5*\dy-\dy*\yb);
  \pdot{A\x.input 1 |- 0,5*\dy-\dy*\yb};
}
\node[and gate,scale=\dy,inputs={normal,normal,normal},rotate=-90] (O) at (\xpo+\dxp*2,-1.25) {};
\foreach \x/\y/\z in {1/1/3,2/2/2,3/1/1} {
  \draw (O.input \x) -- (O.input \x |- 0,-0.85+0.065*\y) -| (A\z.output);
}
\draw (O.output) |- ++(\dy,-0.5*\dy) node[anchor=west]{$A$};
\end{tikzpicture}}
\caption{Verschillende implementaties van dezelfde logische functie.}
\label{fig:differentImplementationsSSD}
\end{figure}
\begin{table}[hbt]
\centering
\begin{tabular}{l|l|rr|rr}
Modus&Implementatie&Kosten&Relatief&Vertraging&Relatief\\\hline
\multirow{2}{*}{Canonieke SOP}&AND-OR&47&100\%&8.6&100\%\\
&NAND-NAND&39&83\%&6.6&77\%\\\hline
\multirow{2}{*}{Canonieke POS}&OR-AND&59&126\%&9.4&109\%\\
&NOR-NOR&49&104\%&7.4&86\%\\\hline
\multirow{2}{*}{Minimale SOP}&AND-OR&16&34\%&6.6&77\%\\
&NAND-NAND&12&26\%&4.6&53\%\\\hline
\multirow{2}{*}{Minimale POS}&OR-AND&17&36\%&6.2&72\%\\
&NOR-NOR&13&28\%&4.2&49\%
\end{tabular}
\caption{Samenvatting van de verschillende implementaties.}
\label{tbl:differentImplementationsSSD}
\end{table}
\section{Rekenkundige basisschakelingen}
\label{s:rekenkundig}
In deze sectie defini\"eren we enkele belangrijke schakelingen voor rekenkundige bewerkingen. We hebben het dan over optellen, aftrekken vermenigvuldigen, delen, modulo rekenen en logische berekeningen. Alvorens we echter met getallen kunnen rekenen, moeten we een manier bedenken om getallen voor te stellen met binaire signalen. Doorheen deze sectie zullen we de voorstelling van getallen regelmatig veranderen om extra functionaliteit toe te voegen.
\subsection{Getallen voorstellen}
In de wereld wordt bij het voorstellen van getallen meestal het Arabisch getalsysteem gehanteerd. Hierbij stellen we een getal voor door een reeks cijfers. Een cijfer op een bepaalde plaats heeft een gewicht dat $r$ keer groter is, dan het volgende cijfer, met $r$ als de \termen{radix} van het getalstelsel. Om tot een eenduidige voorstelling van elk getal te komen wordt de verzameling mogelijke cijfers beperkt tot $r$ elementen. Een getal $D_r$ wordt dus voorgesteld als:
\begin{equation}
D_r=d_{m-1}d_{m-2}\ldots d_0,d_{-1}\ldots d_{-n}=\displaystyle\sum_{i=-n}^{m-1}{r^i\times d_i}
\label{eqn:numberRepresentation}
\end{equation}
Hierbij zijn $d_i$ de cijfers van het getal. Wereldwijd gebruikt men het \termen{decimale stelsel}, andere populaire stelsel zijn het \termen{binair} ($r=2$), \termen{octaal} ($r=8$) en \termen{hexadecimaal} ($r=16$) stelsel. Het binaire stelsel heeft logischerwijs twee mogelijke cijfers. Deze kunnen we voorstellen door $0$ of $1$ op een lijn te zetten. Indien we meer cijfers nodig hebben, zullen we eenvoudigweg meer lijnen voorzien, die elk een cijfer van het getal voorstellen. In de informatica en elektronica maakt men theoretisch vaak gebruik van andere getalstelsel. Dit komt omdat binaire getallen niet bepaald compact zijn\footnote{Een binair getal bestaat uit $3.3$ keer het aantal cijfers van zijn decimale tegenhanger.}. Immers is het omzetten van binaire getallen naar hexadecimale getallen niets anders dan het groeperen van cijfers. Indien men getallen noteert met een andere radix dan de decimale, wordt de radix in decimale notatie als subscript toegevoegd. Zo is $\mbox{2A}_{16}$ het equivalent van $42$.
\subsection{Radix-conversie}
\subsubsection{$r_1\rightarrow r_2$ omzetting met $r_1=r_2^p$}
Een speciaal geval van omzetting treed op indien de bronradix een macht is van de doel matrix. In dat geval kunnen we de omzetting eenvoudig doen door alfabetomzetting. Immers betekent dit dat we elk broncijfer kunnen omzetten naar $p$ doelcijfers. Hierbij dient de sequentie doelcijfers het wiskundig equivalent te zijn van de broncijfers. Een concreet voorbeeld is het omzetten van een hexadecimaal getal in het binair stelsel ($r_1=16=2^4=r_2^4$). Tabel \ref{tbl:radixConversionHexOctBin}
\begin{table}[hbt]
\centering
\subtable[Hex $\leftrightarrow$ Binair]{
\begin{tabular}{r|r||r|r||r|r||r|r}
Hex&Bin&Hex&Bin&Hex&Bin&Hex&Bin\\\hline
0&0000&4&0100&8&1000&C&1100\\
1&0001&5&0101&9&1001&D&1101\\
2&0010&6&0110&A&1010&E&1110\\
3&0011&7&0111&B&1011&F&1111\\
\end{tabular}
}
\subtable[Octaal $\leftrightarrow$ Binair]{
\begin{tabular}{r|r||r|r}
Oct&Bin&Oct&Bin\\\hline
0&000&4&100\\
1&001&5&101\\
2&010&6&110\\
3&011&7&111\\
\end{tabular}
}
\caption{Radix-conversie van hexadecimaal en octaal naar binair.}
\label{tbl:radixConversionHexOctBin}
\end{table}
toont de omzetting van hexadecimale en octale cijfers naar binaire cijfers. Bij wijzen van voorbeeld zetten we $\mbox{B4F}_{16}$ om naar het binaire stelsel\footnote{De verticale strepen ($|$) dienen uitsluitend om educatieve doeleinden, en zijn niet verplicht.}:
\begin{equation}
\mbox{B4F}_{16}=1011|0100|1111=101101001111_2
\end{equation}
\subsubsection{$r_1\rightarrow r_2$ omzetting met $r_1^q=r_2$}
In de omgekeerde situatie willen we een getal van een radix $r_1$ omzetten naar een macht van deze radix. De oplossing ligt dan ook voor de hand: we vertalen groepjes van $q$ broncijfers naar 1 doelcijfer. Een belangrijke opmerking is hoe we de voorstelling met de bronradix onderverdelen in groepen: dit doen we vanaf de komma voor het gehele gedeelte naar links, voor het kommagedeelte naar rechts. Aan de uiteinden van de voorstelling kunnen er soms onvoldoende cijfers aanwezig zijn. In dat geval dienen nullen toegevoegd te worden. Bij wijze van voorbeeld zetten we $1011011111.10001_2$ om naar zijn octale equivalent (zie hiervoor tabel \ref{tbl:radixConversionHexOctBin}):
\begin{equation}
1011011111.10001_2=1|011|011|111.100|01_2=\underline0\underline01|011|011|111.100|01\underline0_2=1337.42_8
\end{equation}
\subsubsection{$r_1\rightarrow r_2$ omzetting met $r_1^q=r_2^p$}
Een logisch gevolg van de voorgaande omzettingen, is dat we ook de mogelijkheid hebben om getallen makkelijk tussen 2 radixen om te zetten indien ze een gemeenschappelijke macht hebben. In dat geval laten we de conversie verlopen langs een derde radix: $r'$. $r'$ vormt een basis waarvan de bron- en doelradix machten zijn. Er geldt dan ook: $r_1=r'^p$ en $r'^q=r_2$. Een concreet voorbeeld is het binaire stelsel die als een tolk kan functioneren tussen het hexadecimale en octale stelsel. Dit illustreren we door $157255_8$ om te zetten naar het hexadecimaal stelsel:
\begin{equation}
\begin{array}{ll}
\mbox{Oct$\rightarrow$Bin}&157255_8=001|101|111|010|101|101_2=1101111010101101_2\\
\mbox{Bin$\rightarrow$Hex}&1101111010101101_2=1101|1110|1010|1101_2=\mbox{DEAD}_{16}
\end{array}
\end{equation}
\subsubsection{Algemene omzetting}
De vorige methodes werkten enkel onder een fundamentele aanname: de ene radix moest een macht van een andere zijn. In de praktijk is dit meestal niet zo, zo willen we vaak decimale getallen omzetten naar hun binair equivalent, of omgekeerd. In deze subsubsectie behandelen we kort een methode om in het algemeen een radixomzetting uit te voeren. Deze omzettingen zijn relatief arbeidintensief, en vereisen bovendien een getalstel waarin we simpele rekenkundige operaties kunnen uitvoeren. Voor mensen is dit over het algemeen het decimaal stelsel, computers werken doorgaans met het binair stelsel\footnote{Dit is niet altijd zo, in de Sovjet-Unie waren in de jaren '50 ternaire computers populair.}. In deze methode zetten we eerst de representatie om naar de representatie waarop we kunnen rekenen. Indien we bijvoorbeeld $\mbox{8C989}_{16}$ willen omzetten naar een radix 36, zullen we dit getal eerst omzetten naar het decimale stelsel, dit doen we door de waarde van de cijfers te vermenigvuldigen met het gewicht van hun positie en deze vervolgens op te tellen:
\begin{equation}
\mbox{8C989}_{16}=8\cdot 16^4+12\cdot 16^3+9\cdot 16^2+8\cdot 16^1+9\cdot 16^0=575881
\end{equation}
Vervolgens bepalen we iteratief elk cijfer, hierbij beginnen we bij het laatste cijfer, dit vinden we door het getal modulo de radix te bereken: $575881\mod36=25=\mbox{P}_{36}$. Hiermee weten we al het laatste cijfers. We trekken vervolgens de bekomen waarde af van het getal, en delen het door de radix. Dit resultaat kunnen we zo verder iteratief manipuleren, om de andere cijfers te berekenen. Indien we 0 uitkomen stopt het algoritme en hebben we de equivalente representatie gevonden. Tabel \ref{tbl:radixConversionExample} illustreert dit principe.
\begin{table}[hbt]
\centering
\begin{tabular}{r|r|l|l|r}
Stap&Getal&Modulo&Volgende getal&Cijfer\\\hline
$1$&$575881$&$575881\mod36=25$&$\left(575881-25\right)/36=15996$&$25=\mbox{P}_{36}$\\
$2$&$15996$&$15996\mod36=12$&$\left(15996-12\right)/36=44$&$12=\mbox{C}_{36}$\\
$3$&$444$&$444\mod36=12$&$\left(444-12\right)/36=12$&$12=\mbox{C}_{36}$\\
$4$&$12$&$12\mod36=12$&$\left(12-12\right)/36=0$&$12=\mbox{C}_{36}$\\
$5$&$0$&$-$&$-$&$-$
\end{tabular}
\caption{Voorbeeld van algemene radix-omzetting.}
\label{tbl:radixConversionExample}
\end{table}
We kunnen dus concluderen dat: $\mbox{8C989}_{16}=\mbox{CCCP}_{36}$.
\subsection{Optellen}
\label{ss:add}
Hoe tellen we nu twee getallen op? In het decimaal stelsel tellen we twee getallen op door middel van cijferen. Hieronder geven we een illustrerend voorbeeld waarbij we $1425+1917$ uitrekenen:
\begin{equation}
\begin{array}{l|lcccc}
\mbox{overdracht $c$}&&1&0&1&\\
x&&1&4&2&5\\
y&+&1&9&1&7\\\hline
\mbox{som $s$}&&3&3&4&2
\end{array}
\end{equation}
Om getallen in het decimaal stelsel te kunnen optellen gebruiken we een repetitieve structuur, waarbij we 200 basisregels moeten onthouden. Zo'n basisregel $f\left(c_i,x_i,y_i\right)=\left(s_i,c_{i+1}\right)$ is een functie die de eventuele \termen{overdracht} of \termen{carry}, en de cijfers van de twee getallen in een bepaalde kolom omzet naar de overdracht van de volgende kolom en de som van deze kolom. Een voorbeeld van zo'n basisregel is $f\left(0,4,9\right)=\left(3,1\right)$. Deze regel wordt in ons voorbeeld gebruikt in de derde kolom vanaf rechts\footnote{We nummeren de kolommen vanaf rechts, analoog aan de getalvoorstelling van vergelijking (\ref{eqn:numberRepresentation})}. Binair optellen gebeurt volledig analoog, we hebben opnieuw met een repetitieve structuur te maken, alleen dienen we nu slechts 8 verschillende regels te onthouden.
\subsubsection{Half adder}
In deze paragraaf synthetiseren we een \termen{half adder (HA)}, die de laatste bit van de twee getallen optelt. Bij het laatste getal is de overdracht sowieso 0, dus daarmee hoeven we geen rekening te houden. We dienen een functie te ontwikkelen die de overdracht van de volgende kolom berekend, en de som. We stellen een waarheidstabel en de bijbehorende Karnaugh-kaarten op in figuur \ref{fig:halfAdder}.
\begin{figure}[hbt]
\centering
\subfigure[Waarheidstabel]{
\begin{tikzpicture}
\draw (0,0) node{\begin{tabular}{cc|cc}
$x_i$&$y_i$&$s_i$&$c_{i+1}$\\\hline
0&0&0&0\\
0&1&1&0\\
1&0&1&0\\
1&1&0&1
\end{tabular}};
\end{tikzpicture}}
\subfigure[Karnaugh-kaarten]{\begin{tikzpicture}
\kkaartb{0}{0}{$s_i$}{$x_i$/$y_i$}{0/0/0/1}
\kkaartb{2.25}{0}{$c_{i+1}$}{$x_i$/$y_i$}{0/1/1/0}
\end{tikzpicture}}
\subfigure[Interface]{\begin{tikzpicture}
\node[halfadder] (HA) at (0,0) {HA};
\draw[->] (HA.co) -- ++(-0.25,0) node[anchor=east]{$c_{i+1}$};
\draw[<-] (HA.x) -- ++(0,0.25) node[anchor=south]{$x_i$};
\draw[<-] (HA.y) -- ++(0,0.25) node[anchor=south]{$y_i$};
\draw[->] (HA.s) -- ++(0,-0.25) node[anchor=north]{$s_i$};
\end{tikzpicture}}
\subfigure[Mogelijke implementatie]{\begin{tikzpicture}[circuit logic US]
\draw (-1.5,-1) rectangle (1.5,1);
\draw[dashed] (-0.5,1) -- ++(0,0.5) node[anchor=south]{$x_i$};
\draw[dashed] (0.5,1) -- ++(0,0.5) node[anchor=south]{$y_i$};
\draw[dashed] (-1.5,0) -- ++(-0.5,0) node[anchor=east]{$c_{i+1}$};
\draw[dashed] (0,-1) -- ++(0,-0.5) node[anchor=north]{$s_i$};
\node[and gate,rotate=-90] (A) at (-0.5,0) {};
\node[xor gate,rotate=-90] (X) at (0.5,0) {};
\draw (X.output) -- (X.output |- 0,-0.75) -| (0,-1);
\draw (A.output) |- (-1.25,-0.75) |- (-1.5,0);
\draw (-0.5,1) |- (A.input 2 |- 0,0.75) -- (A.input 2);
\draw (0.5,1) |- (X.input 1 |- 0,0.625) -- (X.input 1);
\draw (0.5,0.625) -| (A.input 1);
\draw (-0.5,0.75) -| (X.input 2);
\pdot{0.5,0.625}
\pdot{-0.5,0.75}
\end{tikzpicture}
\label{fig:halfAdderImplementation}}
\caption{Half adder (HA).}
\label{fig:halfAdder}
\end{figure}
Deze combinatorische functies kunnen we verwezenlijken met een AND en XOR poort.
\subsubsection{Full adder}
\label{sss:fulladder}
Met een Half adder kunnen we echter geen optellingen uitrekenen. Hiervoor dienen we een component te ontwikkelen die meer functionaliteit biedt: de \termen{full adder (FA)}. Een full adder bevat slechts \'e\'en ingang extra: de overdracht van de vorige opteller $c_i$. Een full adder telt op door een XOR operatie toe te passen op de drie ingangen: $x_i$, $y_i$ en $c_i$, verder is er sprake van overdracht indien twee of meer ingangen 1 zijn. Dit kunnen we eenvoudig implementeren met tweelagen-logica. De waarheidstabellen en Karnaugh-kaarten staan samen met een mogelijke implementatie op figuur \ref{fig:fullAdder}.
\begin{figure}[hbt]
\centering
\subfigure[Waarheidstabel]{
\begin{tikzpicture}
\draw (0,0) node{\begin{tabular}{ccc|cc}
$x_i$&$y_i$&$c_i$&$s_i$&$c_{i+1}$\\\hline
0&0&0&0&0\\
0&0&1&1&0\\
0&1&0&1&0\\
0&1&1&0&1\\
1&0&0&1&0\\
1&0&1&0&1\\
1&1&0&0&1\\
1&1&1&1&1
\end{tabular}};
\end{tikzpicture}}
\subfigure[Karnaugh-kaarten]{\begin{tikzpicture}
\kkaartc{0}{0}{$s_i$}{$x_i$/$y_i$/$c_i$}{0/1/1/0/1/0/0/1}
\kkaartc{2.25}{0}{$c_{i+1}$}{$x_i$/$y_i$/$c_i$}{0/0/0/1/0/1/1/1}
\end{tikzpicture}}
\subfigure[Interface]{\begin{tikzpicture}
\node[fulladder] (FA) at (0,0) {FA};
\draw[<-] (FA.ci) -- ++(0.25,0) node[anchor=west]{$c_i$};
\draw[->] (FA.co) -- ++(-0.25,0) node[anchor=east]{$c_{i+1}$};
\draw[<-] (FA.x) -- ++(0,0.25) node[anchor=south]{$x_i$};
\draw[<-] (FA.y) -- ++(0,0.25) node[anchor=south]{$y_i$};
\draw[->] (FA.s) -- ++(0,-0.25) node[anchor=north]{$s_i$};
\end{tikzpicture}}
\subfigure[Mogelijke implementatie]{\begin{tikzpicture}[circuit logic US]
\draw (-2.5,-1.5) rectangle (1.25,1.75);
\draw[dashed] (-1.25,1.75) -- ++(0,0.5) node[anchor=south]{$x_i$};
\draw[dashed] (0,1.75) -- ++(0,0.5) node[anchor=south]{$y_i$};
\draw[dashed] (-2.5,0.125) -- ++(-0.5,0) node[anchor=east]{$c_{i+1}$};
\draw[dashed] (1.25,0.125) -- ++(0.5,0) node[anchor=west]{$c_i$};
\draw[dashed] (-0.625,-1.5) -- ++(0,-0.5) node[anchor=north]{$s_i$};
\node[nand gate,rotate=-90] (A1) at (-2,0.66) {};
\node[nand gate,rotate=-90] (A2) at (-1.25,0.66) {};
\node[nand gate,rotate=-90] (A3) at (-0.5,0.66) {};
\node[nand gate,inputs={normal,normal,normal},rotate=-90] (O) at (-1.25,-0.66) {};
\draw (A1.output) -- (A1.output |- 0,0) -| (O.input 3);
\draw (A2.output) -- (O.input 2);
\draw (A3.output) -- (A3.output |- 0,0) -| (O.input 1);
\draw (O.output) |- (-2.25,-1.25) |- (-2.5,0.125);
\node[xor gate,rotate=-90] (X1) at (0.5,0.66) {};
\node[xor gate,rotate=-90] (X2) at (0.75,-0.66) {};
\draw (X2.output) -- (X2.output |- 0,-1.25) -| (-0.625,-1.5);
\draw (1.25,0.125) -| (X2.input 1);
\draw (A1.input 2) |- (X1.input 1 |- 0,1.35) -- (X1.input 1);
\draw (A1.input 1) |- (X1.input 2 |- 0,1.225) -- (X1.input 2);
\draw (A2.input 2) -- (A2.input 2 |- 0,1.35);
\draw (A2.input 1) |- (1,1.5) -- (1,0.125);
\draw (A3.input 2) -- (A3.input 2 |- 0,1.225);
\draw (A3.input 1) -- (A3.input 1 |- 0,1.5);
\draw (X1.output) -- (X1.output |- 0,0) -| (X2.input 2);
\draw (-1.25,1.35) -- (-1.25,1.75);
\draw (0,1.225) -- (0,1.75);
\pdot{-1.25,1.35}
\pdot{0,1.225}
\pdot{A2.input 2 |- 0,1.35}
\pdot{A3.input 2 |- 0,1.225}
\pdot{A3.input 1 |- 0,1.5}
\pdot{1,0.125}
\end{tikzpicture}}
\subfigure[Functionele ontbinding]{
\begin{tikzpicture}[circuit logic US]
\coordinate (xi) at (-1.33333,1.75);
\coordinate (yi) at (0.08333,1.75);
\coordinate (ci) at (1.5,0.25);
\coordinate (co) at (-2.75,0.25);
\coordinate (si) at (-0.625,-1.25);
\draw (co |- xi) rectangle (ci |- si);
\draw[dashed] (xi) -- ++(0,0.5) node[anchor=south]{$x_i$};
\draw[dashed] (yi) -- ++(0,0.5) node[anchor=south]{$y_i$};
\draw[dashed] (co) -- ++(-0.5,0) node[anchor=east]{$c_{i+1}$};
\draw[dashed] (ci) -- ++(0.5,0) node[anchor=west]{$c_i$};
\draw[dashed] (si) -- ++(0,-0.5) node[anchor=north]{$s_i$};
\node[halfadder] (HA1) at (-1.5,0.625) {HA};
\node[halfadder] (HA2) at (0.5,0.625) {HA};
\draw (xi) -- ++(0,-0.125) -| (HA1.y);
\draw (yi) -- ++(0,-0.25) -| (HA1.x);
\draw (HA1.s) |- ++(0.875,-0.25) |- (HA2.y |- 0,1.375) -- (HA2.y);
\draw (ci) -- ++(-0.125,0) |- (HA2.x |- 0,1.375) -- (HA2.x);
\draw (HA2.s) -- ++(0,-1.125) -| (si);
\node[or gate,rotate=180] (O) at (-1.5,-0.75) {};
\draw (HA1.co) -- ++(-0.125,0) |- (O.input 2 |- 0,-0.25) -| (O.input 2 -| -1,0) -- (O.input 2);
\draw (HA2.co) -- ++(-0.125,0) |- (O.input 1);
\draw (O.output) -| (co -| -2.625,0) -- (co);
\end{tikzpicture}
\label{fig:fullAdderFunctionalImplementation}
}
\caption{Full adder (FA).}
\label{fig:fullAdder}
\end{figure}
We kunnen ook een implementatie synthetiseren met behulp van functionele ontbinding. Bij een full adder voeren we immers twee optellingen uit, de volgorde speelt hierbij geen rol. Indien minstens \'e\'en van de twee optellingen overdracht genereert, is er sprake van overdracht bij de full adder. We kunnen een full adder dus ook implementeren zoals op figuur \ref{fig:fullAdderFunctionalImplementation}. Indien we de half adders dan implementeren zoals we dit op figuur \ref{fig:halfAdderImplementation}, zien we dat we \'e\'en poort uitsparen. Dit betalen we echter met grotere vertragingen, iets wat bij optellingen met grote getallen niet gewenst is.
\subsubsection{Ripple-carry opteller}
Met behulp van een half adder en $n$ full adders kunnen we vervolgens een opteller realiseren die twee $n+1$ bit getallen optelt. Dit doen we met behulp van een \termen{Ripple-carry opteller}. Figuur \ref{fig:rippleCarryAdder}
\begin{figure}[hbt]
\centering
\begin{tikzpicture}
\node[fulladder] (FA0) at (0,0) {FA$_0$};
\node[fulladder] (FA1) at (-2,0) {FA$_1$};
\node[fulladder] (FA2) at (-4,0) {FA$_2$};
\node[fulladder] (FAi) at (-7,0) {FA$_i$};
\node[fulladder] (FAn) at (-10,0) {FA$_n$};
\node (FAidr) at (-5.5,0) {$\ldots$};
\node (FAidl) at (-8.5,0) {$\ldots$};
\draw (FA0.ci) -- ++(0.5,0) node[anchor=west,scale=0.7]{$c_0=0$};
\draw (FA0.co) -- (FA1.ci);
\draw (FA1.co) -- (FA2.ci);
\draw (FA2.co) -- ++(-0.25,0);
\draw (FAi.ci) -- ++(0.25,0);
\draw (FAi.co) -- ++(-0.25,0);
\draw (FAn.ci) -- ++(0.25,0);
\draw (FAn.co) -- ++(-0.5,0) node[anchor=east,scale=0.7]{$c_n$};
\foreach \i/\l in {0/FA0,1/FA1,2/FA2,i/FAi,n-1/FAn} {
  \draw (\l.x) -- ++(0,0.5) node[anchor=south,scale=0.7]{$y_{\i}$};
  \draw (\l.y) -- ++(0,0.5) node[anchor=south,scale=0.7]{$x_{\i}$};
  \draw (\l.s) -- ++(0,-0.5) node[anchor=north,scale=0.7]{$s_{\i}$};
}
\end{tikzpicture}
\caption{Schematische voorstelling van een $n$-bit Ripple-carry opteller.}
\label{fig:rippleCarryAdder}
\end{figure}
toont hoe dit in z'n werk gaat: we tellen de laatste twee bits op met een half adder, de overige bits tellen we op met full adders. De carry uitgangen van een adder gaat naar de de ingang van de volgende adder. De allerlaatste carry uitgang $c_n$, kan men gebruiken als een \termen{overflow} uitgang. Indien we immers twee $n$-bit getallen met elkaar optellen, kunnen sommige resultaten $n+1$-bit getallen vereisen om nog voorgesteld te kunnen worden. In een ander geval wordt deze uitgang gebruikt om de waarde van de $n+1$-ste bit te bepalen. In dat geval heeft de uitgang meer bits dan de operanden. Uiteraard kunnen we de half adder vervangen door een full adder met als carry ingang 0.
\paragraph{}
Omdat we meestal getallen met een groot aantal bit optellen\footnote{Op de meeste processoren is dat 32 of 64 bit.} is het interessant om het tijdsgedrag van een ripple-carry adder te bekijken. Het kritisch pad gaat logischerwijs van een ingang van de half adder tot de laatste full adder. Indien we de implementaties van de full adder met functionele ontbinding beschouwen blijkt het kritische pad $x_0\rightarrow c_n$ te zijn. In dit geval moet het signaal doorheen 1 XOR, $n$ AND en $n$ OR poorten. Elk van deze poorten heeft twee ingangen, de vertraging is dus bijgevolg:
\begin{equation}
\mbox{vertraging}=2.4n+2.4n+3.2=4.8n+3.2
\end{equation}
Het is mogelijk dat we deze vertraging verder kunnen reduceren, een probleem is echter dat de vertraging een orde $n$ blijft. Voor berekeningen met grote getallen zijn ripple-carry adders dan ook onaanvaardbaar.
\subsubsection{Carry-lookahead opteller}
Een \termen{Carry-lookahead opteller (CLA)} is een component ter vervanging van een full adder. Het is de bedoeling om met behulp van dit component rechtstreeks $c_i$ te berekenen als een functie $c_i\left(c_0,x_0,x_1,\ldots,x_{i-1},y_0,y_1,\ldots,y_{i-1}\right)$. De Carry-lookahead opteller heeft dezelfde ingangen als een full adder: $(x_i,y_i,c_i)$, maar meer uitgangen $(c_{i+1},g_i,p_i,s_i)$. Deze uitgangssignalen hebben volgende functie:
\begin{itemize}
 \item \termen{carry-generate $g_i$}: 1 indien er bij het optellen van $x_i$ en $y_i$ overdracht gegenereerd wordt. Bijgevolg is $g_i=x_iy_i$.
 \item \termen{carry-propagate $p_i$}: 1 indien bij de optelling de overdracht verder gepropageerd zal worden. Bijgevolg $p_i=x_iy_i'+x_i'y_i$.
 \item som $s_i$: het resultaat van de optelling: $s_i=c_ip_i'+c_i'p_i$.
\end{itemize}
De opsomming van de uitgangen geeft dan ook al meteen een mogelijke implementatie, de waarheidstabellen, interface en een mogelijke implementatie staan op figuur \ref{fig:carryLookaheadAdder}.
\begin{figure}[hbt]
\centering
\subfigure[Waarheidstabel]{
\begin{tikzpicture}
\draw (0,0) node{\begin{tabular}{ccc|ccc}
$x_i$&$y_i$&$c_i$&$s_i$&$g_i$&$p_i$\\\hline
0&0&0&0&0&0\\
0&0&1&1&0&0\\
0&1&0&1&0&1\\
0&1&1&0&0&1\\
1&0&0&1&0&1\\
1&0&1&0&0&1\\
1&1&0&0&1&0\\
1&1&1&1&1&0
\end{tabular}};
\end{tikzpicture}}
\subfigure[Karnaugh-kaarten]{\begin{tikzpicture}
\kkaartc{1.125}{3}{$s_i$}{$x_i$/$y_i$/$c_i$}{0/1/1/0/1/0/0/1}
\kkaartc{0}{0}{$g_i$}{$x_i$/$y_i$/$c_i$}{0/0/0/0/0/0/1/1}
\kkaartc{2.25}{0}{$p_i$}{$x_i$/$y_i$/$c_i$}{0/0/1/1/1/1/0/0}
\end{tikzpicture}}
\subfigure[Interface]{\begin{tikzpicture}
\node[cla] (CLA) at (0,0) {};
\draw[<-] (CLA.x) -- ++(0,0.25) node[anchor=south]{$x_i$};
\draw[<-] (CLA.y) -- ++(0,0.25) node[anchor=south]{$y_i$};
\draw[->] (CLA.g) -- ++(0,-0.25) node[anchor=north]{$g_i$};
\draw[->] (CLA.p) -- ++(0,-0.25) node[anchor=north]{$p_i$};
\draw[->] (CLA.s) -- ++(-0.25,0) node[anchor=east]{$s_i$};
\draw[<-] (CLA.c) -- ++(0.25,0) node[anchor=west]{$c_i$};
\end{tikzpicture}}
\subfigure[Mogelijke implementatie]{\begin{tikzpicture}[circuit logic US]
\coordinate (x) at (-0.75,1);
\coordinate (y) at (0.75,1);
\coordinate (xb) at (x |- 0,0.875);
\coordinate (yb) at (y |- 0,0.75);
\coordinate (g) at (x |- 0,-1);
\coordinate (p) at (y |- g);
\coordinate (s) at (-1.25,0);
\coordinate (sb) at (s -| -1.125,0);
\coordinate (c) at (1.25,0);
\coordinate (cb) at (c -| 1.125,0);
\draw (s |- x) rectangle (c |- g);
\node[and gate,rotate=-90] (A) at (g |- 0,0.15) {};
\node[xor gate,rotate=-90] (X0) at (p |- 0,0.15) {};
\node[xor gate,rotate=180] (X1) at (0,-0.55) {};
\draw (x) -- (xb);
\draw (y) -- (yb);
\draw (s) -- (sb);
\draw (c) -- (cb);
\draw (xb) -| (X0.input 2);
\draw (yb) -| (X0.input 1);
\draw (xb) -| (A.input 2);
\draw (yb) -| (A.input 1);
\draw (X0.output) -- (p);
\draw (X1.output) -| (sb);
\draw (X1.input 1) -| (cb);
\draw (X1.input 2) -- (X1.input 2 -| X0.output);
\draw (A.output) -- (g);
\draw[dashed] (x) -- ++(0,0.5) node[anchor=south]{$x_i$};
\draw[dashed] (y) -- ++(0,0.5) node[anchor=south]{$y_i$};
\draw[dashed] (s) -- ++(-0.5,0) node[anchor=east]{$s_i$};
\draw[dashed] (c) -- ++(0.5,0) node[anchor=west]{$c_i$};
\draw[dashed] (g) -- ++(0,-0.5) node[anchor=north]{$g_i$};
\draw[dashed] (p) -- ++(0,-0.5) node[anchor=north]{$p_i$};
\pdot{xb}
\pdot{yb}
\pdot{X0.output |- X1.input 2}
\end{tikzpicture}}
\caption{Carry-Lookahead Opteller (CLA).}
\label{fig:carryLookaheadAdder}
\end{figure}
Een belangrijke opmerking bij dit component is dat we de overdracht $c_{i+1}$ vervolgens kunnen berekenen als $c_{i+1}=g_i+c_ip_i$. De overdacht hangt dus niet rechtstreeks af van $x_i$ of $y_i$, deze eigenschap is belangrijk voor de synthese van een ander component: de CLA-generator.
\subsubsection{CLA-generator}
Om een opteller te realiseren moeten we net als bij de Ripple-carry opteller, moeten we alleen nog de glue voorzien tussen verschillende carry-lookahead optellers. Bij deze optellers is het verhaal echter gecompliceerder: we hebben een extra component nodig om de CLA elementen met elkaar te verbinden: de \termen{CLA-generator}. Een $n$-bit CLA-generator is een component met ingangen $\left(c_0,g_0,\ldots g_{n-1},p_0,\ldots,p_{n-1}\right)$. De component heeft tot doel om de overdracht (carry) te berekenen met een zo klein mogelijk kritisch pad, logischerwijs heeft de component dan ook de uitgangen: $\left(c_1,\ldots c_n,g_{0,n-1},p_{0,n-1}\right)$. De laatste twee uitgangen zullen we later bespreken. Op figuur \ref{fig:CLAGeneratorSchematic} zien we hoe we een CLA-generator aansluiten op de Carry-lookahead optellers.
\begin{figure}[hbt]
\centering
\subfigure[Structuur van de CLA-generator.]{
\begin{tikzpicture}
\draw (-12.75 cm,0 cm) rectangle (0.75 cm,-1.5 cm);
\draw (0.75,-0.75) -- ++(1,0) node[anchor=west,scale=0.7]{$c_0$};
\foreach\x/\i/\j in {0/0/1,1/1/2,2/2/3,4/n-1/n} {
  \node[cla] (CLA\x) at (-3*\x,1) {\tiny{CLA${_{\i}}$}};
  \draw (CLA\x.x) -- (CLA\x.x |- 0,1.85) node[anchor=south,scale=0.7]{$x_{\i}$};
  \draw (CLA\x.y) -- (CLA\x.y |- 0,1.85) node[anchor=south,scale=0.7]{$y_{\i}$};
  \draw[->] (CLA\x.s) -| (-3*\x-1.25,1.85) node[anchor=south,scale=0.7]{$s_{\i}$};
  \draw (CLA\x.g) -- (CLA\x.g |- 0,0);
  \draw (CLA\x.p) -- (CLA\x.p |- 0,0);
  \draw (CLA\x.p |- 0,0) node[anchor=north,port labels]{p$_{\mbox{\tiny{\i}}}$};
  \draw (CLA\x.g |- 0,0) node[anchor=north,port labels]{g$_{\mbox{\tiny{\i}}}$};
  \draw (-3*\x,-1.5) node[anchor=south,port labels]{c$_{\mbox{\tiny{\j}}}$};
  \draw[->] (-3*\x,-1.5) -- ++(0,-0.5) node[anchor=north,scale=0.7]{$c_{\j}$};
}
\draw (0.75,-0.75) node[anchor=east,port labels]{c$_{\mbox{\tiny{0}}}$};
\draw (1.25,-0.75) |- (CLA0.c);
\pdot{1.25,-0.75}
\foreach\x/\y in {1/0,2/1} {
  \pdot{-3*\y,-1.75}
  \draw (-3*\y,-1.75) -| (CLA\x.c -| -3*\x+1.25,0) -- (CLA\x.c);
}
\draw (-12.75,-0.5) node[anchor=west,port labels]{g$_{\mbox{\tiny{0,n-1}}}$};
\draw (-12.75,-1) node[anchor=west,port labels]{p$_{\mbox{\tiny{0,n-1}}}$};
\pdot{-6,-1.75}
\draw (-6,-1.75) -| (-9+1.25,1) -- (-9+0.5,1) node[anchor=east]{$\ldots$};
\draw (CLA4.c) -| (-9-1.25,-1.75) -- (-9-0.5,-1.75) node[anchor=west]{$\ldots$};
\draw (-6.2,-0.75) node{\large{CLA-generator}};
\draw[->] (-12.75,-0.5) -- ++(-0.5,0) node[anchor=east,scale=0.7]{$g_{0,n-1}$};
\draw[->] (-12.75,-1) -- ++(-0.5,0) node[anchor=east,scale=0.7]{$p_{0,n-1}$};
\end{tikzpicture}
\label{fig:CLAGeneratorSchematic}}
\subfigure[Cascade van CLA-generators met $n=9$, $k=3$.]{
\begin{tikzpicture}
\foreach\x in {0,1,...,8} {
  \node[cla,scale=0.75] (CLA\x) at (-1.5*\x,0) {\small{CLA$_{\x}$}};
  \draw (CLA\x.x) -- ++(0,0.5) node[anchor=south,scale=0.7]{$x_{\x}$};
  \draw (CLA\x.y) -- ++(0,0.5) node[anchor=south,scale=0.7]{$y_{\x}$};
  \draw[->] (CLA\x.s) -| ++(-0.075,0.875) node[anchor=south,scale=0.7]{$s_{\x}$};
}
\foreach\x in {0,1,2} {
  \node[clag3,scale=0.75] (CLAG1\x) at (-4.5*\x-1.5,-1.5) {\small{CLA-Generator$_{1,\x}$}};
}
\node[clag3,scale=0.75,minimum width=17.5cm] (CLAG20) at (-6,-3) {\small{CLA-Generator$_{2,0}$}};
\foreach\x/\a/\y in {1/a/0,2/b/0,3/c/0,4/a/1,5/b/1,6/c/1,7/a/2,8/b/2} {
  \draw (CLA\x.c) -- ++(0.15,0) |- (CLAG1\y.c\a |- 0,-2) -- (CLAG1\y.c\a);
}
\foreach\x/\a/\y in {1/a/0,2/b/1} {
  \draw (CLAG1\x.c) -- ++(0.075,0) |- (CLAG20.c\a |- 0,-3.5) -- (CLAG20.c\a);
  \draw (CLAG1\y.pab) -| ++(-0.075,-0.6) -| (CLAG20.p\a);
  \draw (CLAG1\y.gab) -| ++(-0.15,-0.9375) -| (CLAG20.g\a);
}
\foreach\a/\y in {c/2} {
  \draw (CLAG1\y.pab) -| ++(-0.075,-0.6) -| (CLAG20.p\a);
  \draw (CLAG1\y.gab) -| ++(-0.15,-0.9375) -| (CLAG20.g\a);
}
\draw (CLA0.c) -- ++(0.3,0) |- (CLAG20.c);
\draw (CLAG10.c) -- ++(0.6,0) node[anchor=west,scale=0.7]{$c_0$};
\pdot{CLAG10.c -| 0.8625,0}
\foreach\x/\a/\y in {0/a/0,1/b/0,2/c/0,3/a/1,4/b/1,5/c/1,6/a/2,7/b/2,8/c/2} {
  \draw (CLA\x.g) -- (CLAG1\y.g\a);
  \draw (CLA\x.p) -- (CLAG1\y.p\a);
}
\end{tikzpicture}
\label{fig:CLAGeneratorCascade}}
\caption{CLA-generator.}
\label{fig:CLAGenerator}
\end{figure}
Eerst zullen we een methode ontwikkelen om een overdracht $c_i$ te berekenen. Een overdracht kan maar op twee manieren tot stand komen: ofwel wordt deze gegenereerd, ofwel wordt deze gepropageerd. Dit formaliseren we als:
\begin{equation}
c_i=g_{0,i-1}+p_{0,i-1}\cdot c_0
\end{equation}
Een oplettende lezer zal gemerkt hebben dat we hierbij $g$ en $p$ twee indices geven. De rede hiervoor is, dat de overdracht overal kan gegenereerd of gepropageerd worden. We moeten de twee indices dus als de definitie van een bereik zien. De overdacht kan uitsluitend tot de $i$-de bit gepropageerd worden, indien alle bits voor de $i$-de bit propageren, bijgevolg kunnen we stellen:
\begin{equation}
p_{i,j}=\displaystyle\prod_{k=i}^j{p_k}
\end{equation}
Tenslotte wordt er overdacht gegenereerd indien de vorige bit een overdacht genereerde, of een willekeurige bit ervoor, die deze dan tot de $i$-de bit weet te propageren. Dit formaliseren we als:
\begin{equation}
g_{i,j}=g_j+\displaystyle\sum_{k=i}^{j-1}{g_kp_{k+1,j}}=g_j+\displaystyle\sum_{k=i}^{j-1}{\left(g_k\cdot\displaystyle\prod_{l=k+1}^j{p_l}\right)}
\end{equation}
De volledige berekening voor het bepalen van overdacht $c_i$ wordt dan:
\begin{equation}
c_i=g_{i-1}+\displaystyle\sum_{k=0}^{i-2}{\left(g_k\cdot\displaystyle\prod_{l=k+1}^{i-1}{p_l}\right)}+c_0\cdot\displaystyle\prod_{k=0}^{i-1}{p_k}
\end{equation}
De berekening voor $c_i$ kost dan ook een $i+1$-input OR poort, en $i+1$ AND poorten, elk van deze poorten heeft een ander aantal ingangen vari\"erend van 2 tot $i+1$. We kunnen nu opnieuw het kritische pad analyseren. Dit pad is $x_0\rightarrow c_n$. Hierbij gaat het signaal doorheen 1 XOR poort, 1 $n+1$-input AND poort en 1 $n+1$ input OR poort. De totale vertraging is dus:
\begin{equation}
\mbox{vertraging}=3.2+2\left(2+0.4n\right)=0.8n+7.2
\label{eqn:delayCLAGenerator}
\end{equation}
Toegegeven dat de orde van de vertraging nog steeds linear is, maar de factor voor $n$ is met de CLA-generator sterk gereduceerd. We kunnen echter ook een cascade van CLA-generators maken. Indien we twee $n$-bit getallen willen optellen, en we groeperen elke $k$ componenten tot een niveau hoger, hebben we $\left\lceil\log_kn\right\rceil$ niveaus nodig. We kunnen in dat geval vergelijking (\ref{eqn:delayCLAGenerator}) gebruiken om de vertraging te berekenen:
\begin{equation}
\mbox{vertraging}=3.2+\left(2\left\lceil\log_kn\right\rceil-1\right)\times\left(4+0.8k\right)
\end{equation}
Hiermee kunnen we de vertraging reduceren tot een logaritmische orde. Figuur \ref{fig:CLAGeneratorCascade} toont hoe we een dergelijke cascade kunnen bouwen. Met deze cascade wordt ook meteen de rede van de twee extra ingangen duidelijk.
\subsection{Negatieve getallen}
Om getallen te kunnen aftrekken, zullen we het getalbegrip uitbreiden, zodat deze ook negatieve getallen kunnen voorstellen. Negatieve getallen voorstellen kan op verschillende manieren. In onderstaande subsubsecties zullen we de meest populaire bespreken. Hierbij moeten we in rekening houden dat we eventueel de bovenstaande opteller zullen moeten uitbreiden om ook negatieve getallen te kunnen optellen.
\subsubsection{`Sign-Magnitude'-voorstelling}
Het arabische getalstelsel lost dit probleem op door middel van een teken. Een getal bestaat dan uit twee delen: een teken (+ of -) en een grootte. Bij positieve getallen wordt dit teken zelfs meestal niet geplaatst. Deze voorstelling is gekend als de `\termen{sign-magnitude}' voorstelling. Aangezien een getal ofwel positief of negatief is, kunnen we bij binaire een extra bit bij het getal plaatsen. Deze bit is 0 bij een positief getal, en 1 bij een negatief getal. In dat geval wordt dus $\underline{0}1100_2=\underline{+}12_{10}$ en $\underline{1}1100_2=\underline{-}12_{10}$. Een getal met $n$ bits kan dus alle gehele waarden aannemen tussen $-2^{n-1}+1$ en $2^{n-1}-1$. Er zijn echter enkele problemen met deze voorstelling. Allereerst is er nu sprake van twee voorstellingen van $0$: $-0$ en $+0$. Dit leidt tot extra complexiteit bij bijvoorbeeld het vergelijken van twee getallen. Verder wordt optellen en aftrekken van twee getallen gecompliceerd. Er zijn veel testen nodig alvorens we de waardes van de getallen kunnen optellen of aftrekken, wat tot grote vertragingen leidt.
\subsubsection{Complement-voorstellingen}
De meeste voorstellingen van negatieve getallen zijn gebaseerd op complement voorstellingen. Er bestaan twee soorten complement voorstellingen voor een getal $D$ met $m$ cijfers en radix $r$:
\begin{itemize}
 \item \termen{cijfer-complement} of \termen{diminished-radix complement} $D'$: hierbij wordt elk cijfer $i$ vervangen door zijn complement $r-i-1$. Bijvoorbeeld:
 \begin{itemize}
    \item Het $9$-complement van $1337_{10}$ is $8662_{10}$.
    \item Het $1$-complement van $01001011\ 01010011_2$ is $10110100\ 10101100_2$.
 \end{itemize}
 Het \termen{1-complement} wordt soms gebruikt voor het voorstellen van negatieve getallen in het binaire stelsel.
 \item \termen{radix-complement} $D^*=r^m-D$. Bijvoorbeeld:
 \begin{itemize}
    \item Het radix-complement van $1337_{10}$ is $8663_{10}$.
    \item Het radix-complement van $01001011\ 01010011_2$ is $10110100\ 10101101_2$.
 \end{itemize}
 Voor binaire getallen is dit het \termen{2-complement}. Dit is veruit de meest populaire voorstellingswijze.
\end{itemize}
Een belangrijke eigenschap die ook al blijkt uit de voorbeelden is dat $D^*=D'+1$. Deze eigenschap is interessant, omdat het ons help snel het 2-complement te berekenen. Immers is het 1-complement niets anders dan het toepassen van een NOT operatie op alle bits.
\subsubsection{Twee-complement voorstelling}
Omdat de 2-complement voorstelling vrij populair is, zullen we deze verder bespreken. Stel we beschikken over $n$ cijfers in een getalstelsel met radix $r$. In dat geval kunnen we $r^n$ in principe niet voorstellen. Logisch gezien stellen we dit voor door $r^n\equiv 0$. Immers kunnen we stellen dat in dat geval het $n+1$-de cijfer een 1 is, en de andere cijfers 0. Aangezien $D^*=r^n-D$ geldt: $D^*=-D$. Het gevolg is dat we de traditionele optelling kunnen gebruiken, indien we 2-complement getallen met elkaar optellen. of deze getallen nu een teken of niet bevat, is een kwestie van interpretatie, de opteller is hiervan niet afhankelijk. Bovendien kunnen we relatief eenvoudig de negatie van een getal berekenen: we passen op elke bit een NOT operatie toe, en tellen daar vervolgens 1 bij op.
\paragraph{}
Een ander voordeel van de 2-complement voorstelling, is dat er slechts \'e\'en voorstelling van het getal 0 is. Indien we immers het negatief van 0 berekenen bekomen we:
\begin{equation}
D=0000_2\Rightarrow -D\equiv D^*=1111_2+0001_2=\underline{1}0000_2=0000_2
\end{equation}
Tot slot heeft een getal met $n$ bits een bereik van $-2^{n-1}$ tot $2^{n-1}-1$, wat exact \'e\'en element groter is dan de 1-complement tegenhanger\footnote{Dit is uiteraard de plaats die vrijkomt omdat de 2-complement voorstelling slechts \'e\'en voorstelling van 0 heeft.}.
\subsubsection{Betekenis van binaire getallen}
Bij wijze van voorbeeld om het verschil tussen de verschillende voorstellingen duidelijk te maken, zetten we de betekenis van elke $4$-bit getal naast elkaar in tabel \ref{tbl:binaryMeaningSigned}.
\begin{table}[hbt]
\centering
\begin{tabular}{r|rrrr}
Binair&Unsigned&Sign-Magnitude&1-Complement&2-Complement\\\hline
$0000$&$+0$&$+0$&$+0$&$+0$\\
$0001$&$+1$&$+1$&$+1$&$+1$\\
$0010$&$+2$&$+2$&$+2$&$+2$\\
$0011$&$+3$&$+3$&$+3$&$+3$\\
$0100$&$+4$&$+4$&$+4$&$+4$\\
$0101$&$+5$&$+5$&$+5$&$+5$\\
$0110$&$+6$&$+6$&$+6$&$+6$\\
$0111$&$+7$&$+7$&$+7$&$+7$\\
$1000$&$+8$&$-0$&$-7$&$-8$\\
$1001$&$+9$&$-1$&$-6$&$-7$\\
$1010$&$+10$&$-2$&$-5$&$-6$\\
$1011$&$+11$&$-3$&$-4$&$-5$\\
$1100$&$+12$&$-4$&$-3$&$-4$\\
$1101$&$+13$&$-5$&$-2$&$-3$\\
$1110$&$+14$&$-6$&$-1$&$-2$\\
$1111$&$+15$&$-7$&$-0$&$-1$
\end{tabular}
\caption{Betekenis van de binaire getallen.}
\label{tbl:binaryMeaningSigned}
\end{table}
\subsection{Optellen en aftrekken}
Nu we de belangrijkste voorstellingen van negatieve getallen besproken hebben, zullen we voor ieder van deze voorstellingen een schema uitwerken hoe we getallen kunnen optellen en aftrekken. Deze schema's worden weergegeven in figuur \ref{fig:addSubNegSchematic}.
\begin{figure}[hbt]
\centering
\subfigure[Sign-magnitude]{
\begin{tikzpicture}
\node[asmterminal] (E) at (0,0) {Einde};
\node[asmcommand] (C1) at (-3,1) {$m_r=m_2-m_1$\\$s_r=s_2$};
\node[asmcommand] (C2) at (-1,1) {$m_r=m_1-m_2$\\$s_r=s_1$};
\node[asmcommand] (C3) at (1,1) {$m_r=0$\\$s_r=0$};
\node[asmcommand] (C4) at (3,1) {$m_r=m_1+m_2$\\$s_r=s_1$};
\node[asmcondit] (I1) at (-2,2) {$m_1>m_2$};
\node[asmcondit] (I2) at (-0.5,3) {$m_1=m_2$};
\node[asmcondit] (I3) at (1.25,4) {$s_1=s_2$};
\node[asmcommand] (C5) at (0.25,5) {$s_2=s_2'$};
\node[asmterminal] (S1) at (0.25,6) {Begin\\aftrekking};
\node[asmterminal] (S2) at (2.25,6) {Begin\\optelling};
\draw (C1) |- (C4 |- 0,0.5) -- (C4);
\draw (C2) -- (C2 |- 0,0.5);
\draw (C3) -- (C3 |- 0,0.5);
\draw[->] (0,0.5) -- (E);
\draw[->] (I1.west) node[scale=0.6,anchor=south east]{nee} -| (C1.north);
\draw[->] (I1.east) node[scale=0.6,anchor=south west]{ja} -| (C2.north);
\draw[->] (I2.west) node[scale=0.6,anchor=south east]{nee} -| (I1.north);
\draw[->] (I2.east) node[scale=0.6,anchor=south west]{ja} -| (C3.north);
\draw[->] (I3.west) node[scale=0.6,anchor=south east]{nee} -| (I2.north);
\draw[->] (I3.east) node[scale=0.6,anchor=south west]{ja} -| (C4.north);
\draw[->] (C5.south) |- (I3.north |- 0,4.65) -- (I3.north);
\draw[->] (S1.south) -- (C5.north);
\draw (S2.south) |- (I3.north |- 0,4.65);
\end{tikzpicture}
\label{fig:addSubNegSchematicSignedMagnitude}}
\subfigure[1-complement]{
\begin{tikzpicture}
\node[asmterminal] (E) at (0,0) {Einde};
\node[asmcommand] (C1) at (0,1) {Carry aanpassen};
\node[asmcommand] (C2) at (0,2) {$B_r=B_1+B_2$};
\node[asmcommand] (C3) at (-1,3) {$B_2=B_2'$};
\node[asmterminal] (S1) at (-1,4) {Begin\\aftrekking};
\node[asmterminal] (S2) at (1,4) {Begin\\optelling};
\draw[->] (C1.south) -- (E.north);
\draw[->] (C2.south) -- (C1.north);
\draw[->] (C3.south) |- (0,2.5) -- (C2.north);
\draw (S2.south) |- (0,2.5);
\draw[->] (S1.south) -- (C3.north);
\end{tikzpicture}
\label{fig:addSubNegSchematicOneComplement}}
\subfigure[2-complement]{
\begin{tikzpicture}
\node[asmterminal] (E) at (0,0) {Einde};
\node[asmcommand] (C1) at (-1,1) {$B_r=B_1+B_2'+1$};
\node[asmcommand] (C2) at (1,1) {$B_r=B_1+B_2$};
\node[asmterminal] (S1) at (-1,2) {Begin\\aftrekking};
\node[asmterminal] (S2) at (1,2) {Begin\\optelling};
\draw[->] (C1.south) |- (0,0.5) -- (E.north);
\draw (C2.south) |- (0,0.5);
\draw[->] (S1.south) -- (C1.north);
\draw[->] (S2.south) -- (C2.north);
\end{tikzpicture}
\label{fig:addSubNegSchematicTwoComplement}}
\caption{Optelling en aftrekking van gehele getallen.}
\label{fig:addSubNegSchematic}
\end{figure}
\subsubsection{Signed-magnitude voorstelling}
Bij een signed-magnitude voorstelling moeten eerst heel wat testen uitgevoerd worden alvorens we getallen kunnen optellen of aftrekken. In eerste instantie moeten we bij het aftrekken van twee getallen het teken van de tweede operand veranderen, om daarna de bewerking als een optelling verder te verwerken. Bij deze optelling moeten we eerst testen of beide getallen hetzelfde teken hebben. Indien dit niet het geval is, bepaald het getal met de grootste magnitude $m$ het teken van het getal, en moeten de magnitudes van elkaar afgetrokken worden, in het andere geval is er sprake van een gewone optelling. Al deze testen zorgen voor complexe hardware die bovendien ook nog traag werkt, dit maakt signed-magnitude tot een weinig populaire voorstelling. Figuur \ref{fig:addSubNegSchematicSignedMagnitude} geeft de werkwijze schematisch weer.
\subsubsection{1-complement voorstelling}
Optellen en aftrekken met de 1-complement voorstelling is heel wat eenvoudiger. Bij een aftrekking bereken we ook eerst de negatie van de tweede operand, wat een NOT operatie is. En vervolgens kunnen we de twee getallen eenvoudig optellen met onze eerder ge\"implementeerde opteller. Deze optelling volstaat soms echter niet, met het volgende voorbeelden kunnen we illustreren wat er kan fout lopen:
\begin{equation}
\begin{array}{rl|rl|rl}
\begin{array}{rr}
&0101\\
+&1010\\\hline
\textcolor{gray}{0}&1111
\end{array}
&
\begin{array}{rr}
&+5\\
+&-5\\\hline
&-0
\end{array}
&
\begin{array}{rr}
&0110\\
+&1101\\\hline
\textcolor{gray}{1}&0011
\end{array}
&
\begin{array}{rr}
&+6\\
+&-2\\\hline
&3
\end{array}
&
\begin{array}{rr}
&1011\\
+&0011\\\hline
\textcolor{gray}{0}&1110
\end{array}
&
\begin{array}{rr}
&-4\\
+&+3\\\hline
&1
\end{array}
\end{array}
\label{eqn:oneComplementFaultExample}
\end{equation}
Indien we twee getallen optellen met een tegengesteld teken, zodat de uitkomst positief is, zien we dat het resultaat altijd 1 lager uitkomt dan het juiste resultaat. Dit komt omdat de 1-complement voorstelling twee representaties heeft voor 0. Zoals we zien in het eerste voorbeeld, komen we altijd -0 uit bij het optellen van twee tegengestelde getallen. Bijgevolg zal een resultaat van 1 uitkomen op de voorstelling die 1 hoger is dan $-0$: $+0$. Indien de carry $c_n$ dus 1 is, moeten we nog 1 optellen bij het resultaat. Dit kunnen we toepassen op de voorbeelden in vergelijking (\ref{eqn:oneComplementFaultExample}):
\begin{equation}
\begin{array}{rl|rl|rl}
\begin{array}{rr}
&0101\\
+&1010\\\hline
\underline{0}&1111\\
+&000\underline{0}\\\hline
&1110
\end{array}
&
\begin{array}{rr}
&+5\\
+&-5\\\hline
&-0\\
+&+0\\\hline
&-0
\end{array}
&
\begin{array}{rr}
&0110\\
+&1101\\\hline
\underline{1}&0011\\
+&000\underline{1}\\\hline
&00100
\end{array}
&
\begin{array}{rr}
&+6\\
+&-2\\\hline
&+3\\
+&+1\\\hline
&+4
\end{array}
&
\begin{array}{rr}
&1011\\
+&0011\\\hline
\underline{0}&1110\\
+&000\underline{0}\\\hline
&1110
\end{array}
&
\begin{array}{rr}
&-4\\
+&+3\\\hline
&-1\\
+&+0\\\hline
&-1
\end{array}
\end{array}
\label{eqn:oneComplementCorrectExample}
\end{equation}
Het grote nadeel bij deze implementatie is dat we de optelling-aftrekking dus in twee tijden moeten uitvoeren: eerst tellen we ze op en bepalen we de hoogste carry $c_n$, vervolgens tellen we deze carry nog eens bij het resultaat op. Dit veroorzaakt dus een verdubbeling van de vertraging. Figuur \ref{fig:addSubNegSchematicOneComplement} geeft het hele proces schematisch weer.
\subsubsection{2-complement voorstelling}
De 2-complement voorstelling is de beste voor de implementatie van een opteller-aftrekker. Immers kunnen we een optelling eenvoudigweg uitvoeren zoals we die reeds hebben beschreven in subsectie \ref{ss:add}. Bij een optelling dienen we dus eenvoudigweg de twee operatoren optellen: $B_r=B_1+B_2$. Bij een aftrekking moeten we dan eenvoudigweg de negatie van de tweede operand berekenen. De berekening wordt in dat geval $B_r=B_1+B_2'+1$. Deze optelling lijkt misschien tot hetzelfde probleem te leiden als bij de 1-complement voorstelling. Maar we kunnen eenvoudigweg de carry van de laagste bit $c_0$ op 1 zetten. We kunnen vervolgens $B_2'$ berekenen met behulp van XOR poorten. Een XOR poort is dan ook een geprogrammeerde NOT poort. Indien \'e\'en van de ingangen van de XOR poorten 1 is, zal de andere de negatie zijn van de andere ingang. Indien de ingang 0 is, laat de XOR poort de andere ingang door. We kunnen dus een opteller-aftrekker realiseren zoals op figuur \ref{fig:adderSubTwoComplementImplement}.
\begin{figure}[hbt]
\centering
\subfigure[Interface]{
\begin{tikzpicture}[scale=0.75]
\def\sca{0.75};
\node[addsubtractor,scale=\sca] (AS) at (0,0) {adder-subtractor};
\draw[->] (AS.c) -- ++(-0.5,0) node[scale=0.75,anchor=east]{$c_{\mbox{\tiny{out}}}$};
\draw[<-] (AS.s) -- ++(0.5,0) node[scale=0.75,anchor=west]{$s$};
\draw[<-] (AS.X) -- ++(0,0.5) node[scale=0.75,anchor=south]{$X$};
\draw[<-] (AS.Y) -- ++(0,0.5) node[scale=0.75,anchor=south]{$Y$};
\draw[->] (AS.F) -- ++(0,-0.5) node[scale=0.75,anchor=north]{$F$};
\end{tikzpicture}
\label{fig:adderSubTwoComplementInterface}}
\subfigure[Mogelijke implementatie]{\begin{tikzpicture}[circuit logic US,scale=0.8]
\def\sca{0.8};
\foreach\x/\i in {0/0,1/1,2/2,4/n-3,5/n-2,6/n-1} {
  \fill[black!15] (-2*\x-0.85,-0.65) rectangle ++(1.7,2.85);
  \node[fulladder,scale=\sca] (FA\x) at (-2*\x,0){FA$_{\i}$};
  \node[xor gate,scale=\sca,rotate=-90] (X\x) at (FA\x.x |- 0,1.25) {};
  \draw[->] (FA\x.s) -- ++(0,-0.75) node[scale=0.75*\sca,anchor=north]{$f_{\i}$};
  \draw (FA\x.x) -- (X\x.output);
  \draw (X\x.input 2) -- (X\x.input 2 |- 0,2);
  \draw (FA\x.y) -- (FA\x.y |- 0,2.5) node[scale=0.75*\sca,anchor=south]{$x_{\i}$};
  \draw (X\x.input 1) -- (X\x.input 1 |- 0,2.5) node[scale=0.75*\sca,anchor=south]{$y_{\i}$};
}
\foreach\x/\i in {0,1,2,4,5} {
  \pdot{X\x.input 2 |- 0,2};
}
\draw (-6,0) node {$\ldots$};
\draw (-5.5,1.25) node {$\ldots$};
\draw (X4.input 2 |- 0,2) -- (-6.5,2);
\draw (X2.input 2 |- 0,2) -- (-5.5,2);
\draw (FA4.ci) -- (-6.5,0);
\draw (FA2.co) -- (-5.5,0);
\draw[->] (FA6.co) -- ++(-0.75,0) node[scale=0.75*\sca,anchor=east]{$c_n$};
\node[xor gate,scale=\sca,rotate=-90] (XO) at (-13,-1.375) {};
\draw (-11,0) |- (XO.input 1 |- 0,-0.75) -- (XO.input 1);
\pdot{-11,0};
\pdot{XO.input 2 |- 0,0};
\draw[->] (XO.output) -- ++(0,-0.25) node[scale=0.75*\sca,anchor=north]{overflow};
\draw (XO.input 2) -- (XO.input 2 |- 0,0);
\draw (X0.input 2 |- 0,2) -- ++(1.35,0) node[scale=0.75*\sca,anchor=west]{$s$};
\draw (1,2) |- (FA0.ci);
\pdot{1,2}
%\draw (X6.input 2 |- 0,2) -- (-13.5,2) node[scale=0.75*\sca,anchor=east]{$s$};
\foreach\x/\y in {0/1,1/2,4/5,5/6} {
  \draw (X\x.input 2 |- 0,2) -- (X\y.input 2 |- 0,2);
  \draw (FA\x.co) -- (FA\y.ci);
}
\end{tikzpicture}
\label{fig:adderSubTwoComplementImplement}}
\caption{Opteller-aftrekker voor 2-complement getallen.}
\label{fig:adderSubTwoComplement}
\end{figure}
Indien $s=0$ tellen we de twee getallen met elkaar op, anders trekken we de twee getallen van elkaar af. Door een XOR poort op de laatste twee carry-uitgangen $c_{n-1}$ en $c_n$ te plaatsen, kunnen we een controle op overflow inbouwen. Deze component wordt meestal samengevat tot een interface zoals op figuur \ref{fig:adderSubTwoComplementInterface}\footnote{Bemerk op de figuur dat variabele met hoofdletters een rij van in- of uitgangen voorstellen.}. Een schematische werkwijze van optellen en aftrekken met 2-complement voorstelling staat op figuur \ref{fig:addSubNegSchematicTwoComplement}. De 2-complement voorstelling is dan ook populair omdat deze weinig bewerkingen en testen vraagt om sommen en verschillen te berekenen. Bovendien is er weinig extra hardware nodig om een opteller om te bouwen.
\subsection{Arithmetic-Logic Unit (ALU)}
\begin{figure}[hbt]
\centering
\begin{tikzpicture}[circuit logic US]
\foreach \x/\i in {0/0,1/1,3/n-2,4/n-1} {
  \fill[black!15] (-2.5*\x-1.15,-0.7) rectangle ++(2.3,2.9);
  \node[fulladder] (FA\x) at (-2.5*\x,0) {FA$_{\i}$};
  \node[ale] (ALE\x) at (-2.5*\x,1.5) {ALE$_{\i}$};
  \draw (FA\x.x) -- (ALE\x.y);
  \draw (FA\x.y) -- (ALE\x.x);
  \draw (ALE\x.a) -- ++(0,1.5) node[scale=0.75,anchor=south]{$a_{\i}$};
  \draw (ALE\x.b) -- ++(0,1.5) node[scale=0.75,anchor=south]{$b_{\i}$};
  \draw (ALE\x.m) -- (ALE\x.m |- 0,3.1);
  \draw (ALE\x.ia) -- (ALE\x.ia |- 0,2.85);
  \draw (ALE\x.ib) -- (ALE\x.ib |- 0,2.6);
  \draw[->] (FA\x.s) -- ++(0,-0.75) node[scale=0.75,anchor=north]{$f_{\i}$};
}
\foreach \x/\y in {0/1,3/4} {
  \draw (FA\x.co) -- (FA\y.ci);
  \draw (ALE\x.ib |- 0,2.6) -- (ALE\y.ib |- 0,2.6);
  \draw (ALE\x.ia |- 0,2.85) -- (ALE\y.ia |- 0,2.85);
  \draw (ALE\x.m |- 0,3.1) -- (ALE\y.m |- 0,3.1);
}
\foreach \x in {0,1,3} {
  \pdot{ALE\x.ib |- 0,2.6}
  \pdot{ALE\x.ia |- 0,2.85}
  \pdot{ALE\x.m |- 0,3.1}
}
\draw (ALE3.ib |- 0,2.6) -- ++(1.25,0);
\draw (ALE3.ia |- 0,2.85) -- ++(1.25,0);
\draw (ALE3.m |- 0,3.1) -- ++(1.25,0);
\draw (FA3.ci) -- ++(1,0);
\draw (ALE1.ib |- 0,2.6) -- ++(-2.25,0);
\draw (ALE1.ia |- 0,2.85) -- ++(-2.25,0);
\draw (ALE1.m |- 0,3.1) -- ++(-2.25,0);
\draw (FA1.co) -- ++(-1,0);
\draw (ALE0.ib |- 0,2.6) -- (2.25,2.6) node[scale=0.75,anchor=west]{$i_1$};
\draw (ALE0.ia |- 0,2.85) -- (2.25,2.85) node[scale=0.75,anchor=west]{$i_0$};
\draw (ALE0.m |- 0,3.1) -- (2.25,3.1) node[scale=0.75,anchor=west]{$m$};
\node[xor gate,scale=0.75,rotate=-90] (XO) at (-11,-1.25) {};
\node[cig] (CIG) at (1.75,0) {CIG};
\pdot{CIG.ib |- 0,2.6}
\pdot{CIG.ia |- 0,2.85}
\pdot{CIG.m |- 0,3.1}
\draw (CIG.c) -- (FA0.ci);
\draw (CIG.ib) -- (CIG.ib |- 0,2.6);
\draw (CIG.ia) -- (CIG.ia |- 0,2.85);
\draw (CIG.m) -- (CIG.m |- 0,3.1);
\draw (FA4.co) -| (XO.input 2);
\draw (-8.75,0) |- (XO.input 1 |- 0,-0.75) -- (XO.input 1);
\draw[->] (XO.output) -- ++(0,-0.5) node[scale=0.75,anchor=north]{overflow};
\pdot{-8.75,0}
\pdot{FA4.co -| XO.input 2}
\draw[->] (FA4.co -| XO.input 2) -- ++(-0.5,0) node[scale=0.75,anchor=east]{$c_n$};
\draw (-5,0) node{$\ldots$};
\draw (-5,1.5) node{$\ldots$};
\end{tikzpicture}
\caption{Schematisch implementatie van een arithmetic-logic unit (ALU).}
\label{fig:aLUImplementatie}
\end{figure}
Hoewel veel processoren in staat zijn om sommen en verschillen te berekenen zal men meestal geen directe opteller of opteller-aftrekker vinden. Meestal gebruikt men hiervoor een \termen{Arithmetic-Logic Unit (ALU)}. Een ALU is een component die gebaseerd op een opteller allerlei instructies\footnote{Welke instructies is niet echt gespecificeerd. Er bestaan dan ook boeken over het ontwerpen van een goede ALU.} kan uitvoeren. In ons geval beschouwen we 4 rekenkundige (optellen, aftrekken, \termen{increment} en \termen{decrement}) en 4 logische bewerkingen (AND, OR, NOT en identiteit). We bouwen een ALU op ongeveer dezelfde manier zoals we een opteller-aftrekker bouwden uit een opteller. In plaats van XOR poorten gebruiken we een nieuw component: een \termen{Arithmetic-Logic Extender (ALE)}. Een ander component -- de CIG -- berekent welke ingang aan de carry $c_0$ moet worden gegeven. In het algemeen heeft een ALU dus een structuur zoals op figuur \ref{fig:aLUImplementatie} waarbij de ALE en CIG vrij ge\"implementeerd kunnen worden.
\subsubsection{Instructieset}
\label{sss:aLUInstructionSet}
Alvorens we in staat zijn een ALU te maken moeten we een instructieset defini\"eren. Onze instructieset heeft drie\footnote{Omdat we 8 opdrachten gedefinieerd hebben, hebben we een $\log_28=3$ bit instructiewoord nodig.} ingangssignalen die bepalen welke opdracht er moet worden uitgevoerd. Bovendien zijn de 8 opdrachten in te delen in 2 groepen van 4: aritmetisch en logisch. Daarom noemen we de eerste bit van het instructiewoord $m$ voor mode\footnote{0=logisch, 1=aritmetisch.}. Verder wijzen we dan elke opdracht toe aan een bepaald instructiewoord zoals in tabel \ref{tbl:aLUInstructionSet}.
\begin{table}[hbt]
\centering
\begin{tabular}{ccc|c|ccc|l}
%\multicolumn{3}{c|}{Instructie}&\multicolumn{2}{c}{Opdracht}\\
$m$&$i_1$&$i_0$&$F$&$X$&$Y$&$c_0$&\\\hline
0&0&0&$A'$&$A'$&0&0&NOT\\
0&0&1&$A\mbox{ AND }B$&$A\mbox{ AND }B$&0&0&AND\\
0&1&0&$A$&$A$&0&0&Identiteit\\
0&1&1&$A\mbox{ OR }B$&$A\mbox{ OR }B$&0&0&OR\\
1&0&0&$A-1$&$A$&-1&0&Decrement\\
1&0&1&$A+B$&$A$&$B$&0&Optelling\\
1&1&0&$A-B$&$A$&$B'$&1&Aftrekking\\
1&1&1&$A+1$&$A$&$0$&1&Increment\\
\end{tabular}
\caption{Instructieset van een typische arithmetic-logic unit (ALU).}
\label{tbl:aLUInstructionSet}
\end{table}
Gebaseerd op deze instructieset moeten we een ALE en CIG implementeren. De ALE implementeert een functie die $\left(A,B\right)$-waardes afbeeld op $\left(X,Y\right)$-waardes. Deze laatste waardes dienen als invoer  voor de optellers. De CIG ten slotte geeft gebaseerd op het instructiewoord een waarde voor $c_0$. Deze functies staan ook in de instructietabel.
\subsubsection{Synthese van de ALE en CIG}
\begin{figure}[hbt]
\centering
\subfigure[Karnaugh-kaarten]{
\begin{tikzpicture}
\kkaarte{0}{0}{$x_j$}{$m$/$i_1$/$i_0$/$a_j$/$b_j$}{1/1/0/0/0/0/0/1/0/0/1/1/0/1/1/1/0/0/1/1/0/0/1/1/0/0/1/1/0/0/1/1}
\kkaarte{6}{0}{$y_j$}{$m$/$i_1$/$i_0$/$a_j$/$b_j$}{0/0/0/0/0/0/0/0/0/0/0/0/0/0/0/0/1/1/1/1/0/1/0/1/1/0/1/0/0/0/0/0}
\kkaartc{12}{0}{$c_0$}{$m$/$i_1$/$i_0$}{0/0/0/0/0/0/1/1}
\end{tikzpicture}
\label{fig:aLECIGImplKarnaugh}}
\subfigure[Implementatie ALE]{
\begin{tikzpicture}[circuit logic US]
\def\dx{0.85};
\def\dy{1.25};
\def\sc{1};
\draw[dashed] (-0.5,-1.875) rectangle (5.75,2);
\node[and gate,scale=0.85*\sc,inputs={inverted,inverted,inverted,inverted},rotate=-90] (A0) at (\dx*0,0) {};
\node[and gate,scale=0.85*\sc,inputs={normal,normal,normal,inverted},rotate=-90] (A1) at (\dx*1,0) {};
\node[and gate,scale=\sc,inputs={normal,normal,normal},rotate=-90] (A2) at (\dx*2,0) {};
\node[and gate,scale=\sc,inputs={normal,normal},rotate=-90] (A3) at (\dx*3,0) {};
\node[and gate,scale=\sc,inputs={normal,normal},rotate=-90] (A4) at (\dx*4,0) {};
\node[or gate,scale=0.65*\sc,inputs={normal,normal,normal,normal,normal},rotate=-90] (O0) at (\dx*2,\dy*-1) {};
\draw (O0.input 5) -- ++(0,0.3) -| (A0.output);
\draw (O0.input 4) -- ++(0,0.435) -| (A1.output);
\draw (O0.input 3) -- (A2.output);
\draw (O0.input 2) -- ++(0,0.435) -| (A3.output);
\draw (O0.input 1) -- ++(0,0.3) -| (A4.output);
\draw (O0.output) -- (O0.output |- 0,-2) node[scale=0.758,anchor=north] {$x_j$};
\node[and gate,scale=\sc,inputs={inverted,inverted,normal},rotate=-90] (A5) at (\dx*5,0) {};
\node[and gate,scale=\sc,inputs={normal,inverted,normal},rotate=-90] (A6) at (\dx*6,0) {};
\node[or gate,scale=\sc,inputs={normal,normal},rotate=-90] (O1) at (\dx*5.5,\dy*-1) {};
\draw (O1.input 2) -- ++(0,0.435) -| (A5.output);
\draw (O1.input 1) -- ++(0,0.435) -| (A6.output);
\draw (O1.output) -- (O1.output |- 0,-2) node[scale=0.758,anchor=north] {$y_j$};
\foreach\a/\i/\x/\t in {0/4/4/$m$,0/3/3/$i_1$,0/2/2/$i_0$,1/1/0/$b_j$,0/1/1/$a_j$} {
  \draw (A\a.input \i) -- (A\a.input \i |- 0,0.25*\x+0.75) -- (6,0.25*\x+0.75) node[scale=0.75,anchor=west]{\t};
}
\foreach\a/\i/\x in {1/4/4,1/3/3,1/2/2,2/3/2,2/2/1,2/1/0,3/2/3,3/1/1,4/2/4,4/1/1,5/3/4,5/2/2,5/1/0,6/3/4,6/2/3,6/1/0} {
  \draw (A\a.input \i) -- (A\a.input \i |- 0,0.25*\x+0.75);
  \pdot{A\a.input \i |- 0,0.25*\x+0.75};
}
\end{tikzpicture}
\label{fig:aLEImpl}}
\subfigure[Implementatie CIG]{
\begin{tikzpicture}[circuit logic US]
\def\dx{0.85};
\def\dy{1.25};
\def\sc{1};
\draw[dashed] (-1.25,-0.5) rectangle (1.25,0.5);
\node[and gate,scale=\sc,rotate=180] (A) at (-0.5,-0.125) {};
\draw (A.output) -| (-1.125,0) -- (-1.25,0);
\draw (A.input 2) -| ++(0.125,0.375) -| (-0.5,0.5);
\draw (A.input 1) -| (0.5,0.5);
\foreach\x/\t in {0/$m$,1/$i_0$,2/$i_1$} {
  \draw (0.5*\x-0.5,0.5) -- ++(0,0.5) node[scale=0.75,anchor=south]{\t};
}
\draw (-1.25,0) -- ++(-0.5,0) node[scale=0.75,anchor=east]{$c_0$};
\end{tikzpicture}
\label{fig:cIGImpl}}
\caption{Synthese van de ALE en CIG}
\end{figure}
Eenmaal we de instructieset hebben, dienen we enkel nog de ALE en CIG te implementeren. Een ALE is een component met als ingangen het instructiewoord en van elke operand een bit. In ons geval wordt dit dus $\left(m,i_1,i_0,a_j,b_j\right)$. Als uitgangen hebben we de ingangen van \'e\'en van de optellers $\left(x_j,y_j\right)$. We zijn dus in staat om een waarheidstabel en Karnaugh-kaart op te stellen van de ALE. Zoals op figuur \ref{fig:aLECIGImplKarnaugh}. Vervolgens is het enkel een kwestie van implementatie. Op basis van de Karnaugh-kaaren hebben we op figuur \ref{fig:aLEImpl} een AND-OR implementatie gesynthetiseerd.
\paragraph{}
De CIG is enkel afhankelijk van het instructiewoord. Met de instructieset op tabel \ref{tbl:aLUInstructionSet} is dit $\left(m,i_1,i_0\right)$. Een CIG heeft slechts \'e\'en uitgang: carry $c_0$. Op basis van deze instructieset staat op figuur \ref{fig:aLECIGImplKarnaugh} de bijbehorende Karnaugh-kaart. Deze blijkt een eenvoudige AND-operatie te zijn zoals op figuur \ref{fig:cIGImpl}.
\subsection{Vermenigvuldigen}
\begin{figure}[htb]
\centering
\subfigure[$1\times 1$-bit]{\begin{tikzpicture}[circuit logic US]
\node[anchor=east,scale=0.75] (C) at (0,0) {$\begin{array}{lc}
&a_0\\
\times&b_0\\\hline
&a_0b_0
\end{array}$};
\node[and gate,rotate=-90] (A) at (0.5,0) {};
\draw (A.input 1) -- ++(0,0.25) node[scale=0.75,anchor=south]{$a_0$};
\draw (A.input 2) -- ++(0,0.25) node[scale=0.75,anchor=south]{$b_0$};
\draw (A.output) -- ++(0,-0.25) node[scale=0.75,anchor=north]{$a_0\times b_0$};
\end{tikzpicture}}
\subfigure[$3\times 2$-bit]{\begin{tikzpicture}[circuit logic US]
\node[anchor=east,scale=0.75] (C) at (0,1.25) {$\begin{array}{lccccc}
&&&a_2&a_1&a_0\\
\times&&&&b_1&b_0\\\hline
&&&a_2b_0&a_1b_0&a_0b_0\\
+&&a_2b_1&a_1b_1&a_0b_1&\\\hline
&c_4&c_3&c_2&c_1&c_0
\end{array}$};
\begin{scope}[xshift=2 cm]
\node[halfadder,scale=0.75] (HA2) at (0,0) {HA$_2$};
\node[fulladder,scale=0.75] (FA1) at (1.5,0) {FA$_1$};
\node[halfadder,scale=0.75] (HA0) at (3,0) {HA$_0$};
\foreach\x in {0,1} {
  \coordinate (a\x) at (-1.5,3-1.5*\x);
  \draw (a\x) node[anchor=east,scale=0.75] {$b_\x$};
}
\foreach\x in {0,1,2,3,4} {
  \coordinate (c\x) at (4.5-1.5*\x,-0.75);
  \draw (c\x) node[anchor=north,scale=0.75] {$c_\x$};
}
\node[and gate,scale=0.75,rotate=-90] (A00) at (c0 |- 0,2.375) {};
\foreach \ad/\a in {HA0/1,FA1/2,HA2/3} {
  \node[and gate,scale=0.75,rotate=-90] (A1\a) at (\ad.y |- 0,1) {};
  \draw (\ad.y) -- (A1\a.output);
}
\foreach \a in {2,3} {
  \draw (A1\a.input 1) -- (a1 -| A1\a.input 1);
  \pdot{a1 -| A1\a.input 1};
}
\foreach \ad/\a in {HA0/1,FA1/2} {
  \node[and gate,scale=0.75,rotate=-90] (A0\a) at (\ad.x |- 0,2.375) {};
  \draw (\ad.x) -- (A0\a.output);
  \draw (A0\a.input 1) -- (A0\a.input 1 |- a0);
  \pdot{A0\a.input 1 |- a0};
  \coordinate (ai\a) at (A0\a.input 2 |- 0,2.825);
  \pdot{ai\a};
}
\coordinate (ai0) at (A00.input 2 |- 0,2.825);
\pdot{ai0};
\foreach\x/\y in {0/1,1/2,2/3} {
  \draw (ai\x) -| ++(-0.75,-1) -| (A1\y.input 2);
}
\foreach\x in {0,1,2} {
  \coordinate (b\x) at (A0\x.input 2 |- 0,3.25);
  \draw (A0\x.input 2) -- (b\x) node[anchor=south,scale=0.75] {$a_\x$};
}
\draw (a0) -| (A00.input 1);
\draw (a1) -| (A11.input 1);
\draw (A00.output) -- (c0);
\draw (c1) -- (HA0.s);
\draw (c2) -- (FA1.s);
\draw (c3) -- (HA2.s);
\draw (c4) |- (HA2.co);
\draw (HA0.co) -- (FA1.ci);
\draw (FA1.co) -| ++(-0.2,0.6) -| (HA2.x);
\end{scope}
\end{tikzpicture}}
\subfigure[$n\times m$-bit]{\begin{tikzpicture}[circuit logic US]
\def\dh{2};
%\node[nadder,anchor=sd] (NA4) at (0,0) {$n$-opteller$_{m-1}$};
\node[nadder,anchor=sd] (NA3) at (0,\dh) {$n$-opteller$_{m-1}$};
\node[white,nadder,anchor=sd] (NA2) at (NA3.yc |- 0,2*\dh) {$n$-opteller$_{m-3}$};
\node (NA2x) at (NA2) {$\ldots$};
\node[nadder,anchor=sd] (NA1) at (NA2.yc |- 0,3*\dh) {$n$-opteller$_2$};
\node[nadder,anchor=sd] (NA0) at (NA1.yc |- 0,4*\dh) {$n$-opteller$_1$};
\draw (NA3.sa) -- (NA3.sa |- 0,\dh-0.5) node[anchor=north,scale=0.75]{$s_{m-1}$};
\draw (NA3.sb) -- (NA3.sb |- 0,\dh-0.5) node[anchor=north,scale=0.75]{$s_m$};
\draw (NA3.sd) -- (NA3.sd |- 0,\dh-0.5) node[anchor=north,scale=0.75]{$s_{m+n-2}$};
\draw (NA3.c) -| ++ (-0.25,-0.5) -| (NA3.xd |- 0,\dh-0.5) node[anchor=north,scale=0.75]{$s_{m+n-1}$};
\foreach \xa/\xb in {0/1,1/2,2/3} {
  \draw (NA\xa.sb) -- (NA\xb.ya);
  \draw (NA\xa.sc) -- (NA\xb.yb);
  \draw (NA\xa.sd) -- (NA\xb.yc);
  \draw (NA\xa.c) -| ++ (-0.25,-0.5) -| (NA\xb.yd);
}
\foreach \xa/\i in {0/1,1/2} {
  %\draw (NA\xa.yd) -- ++(0,0.25) node[anchor=south,scale=0.75]{$0$};
  \draw (NA\xa.sa) |- ++(0.25,-0.5*\dh+0.125) |- ++(-0.25,-\dh+0.5) |- (NA\xa.sa |- 0,\dh-0.5) node[anchor=north,scale=0.75]{$s_{\i}$};
}
\foreach \xa/\t in {0/1,1/2,3/m-2} {
  \node[and gate,rotate=-90,scale=0.5] (Aa\xa) at (NA\xa.xa |- 0,4*\dh+1.125-\dh*\xa) {};
  \node[and gate,rotate=-90,scale=0.5] (Ab\xa) at (NA\xa.xb |- 0,4*\dh+1.125-\dh*\xa) {};
  \node[scale=0.75] (Ac\xa) at (NA\xa.xc |- 0,4*\dh+1.125-\dh*\xa) {$\ldots$};
  \node[and gate,rotate=-90,scale=0.5] (Ad\xa) at (NA\xa.xd |- 0,4*\dh+1.125-\dh*\xa) {};
  \draw (Aa\xa.input 2) |- (0,4*\dh+1.375-\dh*\xa -| NA3.west) node[scale=0.75,anchor=east]{$b_{\t}$};
  \draw (Ab\xa.input 2) -- (Ab\xa.input 2 |- 0,4*\dh+1.375-\dh*\xa);
  \pdot{Ab\xa.input 2 |- 0,4*\dh+1.375-\dh*\xa};
  \draw (Ad\xa.input 2) -- (Ad\xa.input 2 |- 0,4*\dh+1.375-\dh*\xa);
  \pdot{Ad\xa.input 2 |- 0,4*\dh+1.375-\dh*\xa};
  \draw (NA\xa.xa) -- (Aa\xa.output);
  \draw (NA\xa.xb) -- (Ab\xa.output);
  \draw (NA\xa.xd) -- (Ad\xa.output);
  \draw (Aa\xa.input 1) -- ++(0,0.25) node[anchor=south,scale=0.75]{$a_0$};
  \draw (Ab\xa.input 1) -- ++(0,0.25) node[anchor=south,scale=0.75]{$a_1$};
  \draw (Ad\xa.input 1) -- ++(0,0.25) node[anchor=south,scale=0.75]{$a_{n-1}$};
}
\node[and gate,rotate=-90,scale=0.5] (Aya0) at (NA0.ya |- 0,4*\dh+1.375) {};
\node[and gate,rotate=-90,scale=0.5] (Ayb0) at (NA0.yb |- 0,4*\dh+1.375) {};
\node[scale=0.75] (Ayc0) at (NA0.yc |- 0,4*\dh+1.375) {$\ldots$};
\draw (Aya0.input 2) |- (0,4*\dh+1.75 -| NA3.west) node[scale=0.75,anchor=east]{$b_{0}$};
\draw (Ayb0.input 2) -- (Ayb0.input 2 |- 0,4*\dh+1.75);
\pdot{Ayb0.input 2 |- 0,4*\dh+1.75};
\draw (NA0.ya) -- (Aya0.output);
\draw (NA0.yb) -- (Ayb0.output);
\draw (NA0.yd) -- (NA0.yd |- Ayb0.output) node[anchor=south,scale=0.75]{$0$};
\draw (Aya0.input 1) -- ++(0,0.25) node[anchor=south,scale=0.75]{$a_1$};
\draw (Ayb0.input 1) -- ++(0,0.25) node[anchor=south,scale=0.75]{$a_2$};
\end{tikzpicture}}
\caption{Parallelle vermenigvuldigers.}
\label{fig:parallelMultipliers}
\end{figure}
In deze subsectie bouwen we een vermenigvuldiger gebaseerd op de manier hoe men met behulp van cijferen twee getallen vermenigvuldigt. We zullen hierbij enkel natuurlijke getallen beschouwen, later breiden we dit uit naar gehele getallen. Vergelijking (\ref{eqn:multiplyNormal}) toont de vermenigvuldiging van twee decimale getallen:
\begin{equation}
\begin{array}{lcccccc}
&&&1&4&2&5\\
\times&&&&3&6&5\\\hline
&&&7&1&2&5\\
&&8&5&5&0&\\
+&4&2&7&5&&\\\hline
&5&2&0&1&2&5
\end{array}
\label{eqn:multiplyNormal}
\end{equation}
We kunnen dus twee getallen vermenigvuldigen door het eerste getal telkens te vermenigvuldigen met \'e\'en van de cijfers van het onderste getal en na deze op de juiste manier te hebben uitgelijnd op te tellen. Deze methode werkt ook binaire zoals blijkt uit vergelijking (\ref{eqn:multiplyBinary}):
\begin{equation}
\begin{array}{lcccccc}
&&&1&0&1&1\\
\times&&&&1&0&1\\\hline
&&&1&0&1&1\\
&&0&0&0&0&\\
+&1&0&1&1&&\\\hline
&1&1&0&1&1&1
\end{array}
\label{eqn:multiplyBinary}
\end{equation}
Een groot voordeel van vermenigvuldigingen met binaire getallen is dat een cijfer slechts 1 of 0 kan zijn. Indien het 0 is tellen we niets op bij het resultaat, anders tellen we het eerste getal op, die voldoende naar links is opgeschoven. Deze realisatie noemt men een \termen{parallelle vermenigvuldiger}. Figuur \ref{fig:parallelMultipliers} toont enkele voorbeelden van parallelle vermenigvuldigers. Zo zien we . Naast parallelle vermenigvuldigers bestaan er ook andere implementaties die sneller of werken en goedkoper te realiseren zijn. De kostprijs om een getal met $m$ bits te vermenigvuldigen met een getal van $n$ bits is immers $\bigoh{nm}$. Met een sequenti\"ele schakeling kunnen we de kosten sterk reduceren en zelfs de vermenigvuldiging versnellen. Vermenigvuldigen met een macht van 2 kunnen we bovendien berekenen met schuifoperators die in subsectie \ref{ss:shiftoperators} aan bod komen.
\subsubsection{2-complement vermenigvuldiger}
Het uitbreiden van deze vermenigvuldiger naar gehele getallen met 2-complement voorstelling is geen sinecure. In de praktijk zal men dan ook eerst de getallen omzetten naar sign-magnitude voorstelling alvorens we de twee getallen vermenigvuldigen. In dat geval kunnen we eenvoudigweg de waardes van de twee getallen vermenigvuldigen en passen we een XOR operatie toe op de tekenbit van de getallen. Het resultaat zetten we vervolgens weer om naar 2-complement voorstelling. Er bestaat uiteraard wel een vermenigvuldiger voor een 2-complement voorstelling. Deze zullen we enkel op hoog niveau beschrijven door middel van een voorbeeld. We rekenen hierbij $-2\times -3$ uit op 4-bit voorstelling en het resultaat stellen we voor op een 8-bit voorstellingen zoals in onderstaande vergelijking:
\begin{equation}
\begin{array}{l|r}
 \begin{array}{lr}
&-2\\
\times&-3\\\hline
&+6
\end{array}&\begin{array}{lr}
&1110\\
\times&1101\\\hline
&00000110
\end{array}
\end{array}
\end{equation}
De oplossing bestaat eruit om $-2$ te zien in zijn binaire vorm als $-8+4+2=-2$. Elk van de bits dient afzonderlijk vermenigvuldigt te worden, en vervolgens opgeteld te worden zoals in onderstaande vermenigvuldiging:
\begin{equation}
\begin{array}{lcr}
\begin{array}{lrcr}
&2&\times&-3\\
+&4&\times&-3\\
+&-8&\times&-3\\\hline
&-2&\times&-3
\end{array}&\begin{array}{c}
\Rightarrow\\
\Rightarrow\\
\Rightarrow\\
\Rightarrow
\end{array}&\begin{array}{lr}
&1111\ 1010\\
+&1111\ 0100\\
+&0001\ 1000\\\hline
&0000\ 0110
\end{array}
\end{array}
\end{equation}
\subsection{Andere courante bewerkingen}
\subsubsection{Delen}
Analoog aan vermenigvuldigen kunnen we ook een \termen{deling} in een binaire voorstelling gebaseerd op cijferen. We geven een voorbeeld van een binaire deling maar vermelden geen details over het realiseren van een schakeling:
\begin{equation}
\begin{array}{r|l}
1011\ 1010&1110\\\hline
\underline{111\ 0}\textcolor{white}{000}&1101\\
100\ 1010&\\
\underline{11\ 10}\textcolor{white}{00}&\\
1\ 0010&\\
\underline{0\ 000}\textcolor{white}{0}&\\
1\ 0010&\\
\underline{11\ 10}&\\
100&
\end{array}
\end{equation}
Door dus herhaaldelijk het resterende deeltal te vergelijken met een verschuifde deler en indien deze groter is, deze af te trekken, kunnen we de rest en de het quoti\"ent berekenen. Het aantal cycli is dus ook tevens het aantal bits van het quoti\"ent. Deze manier van implementatie is dan ook de meest populaire, en wordt vaak gebruikt. Indien we delen door een macht van 2, kunnen we het getal berekenen met een schuifoperatie (zie subsectie \ref{ss:shiftoperators}).
\subsubsection{Modulo rekenen}
Een andere populaire berekening is \termen{modulo} (ook wel bekend als \termen{$\mod$}). Modulo-rekening is sterk gerelateerd aan, maar niet equivalent aan \termen{remainder}-rekenen (ofwel \termen{$\rem$}). We defini\"eren de remainder als:
\begin{equation}
a\rem b\equiv a-\left\lfloor a/b\right\rfloor\times b
\end{equation}
De modulo-operatie daarentegen houdt rekening met het teken van zowel $a$ als $b$:
\begin{equation}
a\mod b\equiv \left\{\begin{array}{lcl}
a\rem b&\ifun&a\cdot b>0\\
\left(a\rem b\right)+b&\ifun&a\cdot b<0
\end{array}\right.
\end{equation}
Net als bij andere operaties, heeft ook de modulo-operatie een speciaal geval bij een macht van 2: in dat geval kunnen we een AND operatie uitvoeren op het getal en de macht gedecrementeerd. Bijvoorbeeld:
\begin{equation}
1653\mod 16=1653\mbox{ AND }15=5
\end{equation}
\subsection{Vaste komma getallen}
We zullen het getal verder uitbreiden naar getallen met vaste komma in deze subsectie en getallen met vlottende komma in subsectie \ref{ss:floatingPoints}. Ten slotte zullen we nog enkele andere voorstellingen van gegevens bekijken in subsectie \ref{ss:otherRepresentations}.
\paragraph{}
\termen{Vaste komma voorstelling} (ofwel \termen{fixed point}) lost het probleem van het zetten van een komma op door gewoon een bit af te spreken waarna men een komma plaatst, deze positie staat vast. Bijgevolg is er geen voorstelling van de komma zelf nodig. Indien we dus bijvoorbeeld een 8-bit getal beschouwen, kunnen we de eerste vier bit voorstellen als het \termen{geheel deel}, de overige vier bit behoren dan tot het \termen{fractionele gedeelte}. Dit formaliseren we met de notatie \termen{$\fix\left<i,f\right>$} waarbij $i$ het aantal bits voorstellen behorend tot het gehele deel, en $f$ het aantal bits tot het fractionele gedeelte. Zo stelt $1110010$ in $\fix\left<4,3\right>$, $14.25$ voor. De eerste vier bits stellen immers $14$ voor, de laatste drie stellen $2$ voor, deze delen we vervolgens door $2^f$.
\subsubsection{Aantal bits voor fouteloze voorstelling}
In de vorige secties hebben we al aandacht besteed aan het aantal bits die we moeten reserveren om de resultaten van bewerkingen zonder verlies te kunnen blijven voorstellen. Bij een optelling van 2 vaste komma getallen moet het resultaat voorgesteld worden op een vaste komma voorstelling waarbij we het geheel en fractioneelgedeelte voorstellen met het maximum aantal bits van de operanden. Bovendien dienen we \'e\'en bit toe te voegen aan het geheel gedeelte, ofwel formeler:
\begin{equation}
\fix\left<i_1,f_1\right>+\fix\left<i_2,f_2\right>=\fix\left<\max\left(i_1,i_2\right)+1,\max\left(f_1,f_2\right)\right>
\end{equation}
Wat we verder kunnen veralgemenen tot:
\begin{equation}
\displaystyle\sum_{k=1}^{n}{\fix\left<i_k,f_k\right>}=\fix\left<\left\lceil\log_2n\right\rceil+\max_{k=1}^{n}{i_k},\max_{k=1}^{n}{f_k}\right>
\end{equation}
Bij vermenigvuldigingen moeten we het aantal bits optellen van zowel het gehele als fractionele gedeelte:
\begin{equation}
\fix\left<i_1,f_1\right>\times\fix\left<i_2,f_2\right>=\fix\left<i_1+i_2,f_1+f_2\right>
\end{equation}
Of algemener:
\begin{equation}
\displaystyle\prod_{k=1}^{n}{\fix\left<i_k,f_k\right>}=\fix\left<\displaystyle\sum_{k=1}^{n}{i_k},\displaystyle\sum_{k=1}^{n}{f_k}\right>
\end{equation}
\subsection{Vlottende komma getallen}
\label{ss:floatingPoints}
Indien we echter niet weten tot welke grootorde het getal in kwestie zal behoren, in dat geval zouden we vaste getallen met een groot aantal bits moeten gebruiken, waarbij bovendien in de meeste gevallen slechts een kleine hoeveelheid bits nuttig zijn. In deze gevallen gebruiken we \termen{vlottende komma voorstelling} (ook wel \termen{zwevendekommagetal}, \termen{drijvendekommagetal} of \termen{floating-point number} genoemd). Vlottende komma getallen worden voorgesteld door een sequentie van bits die in drie groepen worden onderverdeeld:
\begin{itemize}
 \item \termen{sign-bit $s$}: 1-bit die het teken van het getal bepaald. De vlottende kommavoorstelling is dus vergelijkbaar met de signed-magnitude voorstelling.
 \item \termen{exponent $E$}: een getal die bepaald met hoeveel plaatsen het getal moet worden opgeschoven. Dit deel wordt voorgesteld in `\termen{excess}'-formaat.
 \item \termen{mantisse $M$}: de getal onafhankelijk van het schuiven van plaatsen. De voorstelling van de mantisse zelf is $\fix\left<1,m-1\right>$.
\end{itemize}
Het getal die we hierbij willen voorstellen is gelijk aan:
\begin{equation}
N=\left(-1\right)^s\times\mbox{mantisse}\times r^{\mbox{\tiny{exponent}}}
\end{equation}
met radix $r$. Net als bij een vaste komma getal geven noteren we een familie van vlottende komma getallen als \termen{$\float\left<m,e\right>$} met $m$ het aantal bits van de mantisse en $e$ het aantal bits van de exponent. Verder onderscheiden we nog twee families van vlottende komma voorstellingen:
\begin{itemize}
 \item \termen{Genormaliseerde vlottende komma voorstelling}: hierbij leggen we extra constraint op betreft de mantisse: $1\leq\mbox{mantisse}<r$. Omdat in dat geval de eerste bit dus altijd gelijk is aan 1, behoren enkel de bits na de komma tot de voorstelling. In dat geval denken we er de 1 dus bij.
 \item \termen{Niet-genormaliseerde vlottende komma voorstelling}: hierbij is deze bovengenoemde constraint dus niet van toepassing. Niet-genormaliseerde vlottende komma getallen kunnen meer getallen voorstellen dan hun tegenhanger, maar hebben een groot nadeel: men kan een getal op verschillende manieren voorstellen: eenzelfde getal kan op verschillende manieren voorgesteld worden.
\end{itemize}
\subsubsection{Underflow en overflow}
Naast overflow introduceert de vlottende kommavoorstelling nog een bijkomend fenomeen: ``\termen{underflow}''. Underflow treed op op het moment dat een getal niet meer voorgesteld kan worden omdat de absolute waarde te klein geworden is. Figuur \ref{fig:floatingPointUnderflow} geeft dit fenomeen schematisch weer van een \texttt{Single} (zie volgende paragraaf).
\begin{figure}[hbt]
\centering
\begin{tikzpicture}[decoration={brace}]
\begin{scope}
\fill[black!35] (-2.5,-0.1) rectangle (-0.5,0.1);
\fill[black!35] (-0.05,-0.1) rectangle (0.05,0.1);
\fill[black!35] (0.5,-0.1) rectangle (2.5,0.1);
\draw[thick,->] (-2.85,0) -- (2.85,0);
\draw[thick] (-2.5,-0.1) -- ++(0,0.25);
\draw[thick] (-0.5,-0.1) -- ++(0,0.25);
\draw[thick] (0.5,-0.1) -- ++(0,0.25);
\draw[thick] (2.5,-0.1) -- ++(0,0.25);
\fill (0,0) circle (0.05 cm);
\draw [decorate] (-0.05,-0.15) -- ++(-0.45,0);
\draw [decorate] (0.5,-0.15) -- ++(-0.45,0);
\draw (-2.5,0.1) node[scale=0.55,anchor=south]{$\approx-2^{128}$};
\draw (-0.5,0.1) node[scale=0.55,anchor=south]{$-2^{-149}$};
\draw (0.5,0.1) node[scale=0.55,anchor=south]{$2^{-149}$};
\draw (2.5,0.1) node[scale=0.55,anchor=south]{$\approx2^{128}$};
\draw (0,-0.25) node[scale=0.55,anchor=north]{underflow};
\draw (0,-0.5) node[scale=0.75,anchor=north]{Genormaliseerde vlottende komma};
\end{scope}
\begin{scope}[xshift=6cm]
\fill[black!35] (-2.5,-0.1) rectangle (-0.35,0.1);
\fill[black!35] (-0.05,-0.1) rectangle (0.05,0.1);
\fill[black!35] (0.35,-0.1) rectangle (2.5,0.1);
\draw[thick,->] (-2.85,0) -- (2.85,0);
\draw[thick] (-2.5,-0.1) -- ++(0,0.25);
\draw[thick] (-0.35,-0.1) -- ++(0,0.25);
\draw[thick] (0.35,-0.1) -- ++(0,0.25);
\draw[thick] (2.5,-0.1) -- ++(0,0.25);
\fill (0,0) circle (0.05 cm);
\draw [decorate] (0.35,-0.15) -- ++(-0.3,0);
\draw [decorate] (-0.05,-0.15) -- ++(-0.3,0);
\draw (-2.5,0.1) node[scale=0.55,anchor=south]{$\approx-2^{128}$};
\draw (-0.35,0.1) node[scale=0.55,anchor=south]{$-2^{-126}$};
\draw (0.35,0.1) node[scale=0.55,anchor=south]{$2^{-126}$};
\draw (2.5,0.1) node[scale=0.55,anchor=south]{$\approx2^{128}$};
\draw (0,-0.25) node[scale=0.55,anchor=north]{underflow};
\draw (0,-0.5) node[scale=0.75,anchor=north]{Niet-genormaliseerde vlottende komma};
\end{scope}
\end{tikzpicture}
\caption{Underflow van een vlottende komma voorstelling.}
\label{fig:floatingPointUnderflow}
\end{figure}
\subsubsection{IEEE-formaat voor vlottende komma getallen}
Nu we de vlottende komma voorstelling theoretisch beschreven hebben, zullen we het formaat gedefinieerd door IEEE bespreken. De floating point wordt beschreven in IEEE 754-1985\footnote{De opvolger van deze standaard is IEEE 754-2008\cite{5976968}. Hierbij worden ook halve en viervoudige precisie gedefinieerd.}\cite{30711}. Hierbij maakt met een onderscheid tussen vlottende komma getallen met \termen{enkele precisie} (32-bit en beter bekend onder de term \termen{\texttt{float}} of \termen{\texttt{Single}}) en \termen{dubbele precisie} (64-bit en beter bekend als \termen{\texttt{double}}). Tabel \ref{tbl:iEEEFloatingPointFormat} toont de verdeling van de beschikbare bits onder de mantisse en exponent.
\begin{table}[hbt]
\centering
\subtable[Bit-indeling]{\begin{tabular}{l|crrr}
Precisie&grootte&$e$&$m$&$B$\\\hline
\texttt{float}&32-bit&8&23&127\\
\texttt{double}&64-bit&11&52&1023\\
\end{tabular}
\label{tbl:iEEEFloatingPointFormat}}
\subtable[Getalvoorstelling]{\begin{tabular}{l|ccc}
&$E=0$&$0<E<2^e-1$&$E=2^e-1$\\\hline
$M=0$&$0$&$\left(-1\right)^s\times\underline{1}.M\times2^{E-B}$&$\left(-1\right)^s\times\infty$\\
$M\neq 0$&$\left(-1\right)^s\times\underline{0}.M\times2^{1-B}$&$\left(-1\right)^s\times\underline{1}.M\times2^{E-B}$&NaN\\
\end{tabular}
\label{tbl:iEEEFloatingNumberRepresentation}}
\caption{IEEE 754-1985 Floating Point.}
\label{tbl:iEEEFloatingPoint}
\end{table}
Elk van deze formaten heeft uiteraard radix $r=2$. IEEE 754-1985 vermeld ook hoe een vlottende komma moet worden uitgelezen. Tabel \ref{tbl:iEEEFloatingNumberRepresentation} geeft hierbij weer hoe we het getal moeten interpreteren afhankelijk van de waarde van de exponent $E$ en mantisse $M$. Bovendien wordt bij de IEEE 754 een extra variabele ingevoerd: \termen{excess-bias $B$}. Tot op heden is $B=2^{e-1}-1$, en is de range van de exponent dus van $-2^{e-1}+2$ tot $2^{e-1}-1$. Eventueel zou men later kunnen afwijken en bijvoorbeeld meer positieve exponenten dan negatieve kunnen toelaten. Verder definieert het formaat nog enkele speciale variabelen:
\begin{itemize}
 \item Nul (0): dit getal kan eigenlijk niet weergegeven worden met ons eerdere definitie van een genormaliseerde vlottende komma. De getalexperts bij Intel bedachten dan ook de regel dat indien $E=0$, we het getal in niet genormaliseerde getalvoorstelling zien.
 \item \termen{Negatief oneindig $-\infty$}/\termen{Positief oneindig $+\infty$}: Indien een operatie tot een overflow leid, zal het resultaat worden voorgesteld als oneindig. Oneindig gedraagt zich ongeveer hetzelfde als zijn wiskundig equivalent.
 \item \termen{Not a Number (NaN)}: In tegenstelling tot gehele getallen die bij bijvoorbeeld een deling door 0 een fout\footnote{En een interrupt bij de meeste processoren.} veroorzaken, zullen ongeldige bewerkingen zoals delen door 0, en het verschil van oneindig en oneindig resulteren in Not a Number.
\end{itemize}
In tabel \ref{tbl:iEEEFloatingNumberRepresentation} zien we ten slotte ook dat enkele bits onderlijnd worden. Dit zijn de zogenaamde \termen{verborgen bits}. Deze bits zijn geen onderdeel van de mantisse, maar dienen we er wel denkbeeldig aan toe te voegen om het getal te kunnen interpreteren.
\subsubsection{Rekenen met vlottende komma}
In deze subsubsectie bespreken we kort hoe we kunnen rekenen met getallen met vlottende komma. De details kan men vinden in \cite[\S4]{hyde2004write} maar enkel een korte introductie behoort tot dit vak.
\paragraph{Optelling}
Het optellen van twee vlottende komma getallen verloopt in drie stappen:
\begin{enumerate}
 \item Eerst denormaliseren we de twee getallen door de exponenten gelijk te maken. Deze stap is geen sinecure omdat we bij dit proces zoveel mogelijk informatie willen blijven behouden (Indien we dus enkel bij \'e\'en van de twee getallen de exponent aanpassen, lopen we kans dat zijn mantisse alle informatie verliest).
 \item Vervolgens tellen we de twee mantisses met elkaar op.
 \item Het resultaat dienen we vervolgens weer te normaliseren.
\end{enumerate}
IEEE 754 definieert het formaat van een vlottende komma, de wiskunde operaties zelf behoren hier echter niet bij. Intel heeft bijvoorbeeld gepatenteerde technologie waarbij het getal eerst wordt omgezet naar een equivalent met grotere mantisse om meer precisie te garanderen. In de praktijk komt het er dus op neer dat IEEE 754 niet specificeert wat het resultaat moet zijn na een wiskundige bewerking.
\paragraph{Vermenigvuldiging}
Vermenigvuldigen is bij vlottende komma simpeler dan optellen:
\begin{enumerate}
 \item We vermenigvuldigen de mantisse zoals we dit hebben gedaan bij vermenigvuldigen van natuurlijke getallen. We passen een XOR-operatie toe op de tekenbit van de twee getallen, en we tellen de exponenten met elkaar op\footnote{Optellen bij het excess-formaat verloopt anders, we dienen immers \'e\'enmaal de excess-bias van de som af te trekken.}.
 \item We normaliseren het resultaat. Dit betekent dat ofwel de exponent hetzelfde blijft, ofwel met \'e\'en moet worden opgehoogd.
\end{enumerate}
\subsection{Andere voorstellingen van gegevens}
\label{ss:otherRepresentations}
Naast de hierboven beschreven voorstellingen voor natuurlijke-, gehele- en kommagetallen. Bestaan er nog andere voorstelling. We stellen er in deze subsectie twee voor: \termen{Binary Coded Decimal (BCD)} is een alternatief formaat om natuurlijke getallen voor te stellen. Dit kan eventueel uitgebreid worden tot gehele- en kommagetallen. De \termen{American Standard Code for Information Interchange (ASCII)} is een formaat die hoofdzakelijk bedoelt is om tekst op te slaan.
\subsubsection{Binary Coded Decimal (BCD)}
In de vorige subsecties hebben we geen aandacht besteed aan het omzetten van een getal naar tekst. Deze omzetting is vrij arbeidsintensief en kost dus ook veel hardware. Bovendien stuiten we op nog een probleem: heel wat decimale kommagetallen kunnen niet voorgesteld worden in het binair getalstelsel. Zo bestaat er geen enkel binaire vorm die $0.3$ voorstelt. In dat geval moeten we het getal zo goed mogelijk benaderen\footnote{Het is echter niet omdat het binair stelsel er niet in slaagt om alle decimale getallen voor te stellen dat het minder accuraat is, integendeel: het binair stelsel is accurater.}. Een oplossing voor deze twee problemen is het Binary Coded Decimal (BCD) systeem. We kunnen elke decimaal cijfer voorstellen met 4-bit. We kunnen dus een decimaal getal voorstellen, door voor ieder cijfers zijn binair equivalent te gebruiken. Dit betekent dus dat we een decimaal getal met $n$ cijfers voorstellen met $4n$ bit. Elk van deze groepen bits kan dus alleen waardes van $0$ tot $9$ aannemen. Bijvoorbeeld: $1425_{10}=0101\_1001\_0001_2=0001\_0100\_0010\_0101_{\mbox{\tiny{BCD}}}$. Een groot nadeel van BCD voorstelling is dat bewerkingen heel wat complexe worden. Tabel \ref{tbl:bCDConversion} toont de omzetting van een decimaal cijfer naar het BCD equivalent.
\begin{table}[hbt]
\centering
\begin{tabular}{c|c}
Decimaal&BCD\\\hline
0&0000\\
1&0001\\
2&0010\\
3&0011\\
4&0100\\
5&0101\\
6&0110\\
7&0111\\
8&1000\\
9&1001
\end{tabular}
\caption{Decimale cijfers en hun BCD equivalent.}
\label{tbl:bCDConversion}
\end{table}
\paragraph{Bewerkingen met BCD}
\begin{figure}[hbt]
\centering
\subfigure[Schematische voorstelling]{\begin{tikzpicture}
\node[dadder,scale=0.75] (A0) at (0,0) {$4$-bit opteller};
\node[dadder,anchor=ya,scale=0.75] (A1) at (A0.sa |- 0,-2) {$4$-bit opteller};
\draw (A0.sa) -- (A1.ya);
\draw (A0.sb) -- (A1.yb);
\draw (A0.sc) -- (A1.yc);
\draw (A0.sd) -- (A1.yd);
\draw (A0.ci) -- ++(0.25,0) node[scale=0.75,anchor=west]{$c_i$};
\draw (A1.xd) -- ++(0,0.125) node[scale=0.75,anchor=south]{$0$};
\draw (A1.xa) -- ++(0,0.125) node[scale=0.75,anchor=south]{$0$};
\draw (A1.xb) -- ++(0,0.5) -| (A1.xc);
\draw[->] (A1.xc |- 0,-1.5) -- ++(-1.5,0) node[scale=0.75,anchor=east]{$c_o$};
\draw (-3.5,-0.375) rectangle ++(0.75,-0.75);
\draw (-3.125,-0.75) node[scale=0.75]{$>9$};
\draw (A0.co) node[scale=0.75,anchor=north east]{$c$} -| (-3.125,-0.375);
\pdot{A1.xc |- 0,-1.5};
\draw (-3.125,-1.125) -- (-3.125,-1.5);
\foreach \a/\i/\s in {sa/4/0,sb/3/1,sc/2/2,sd/1/3} {
  \draw (A0.\a |- 0,-0.375-0.15*\i) node[scale=0.75,anchor=west]{$z_\s$} -- (-2.75,-0.375-0.15*\i);
  \pdot{A0.\a |- 0,-0.375-0.15*\i}
}
\draw (A0.xa) -- ++(0,0.25) node[scale=0.75,anchor=south]{$x_0$};
\draw (A0.xb) -- ++(0,0.25) node[scale=0.75,anchor=south]{$x_1$};
\draw (A0.xc) -- ++(0,0.25) node[scale=0.75,anchor=south]{$x_2$};
\draw (A0.xd) -- ++(0,0.25) node[scale=0.75,anchor=south]{$x_3$};
\draw (A0.ya) -- ++(0,0.25) node[scale=0.75,anchor=south]{$y_0$};
\draw (A0.yb) -- ++(0,0.25) node[scale=0.75,anchor=south]{$y_1$};
\draw (A0.yc) -- ++(0,0.25) node[scale=0.75,anchor=south]{$y_2$};
\draw (A0.yd) -- ++(0,0.25) node[scale=0.75,anchor=south]{$y_3$};
\draw[->] (A1.sa) -- ++(0,-0.25) node[scale=0.75,anchor=north]{$s_0$};
\draw[->] (A1.sb) -- ++(0,-0.25) node[scale=0.75,anchor=north]{$s_1$};
\draw[->] (A1.sc) -- ++(0,-0.25) node[scale=0.75,anchor=north]{$s_2$};
\draw[->] (A1.sd) -- ++(0,-0.25) node[scale=0.75,anchor=north]{$s_3$};
\pdot{-3.125,-1.5};
\end{tikzpicture}
\label{fig:bCDAdderSchematic}}
\subfigure[Karnaugh-kaart]{
\begin{tikzpicture}
\kkaarte{0}{0}{$c_o$}{$c$/$z_3$/$z_2$/$z_1$/$z_0$}{0/0/0/0/0/0/0/0/0/0/1/1/1/1/1/1/1/1/1/-/-/-/-/-/-/-/-/-/-/-/-/-}
\end{tikzpicture}
\label{fig:bCDAdderKarnaugh}}
\subfigure[Implementatie vergelijker]{
\begin{tikzpicture}[circuit logic US]
\draw (-1,-1) rectangle (1,1);
\node[or gate,rotate=-90,inputs={normal,normal,normal},scale=0.75] (o0) at (-0.33,-0.5) {};
\node[and gate,rotate=180,scale=0.75] (a0) at (0.33,0) {};
\node[and gate,rotate=180,scale=0.75] (a1) at (0.33,0.5) {};
\draw (o0.input 1) |- (a0.output);
\draw (o0.input 2) |- (a1.output);
\draw (o0.input 3) -- (o0.input 3 |- 0,1.25) node[scale=0.75,anchor=south]{$c$};
\draw (1,-0.5) -- (1.25,-0.5) node[scale=0.75,anchor=west]{$z_0$};
\draw (a0.input 1) -- (a0.input 1 -| 1.25,0) node[scale=0.75,anchor=west]{$z_1$};
\draw (a1.input 1) -- (a1.input 1 -| 1.25,0) node[scale=0.75,anchor=west]{$z_2$};
\draw (a1.input 2) -- (a1.input 2 -| 1.25,0) node[scale=0.75,anchor=west]{$z_3$};
\draw (a0.input 2) -| (a1.input 2 -| 0.875,0);
\pdot{a1.input 2 -| 0.875,0}
\draw[->] (o0.output) -- (o0.output |- 0,-1.25) node[scale=0.75,anchor=north]{$c_o$};
\end{tikzpicture}
\label{fig:bCDAdderComparator}}
\caption{Mogelijke implementatie van een BCD opteller.}
\label{fig:bCDAdder}
\end{figure}
Ter illustratie zullen we tonen hoe we een opteller kunnen realiseren voor het BCD formaat. Hierbij streven we niet zozeer naar een minimale implementatie, maar tonen we dat een optelling in ieder geval complexer is dan bij zijn binair equivalent. Bij een optelling dienen we immers elke decimaal afzonderlijk op te tellen. Net zoals een half- en full adder bit per bit optelden, dienen we nu per decimaal een 4-bit opteller te realiseren. Indien het resultaat van deze optelling echter groter is dan 9, dienen we dit resultaat verder aan te passen, zodat het resultaat 10 minder is, en we de overdacht (carry) doorgeven. Voor dit eerste hebben we nogmaals een 4-bit opteller nodig. Verder hebben we een component nodig die controleert of het eerste resultaat groter is dan 9. Dit component heeft niet alleen een carry als uitgang, maar moet ook invoer genereren voor de tweede opteller. Indien het eerste resultaat immers groter is dan 9, tellen we er nog eens 6 bij op. Dit betekent dat bij een tussenresultaat van 10, de tweede opteller tot 16 komt wat dus gebaseerd op de laatste 4 bits in 0 resulteert. Figuur \ref{fig:bCDAdderSchematic} toont schematisch hoe we twee decimale cijfers kunnen optellen. Indien we deze structuur voor ieder cijfer herhalen, realiseren we een BCD opteller. De vergelijker synthetiseren we met behulp van Karnaugh-kaarten zoals op figuur \ref{fig:bCDAdderKarnaugh}. Merk op dat het resultaat van de eerst opteller hoogstens 18 is, en we dus voor de andere waarden don't cares kunnen gebruiken. Een mogelijke synthese van deze vergelijker staat op figuur \ref{fig:bCDAdderComparator}. Tot slot merken we op dat we bij het bepalen van het negatieve getal van een BCD niet het 2-complement moeten berekenen, maar het 10-complement. Ook deze stap vereist extra hardware.
\subsubsection{American Standard Code for Information Interchange (ASCII)}
\begin{table}[hbt]
\centering
\begin{tabular}{c|cccccccc}
&000&001&010&011&100&101&110&111\\\hline
0000	&\verb+NUL+	&\verb+DLE+	&\verb+SP+	&\verb+0+	&\verb+@+	&\verb+P+	&\verb+`+	&\verb+p+\\
0001	&\verb+SOH+	&\verb+DC1+	&\verb+!+	&\verb+1+	&\verb+A+	&\verb+Q+	&\verb+a+	&\verb+q+\\
0010	&\verb+STX+	&\verb+DC2+	&\verb+"+	&\verb+2+	&\verb+B+	&\verb+R+	&\verb+b+	&\verb+r+\\
0011	&\verb+ETX+	&\verb+DC3+	&\verb+#+	&\verb+3+	&\verb+C+	&\verb+S+	&\verb+c+	&\verb+s+\\

0100	&\verb+EOT+	&\verb+DC4+	&\verb+$+	&\verb+4+	&\verb+D+	&\verb+T+	&\verb+d+	&\verb+t+\\
0101	&\verb+ENQ+	&\verb+NAK+	&\verb+%+	&\verb+5+	&\verb+E+	&\verb+U+	&\verb+e+	&\verb+u+\\
0110	&\verb+ACK+	&\verb+SYN+	&\verb+&+	&\verb+6+	&\verb+F+	&\verb+V+	&\verb+f+	&\verb+v+\\
0111	&\verb+BEL+	&\verb+ETB+	&\verb+'+	&\verb+7+	&\verb+G+	&\verb+W+	&\verb+g+	&\verb+w+\\

1000	&\verb+BS+	&\verb+CAN+	&\verb+(+	&\verb+8+	&\verb+H+	&\verb+X+	&\verb+h+	&\verb+x+\\
1001	&\verb+HT+	&\verb+EM+	&\verb+)+	&\verb+9+	&\verb+I+	&\verb+Y+	&\verb+i+	&\verb+y+\\
1010	&\verb+LF+	&\verb+SUB+	&\verb+*+	&\verb+:+	&\verb+J+	&\verb+Z+	&\verb+j+	&\verb+z+\\
1011	&\verb+VT+	&\verb+ESC+	&\verb/+/	&\verb+;+	&\verb+K+	&\verb+[+	&\verb+k+	&\verb+{+\\

1100	&\verb+FF+	&\verb+FS+	&\verb+,+	&\verb+<+	&\verb+L+	&\verb+\+	&\verb+l+	&\verb+|+\\
1101	&\verb+CR+	&\verb+GS+	&\verb+-+	&\verb+=+	&\verb+M+	&\verb+]+	&\verb+m+	&\verb+}+\\
1110	&\verb+SO+	&\verb+RS+	&\verb+.+	&\verb+>+	&\verb+N+	&\verb+^+	&\verb+n+	&\verb+~+\\
1111	&\verb+SI+	&\verb+US+	&\verb+/+	&\verb+?+	&\verb+O+	&\verb+_+	&\verb+o+	&\verb+DEL+
\end{tabular}
\caption{ASCII standaard.}
\label{tbl:aSCIIStandard}
\end{table}
Een standaard om tekst voor te stellen is de American Standard Code for Information Interchange (ASCII). ASCII reserveert 7 bit per karakter. Deze 7 bit zorgen voor 128 mogelijke karakters. De karakterset bestaat uit de Romaanse letters, Arabische cijfers, diverse leestekens en enkele functiesymbolen. De toewijzing van deze symbolen vertoond enige logica om bijvoorbeeld kleine letters naar hoofdletters om te zetten, en binaire getallen in hun ASCII equivalent. Tabel \ref{tbl:aSCIIStandard} toont de ASCII-karakters met hun bijbehorende binaire code. ASCII bevat geen ondersteuning voor cyrillisch, Arabisch,... Unicode is een standaard die 8, 16 of 32 bit per karakter reserveert en die dit probleem oplost. Anno 2011 is $11\%$ van de binaire waarden toegewezen.
\section{Andere basisschakelingen}
\label{s:andereBasis}
Naast de rekenkundige schakelingen in sectie \ref{s:rekenkundig} zullen we in schema's ook geregeld enkele andere bouwstenen terugvinden. In deze sectie zullen we de meest voornaamste bespreken: In subsectie \ref{ss:multiplexer} bespreken we de multiplexer. Daarna bespreken we in \ref{ss:decoder} en \ref{ss:demultiplexer} twee verwante bouwstenen: de decoder en demultiplexer. Het omgekeerde van de decoder, de encoder behandelen we in \ref{ss:encoder}. We eindigen met de vergelijker in \ref{ss:comparator} en de reeds aangehaalde schuifoperatie in subsectie \ref{ss:shiftoperators}.
\subsection{Multiplexer}
\label{ss:multiplexer}
Een \termen{multiplexer}, \termen{selector} of \termen{MUX} is een component die bij $n$ \termen{selectie-ingangen $s_i$} en $2^n$ \termen{data-ingangen $d_i$}, de data-ingang met index $S$ op de uitgang zet. Hierbij is $S$ de waarde die voorgesteld wordt door de selectie-ingangen. In een blokschema word dit component voorgesteld door een trapezium, waarbij de data-ingangen aan de lange zijde staan, de uitgangen aan de korte zijde, en de selectie-ingangen aan \'e\'en van de schuine zijden zoals op figuur \ref{fig:multiplexerInterface}. Deze figuur toont een 4-naar-1 MUX. Eventueel worden aan de andere kant ook uitgangen toegevoegd, deze zijn identiek aan de selectie-ingangen aan de ene kant en dienen enkel om het blokschema overzichtelijker te maken. Figuren \ref{fig:multiplexerTruthTable} en \ref{fig:multiplexerSchema} tonen respectievelijk de eerder informeel besproken waarheidstabel en een mogelijke implementatie van dit component.
\begin{figure}[hbt]
\centering
\subfigure[Interface]{
  \begin{tikzpicture}
  \node[mux4to1] (I) at (0,0) {};
  \draw[<-] (I.selin0) -- (I.selin0 -| -1,0) node[scale=0.75,anchor=east]{$s_0$};
  \draw[<-] (I.selin1) -- (I.selin1 -| -1,0) node[scale=0.75,anchor=east]{$s_1$};
  \draw[->] (I.selout0) -- (I.selout0 -| 1,0) node[scale=0.75,anchor=west]{$s_0$};
  \draw[->] (I.selout1) -- (I.selout1 -| 1,0) node[scale=0.75,anchor=west]{$s_1$};
  \draw[->] (I.output) -- ++(0,-0.25) node[scale=0.75,anchor=north]{$f$};
  \draw[<-] (I.data0) -- ++(0,0.25) node[scale=0.75,anchor=south]{$d_0$};
  \draw[<-] (I.data1) -- ++(0,0.25) node[scale=0.75,anchor=south]{$d_1$};
  \draw[<-] (I.data2) -- ++(0,0.25) node[scale=0.75,anchor=south]{$d_2$};
  \draw[<-] (I.data3) -- ++(0,0.25) node[scale=0.75,anchor=south]{$d_3$};
  \end{tikzpicture}
  \label{fig:multiplexerInterface}
}
\subfigure[Waarheidstabel] {
\begin{tikzpicture}
\draw (0,0) node[scale=0.95]{\begin{tabular}{cc|c}
$s_0$&$s_1$&$f$\\\hline
0&0&$d_0$\\
0&1&$d_1$\\
1&0&$d_2$\\
1&1&$d_3$
\end{tabular}};
\end{tikzpicture}
\label{fig:multiplexerTruthTable}}
\subfigure[Mogelijke Implementatie]{
%\begin{tikzpicture}[circuit logic US,scale=1,xscale=0.8]
%\draw[dashed] (-3.5,2.25) -- (3.5,2.25) -- (2.25,-0.25) -- (-2.25,-0.25) -- cycle;
%\node[or gate, inputs={normal,normal,normal,normal},scale=0.7,rotate=-90] (O) at (0,0.25) {};
%\foreach\d/\dga/\dgb in {0/inverted/inverted,1/inverted/normal,2/normal/inverted,3/normal/normal} {
%  \node[and gate, inputs={\dga,\dgb,normal},scale=0.7,rotate=-90] (A\d) at (2.25-1.5*\d,1.25) {};
%  \draw (A\d.input 3) -- (A\d.input 3 |- 0,2.25);
%  \draw (A\d.input 1) -- (A\d.input 1 |- 0,1.75);
%  \draw (A\d.input 2) -- (A\d.input 2 |- 0,2);
%  \draw[dashed] (A\d.input 3 |- 0,2.25) -- ++(0,0.25) node[anchor=south]{$d_\d$};
%}
%\foreach\s/\g in {0/1,1/2} {
%  \draw[dashed] (-3-1/6-\s/6,1.75+0.25*\s) -- (-3.75,1.75+0.25*\s) node[anchor=east]{$s_\s$};
%  \draw (-3-1/6-\s/6,1.75+0.25*\s) -- (A0.input \g |- 0,1.75+0.25*\s);
%}
%\foreach\d/\g/\y in {0/1/0,1/2/1,2/3/1,3/4/0} {
%  \draw (A\d.output) -- ++(0,-0.25+\y*0.1) -| (O.input \g);
%}
%\draw (O.output) -- (O.output |- 0,-0.25);
%\draw[dashed] (O.output |- 0,-0.25) -- (O.output |- 0,-0.5) node[anchor=north]{$f$};
%\end{tikzpicture}
\begin{tikzpicture}[circuit logic US,scale=4]
\draw[dashed] (-0.5,-0.3) -- (-0.8,0.3) -- (0.8,0.3) -- (0.5,-0.3) -- cycle;
\node[or gate,scale=0.2*0.85,rotate=-90,inputs={normal,normal,normal,normal}] (O) at (0,-0.15) {};
\node[and gate,scale=0.2,rotate=-90,inputs={inverted,normal,inverted}] (A0) at (0.45,0.09) {};
\node[and gate,scale=0.2,rotate=-90,inputs={normal,normal,inverted}] (A1) at (0.16,0.09) {};
\node[and gate,scale=0.2,rotate=-90,inputs={inverted,normal,normal}] (A2) at (-0.16,0.09) {};
\node[and gate,scale=0.2,rotate=-90,inputs={normal,normal,normal}] (A3) at (-0.48,0.09) {};
\coordinate (Fdl) at (0,-0.3);
\coordinate (Sia) at (-0.6,-0.1);
\coordinate (Sib) at (-0.7,0.1);
\coordinate (Soa) at (0.6,-0.1);
\coordinate (Sob) at (0.7,0.1);
\coordinate (Dia) at (A0.output |- 0,0.3);
\coordinate (Dib) at (A1.output |- 0,0.3);
\coordinate (Dic) at (A2.output |- 0,0.3);
\coordinate (Did) at (A3.output |- 0,0.3);
\draw (Sib) |- (Sob |- 0,0.265) -- (Sob);
\draw (Sia) |- (Soa |- 0,0.225) -- (Soa);
\draw (Dia) -- (A0.input 2);
\draw (Dib) -- (A1.input 2);
\draw (Dic) -- (A2.input 2);
\draw (Did) -- (A3.input 2);
\draw[dashed] (Fdl) -- ++(0,-0.1) node[scale=0.75,anchor=north]{$f$};
\draw[dashed] (Dia) -- ++(0,0.1) node[scale=0.75,anchor=south]{$d_0$};
\draw[dashed] (Dib) -- ++(0,0.1) node[scale=0.75,anchor=south]{$d_1$};
\draw[dashed] (Dic) -- ++(0,0.1) node[scale=0.75,anchor=south]{$d_2$};
\draw[dashed] (Did) -- ++(0,0.1) node[scale=0.75,anchor=south]{$d_3$};
\draw (A0.input 1) -- (A0.input 1 |- 0,0.225);
\draw (A1.input 1) -- (A1.input 1 |- 0,0.225);
\draw (A2.input 1) -- (A2.input 1 |- 0,0.225);
\draw (A3.input 1) -- (A3.input 1 |- 0,0.225);
\draw (A0.input 3) -- (A0.input 3 |- 0,0.265);
\draw (A1.input 3) -- (A1.input 3 |- 0,0.265);
\draw (A2.input 3) -- (A2.input 3 |- 0,0.265);
\draw (A3.input 3) -- (A3.input 3 |- 0,0.265);
\draw[dashed] (Sib) -- (-0.9,0 |- Sib) node[scale=0.75,anchor=east]{$s_1$};
\draw[dashed] (Sia) -- (-0.9,0 |- Sia) node[scale=0.75,anchor=east]{$s_0$};
\draw[dashed] (Sob) -- (0.9,0 |- Sib) node[scale=0.75,anchor=west]{$s_1$};
\draw[dashed] (Soa) -- (0.9,0 |- Sia) node[scale=0.75,anchor=west]{$s_0$};
\draw (A3.output) -- (A3.output |- 0,-0.03) -| (O.input 4);
\draw (A2.output) -- (A2.output |- 0,-0.01) -| (O.input 3);
\draw (A1.output) -- (A1.output |- 0,-0.01) -| (O.input 2);
\draw (A0.output) -- (A0.output |- 0,-0.03) -| (O.input 1);
\draw (Fdl) -- (O.output);
\begin{scope}[scale=0.25]
\pdot{A0.input 1 |- 0,0.9}
\pdot{A1.input 1 |- 0,0.9}
\pdot{A2.input 1 |- 0,0.9}
\pdot{A3.input 1 |- 0,0.9}
\pdot{A0.input 3 |- 0,1.06}
\pdot{A1.input 3 |- 0,1.06}
\pdot{A2.input 3 |- 0,1.06}
\pdot{A3.input 3 |- 0,1.06}
\end{scope}
\end{tikzpicture}
\label{fig:multiplexerSchema}}
\subfigure[Cascade]{
\begin{tikzpicture}[circuit logic US]
\node[mux4to1] (A) at (6,0) {};
\draw (A.selin1) -- (A.selin1 -| 1.5,0) node[scale=0.65,anchor=east]{$s_3$};
\draw (A.selin0) -- (A.selin0 -| 1.5,0) node[scale=0.65,anchor=east]{$s_2$};
\draw (A.output) -- ++(0,-0.25) node[scale=0.65,anchor=north]{$f$};
\foreach \x/\s in {0/0,1/1,2/1,3/0} {
  \node[mux4to1] (B\x) at (9-2*\x,1) {};
  \draw (B\x.output) -- ++(0,-0.25+0.125*\s) -| (A.data\x);
}
\foreach \x/\g/\i in {0/0/0,0/1/1,0/2/2,0/3/3,1/0/4,1/1/5,1/2/6,1/3/7,2/0/8,2/1/9,2/2/10,2/3/11,3/0/12,3/1/13,3/2/14,3/3/15} {
  \draw (B\x.data\g) -- ++(0,0.25) node[scale=0.65,anchor=south]{$d_{\i}$};
}
\foreach \x/\xa in {3/2,2/1,1/0} {
  \draw (B\x.selout0) -- (B\xa.selin0);
  \draw (B\x.selout1) -- (B\xa.selin1);
}
\draw (B3.selin1) -- (B3.selin1 -| 1.5,0) node[scale=0.65,anchor=east]{$s_1$};
\draw (B3.selin0) -- (B3.selin0 -| 1.5,0) node[scale=0.65,anchor=east]{$s_0$};
\end{tikzpicture}
\label{fig:multiplexerCascade}}
\caption{Multiplexer.}
\label{fig:multiplexer}
\end{figure}
\subsubsection{Cascade}
We kunnen uiteraard multiplexers met een verschillende parameter $n$ bouwen. Het probleem is echter dat deze waarden erg kunnen vari\"eren, wat weinig interessant is voor massaproductie. Daarom zal men veelal met een \termen{cascade} werken. Figuur \ref{fig:multiplexerCascade} toont een 16-naar-1 MUX gebouwd met behulp van 5 4-naar-1 multiplexers. We kunnen dit uiteraard veralgemenen: Indien we een $2^{n\times m}$-naar-1 multiplexer willen bouwen, kunnen we dit met $n$ niveaus van $2^m$-naar-1 multiplexers. Het aantal multiplexers die we in dat geval nodig hebben is:
\begin{equation}
\mbox{aantal multiplexers}=\displaystyle\frac{2^{n\times m}-1}{2^m-1}
\end{equation}
\subsection{Decoder}
\label{ss:decoder}
Een andere schakeling die sterk gerelateerd is aan een multiplexer is een \termen{decoder}. Een decoder beschikt over een \termen{enable-ingang $e$}, en $n$ \termen{adres-ingangen $a_i$}. De uitvoer bestaat uit $2^n$ \termen{selectie-uitgangen $s_i$}. Een decoder zal indien er een 1 aangelegd wordt op de enable-ingang $e$, een 1 op de uitgang zetten met index $A$. Hierbij staat $A$ voor de binaire waarde die voorgesteld wordt door de adres-ingangen $a_i$. In een blokschema wordt een decoder voorgesteld zoals op figuur \ref{fig:decoderInterface} door een rechthoek. In dit geval een 2-naar-4 decoder. Bovenaan staat de adres-ingangen $a_i$, aan de zijkant de enable-ingang $e$, en onderaan de uitgangen $s_i$.
\begin{figure}[hbt]
\centering
\subfigure[Interface]{
\begin{tikzpicture}
  \node[decoder2to4] (I) at (0,0) {Decoder};
  \draw[<-] (I.enable) -- ++(-0.25,0) node[scale=0.75,anchor=east]{$e$};
  \draw[->] (I.s0) -- ++(0,-0.25) node[scale=0.75,anchor=north]{$s_0$};
  \draw[->] (I.s1) -- ++(0,-0.25) node[scale=0.75,anchor=north]{$s_1$};
  \draw[->] (I.s2) -- ++(0,-0.25) node[scale=0.75,anchor=north]{$s_2$};
  \draw[->] (I.s3) -- ++(0,-0.25) node[scale=0.75,anchor=north]{$s_3$};
  \draw[<-] (I.a0) -- ++(0,0.25) node[scale=0.75,anchor=south]{$a_0$};
  \draw[<-] (I.a1) -- ++(0,0.25) node[scale=0.75,anchor=south]{$a_1$};
\end{tikzpicture}
\label{fig:decoderInterface}
}
\subfigure[Waarheidstabel] {
\begin{tikzpicture}
\draw (0,0) node[scale=0.75]{\begin{tabular}{ccc|cccc}
$e$&$a_0$&$a_1$&$s_0$&$s_1$&$s_2$&$s_3$\\\hline
0&-&-&0&0&0&0\\
1&0&0&1&0&0&0\\
1&0&1&0&1&0&0\\
1&1&0&0&0&1&0\\
1&1&1&0&0&0&1\\
\end{tabular}};
\end{tikzpicture}
\label{fig:decoderTruthTable}}
\subfigure[Mogelijke implementatie]{
\begin{tikzpicture}[circuit logic US,scale=3]
\draw[dashed] (-0.8,-0.3) rectangle (0.8,0.3);
\draw[black!30] (0,0) node[scale=3]{Decoder};
\coordinate (Ei) at (-0.8,0);
\coordinate (Aia) at (0.26666,0.3);
\coordinate (Aib) at (-0.26666,0.3);
\coordinate (Soa) at (0.48,-0.3);
\coordinate (Sob) at (0.16,-0.3);
\coordinate (Soc) at (-0.16,-0.3);
\coordinate (Sod) at (-0.48,-0.3);
\node[and gate,scale=0.27,rotate=-90,inputs={normal,inverted,inverted},anchor=output] (A0) at (Soa |- 0,-0.25) {};
\node[and gate,scale=0.27,rotate=-90,inputs={normal,normal,inverted},anchor=output] (A1) at (Sob |- 0,-0.25) {};
\node[and gate,scale=0.27,rotate=-90,inputs={normal,inverted,normal},anchor=output] (A2) at (Soc |- 0,-0.25) {};
\node[and gate,scale=0.27,rotate=-90,inputs={normal,normal,normal},anchor=output] (A3) at (Sod |- 0,-0.25) {};
\draw (Ei) -| (-0.7,0.25) -| (A0.input 1);
\draw (A1.input 1 |- 0,0.25) -- (A1.input 1);
\draw (A2.input 1 |- 0,0.25) -- (A2.input 1);
\draw (A3.input 1 |- 0,0.25) -- (A3.input 1);
\draw (A1.input 2 |- 0,0.2) -- (A1.input 2);
\draw (A2.input 2 |- 0,0.2) -- (A2.input 2);
\draw (A1.input 3 |- 0,0.15) -- (A1.input 3);
\draw (A2.input 3 |- 0,0.15) -- (A2.input 3);
\draw (Aia) -- (Aia |- 0,0.2);
\draw (Aib) -- (Aib |- 0,0.15);
\draw (A3.input 2) -- (A3.input 2 |- 0,0.2) -| (A0.input 2);
\draw (A3.input 3) -- (A3.input 3 |- 0,0.15) -| (A0.input 3);
\draw (A0.output) -- (Soa);
\draw (A1.output) -- (Sob);
\draw (A2.output) -- (Soc);
\draw (A3.output) -- (Sod);
\draw[dashed] (Soa) -- ++(0,-0.1) node[scale=0.75,anchor=north]{$s_0$};
\draw[dashed] (Sob) -- ++(0,-0.1) node[scale=0.75,anchor=north]{$s_1$};
\draw[dashed] (Soc) -- ++(0,-0.1) node[scale=0.75,anchor=north]{$s_2$};
\draw[dashed] (Sod) -- ++(0,-0.1) node[scale=0.75,anchor=north]{$s_3$};
\draw[dashed] (Aia) -- ++(0,0.1) node[scale=0.75,anchor=south]{$a_0$};
\draw[dashed] (Aib) -- ++(0,0.1) node[scale=0.75,anchor=south]{$a_1$};
\draw[dashed] (Ei) -- ++(-0.1,0) node[scale=0.75,anchor=east]{$e$};
\begin{scope}[scale=0.333]
\pdot{A1.input 1 |- 0,0.75}
\pdot{A2.input 1 |- 0,0.75}
\pdot{A3.input 1 |- 0,0.75}
\pdot{A1.input 2 |- 0,0.6}
\pdot{A2.input 2 |- 0,0.6}
\pdot{A1.input 3 |- 0,0.45}
\pdot{A2.input 3 |- 0,0.45}
\pdot{Aia |- 0,0.6}
\pdot{Aib |- 0,0.45}
\end{scope}
\end{tikzpicture}
\label{fig:decoderSchema}}
\subfigure[Cascade]{
\begin{tikzpicture}[circuit logic US]
\node[decoder2to4] (A) at (0,0) {Decoder};
\draw (A.a0) -- ++(0,0.25) node[anchor=south,scale=0.75]{$a_2$};
\draw (A.a1) -- ++(0,0.25) node[anchor=south,scale=0.75]{$a_3$};
\draw (A.enable) -- ++(-0.25,0) node[anchor=east,scale=0.75]{$e$};
\foreach\x/\y in {0/1,1/0,2/0,3/1} {
  \node[decoder2to4] (B\x) at (3-2*\x,-1.5) {Decoder};
  \draw (B\x.a0) -- ++(0,0.125) node[anchor=south,scale=0.75]{$a_0$};
  \draw (B\x.a1) -- ++(0,0.125) node[anchor=south,scale=0.75]{$a_1$};
  \draw (B\x.enable) -- ++(-0.2,0) |- (A.s\x |- 0,-0.7+0.2*\y) -- (A.s\x);
}
\foreach \x/\y/\z in {0/0/0,0/1/1,0/2/2,0/3/3,1/0/4,1/1/5,1/2/6,1/3/7,2/0/8,2/1/9,2/2/10,2/3/11,3/0/12,3/1/13,3/2/14,3/3/15} {
  \draw (B\x.s\y) -- ++(0,-0.25) node[anchor=north,scale=0.75]{$s_{\z}$};
}
\end{tikzpicture}
\label{fig:decoderCascade}}
\caption{Decoder.}
\label{fig:decoder}
\end{figure}
Op figuur \ref{fig:decoderTruthTable} staat de waarheidstabel voor dit component. Figuur \ref{fig:decoderSchema} toont een mogelijke implementatie. Decoders worden hoofdzakelijk gebruikt voor het decoderen van adressen.
\subsubsection{Cascade}
Net als bij multiplexer kunnen we in plaats van een heel arsenaal aan decoders aan te bieden door middel van een cascade een nieuwe decoder bouwen, zoals op figuur \ref{fig:decoderCascade}. Hier bouwen we een 4-naar-16 decoder met 5 2-naar-4 decoders. In het algemeen kunnen we een $nm$-naar-$2^{nm}$ decoder bouwen met $n$ niveaus van $m$-naar-$2^m$ decoders. In totaal hebben we dus volgend aantal decoders nodig:
\begin{equation}
\mbox{aantal decoders}=\displaystyle\frac{2^{n\times m}-1}{2^m-1}
\end{equation}
\subsubsection{Alternatieve implementatie voor Multiplexers}
Decoders worden ook gebruikt voor de synthese van bijvoorbeeld multiplexers. Figuur \ref{fig:decoderMultiplexerAnd} toont een manier om op basis van AND-poorten en een decoder een multiplexer te bouwen. We kunnen echter de OR-poort weglaten en de AND-poorten vervangen door 3-state buffers zoals op figuur \ref{fig:decoderMultiplexerTriState}. In dat geval hebben we een schakeling gerealiseerd die we een \termen{bus} noemen. Bussen zijn multiplexers waarbij we toelaten dat de ingangen gedistribueerd zijn over de verschillende delen van de schakeling. Elk van deze ingangen dienen we eenvoudigweg een uitgang van de decoder toe toe te wijzen, en de uitgang met een 3-state buffer verbinden met de bus. Bussen hebben verschillende voordelen:
\begin{itemize}
 \item We kunnen makkelijk het aantal ingangen uitbreiden. We dienen enkel over een decoder met voldoende grote $n$ te beschikken en een tri-state buffer per ingang.
 \item Indien we de traditionele Multiplexer gebruiken zal bij een groot aantal ingangen de fan-in van de OR-poort toenemen. Bovendien moeten alle ingangen van de multiplexer dicht bij elkaar staan.
 \item 3-state buffers zijn meestal gratis op een FPGA. Elk logisch blok (LB) heeft immers minstens \'e\'en uitgang langs een 3-state buffer die verbonden is met een lange lijn.
\end{itemize}
\begin{figure}[hbt]
\centering
\subfigure[Muliplexer]{
\begin{tikzpicture}[circuit logic US,scale=4]
\coordinate (Fdl) at (0,-0.3);
\coordinate (Sia) at (-0.6,-0.1);
\coordinate (Sib) at (-0.7,0.1);
\draw[dashed] (-0.5,-0.3) -- (-0.8,0.3) -- (0.8,0.3) -- (0.5,-0.3) -- cycle;
\node[decoder2to4,scale=0.75,rotate=90] (D) at (-0.50,0) {Decoder};
\draw (D.enable) -- ++(0,-0.05) node[anchor=north,scale=0.75]{$1$};
\draw (D.a0) -| (Sia);
\draw (D.a1) -- ++(-0.025,0) |- (Sib);
\node[or gate,scale=0.2*0.85,rotate=-90,inputs={normal,normal,normal,normal}] (O) at (0,-0.15) {};
\node[and gate,scale=0.2,rotate=-90,inputs={normal,normal}] (A0) at (0.24,0.09) {};
\node[and gate,scale=0.2,rotate=-90,inputs={normal,normal}] (A1) at (0.08,0.09) {};
\node[and gate,scale=0.2,rotate=-90,inputs={normal,normal}] (A2) at (-0.08,0.09) {};
\node[and gate,scale=0.2,rotate=-90,inputs={normal,normal}] (A3) at (-0.24,0.09) {};
\coordinate (Dia) at (A0.input 1 |- 0,0.3);
\coordinate (Dib) at (A1.input 1 |- 0,0.3);
\coordinate (Dic) at (A2.input 1 |- 0,0.3);
\coordinate (Did) at (A3.input 1 |- 0,0.3);
\draw (Dia) -- (A0.input 1);
\draw (Dib) -- (A1.input 1);
\draw (Dic) -- (A2.input 1);
\draw (Did) -- (A3.input 1);
\draw (A0.input 2) |- (-0.41,0.27) |- (D.s0);
\draw (A1.input 2) |- (-0.38,0.24) |- (D.s1);
\draw (A2.input 2) |- (-0.35,0.21) |- (D.s2);
\draw (A3.input 2) |- (-0.32,0.18) |- (D.s3);
\draw[dashed] (Fdl) -- ++(0,-0.1) node[scale=0.75,anchor=north]{$f$};
\draw[dashed] (Dia) -- ++(0,0.1) node[scale=0.75,anchor=south]{$d_0$};
\draw[dashed] (Dib) -- ++(0,0.1) node[scale=0.75,anchor=south]{$d_1$};
\draw[dashed] (Dic) -- ++(0,0.1) node[scale=0.75,anchor=south]{$d_2$};
\draw[dashed] (Did) -- ++(0,0.1) node[scale=0.75,anchor=south]{$d_3$};
\draw[dashed] (Sib) -- (-0.9,0 |- Sib) node[scale=0.75,anchor=east]{$s_1$};
\draw[dashed] (Sia) -- (-0.9,0 |- Sia) node[scale=0.75,anchor=east]{$s_0$};
\draw (A3.output) -- (A3.output |- 0,-0.03) -| (O.input 4);
\draw (A2.output) -- (A2.output |- 0,-0.01) -| (O.input 3);
\draw (A1.output) -- (A1.output |- 0,-0.01) -| (O.input 2);
\draw (A0.output) -- (A0.output |- 0,-0.03) -| (O.input 1);
\draw (Fdl) -- (O.output);
\end{tikzpicture}
\label{fig:decoderMultiplexerAnd}}
\subfigure[Bus]{
\begin{tikzpicture}[circuit logic US,scale=4]
\coordinate (Fdl) at (0,-0.3);
\coordinate (Sia) at (-0.6,-0.1);
\coordinate (Sib) at (-0.7,0.1);
\foreach \x in {0,1,2,3} {
  \fill[black!30] (0.18-0.16*\x,0.1) rectangle (0.30-0.16*\x,-0.27);
}
\draw[dashed] (-0.5,-0.3) -- (-0.8,0.3) -- (0.8,0.3) -- (0.5,-0.3) -- cycle;
\node[decoder2to4,scale=0.75,rotate=90] (D) at (-0.50,0) {Decoder};
\draw (D.enable) -- ++(0,-0.05) node[anchor=north,scale=0.75]{$1$};
\draw (D.a0) -| (Sia);
\draw (D.a1) -- ++(-0.025,0) |- (Sib);
\node[tris,scale=0.75,rotate=-90] (T0) at (0.24,0) {};
\node[tris,scale=0.75,rotate=-90] (T1) at (0.08,0) {};
\node[tris,scale=0.75,rotate=-90] (T2) at (-0.08,0) {};
\node[tris,scale=0.75,rotate=-90] (T3) at (-0.24,0) {};
\draw (T0.z) -- (T0.z |- 0,-0.25);
\draw (T1.z) -- (T1.z |- 0,-0.25);
\draw (T2.z) -- (T2.z |- 0,-0.25);
\draw (T3.z) -- (T3.z |- 0,-0.25);
\draw (Fdl |- 0,-0.25) -- (Fdl);
\draw[thick] (T0.z |- 0,-0.25) node[anchor=west,scale=0.75]{bus} -- (T3.z |- 0,-0.25);
\coordinate (Dia) at (T0.z |- 0,0.3);
\coordinate (Dib) at (T1.z |- 0,0.3);
\coordinate (Dic) at (T2.z |- 0,0.3);
\coordinate (Did) at (T3.z |- 0,0.3);
\draw (Dia) -- (T0.x);
\draw (Dib) -- (T1.x);
\draw (Dic) -- (T2.x);
\draw (Did) -- (T3.x);
\draw (T3.c) -- (T3.c -| -0.32,0) |- (D.s3);
\foreach \x in {0,1,2} {
  \draw (T\x.c) -| (0.16-0.16*\x,0.25-0.03*\x) -| (D.s\x -| -0.41+0.03*\x,0) -- (D.s\x);
}
\draw[dashed] (Fdl) -- ++(0,-0.1) node[scale=0.75,anchor=north]{$f$};
\draw[dashed] (Dia) -- ++(0,0.1) node[scale=0.75,anchor=south]{$d_0$};
\draw[dashed] (Dib) -- ++(0,0.1) node[scale=0.75,anchor=south]{$d_1$};
\draw[dashed] (Dic) -- ++(0,0.1) node[scale=0.75,anchor=south]{$d_2$};
\draw[dashed] (Did) -- ++(0,0.1) node[scale=0.75,anchor=south]{$d_3$};
\draw[dashed] (Sib) -- (-0.9,0 |- Sib) node[scale=0.75,anchor=east]{$s_1$};
\draw[dashed] (Sia) -- (-0.9,0 |- Sia) node[scale=0.75,anchor=east]{$s_0$};
\end{tikzpicture}
\label{fig:decoderMultiplexerTriState}}
\caption{Multiplexer en bus gesynthetiseerd met decoders.}
\label{fig:decoderMultiplexer}
\end{figure}
\subsection{Demultiplexer}
\label{ss:demultiplexer}
Het inverse van een $2^n$-naar-1 multiplexer is een 1-naar-$2^n$ \termen{demultiplexer} (ook wel \termen{demux} genoemd). Dit component bestaat dan ook logischerwijs uit 1 \termen{data-ingang $d$} en $n$ \termen{selectie-ingangen $a_i$}. Het component beschikt verder over $2^n$ uitgangen $s_i$. Logischerwijs zetten we de waarde van de data-ingang $d$ op de uitgang met de index die wordt voorgesteld door de selectie-ingangen. Een aandachtige lezer zal misschien al opgemerkt hebben dat dit probleem in wezen niet veel verschilt van de constructie van een decoder. Sterker nog, we hoeven niets aan te passen, het is alleen een kwestie van een andere interface. Figuur \ref{fig:demultiplexer} toont de interface van een demultiplexer en de equivalentie met een decoder. In tegenstelling tot de decoder zetten we de selectie-ingangen aan de zijkant en de data-ingang aan de bovenkant.
\begin{figure}[hbt]
\centering
\begin{tikzpicture}
\node[demux1to4] (DM) at (0,0) {Demux};
\draw (DM.a0) -- ++(-0.25,0) node[scale=0.75,anchor=east]{$a_0$};
\draw (DM.a1) -- ++(-0.25,0) node[scale=0.75,anchor=east]{$a_1$};
\draw (DM.s0) -- ++(0,-0.25) node[scale=0.75,anchor=north]{$s_0$};
\draw (DM.s1) -- ++(0,-0.25) node[scale=0.75,anchor=north]{$s_1$};
\draw (DM.s2) -- ++(0,-0.25) node[scale=0.75,anchor=north]{$s_2$};
\draw (DM.s3) -- ++(0,-0.25) node[scale=0.75,anchor=north]{$s_3$};
\draw (DM.data) -- ++(0,0.25) node[scale=0.75,anchor=south]{$d$};
\node (E) at (2,0) {$\equiv$};
\node[decoder2to4] (DC) at (4,0) {Decoder};
\draw (DC.a0) -- ++(0,0.25) node[scale=0.75,anchor=south]{$a_0$};
\draw (DC.a1) -- ++(0,0.25) node[scale=0.75,anchor=south]{$a_1$};
\draw (DC.s0) -- ++(0,-0.25) node[scale=0.75,anchor=north]{$s_0$};
\draw (DC.s1) -- ++(0,-0.25) node[scale=0.75,anchor=north]{$s_1$};
\draw (DC.s2) -- ++(0,-0.25) node[scale=0.75,anchor=north]{$s_2$};
\draw (DC.s3) -- ++(0,-0.25) node[scale=0.75,anchor=north]{$s_3$};
\draw (DC.enable) -- ++(-0.25,0) node[scale=0.75,anchor=east]{$d$};
\end{tikzpicture}
\caption{Demultiplexer}
\label{fig:demultiplexer}
\end{figure}
\paragraph{}
Demultiplexers worden maar zeer zelden gebruikt, kun doel is immers om data op een bepaalde lijn te plaatsen, terwijl men op de andere lijnen een 0 aanlegt. In de praktijk is het aanleggen van de data op alle lijnen meestal geen probleem. In dat geval kunnen we dus de demultiplexer eenvoudigweg vervangen door een draad die de data-ingang met alle uitgangen verbindt.
\subsection{Encoder}
\label{ss:encoder}
Ook de decoder heeft een invers: een $2^n$-naar-$n$ \termen{encoder}. Een encoder bevat $2^n$ \termen{data-ingangen $d_i$}. Het is de bedoeling dat de encoder afhankelijk van de lijn waarop een 1 staat de index weergeeft op de \termen{selectie-uitgangen $s_i$}. Verder bevat de encoder ook nog een extra uitgang: de \termen{any-uitgang $a$}, op deze uitgang wordt 1 aangelegd indien \'e\'en van de data-ingangen een 1 vertoont, anders wordt er een 0 op de any-uitgang aangelegd. We kunnen dit gedrag samenvatten in een waarheidstabel zoals in tabel \ref{tbl:truthTableEncoderNormal} voor een 4-naar-2 encoder.
\begin{table}[hbt]
\centering
\subtable[Encoder]{
\begin{tabular}{cccc|ccc}
$d_3$&$d_2$&$d_1$&$d_0$&$a$&$f_1$&$f_0$\\\hline
0&0&0&0&0&-&-\\
0&0&0&1&1&0&0\\
0&0&1&0&1&0&1\\
0&1&0&0&1&1&0\\
1&0&0&0&1&1&1\\
\end{tabular}
\label{tbl:truthTableEncoderNormal}
}
\subtable[Prioriteitsencoder]{
\begin{tabular}{cccc|ccc}
$d_3$&$d_2$&$d_1$&$d_0$&$a$&$f_1$&$f_0$\\\hline
0&0&0&0&0&-&-\\
0&0&0&1&1&0&0\\
0&0&1&-&1&0&1\\
0&1&-&-&1&1&0\\
1&-&-&-&1&1&1\\
\end{tabular}
\label{tbl:truthTablePriorityEncoder}
}
\caption{Waarheidtabellen van een encoder en prioriteitsencoder.}
\end{table}
Merk op dat de tabel niet alle strikt mogelijke ingangen toont. We nemen aan dat aan de ingang enkel geldige toestanden verschijnen. Indien dit niet zo is, staat het in principe vrij om elke uitgang aan te leggen. De ontbrekende rijen bevatten dus don't cares op alle uitgangen. Op basis van deze tabel kunnen we eventueel Karnaugh-kaarten maken en een implementatie voorstellen:
\begin{equation}
\left\{
\begin{array}{l}
a=d_3+d_2+d_1+d_0\\
f_1=d_3+d_1\\
f_0=d_3+d_2
\end{array}
\right.
\end{equation}
We kunnen deze schakeling dus eenvoudig realiseren met \'e\'en OR-poort per uitgang.
\paragraph{Prioriteitsencoder}
Een variant van de encoder is de \termen{prioriteitsencoder}. Een prioriteitsencoder biedt een antwoord op de ongeldige ingangen van de encoder. Hierbij telt niet d\'e index van de data-ingang waarop een 1 wordt aangelegd. Maar de hoogste index van alle datalijnen met 1. Tabel \ref{tbl:truthTablePriorityEncoder} formaliseert dit. Deze schakeling kunnen we als volgt implementeren:
\begin{equation}
\left\{
\begin{array}{l}
a=d_3+d_2+d_1+d_0\\
f_1=d_3+d_2'd_1\\
f_0=d_3+d_2
\end{array}
\right.
\end{equation}
Wat overeenkomt met \'e\'en extra AND- en NOT-poort. De interface voor beide schakelingen staat beschreven op figuur \ref{fig:encoderInterface}.
\begin{figure}[hbt]
\centering
\subfigure[Interfaces]{
\begin{tikzpicture}
\node[encoder4to2] (E) at (0,0) {Encoder};
\draw[<-] (E.d0) -- ++(0,0.25) node[anchor=south,scale=0.75]{$d_0$};
\draw[<-] (E.d1) -- ++(0,0.25) node[anchor=south,scale=0.75]{$d_1$};
\draw[<-] (E.d2) -- ++(0,0.25) node[anchor=south,scale=0.75]{$d_2$};
\draw[<-] (E.d3) -- ++(0,0.25) node[anchor=south,scale=0.75]{$d_3$};
\draw[->] (E.f0) -- ++(0,-0.25) node[anchor=north,scale=0.75]{$f_0$};
\draw[->] (E.f1) -- ++(0,-0.25) node[anchor=north,scale=0.75]{$f_1$};
\draw[->] (E.any) -- ++(-0.25,0) node[anchor=east,scale=0.75]{$a$};
\node[encoder4to2,text width=1.6 cm] (PE) at (3,0) {Prioriteits-encoder};
\draw[<-] (PE.d0) -- ++(0,0.25) node[anchor=south,scale=0.75]{$d_0$};
\draw[<-] (PE.d1) -- ++(0,0.25) node[anchor=south,scale=0.75]{$d_1$};
\draw[<-] (PE.d2) -- ++(0,0.25) node[anchor=south,scale=0.75]{$d_2$};
\draw[<-] (PE.d3) -- ++(0,0.25) node[anchor=south,scale=0.75]{$d_3$};
\draw[->] (PE.f0) -- ++(0,-0.25) node[anchor=north,scale=0.75]{$f_0$};
\draw[->] (PE.f1) -- ++(0,-0.25) node[anchor=north,scale=0.75]{$f_1$};
\draw[->] (PE.any) -- ++(-0.25,0) node[anchor=east,scale=0.75]{$a$};
\end{tikzpicture}
\label{fig:encoderInterface}
}
\subfigure[Cascade]{
\begin{tikzpicture}
\foreach \x in {0,1,2,3} {
  \node[encoder4to2,text width=1.6 cm] (Ea\x) at (-3*\x,0) {Prioriteits-encoder};
}
\node[encoder4to2,text width=1.6 cm] (Eb3) at (-9,-3) {Prioriteits-encoder};
\node[mux4to1] (Mb2) at (-6,-3) {};
\node[mux4to1] (Mb1) at (-3,-3) {};
\foreach \x/\y/\z in {0/0/0,1/1/1,2/2/1,3/1/0} {
  \draw (Ea\x.any) -| (-3*\x-1.4,-2.4+0.1*\y) -| (Eb3.d\x);
  \draw (Ea\x.f1) -- (Ea\x.f1 |- 0,-1.6+0.1*\z) -| (Mb2.data\x);
  \draw (Ea\x.f0) -- (Ea\x.f0 |- 0,-1+0.1*\z) -| (Mb1.data\x);
}
\draw[<-] (Ea0.d0) -- ++(0,0.25) node[scale=0.75,anchor=south]{$d_0$};
\draw[<-] (Ea0.d1) -- ++(0,0.25) node[scale=0.75,anchor=south]{$d_1$};
\draw[<-] (Ea0.d2) -- ++(0,0.25) node[scale=0.75,anchor=south]{$d_2$};
\draw[<-] (Ea0.d3) -- ++(0,0.25) node[scale=0.75,anchor=south]{$d_3$};
\draw[<-] (Ea1.d0) -- ++(0,0.25) node[scale=0.75,anchor=south]{$d_4$};
\draw[<-] (Ea1.d1) -- ++(0,0.25) node[scale=0.75,anchor=south]{$d_5$};
\draw[<-] (Ea1.d2) -- ++(0,0.25) node[scale=0.75,anchor=south]{$d_6$};
\draw[<-] (Ea1.d3) -- ++(0,0.25) node[scale=0.75,anchor=south]{$d_7$};
\draw[<-] (Ea2.d0) -- ++(0,0.25) node[scale=0.75,anchor=south]{$d_8$};
\draw[<-] (Ea2.d1) -- ++(0,0.25) node[scale=0.75,anchor=south]{$d_9$};
\draw[<-] (Ea2.d2) -- ++(0,0.25) node[scale=0.75,anchor=south]{$d_{10}$};
\draw[<-] (Ea2.d3) -- ++(0,0.25) node[scale=0.75,anchor=south]{$d_{11}$};
\draw[<-] (Ea3.d0) -- ++(0,0.25) node[scale=0.75,anchor=south]{$d_{12}$};
\draw[<-] (Ea3.d1) -- ++(0,0.25) node[scale=0.75,anchor=south]{$d_{13}$};
\draw[<-] (Ea3.d2) -- ++(0,0.25) node[scale=0.75,anchor=south]{$d_{14}$};
\draw[<-] (Ea3.d3) -- ++(0,0.25) node[scale=0.75,anchor=south]{$d_{15}$};
\draw[->] (Eb3.any) -- ++(-0.25,0) node[scale=0.75,anchor=east]{$a$};
\draw[->] (Eb3.f1) -- (Eb3.f1 |- 0,-4.5) node[scale=0.75,anchor=north]{$f_3$};
\draw[->] (Eb3.f0) -- (Eb3.f0 |- 0,-4.5) node[scale=0.75,anchor=north]{$f_2$};
\draw[->] (Mb2.output) -- (Mb2.output |- 0,-4.5) node[scale=0.75,anchor=north]{$f_1$};
\draw[->] (Mb1.output) -- (Mb1.output |- 0,-4.5) node[scale=0.75,anchor=north]{$f_0$};
\draw (Eb3.f1 |- 0,-3.8) -- (-4.5,-3.8) |- (Mb1.selin1);
\draw (-7.5,-3.8) |- (Mb2.selin1);
\draw (-7.3,-4) |- (Mb2.selin0);
\draw (Eb3.f0 |- 0,-4) -- (-4.3,-4) |- (Mb1.selin0);
\pdot{Eb3.f1 |- 0,-3.8};
\pdot{-7.5,-3.8};
\pdot{-7.3,-4};
\pdot{Eb3.f0 |- 0,-4};
\end{tikzpicture}
\label{fig:encoderCascade}
}
\caption{Encoder en Prioriteitsencoder.}
\end{figure}
De interface is opnieuw een rechthoek. De data-ingangen staan bovenaan. De selectie-uitgangen onderaan, en de any-uitgang aan de linkerkant.
\subsubsection{Prioriteitsencoders cascaderen}
Ook bij prioriteitsencoders gaan we even in het op het cascaderend karakter. Ook encoders laten zich eenvoudig cascaderen, mits we ook extra multiplexers gebruiken. In figuur \ref{fig:encoderCascade} bouwen we een 16-naar-4 prioriteitsencoder met 4 4-naar-2 prioriteitsencoders en 2 4-naar-1 multiplexers.
\subsection{Vergelijker}
\label{ss:comparator}
Een \termen{vergelijker} of \termen{comparator} is een component die in staat is om uitspraken te doen over hoe twee getallen zich tot elkaar verhouden. Meestal is een vergelijker in staat om een uitspraak te doen over 4 mogelijke verhoudingen van de getallen $X$ en $Y$:
\begin{equation}
\left\{\begin{array}{c}
X<Y\\
X>Y\\
X=Y\\
X\neq Y
\end{array}\right.
\end{equation}
Op basis van de eerste twee relaties, kunnen we uitspraken doen over de laatste twee. Op basis van deze vaststelling synthetiseren we een comparator. Een comparator bevat 4 ingangen ($x_0$ ofwel $g_{\mbox{\tiny{in}}}$, $y_0$ ofwel $l_{\mbox{\tiny{in}}}$, $x_1$ en $y_1$). De twee uitgangen ($g$ en $l$) geven respectievelijk weer of het getal $X>Y$ en $X<Y$. Figuur \ref{fig:comparatorInterface} toont de interface van een vergelijker.
\begin{figure}[hbt]
\centering
\subfigure[Interface]{
\begin{tikzpicture}
\node[comp] (C) at (0,0) {Comp};
\draw[<-] (C.x0) -- ++(0.25,0) node[anchor=west,scale=0.75]{$x_0/g_{\mbox{\tiny{in}}}$};
\draw[<-] (C.y0) -- ++(0.25,0) node[anchor=west,scale=0.75]{$y_0/l_{\mbox{\tiny{in}}}$};
\draw[<-] (C.x1) -- ++(0,0.25) node[anchor=south,scale=0.75]{$x_1$};
\draw[<-] (C.y1) -- ++(0,0.25) node[anchor=south,scale=0.75]{$y_1$};
\draw[->] (C.g) -- ++(-0.25,0) node[anchor=east,scale=0.75]{$g$};
\draw[->] (C.l) -- ++(-0.25,0) node[anchor=east,scale=0.75]{$l$};
\end{tikzpicture}
\label{fig:comparatorInterface}
}
\subfigure[Karnaugh-kaarten]{
\begin{tikzpicture}
\kkaartd{0}{0}{$g$}{$x_1$/$y_1$/$x_0$/$y_0$}{0/0/1/0/0/0/0/0/1/1/1/1/0/0/1/0};
\kkaartd{3}{0}{$l$}{$x_1$/$y_1$/$x_0$/$y_0$}{0/1/0/0/1/1/1/1/0/0/0/0/0/1/0/0};
\end{tikzpicture}
\label{fig:comparatorKarnaugh}
}
\subfigure[Lineare cascade]{
\begin{tikzpicture}[scale=0.75]
\node[comp,scale=0.75,draw=white] (C0) at (0,0) {};
\foreach \x in {1,...,7} {
  \node[comp,scale=0.75] (C\x) at (-1.5*\x,0) {Comp};
  \draw[<-] (C\x.x1) -- ++(0,0.25) node[anchor=south,scale=0.75]{$x_{\x}$};
  \draw[<-] (C\x.y1) -- ++(0,0.25) node[anchor=south,scale=0.75]{$y_{\x}$};
}
\foreach \x/\y in {1/2,2/3,3/4,4/5,5/6,6/7} {
  \draw (C\x.g) -- (C\y.x0);
  \draw (C\x.l) -- (C\y.y0);
}
\draw[<-] (C1.x0) -| (C0.x1) -- ++(0,0.25) node[anchor=south,scale=0.75]{$x_0$};
\draw[<-] (C1.y0) -| (C0.y1) -- ++(0,0.25) node[anchor=south,scale=0.75]{$y_0$};
\draw[->] (C7.g) -- ++(-0.25,0) node[anchor=east,scale=0.75]{$g$};
\draw[->] (C7.l) -- ++(-0.25,0) node[anchor=east,scale=0.75]{$l$};
\end{tikzpicture}
\label{fig:comparatorCascadeLinear}
}
\subfigure[Hi\"erarchische cascade]{
\begin{tikzpicture}[scale=0.75]
\foreach \x in {1,3,5,7} {
  \node[comp,scale=0.75] (C\x) at (-1.5*\x,0) {Comp};
  \draw[<-] (C\x.x1) -- ++(0,0.25) node[anchor=south,scale=0.75]{$x_{\x}$};
  \draw[<-] (C\x.y1) -- ++(0,0.25) node[anchor=south,scale=0.75]{$y_{\x}$};
}
\foreach \x/\y in {0/1,2/3,4/5,6/7} {
  \node[comp,scale=0.75,draw=white] (C\x) at (-1.5*\x,0) {};
  \draw[<-] (C\y.x0) -| (C\x.x1) -- ++(0,0.25) node[anchor=south,scale=0.75]{$x_{\x}$};
  \draw[<-] (C\y.y0) -| (C\x.y1) -- ++(0,0.25) node[anchor=south,scale=0.75]{$y_{\x}$};
}
\foreach \x/\y in {3/1,7/5} {
  \node[comp,scale=0.75] (Cb\x) at (-1.5*\x,-2) {Comp};
  \draw (C\x.g) -| ++(-0.4,-1.2) -| (Cb\x.x1);
  \draw (C\x.l) -| ++(-0.2,-0.6) -| (Cb\x.y1);
  \draw (C\y.g) -- ++(-0.4,0) |- (Cb\x.x0);
  \draw (C\y.l) -- ++(-0.2,0) |- (Cb\x.y0);
}
\foreach \x/\y in {7/3} {
  \node[comp,scale=0.75] (Cc\x) at (-1.5*\x,-4) {Comp};
  \draw (Cb\x.g) -| ++(-0.4,-1.2) -| (Cc\x.x1);
  \draw (Cb\x.l) -| ++(-0.2,-0.6) -| (Cc\x.y1);
  \draw (Cb\y.g) -- ++(-0.4,0) |- (Cc\x.x0);
  \draw (Cb\y.l) -- ++(-0.2,0) |- (Cc\x.y0);
}
\draw[->] (Cc7.g) -- ++(-0.25,0) node[anchor=east,scale=0.75]{$g$};
\draw[->] (Cc7.l) -- ++(-0.25,0) node[anchor=east,scale=0.75]{$l$};
\end{tikzpicture}
\label{fig:comparatorCascadeHierarchical}
}
\caption{Vergelijker}
\end{figure}
Op basis van de gedragsbeschrijving kunnen we een waarheidstabel en Karnaugh-kaarten opstellen zoals op figuur \ref{fig:comparatorKarnaugh}. Op basis van deze kaarten synthetiseren we volgende formules\footnote{Merk de dualiteit op: beide formules hebben dezelfde variabelen, alleen zijn alle atomen ge\"inverteerd.}:
\begin{equation}
\left\{\begin{array}{l}
g=x_1y_1'+x_0x_1y_0'+x_0y_0'y_1'\\
l=x_1'y_1+x_0'x_1'y_0+x_0'y_0y_1
\end{array}\right.
\end{equation}
\subsubsection{Vergelijkers cascaderen}
Hoe vergelijken we nu twee getallen die uit meer dan 2-bit bestaan? Net als bij een ripple-carry opteller, kunnen we door verschillende vergelijkers aan elkaar te schakelen, een vergelijker met een groter aantal bits bouwen. In dat geval bekomen we een structuur zoals op figuur \ref{fig:comparatorCascadeLinear}. Hiermee komen we echter tot hetzelfde probleem als bij een ripple-carry opteller: de vertraging schaalt linear met het aantal bits. We kunnen echter de vergelijkers ook in een hi\"erarchische structuur kunnen we de vertraging beperken tot een logaritmische orde zoals op figuur \ref{fig:comparatorCascadeHierarchical}.
\subsubsection{Speciale gevallen}
\paragraph{Testen op gelijkheid}
We kunnen twee getallen altijd vergelijken met een vergelijker zoals eerder beschreven. Soms willen we echter twee getallen testen op specifieke eigenschappen. Bijvoorbeeld of twee getallen gelijk zijn. Dit kunnen we natuurlijk realiseren met een vergelijker, maar dit introduceert extra hardware en vertraging. In dat geval kunnen we gebruik maken van een rij van XNOR-poorten waarbij elke poort \'e\'en bit van het ene en het andere getal als invoer krijgt. De uitgangen van al deze XNOR-poorten laten we vervolgens door een OR-poort gaan. De uitvoer van deze OR poort toont ons dan of $X$ gelijk is aan $Y$. Figuur \ref{fig:comparatorEquality} toont een realisatie van dit concept voor 2 $n$-bit getallen.
\begin{figure}[hbt]
\centering
\subfigure[Gelijkheid]{
\begin{tikzpicture}[circuit logic US]
\node[or gate,inputs={normal,normal,normal,normal,normal,normal,normal},scale=0.5,rotate=-90] (O) at (-3,-1.5) {};
\node[scale=0.75] (P) at (-5,0) {$\ldots$};
\draw (O.output) -- ++(0,-0.25) node[scale=0.75,anchor=north]{$X=Y$};
\foreach \x in {0,1,2,3,4,6} {
  \coordinate (y\x) at (0.25-\x,0.75);
  \coordinate (x\x) at (-0.25-\x,0.75);
}
\foreach \x/\y/\z/\t in {0/1/3/0,1/2/2/1,2/3/1/2,3/4/0/3,4/5/1/4,6/7/3/n-1} {
  \node[xnor gate,rotate=-90] (XN\x) at (-\x,0) {};
  \draw (XN\x.input 1) -- ++(0,0.2) -| (y\x) node[scale=0.75,anchor=south]{$y_{\t}$};
  \draw (XN\x.input 2) -- ++(0,0.2) -| (x\x) node[scale=0.75,anchor=south]{$x_{\t}$};
  \draw (XN\x.output) -- ++(0,-0.1*\z) -| (O.input \y);
}
\end{tikzpicture}
\label{fig:comparatorEquality}}
\subfigure[Testen met constanten]{
\begin{tikzpicture}[circuit logic US]
\node[nor gate,inputs={normal,normal,normal,normal,normal}] (O1) at (0,0) {};
\draw (O1.output) -- ++(0.25,0) node[scale=0.75,anchor=west]{$X=0$};
\node[and gate,inputs={normal,normal,normal,normal,normal}] (O2) at (0,-1.5) {};
\draw (O2.output) -- ++(0.25,0) node[scale=0.75,anchor=west]{$X=2^n-1$};
\node[not gate] (O3) at (0,-3) {};
\draw (O3.input) -- ++(-0.25,0) node[scale=0.75,anchor=east]{$x_0$};
\draw (O3.output) -- ++(0.25,0) node[scale=0.75,anchor=west]{$\mbox{even}\left(X\right)$};
\foreach \x/\t in {5/$x_0$,4/$x_1$,3/$x_2$,2/$\ldots$,1/$x_{n-1}$} {
  \draw (O1.input \x) -- ++(-0.25,0) node[scale=0.75,anchor=east]{\t};
  \draw (O2.input \x) -- ++(-0.25,0) node[scale=0.75,anchor=east]{\t};
}
\node[or gate,inputs={normal,normal,normal,normal,normal}] (O4) at (4,0) {};
\draw (O4.output) -- ++(0.25,0) node[scale=0.75,anchor=west]{$X\geq 2^k$};
\node[nand gate,inputs={normal,normal,normal,normal,normal}] (O5) at (4,-1.5) {};
\draw (O5.output) -- ++(0.25,0) node[scale=0.75,anchor=west]{$X<2^n-2^k$};
\draw (O5.output |- 0,-3) -- (O5.input 1 |- 0,-3);
\draw (O5.input 1 |- 0,-3) -- ++(-0.25,0) node[scale=0.75,anchor=east]{$x_0$};
\draw (O5.output |- 0,-3) -- ++(0.25,0) node[scale=0.75,anchor=west]{$\mbox{oneven}\left(X\right)$};
\foreach \x/\t in {5/$x_k$,4/$x_{k+1}$,3/$x_{k+2}$,2/$\ldots$,1/$x_{n-1}$} {
  \draw (O4.input \x) -- ++(-0.25,0) node[scale=0.75,anchor=east]{\t};
  \draw (O5.input \x) -- ++(-0.25,0) node[scale=0.75,anchor=east]{\t};
}
\end{tikzpicture}
\label{fig:comparatorConstants}}
\caption{Speciale gevallen van vergelijkers}
\end{figure}
\paragraph{Vergelijken met constanten}
We willen een getal niet altijd vergelijken met andere getal, maar soms met een constante. In dat geval zouden we natuurlijk ook van de vergelijker kunnen gebruikmaken, en de $Y$ zelf samenstellen. Aangezien $Y$ echter op voorhand gekend is spreekt het voor zichzelf dat we de implementatie echter grondig kunnen verbeteren. Bovendien kunnen we ook testen ontwerpen die we niet kunnen uitvoeren met een vergelijker\footnote{Bijvoorbeeld deelbaarheid door 2.}. Dergelijke componenten bouwen getuigd verder ook van enig inzicht. Alle speciale gevallen beschouwen is moeilijk. Op figuur \ref{fig:comparatorConstants} geven we enkele voorbeelden.
\subsection{Schuifoperator}
\label{ss:shiftoperators}
Een laatste set van operaties die vaak gebruikt wordt zijn \termen{schuifoperaties}. Vele processoren implementeren schuifoperaties en bijvoorbeeld de ARM processor laat toe om tijdens elke operatie de operand over enkele plaatsen te schuiven. Een eerste probleem met schuifoperaties is dat er verschillende vormen van schuifoperaties bestaan. In het algemeen houdt een schuifoperatie in dat de waarde van de bit die eerst op plaats $i$ stond, nu op een plaats $i+m$ staat. Of formeler: $Y$ is het resultaat van een schuifoperatie van $X$ over $m$ plaatsen naar links als
\begin{equation}
\forall i\in\NN:i\in\left[0,n\right]\wedge i+m\in\left[0,n\right]\Rightarrow y_{i+m}=x_i
\end{equation}
Het probleem is wat we doen met de bits die buiten de grenzen vallen, en met wat we de nieuwe plaatsen opvullen. Voor dit probleem bestaan drie populaire oplossingen:
\begin{itemize}
 \item Bij het \termen{schuiven} negeren we de bits die buiten de grenzen vallen. Er bestaan twee vormen van schuiven die vari\"eren in wat we op de vrijgekomen plaatsen zetten:
 \begin{itemize}
   \item \termen{Logisch schuiven}: indien we logisch schuiven hebben we een extra parameter nodig, namelijk welke waarde we op de nieuwe plaatsen zetten. Soms is dit zelfs een rij van bits. We maken dus de invoer groot genoeg om geen vrije plaatsen meer over te houden.
   \item \termen{Aritmetisch schuiven}: hierbij willen we eigenlijk een wiskundige functie realiseren namelijk vermenigvuldigen of delen met een macht van 2. Hoe we dus aritmetisch schuiven hangt vooral af van de getalvoorstelling. Indien we naar links schuiven betekent dit meestal dat we de nieuwe plaatsen vullen met nullen. Indien we naar rechts schuiven zal bij een 2-complement voorstelling de hoogste bit (MSB) van het originele getal ingevoegd worden. Bij een ``unsigned'' voorstelling vullen we de vrije plaatsen ook met nullen op. Aritmetisch schuiven is vrij populair in programmeertalen. Talen die behoren tot de \verb|C/C++/C#/Java| familie introduceren daarom 2 functies: voor links (\verb+<<+) en rechts (\verb+>>+) aritmetisch schuiven deze worden gedefinieerd als:
\begin{equation}
\left\{\begin{array}{l}
X\verb+<<+M\equiv X\times 2^M\\
X\verb+>>+M\equiv X\div 2^M
\end{array}\right.
\end{equation}
 \end{itemize}
 \item Bij het \termen{roteren} vangen we de bits op die er langs \'e\'en kant afvallen en plaatsen we deze op de vrijgekomen plaatsen aan de andere kant.
\end{itemize}
\subsubsection{Implementatie van schuifoperaties}
In deze subsubsectie zullen we twee schakelingen realiseren. De eerste is een component die alle schuifoperaties kan realiseren op 4-bit getallen. Maar slechts over 1 plaats naar links of rechts. Het tweede component is 8-bit \termen{barrel left rotator}. Dit component roteert 8-bit getallen naar links over een variabel aantal plaatsen. Merk op dat we bij een rotatie het eigenlijk niet uitmaakt in welke richting we roteren: een $n$-bit getal $m$ plaatsen naar links roteren is net hetzelfde als het getal $n-m$ plaatsen naar rechts roteren. Dit tweede component illustreert verder hoe we schuifoperaties effici\"ent over meerdere posities kunnen uitvoeren.
\paragraph{Schuifoperaties over 1 bit}
Indien we een component maken die meerdere operaties kan uitvoeren moeten we altijd eerst een instructieset defini\"eren, zoals in subsubsectie \ref{sss:aLUInstructionSet}. De instructieset dient zowel een onderscheid te maken tussen schuiven of roteren, aritmetisch of logisch\footnote{Enkel in geval van schuiven is dit relevant.} en links of rechts. Verder willen we ook een bit voorzien die aangeeft of er \"uberhaupt een schuifoperatie moet worden uitgevoerd. We zullen dus een 4-bit instructieset gebruiken. Tabel \ref{tbl:shiftInstructionSet} toont de betekenis van elke bit.
\begin{table}[hbt]
\centering
\begin{tabular}{c|cc}
Signaal&0&1\\\hline
$s_3$&geen schuifoperatie&schuifoperatie\\
$s_2$&links&rechts\\
$s_1$&schuiven&roteren\\
$s_0$&aritmetisch&logisch
\end{tabular}
\caption{Instructieset voor de schuifoperaties over 1 bit.}
\label{tbl:shiftInstructionSet}
\end{table}
Op basis van deze instructieset kunnen we nu een component bouwen. Voor de berekening van de uitgangsbits gebruiken we multiplexers. Immers kan elke uitgang $y_i$ maar drie mogelijke uitgangen hebben: $x_{i-1}$, $x_i$ en $x_{i+1}$. Bij de uitgangen aan de rand dienen we alleen een andere interpretatie voor sommige $x_j$ te vinden. Welke van deze drie ingangen we kiezen hangt verder alleen af van twee bits uit het instructiewoord: $s_3$ en $s_2$. In het geval dat $s_3=0$ geldt $y_i=x_i$. Bijgevolg zetten we $x_i$ zowel op de $d_0$ en $d_1$ ingang van de multiplexer voor $y_i$. Indien $s_3=1$ en $s_2=0$ schuiven we naar links. We zetten dus $x_{i+1}$ en $x_{i-1}$ respectievelijk op de $d_2$ en $d_3$ ingangen. Vervolgens dienen we nog de randgevallen op te lossen. Deze randgevallen beslaan enkel $x_{i+1}$ voor $y_3$ en  $x_{i-1}$ voor $y_0$. Deze waarden zullen we respectievelijk noteren als $x_4$ en $x_{-1}$ en vallen dus buiten de grenzen van de ingangen. In geval van rotatie geldt:
\begin{equation}
\begin{array}{ll}
\left\{\begin{array}{l}
x_{-1}=x_3\\
x_4=x_0
\end{array}\right.&\mbox{(rotatie)}
\end{array}
\end{equation}
Dit dwingen we af met twee 2-naar-1 multiplexers. Deze multiplexers hebben als schakelelement instructiebit $s_1$. We dienen nu enkel nog het geval te behandelen waarin we schuiven. Indien we logisch schuiven, schuiven we de bits $L_{\mbox{\tiny{in}}}$ en $R_{\mbox{\tiny{in}}}$ in. Bij arithmetisch zullen we aan de rechterzijde een 0 inschuiven. Aan de linkerkant is dit ook het geval tenzij we schuiven met een 2-complement voorstelling. In dat laatste geval bepaald de hoogste bit immers ook het teken van het getal. In dat geval moeten we de waarde van de hoogste bit dus nogmaals inschuiven. We kunnen bovenstaande beschrijvingen formaliseren tot volgende formules:
\begin{equation}
\begin{array}{ll}
\left\{\begin{array}{l}
x_{-1}=R_{\mbox{\tiny{in}}}\\
x_4=L_{\mbox{\tiny{in}}}
\end{array}\right.&\mbox{(schuiven, aritmetisch)}\\\\\left\{\begin{array}{l}
x_{-1}=0\\
x_4=0
\end{array}\right.&\mbox{(schuiven, logisch, unsigned)}\\\\\left\{\begin{array}{l}
x_{-1}=0\\
x_4=x_3
\end{array}\right.&\mbox{(schuiven, logisch, 2-complement)}
\end{array}
\end{equation}
Deze logica kunnen we vervolgens implementeren met OR-AND-poorten. Het resultaat van deze volledige implementatie staat op figuur \ref{fig:shiftImplementation}.
\begin{figure}[hbt]
\centering
\begin{tikzpicture}[circuit logic US]
\def\xxi{4};
\foreach \x in {0,...,3} {
  \node[mux4to1] (M\x) at (-2*\x,0) {};
  \draw (M\x.output) -- ++(0,-0.25) node[scale=0.75,anchor=north]{$y_{\x}$};
  \draw (M\x.data1) |- (M\x.data0 |- 0,0.5);
  \pdot{M\x.data0 |- 0,0.5};
  \draw (M\x.data0) -- (M\x.data0 |- 0,\xxi) node[scale=0.75,anchor=south]{$x_{\x}$};
}
\node[mux2to1] (MR) at (1.5,1.5) {};
\node[mux2to1] (ML) at (-7,1.5) {};
\draw (ML.selout0) -- (MR.selin0);
\draw (ML.selin0) -- (-8.5,0 |- ML.selin0) node[scale=0.75,anchor=east]{$s_1$};
\draw (M3.selin0) -- (-8.5,0 |- M3.selin0) node[scale=0.75,anchor=east]{$s_2$};
\draw (M3.selin1) -- (-8.5,0 |- M3.selin1) node[scale=0.75,anchor=east]{$s_3$};
\foreach \x/\y/\z in {0/1/1,1/2/0,2/3/1} {
  \draw (M\x.selin0) -- (M\y.selout0);
  \draw (M\x.selin1) -- (M\y.selout1);
  \draw (M\y.data0 |- 0,0.75) -| (M\x.data2);
  \pdot{M\y.data0 |- 0,0.75};
  \draw (M\x.data0 |- 0,1+0.25*\z) -| (M\y.data3);
  \pdot{M\x.data0 |- 0,1+0.25*\z};
}
\draw (ML.output) -- (ML.output |- 0,1) -| (M3.data2);
\draw (MR.output) -- (MR.output |- 0,1) -| (M0.data3);
\draw (ML.data1) |- (M0.data0 |- 0,1.75);
\pdot{M0.data0 |- 0,1.75};
\draw (MR.data1) |- (M3.data0 |- 0,2);
\pdot{M3.data0 |- 0,2};
\node[and gate,rotate=-90,anchor=output,scale=0.75] (AR) at (MR.data0 |- 0,2) {};
\draw (AR.output) -- (MR.data0);
\draw (AR.input 1) -- (AR.input 1 |- 0,\xxi) node[anchor=south,scale=0.75]{$R_{\mbox{\tiny{in}}}$};
\node[or gate,rotate=-90,anchor=output,scale=0.75] (OL) at (ML.data0 |- 0,1.875) {};
\draw (OL.output) -- (ML.data0);
\node[and gate,rotate=-90,anchor=north,scale=0.75] (ALa) at (OL.input 2 |- 0,3.1) {};
\node[and gate,rotate=-90,inputs={normal,normal,inverted},anchor=south,scale=0.75] (ALb) at (OL.input 1 |- 0,3.1) {};
\draw (ALa.output) -- ++(0,-0.125) -| (OL.input 2);
\draw (ALb.output) -- ++(0,-0.125) -| (OL.input 1);
\draw (ALa.input 2) |- (AR.input 2 |- 0,3.75) -- (AR.input 2);
\draw (ALa.input 2 |- 0,3.75) -- (-8.5,3.75) node[scale=0.75,anchor=east]{$s_0$};
\pdot{ALa.input 2 |- 0,3.75};
\draw (ALb.input 3 |- 0,3.75) -- (ALb.input 3);
\pdot{ALb.input 3 |- 0,3.75};
\draw (ALb.input 1) |- (M3.data0 |- 0,3.5);
\pdot{M3.data0 |- 0,3.5};
\draw (ALa.input 1) -- (ALa.input 1 |- 0,\xxi) node[anchor=south,scale=0.75]{$L_{\mbox{\tiny{in}}}$};
\draw (ALb.input 2) -- (ALb.input 2 |- 0,\xxi) node[anchor=west,scale=0.75,rotate=90]{2-comp};
\end{tikzpicture}
\caption{Implementatie van een schuifoperator over 1 bit.}
\label{fig:shiftImplementation}
\end{figure}
\paragraph{8-bit barrel left rotator}
In de vorige paragraaf hebben we een schuifoperator gebouwd die 1-bit kan schuiven. Door verschillende van deze schuifoperatoren na elkaar te plaatsen kunnen we schuiven over meerdere plaatsen. Toch is dit niet erg praktisch: om $m$ plaatsen te schuiven zouden we $m$ van deze schuifoperatoren na elkaar moeten plaatsen, wat grote vertragingen zou impliceren. In deze paragraaf zullen we een rotator bouwen die de vertraging beperkt tot $\log_2 m$, met $m$ het maximaal aantal plaatsen dat kan opgeschoven worden. Het concept is echter ook toepasbaar op algemene schuifoperaties. Alleen is de implementatie te complex voor pedagogische doeleinden. Figuur \ref{fig:hBitBarrelLeftRotatorImplementation} geeft het concept mooi weer.
\begin{figure}[hbt]
\centering
\begin{tikzpicture}
\foreach \y in {0,1,2} {
  \foreach \x in {0,...,7} {
    \node[mux2to1] (M\y\x) at (-1.5*\x,1.5*\y) {};
  }
  \foreach \a/\b in {0/1,1/2,2/3,3/4,4/5,5/6,6/7} {
   \draw (M\y\b.selout0) -- (M\y\a.selin0);
  }
  \draw (M\y7.selin0) -- (M\y7.selin0 -| -11.5,0) node[scale=0.75,anchor=east]{$s_{\y}$};
}
\foreach \y/\yi in {0/1,1/2} {
  \foreach \x in {0,...,7} {
    \draw (M\yi\x.output) -- (M\y\x.data0);
    \pdot{$(M\yi\x.output)!0.25!(M\y\x.data0)$};
  }
}
\foreach \x/\xa/\xb/\xc in {0/4/2/1,1/5/3/2,2/6/4/3,3/7/5/4,4/0/6/5,5/1/7/6,6/2/0/7,7/3/1/0} {
  \draw (M2\x.data0) -- (M2\x.data0 |- 0,4.5) node[scale=0.75,anchor=south]{$x_{\x}$};
  \pdot{M2\x.data0 |- 0,4.25};
  \draw (M0\x.output) -- (M0\x.output |- 0,-0.5) node[scale=0.75,anchor=north]{$y_{\x}$};
  \draw (M2\x.data0 |- 0,4.25) -- (M2\xa.data1);
  \draw ($(M2\x.output)!0.25!(M1\x.data0)$) -- (M1\xb.data1);
  \draw ($(M1\x.output)!0.25!(M0\x.data0)$) -- (M0\xc.data1);
}
\end{tikzpicture}
\caption{Implementatie van een 8-bit barrel left rotator.}
\label{fig:hBitBarrelLeftRotatorImplementation}
\end{figure}
Deze barrel left rotator is gebaseerd op het additief principe. Het additief principe stelt dat indien we een getal willen schuiven of roteren over $m$ bits, we dit ook kunnen verwezenlijken door eerst te schuiven/roteren over $m_1$ bits en vervolgens over $m_2$ bits met $m=m_1+m_2$. Indien we onze rotator bouwen volgens een structuur waarbij elke niveau $i$ een rotatie uitvoert over $2^i$ plaatsen kunnen we elke rotatie-operatie uitvoeren. Op figuur \ref{fig:hBitBarrelLeftRotatorImplementation} zien we dat het niveau $s_2$, vier plaatsen naar links roteert. Het volgende niveau twee en het laatste \'e\'en. Elk van deze niveaus heeft dezelfde vertraging. Indien we dus een $n$-bit barrel left rotator willen bouwen die maximaal $m$ plaatsen kan opschuiven moeten we dus $\left\lceil\log_2m\right\rceil$ niveaus synthetiseren. Een niveau $i$ roteert $2^i$ plaatsen voor $\forall i=0,1,\ldots,\left\lceil\log_2m\right\rceil-1$. Dit concept is eenvoudig te veralgemenen naar een volledige schuifoperator. Merk op dat het aantal niveaus dus theoretisch niet afhangt van het aantal bit in het getal. Op de meeste processoren neemt men echter $m=n$. Een andere mooie eigenschap is dat de volgorde van de niveaus niet relevant is.
%\chapter{Sequenti\"ele Schakelingen (Schakelingen met geheugen)}
\label{ch:SeqComp}
\chapterquote{We denken dat sommige mensen intelligent zijn, terwijl ze alleen maar een goed geheugen hebben.}{Fr\'ed\'eric Dard, Frans humoristisch schrijver (1921-2000)}
\begin{chapterintro}
??
\end{chapterintro}
\minitoc[n]
\section{Terminologie}
Een \termen{Sequenti\"ele schakeling} is een schakeling waarbij niet alleen de ingangen $X$ van belang zijn, maar ook de toestand van deze schakeling. De \termen{toestand $S$} wordt bepaald door de ingang, en door de vorige toestand $S_{\mbox{\tiny{prev.}}}$. In de wiskunde wordt een dergelijke constructie beschreven als een \termen{eindige-toestanden machine} of \termen{finite state machine (FSM)}. Om dit te verwezenlijken hebben we een \termen{geheugencomponent} nodig, een component die de toestand bijhoudt. Dit kunnen we bijvoorbeeld verwezenlijken met een condensator. In DRAM geheugens wordt dergelijke implementatie gebruikt. Het probleem met een condensator is dat de lading na verloop van tijd verloren gaat, wat tot dynamische logica leidt. In DRAM wordt dit opgelost door de condensatoren in kwestie terug op te laden. Een alternatief zijn componenten met \termen{positieve terugkoppeling}, doorgaans zijn deze bekend als \termen{flipflop} of \termen{register}. In sectie \ref{s:memoryBlocks} zullen we verschillende van deze bouwblokken bespreken.
\subsection{Classificatie van sequenti\"ele schakelingen}
\label{ss:classificationSequential}
Doorgaans kunnen we sequenti\"ele schakelingen onderverdelen volgens twee classificatiesystemen: gebaseerd op de uitgangsfunctie $F$, en de synchroniciteit van de schakeling. Wat betreft de uitgangsfunctie beschouwen we twee types:
\begin{itemize}
 \item \termen{Toestandsgebonden sequenti\"ele schakelingen}: In dit geval hangt de uitgangsfunctie enkel af van de toestand op dat moment. In dat geval is de uitgangsfunctie $F\left(S\right)$. Dit concept is ook bekend als de \termen{Moore machine} ofwel \termen{Moore-FSM}. Bij een Moore machine is de invloed van de ingang dus met vertraging te zien, vermits de schakeling eerst van toestand moet veranderen.
 \item \termen{Inputgebonden sequenti\"ele schakelingen}: Hierbij wordt de signatuur van de uitgangsfunctie uitgebreid. De uitgang is zowel afhankelijk van de toestand $S$ als van de ingang $X$. Dit betekent dat indien de ingang verandert, maar de toestand niet de uitvoer $F\left(S,X\right)$ ook verandert. Dit concept wordt ook wel de \termen{Mealy machine} ofwel \termen{Mealy-FSM} genoemd. Elke Moore machine is dus ook een Mealy machine waar de invoer geen onderdeel uitmaakt van de logica die de uitgang berekend.
\end{itemize}
Verder maken we ook een onderscheid in schakelingen inzake synchroniciteit, ook hier beschouwen we twee soorten:
\begin{itemize}
 \item \termen{Asynchrone sequenti\"ele schakelingen}: Hierbij verandert uitgang $F$ en toestand $S$ wanneer de ingang $X$ verandert. We zullen asynchrone schakelingen kort in sectie \ref{s:asynchroneSequence}, bovendien zullen we veronderstellen dat slechts \'e\'en bit tegelijk verandert aan de ingang.
 \item \termen{Synchrone sequenti\"ele schakelingen}: Hierbij veranderen de uitgang $F$ en toestand $S$ enkel op het moment de zogenoemde \termen{klokingang} verandert.
Veruit de meeste sequenti\"ele schakelingen worden gebouwd op basis van een klok. In sectie \ref{s:synchroneSequence} zullen we uitgebreid synchrone schakelingen en hun synthese bespreken.
\end{itemize}
\subsection{Terminologie van het kloksignaal}
Bij synchrone schakelingen maken we gebruik van een klokingang. Op deze klokingang staat periodiek een 0 gevolgd door een 1. Om dit kloksignaal te beschrijven maken we gebruik van de volgende terminologie:
\begin{itemize}
 \item \termen{Klokperiode}: de tijd tussen twee opeenvolgende klokovergangen van 0 naar 1.
 \item \termen{Klokfrequentie}: het aantal klokperiodes per seconden of formeler:
\begin{equation}
\mbox{klokfrequentie}=\displaystyle\frac{1}{\mbox{klokperiode}}
\end{equation}
 \item ``\termen{Duty cycle}'': fragment van de klokcyclus die dat de klok op 1 staat.
\begin{equation}
\mbox{duty cycle}=\displaystyle\frac{\mbox{tijd klok is 1}}{\mbox{klokperiode}}
\end{equation}
De klok staat dus niet per definitie even lang op 0 als op 1.
 \item \termen{Stijgende flank} (ook wel \termen{rising edge} genoemd): de klokovergang waarbij de klok van 0 naar 1 gaat.
 \item \termen{Dalende flank} (ook wel \termen{falling edge} genoemd): de klokovergang waarbij de klok van 1 naar 0 gaat.
\end{itemize}
\section{Bouwblokken}
\label{s:memory}
\label{s:memoryBlocks}
In deze sectie zullen we de bouwblokken van geheugen ontwikkelen. Hierbij zullen we in subsectie \ref{ss:flipflop} eerst het basisblok ontwikkelen om \'e\'en bit te onthouden: de flipflop. Door flipflops te groeperen kunnen we een groter geheugen ontwikkelen. Dergelijke geheugens worden registers genoemd. Registers kunnen meestal maar een beperkt aantal flipflops tegelijk aanspreken. Registers bespreken we in subsectie \ref{ss:registers}. Een andere populaire toepassing van geheugens is een teller. We ontwikkelen een tellerschakeling in subsectie \ref{ss:counters}.
\subsection{De flipflop}
\label{ss:flipflop}
Bij de synthese van een flipflop zullen we een eerste positief feedback component introduceren: de \termen{set-reset latch}. Vervolgens zullen we deze latch uitbreiden en verschillende varianten bespreken. Een probleem met latches is het zogenaamde \termen{transparantie-probleem}. Dit probleem zullen we oplossen door extra logica met deze latch te verbinden wat resulteert in een \termen{flipflop}. Er zijn verschillende varianten van flipflops om verschillende toepassingen te ondersteunen, we eindigen met een beknopt overzicht van de verschillende flipflops.
\subsubsection{De latch}
\begin{figure}[hbt]
\centering
\importtikzsubfigure{setResetLatch-nor}{Implementatie met NOR-poorten.}
\importtikzsubfigure{setResetLatch-nand}{Implementatie met NAND-poorten.}
\caption{Set-reset latch.}
\figlab{setResetLatch}
\end{figure}
\figref{setResetLatch} toont de NOR- en NAND-implementatie van de zogenaamde set-reset latch (ofwel \termen{SR-latch}). Dit component werkt met positieve terugkoppeling waarbij de uitgangen dus ook een deel van de ingangen vormen. We onthouden de uitvoer door bepaalde waarden aan de ingang aan te leggen, die resulteren in het opnieuw bekomen van dezelfde uitvoer. Bij de NOR-implementatie is dit $\left(S,R\right)=\left(0,0\right)$. De NAND-implementatie is in feite de volledig duale vorm en werkt volledig equivalent, alleen zijn $S$ en $R$ hier laag actieve signalen (zie \ref{s:negativeLogic}). We zouden ook negatieve logica aan $Q$ en $Q_n$ moeten toekennen maar omdat altijd geldt $Q_n=Q'$ kunnen we ook eenvoudigweg de twee uitgangen omdraaien en bekomen we positieve logica aan de uitgang. De stabiele invoer van de NAND-implementatie is dan ook $\left(S^*,R^*\right)=\left(1,1\right)$. Indien we \'e\'en van de ingangen laten afwijken van de stabiele toestand, komen we in een onstabiele toestand. Deze stabiele toestand kan maar op \'e\'en manier terug stabiel worden. Op deze manier kunnen we een nieuwe waarde toekennen aan de latch. Deze waarde zal ook verder gelden nadat de ingangen weer stabiel zijn. Hierdoor kent men ook de termen \termen{set} (1 in het geheugen) en \termen{reset} (0 in het geheugen) toe aan de ingangen. Indien beide ingangen afwijken van de stabiele ingangen is het gedrag niet meer bepaald. In dat geval zal de uiteindelijke waarde afhangen van de implementatie van de latch en de vertraging van de poorten. Bij identieke vertragingen leidt dit tot een oscillerend effect wat ook te zien is op de tijdsgrafieken op figuur \ref{fig:setResetLatch}. In een minder ideaal systeem zal de snelste poort\footnote{Uiteraard zijn de twee poorten in theorie even traag, in de praktijk zullen er altijd kleine verschillen zijn.} de uiteindelijke vertraging bepalen. Dit gedrag noemt men een ``\termen{race}'' (zie \ref{term:race}).
\paragraph{Geklokte SR-latch}
Bij de SR-latch is er geen sprake van een klokingang. We kunnen immers ook geheugens zonder klok gebruiken in bijvoorbeeld asynchrone schakelingen. Door extra hardware voor de SR-latch te plaatsen, kunnen we een \termen{geklokte SR-latch} bouwen. \figref{clockedSRLatch} toont hoe we dit kunnen realiseren. Zolang het kloksignaal op 0 staat zal het geheugen zijn waarde behouden, indien het kloksignaal op 1 staat gelden dezelfde regels als bij een SR-latch.
\begin{figure}[hbt]
\centering
\importtikzsubfigure{clockedSRLatch-gates}{Met AND- en NOR-poorten.}
\importtikzsubfigure{clockedSRLatch-nand}{Met NAND-poorten.}
\importtikzsubfigure{clockedSRLatch-trans}{Overgangstabel.}
\caption{Geklokte SR-latch.}
\figlab{clockedSRLatch}
\end{figure}
\paragraph{Geklokte D-latch}
Indien we elke klokflank een nieuwe waarde in de latch willen opslaan loont het meestal de moeite om de geklokte SR-latch om te vormen tot een \termen{geklokte D-latch}. Een geklokte D-latch zoals op figuur \ref{fig:clockedDLatch} bouwt meestal extra logica rond een geklokte SR-latch die met de \termen{data-ingang $D$} de $S$ en $R$ ingang aanstuurt. D-latches zijn populair bij schakelingen waarbij bij iedere klokflank een nieuwe waarde wordt ingelezen. Soms wordt dan echter ook een SR-latch gebruikt omdat dit de logica rond deze geheugenmodules soms kan vereenvoudigen.
\begin{figure}[hbt]
\centering
\importtikzsubfigure{clockedDLatch-impl}{Implementatie.}
\importtikzsubfigure{clockedDLatch-trans}{Overgangstabel.}
\caption{Geklokte D-latch.}
\figlab{clockedDLatch}
\end{figure}
\paragraph{Set-up- en houdtijd}
Aan de hand van de implementatie van de geklokte D-latch op figuur \ref{fig:clockedDLatch} zullen we de vertragingen van verschillende signalen berekenen:
\begin{equation}
\begin{array}{ccl}
D\rightarrow Q&:&\left\{\begin{array}{lll}
t_{HL}=1.4+1.4=2.8&\ifun&D=0\wedge\mbox{Clk}=1\\
t_{LH}=1+1.4+1.4+1.4=5.2&\ifun&D=1\wedge\mbox{Clk}=1
\end{array}\right.\\\\
D\rightarrow Q_n&:&\left\{\begin{array}{lll}
t_{HL}=1+1.4+1.4=3.8&\ifun&D=0\wedge\mbox{Clk}=1\\
t_{LH}=1.4+1.4+1.4=4.2&\ifun&D=1\wedge\mbox{Clk}=1
\end{array}\right.\\\\
\mbox{Clk}\rightarrow Q&:&\left\{\begin{array}{lll}
t_{HL}=1.4+1.4+1.4=4.2&\ifun&D=0\wedge\mbox{Clk}=1\\
t_{LH}=1.4+1.4=2.8&\ifun&D=1\wedge\mbox{Clk}=1
\end{array}\right.\\\\
\mbox{Clk}\rightarrow Q_n&:&\left\{\begin{array}{lll}
t_{HL}=1.4+1.4+1.4=4.2&\ifun&D=0\wedge\mbox{Clk}=1\\
t_{LH}=1.4+1.4=2.8&\ifun&D=1\wedge\mbox{Clk}=1
\end{array}\right.
\end{array}
\label{eqn:dLatchDelays}
\end{equation}
We kunnen dergelijke vertragingen grafisch weergeven zoals op de tijdsgrafieken op figuur \ref{fig:timebehavDLatch}.
\begin{figure}[hbt]
\centering
\subfigure[Set-up-tijd??]{%TODO: ongeldige toestand op figuur zetten
\begin{tikzpicture}
\timebehav{0.00}{0.00}{6.70}{7}{0.80}{0/$Qn$,1/$Q$,2/$R*$,3/$S*$,4/$Dn$,5/$Clk$,6/$D$}{0/0.00/1/5.20/0, 0/5.20/0/8.00/0, 1/0.00/0/3.80/1, 1/3.80/1/8.00/1, 2/0.00/0/3.40/1, 2/3.40/1/8.00/1, 3/0.00/1/2.40/0, 3/2.40/0/5.40/1, 3/5.40/1/8.00/1, 4/0.00/1/2.00/0, 4/2.00/0/8.00/0, 5/0.00/1/4.00/0, 5/4.00/0/8.00/0, 6/0.00/0/1.00/1, 6/1.00/1/8.00/1}{3.80/1.40/1/0, 2.40/1.40/3/1, 2.00/1.40/4/2, 1.00/1.40/6/3, 4.00/1.40/5/3, 1.00/1.00/6/4};
\timebehav{8.00}{0.00}{6.70}{7}{0.80}{0/$Qn$,1/$Q$,2/$R*$,3/$S*$,4/$Dn$,5/$Clk$,6/$D$}{0/0.00/1/5.20/0, 0/5.20/0/5.70/1, 0/5.70/1/8.00/0, 0/8.00/0/8.00/0, 1/0.00/0/3.80/1, 1/3.80/1/4.30/0, 1/4.30/0/6.60/1, 1/6.60/1/7.10/0, 1/7.10/0/8.00/0, 2/0.00/0/2.90/1, 2/2.90/1/8.00/1, 3/0.00/1/2.40/0, 3/2.40/0/2.90/1, 3/2.90/1/8.00/1, 4/0.00/1/2.00/0, 4/2.00/0/8.00/0, 5/0.00/1/1.50/0, 5/1.50/0/8.00/0, 6/0.00/0/1.00/1, 6/1.00/1/8.00/1}{3.80/1.40/1/0, 4.30/1.40/1/0, 6.60/1.40/1/0, 2.40/1.40/3/1, 2.90/1.40/3/1, 5.20/1.40/0/1, 5.70/1.40/0/1, 1.50/1.40/5/2, 1.00/1.40/6/3, 1.50/1.40/5/3, 1.00/1.00/6/4};
\end{tikzpicture}
\figlab{timebehavDLatchSetup}
}
\subfigure[Houdtijd??]{%TODO: afwerken
\begin{tikzpicture}
\timebehav{0.00}{0.00}{6.70}{7}{0.80}{0/$Qn$,1/$Q$,2/$R*$,3/$S*$,4/$Dn$,5/$Clk$,6/$D$}{0/0.00/1/5.20/0, 0/5.20/0/7.70/1, 0/7.70/1/8.00/0, 0/8.00/0/8.00/0, 1/0.00/0/3.80/1, 1/3.80/1/6.30/0, 1/6.30/0/6.60/1, 1/6.60/1/8.00/1, 2/0.00/0/3.40/1, 2/3.40/1/8.00/1, 3/0.00/1/2.40/0, 3/2.40/0/4.90/1, 3/4.90/1/8.00/1, 4/0.00/1/2.00/0, 4/2.00/0/4.50/1, 4/4.50/1/8.00/1, 5/0.00/1/4.00/0, 5/4.00/0/8.00/0, 6/0.00/0/1.00/1, 6/1.00/1/3.50/0, 6/3.50/0/8.00/0}{3.80/1.40/1/0, 6.30/1.40/1/0, 6.60/1.40/1/0, 2.40/1.40/3/1, 4.90/1.40/3/1, 5.20/1.40/0/1, 2.00/1.40/4/2, 1.00/1.40/6/3, 3.50/1.40/6/3, 1.00/1.00/6/4, 3.50/1.00/6/4};
\timebehav{8.00}{0.00}{6.70}{7}{0.80}{0/$Qn$,1/$Q$,2/$R*$,3/$S*$,4/$Dn$,5/$Clk$,6/$D$}{0/0.00/1/5.20/0, 0/5.20/0/8.00/0, 1/0.00/0/3.80/1, 1/3.80/1/8.00/1, 2/0.00/0/3.40/1, 2/3.40/1/8.00/1, 3/0.00/1/2.40/0, 3/2.40/0/5.20/1, 3/5.20/1/8.00/1, 4/0.00/1/2.00/0, 4/2.00/0/4.80/1, 4/4.80/1/8.00/1, 5/0.00/1/4.00/0, 5/4.00/0/8.00/0, 6/0.00/0/1.00/1, 6/1.00/1/3.80/0, 6/3.80/0/8.00/0}{3.80/1.40/1/0, 2.40/1.40/3/1, 2.00/1.40/4/2, 1.00/1.40/6/3, 3.80/1.40/6/3, 1.00/1.00/6/4, 3.80/1.00/6/4};
\end{tikzpicture}
\figlab{timebehavDLatchHold}
}
\caption{Tijdsgrafieken van een D-latch.}
\figlab{timebehavDLatch}
\end{figure}
Deze grafische voorstelling toont twee bekende problemen die veroorzaakt worden door het niet respecteren van twee parameters:
\begin{itemize}
 \item \termen{Set-up-tijd}: De tijd alvorens de het actieve kloksignaal wordt verlaten waarin de waarde op de data-ingang niet meer mag wijzigen. Bij een geklokte $D$-latch zoals op figuur \ref{fig:clockedDLatch} is dit:
\begin{equation}
\begin{array}{cr}
t_{\mbox{\tiny{set-up}}}=t_{pHL}\left(\mbox{inverter}\right)&\mbox{(vertraging van een hoog-naar-laag signaal door een inverter)}
\end{array}
\end{equation}
\figref{timebehavDLatchSetup} toont twee senarios. Bij het eerste wordt de set-up tijd gerespecteerd, bij het tweede faalt de toekenning. Dit komt omdat bij dit scenario $S^*$ weer hoog wordt alvorens $R^*$ een hoog signaal aanlegt. Indien we de vertragingen van de poorten doorrekenen komen we uit dat de vertraging van aan de inverter een cruciale rol speelt. Dit is ook enigszins logisch: indien we een 0 aan de data-ingang $D$ aanleggen zal gedurende deze periode 1 op zowel de $S$ als $R$ van de geklokte SR-latch worden aangelegd, wat eigenlijk een ongeldige invoer is.
 \item \termen{Houdtijd}: We moeten niet alleen het signaal op tijd aanleggen voor de klok een laag signaal aanlegt. Meestal moeten we het signaal daarna nog een tijdje laten staan om te vermijden dat de latch alsnog een foute waarde aanneemt. \figref{timebehavDLatchHold} toont opnieuw twee scenarios.??%TODO: afwerken
\end{itemize}
\paragraph{Metastabiliteit}
Een ander probleem dat komt kijken bij latches is de \termen{metastabiliteit}. Dit probleem treedt op bij de twee poorten\footnote{NAND- of NOR-implementatie maakt niet uit.} van de SR-latch en dus bijgevolg alle afgeleide latches. Als we bij de NOR-implementatie $\left(S,R\right)=\left(0,0\right)$ aanleggen, of bij een NAND-implementatie $\left(S^*,R^*\right)=\left(1,1\right)$, kunnen we deze poorten modelleren als NOT-poorten zoals op figuur \ref{fig:metastabilityNotGates}.
\begin{figure}[hbt]
\centering
\subfigure[Implementatie.]{
\begin{tikzpicture}[circuit logic US,scale=1.2]
\node[not gate] (NO0) at (0,0.65) {};
\node[not gate] (NO1) at (0,-0.65) {};
\draw (NO1.output) -- (NO1.output -| 0.85,0) node[anchor=west,scale=0.75]{$x$} -- (0.85,-0.4) -- (-0.75,0.4) -- (NO0.input -| -0.75,0) -- (NO0.input);
\draw (NO0.output) -- (NO0.output -| 0.85,0) node[anchor=west,scale=0.75]{$y$} -- (0.85,0.4) -- (-0.75,-0.4) -- (NO1.input -| -0.75,0) -- (NO1.input);
\end{tikzpicture}
\figlab{metastabilityNotGates}}
\subfigure[Transfer-functies.]{
\begin{tikzpicture}
\draw[thick,->] (-0.1,0) -- (2.2,0) node[anchor=north east,scale=0.75]{$x$};
\draw[thick,->] (0,-0.1) -- (0,2.2) node[anchor=north east,scale=0.75]{$y$};
\draw [samples=100,smooth,domain=0.1:1.9,variable=\x,dashed] plot (\x,{1.75-1.5/(1+exp(-8*\x+8))});
\draw [samples=100,smooth,domain=0.1:1.9,variable=\x] plot ({1.75-1.5/(1+exp(-8*\x+8))},\x);
\pdot{1,1};
\pdot{0.2538,1.7462};
\pdot{1.7462,0.2538};
\end{tikzpicture}
\figlab{metastabilityNotFunctions}}
\subfigure[Bal-en-heuvel-analogie.]{
\begin{tikzpicture}
\draw[pattern=vertical lines] [samples=100,smooth,domain=-2:2,variable=\x] plot (\x,{cos(114.591559029*\x)}) -- (2,-1.3) -- (-2,-1.3) -- cycle;
\filldraw[fill=black!20,draw=black] (0,1.125) circle (0.125 cm);
\filldraw[fill=black!20,draw=black] (-1.570796327,-0.875) circle (0.125 cm);
\filldraw[fill=black!20,draw=black] (1.570796327,-0.875) circle (0.125 cm);
\end{tikzpicture}
\figlab{metastabilityAnalogy}}
\caption{Metastabiliteit.}
\figlab{metastability}
\end{figure}
Indien we deze schakeling logisch analyseren zien we twee mogelijke stabiele oplossingen: waarbij ofwel $x$ ofwel $y$ 1 is, en de andere 0. Deze waarden zijn ook de enige die we beschouwen indien we de NOT-poort louter als logisch component zien. We implementeren deze poorten echter met behulp van transistoren, bijgevolg behoudt de poort een zeker analoog karakter. Op de grafiek op figuur \ref{fig:metastabilityNotFunctions} geven we de transfer-functie van de twee NOT-poorten weer. De stippelijn geeft de transfer-functie van de bovenste NOT-poort weer, de volle lijn de onderste. We zien zoals verwacht de twee \termen{stabiele toestanden}. We bemerken echter ook een \termen{metastabiele toestand}. Op het moment dat op $x$ of $y$ een kleine hoeveelheid ruis wordt aangebracht zullen de poorten dit effect versterken en zal zal de schakeling in een stabiele toestand terechtkomen. Het probleem is echter dat we uiteraard niet weten hoelang dit zal duren. We gaan er echter vanuit dat de kans dat na een bepaald tijdstip er nog onvoldoende ruis is opgetreden Poisson-verdeeld is\footnote{De meeste kansverdelingen op het voorkomen van een gebeurtenis zijn Poisson-verdeeld.}. We formaliseren dus tot:
\begin{equation}
p\left(\mbox{nog in metastabiele toestand na $t$}\right)=e^{-t/\tau}
\end{equation}
De \termen{tijdsconstante $\tau$} is hierbij afhangen van twee factoren:
\begin{itemize}
 \item De hoeveelheid ruis: hoe meer ruis hoe lager de tijdsconstante.
 \item De stijlheid van de curves rond de metastabiele toestand: hoe stijler hoe lager de tijdsconstante.
\end{itemize}
Een latch kan in een metastabiele toestand komen door een zogenaamde \termen{marginale triggering}: een schending van de set-up- of houdtijd of van de minimale pulsbreedte\footnote{De tijd dat een signaal wordt aangelegd.}. In deze gevallen kan een overgang tussen twee stabiele toestanden worden onderbroken. Indien dit gebeurt op het moment dat men net voorbij de metastabiele toestand passeert treedt dit probleem op. Een latch komt dan ook bij elke overgang kortstondig in een metastabiele toestand. Ook een tijdje in een metastabiele toestand blijven is geen probleem. Zolang deze toestand niet meer actief is wanneer we het signaal gaan gebruiken zullen er geen problemen optreden. Een populaire voorstelling van metastabiliteit is de zogenaamde \termen{bal-en-heuvel-analogie}. In deze analogie beschrijven we een heuvel zoals op figuur \ref{fig:metastabilityAnalogy}. Een bal kan op deze heuvel in drie toestanden een evenwicht bereiken: twee toestanden in een dal (passief evenwicht) en een metastabiele toestand op de heuvel (actief evenwicht). Bij asynchrone circuits is metastabiliteit een veel voorkomend fenomeen. Zeker wanneer de klokfrequentie geen veelvoud is van de frequentie waarmee de ingang omwisselt. In dat geval bestaat de oplossing van het probleem eerder uit ``hoe ga ik om met metastabiliteit?'', in plaats van ``hoe los ik de metastabiliteit op?''.
\subsubsection{De flipflop}
Latches kunnen we gebruiken bij het opslaan van \'e\'en bit. Indien we een geklokte latch aan een kloksignaal hangen zijn er twee toestanden afhankelijk van het niveau van het kloksignaal:
\begin{itemize}
 \item Indien het kloksignaal hoog is $\mbox{Clk}=1$ is de latch \termen{transparant}. De latch neemt de waarde over die aan de ingang staat.
 \item Indien het kloksignaal laag is $\mbox{Clk}=0$ onthoudt de latch de laatste waarde die aan de ingang stond toen de latch transparant was.
\end{itemize}
Latches geven echter problemen op het moment we verschillende latches na elkaar willen hangen. In dat geval zullen immers alle latches transparant zijn op hetzelfde moment. Hierdoor zal de laatste latch de waarde aannemen die op de eerste latch wordt aangelegd\footnote{Bij een groot aantal latches zal het signaal door vertragingen en set-up- en houdtijd uiteraard maar door een beperkt aantal latches in \'e\'en klokflank propageren.}. Dit probleem wordt ook wel het \termen{transparantie-probleem} genoemd. Meestal willen we echter bij een sequentie van een aantal geheugencomponenten afdwingen dat de informatie door \'e\'en geheugencomponent per klokflank propageert. De flipflop is een geheugencomponent die in tegenstelling tot de latch \termen{flankgevoelig} is. Dit betekent dat de flipflop enkel transparant is op het moment dat de klok van 0 naar 1 gaat, in tegenstelling tot een latch die transparant is gedurdende de volledige periode dat de klok hoog is. Om dit te realiseren zijn er doorgaans twee methodes:
\begin{itemize}
 \item De \termen{master-slave flipflop}.
 \item De \termen{edge-triggered flipflop}.
\end{itemize}
We zullen deze twee verschillende technieken in de volgende paragrafen toelichten.
\paragraph{Master-slave flipflop}
Een master-slave flipflop maakt gebruik van twee latches die beurtelings transparant zijn. De master (eerste latch) is transparant wanneer het kloksignaal laag is, de tweede latch is transparant bij een hoog kloksignaal. Gegroepeerd vormen we dus een geheugen die de laatste waarde opslaat die aan de ingang stond op het moment dat het kloksignaal laag was, en deze verder propageert op het moment dat het kloksignaal hoog is. \figref{masterSlaveFlipflop} toont dit concept samen met een tijdsgrafiek.??
\begin{figure}[hbt]
\centering
\subfigure[Implementatie.]{
\begin{tikzpicture}[circuit logic US]
\def\cly{-1.125};
\def\nff{2};
\def\nffd{1};
\def\nffi{3};
\foreach\x in {0,...,\nff} {
  \filldraw[draw=black,dashed,fill=black!20] (4*\x-1.125,\cly-0.125) rectangle (4*\x+2.425,1.25);
  \node[cldlatchmaster,scale=0.75] (M\x) at (4*\x,0) {};
  \node[cldlatch,scale=0.75] (S\x) at (4*\x+1.75,0) {};
  \draw (M\x.Clk) -| (4*\x-1,\cly);
  \node[anchor=south] (Mt\x) at (M\x.north) {Master};
  \node[anchor=south] (St\x) at (S\x.north) {Slave};
  \pdot{4*\x-1,\cly};
  \draw (M\x.Q) -- (S\x.D);
  \draw (S\x.Clk) -| (4*\x+0.75,\cly);
}
\foreach\xd/\x in {0/1,1/2} {
  \draw (S\xd.Q) to node[midway,above,scale=0.75]{$Q_{\x}$} (M\x.D);
  \pdot{0.75+4*\xd,\cly};
}
\draw (0.75+4*\nff,\cly) -- (-1.5,\cly) node[scale=0.75,anchor=east]{Clk};
\draw (M0.D) -- (M0.D -| -1.5,0) node[scale=0.75,anchor=east]{$D$};
\draw (S\nff.Q) -- (S\nff.Q -| 4*\nff+3,0) node[scale=0.75,anchor=west]{$Q_{\nffi}$};
\end{tikzpicture}}
\subfigure[Interface]{
\begin{tikzpicture}
\node[dff] (DFF) at (0,0) {};
\end{tikzpicture}}
\caption{Master-slave flipflop.}
\figlab{masterSlaveFlipflop}
\end{figure}
\paragraph{Edge-triggered flipflop}
Een Edge-trigger flipflop maakt gebruik van een structuur die we kunnen groeperen als drie latches. Deze schakeling stelt ons in staat om om het signaal op te slaan die aan de ingang staat op het moment dat de klok van 0 naar 1 gaat. \figref{edgeTriggeredFlipflop} toont een basis en meer uitgebreide implementatie samen met een tijdsgrafiek.
\begin{figure}[hbt]
\centering
\subfigure[Basisimplementatie.]{
\begin{tikzpicture}[circuit logic US]
\def\dx{-1.75};
\def\dy{1.35};
\node[nand gate] (NA10) at (0,0.65) {};
\node[nand gate] (NA11) at (0,-0.65) {};
\draw (NA11.output -| 0.85,0) -- (0.85,-0.4) -- (-0.75,0.4) -- (NA10.input 2 -| -0.75,0) -- (NA10.input 2);
\draw (NA10.output -| 0.85,0) -- (0.85,0.4) -- (-0.75,-0.4) -- (NA11.input 1 -| -0.75,0) -- (NA11.input 1);
\draw (NA10.output) -- (NA10.output -| 1.1,0) node[anchor=west,scale=0.75]{$Q_n$};
\draw (NA11.output) -- (NA11.output -| 1.1,0) node[anchor=west,scale=0.75]{$Q$};
\pdot{NA10.output -| 0.85,0};
\pdot{NA11.output -| 0.85,0};
\node[nand gate] (NA01) at (NA10.input 1 -| \dx,0) {};
\node[nand gate,inputs={normal,normal,normal}] (NA02) at (NA11.input 2 -| \dx,0) {};
\draw (NA01.output) -- (NA10.input 1);
\draw (NA02.output) -- (NA11.input 2);
\node[nand gate] (NA00) at (\dx,2) {};
\node[nand gate] (NA03) at (\dx,-2) {};
\draw (NA01.output -| \dx+0.85,0) -- (\dx+0.85,\dy-0.3) -- (\dx-0.75,\dy+0.3) -- (NA00.input 2 -| \dx-0.75,0) -- (NA00.input 2);
\draw (NA00.output) -- (NA00.output -| \dx+0.85,0) -- (\dx+0.85,\dy+0.3) -- (\dx-0.75,\dy-0.3) -- (NA01.input 1 -| \dx-0.75,0) -- (NA01.input 1);
\draw (NA03.output) -- (NA03.output -| \dx+0.85,0) -- (\dx+0.85,-\dy-0.3) -- (\dx-0.75,-\dy+0.3) -- (NA02.input 3 -| \dx-0.75,0) -- (NA02.input 3);
\draw (NA02.output -| \dx+0.85,0) -- (\dx+0.85,-\dy+0.4) -- (\dx-0.75,-\dy-0.4) -- (NA03.input 1 -| \dx-0.75,0) -- (NA03.input 1);
\draw (NA01.output -| \dx+0.85,0) -- (\dx+0.85,0.4) -- (\dx-0.75,-0.4) -- (NA02.input 1 -| \dx-0.75,0) -- (NA02.input 1);
\draw (NA00.output -| \dx+0.85,0) node[anchor=west,scale=0.75]{$A$};
\pdot{NA01.output -| \dx+0.85,0};
\draw (NA01.output -| \dx+0.85,0) node[anchor=south west,scale=0.75]{$S^*$};
\pdot{NA02.output -| \dx+0.85,0};
\draw (NA02.output -| \dx+0.85,0) node[anchor=north west,scale=0.75]{$R^*$};
\pdot{NA02.input 3 -| \dx-0.75,0};
\draw (NA03.output -| \dx+0.85,0) node[anchor=west,scale=0.75]{$B$};
\draw (NA02.input 3 -| \dx-0.75,0) -- ++(-0.5,0) |- (NA00.input 1);
\draw (NA02.input 2) -- (NA02.input 2 -| \dx-0.75,0) -- ++(-0.25,0) |- (NA01.input 2);
\draw (\dx-1,0) -- ++(-0.5,0) node[anchor=east,scale=0.75]{Klok Clk};
\pdot{\dx-1,0};
\draw (NA03.input 2) -- (NA03.input 2 -| \dx-1.5,0) node[anchor=east,scale=0.75]{$D$};
\end{tikzpicture}
\figlab{edgeTriggeredFlipflopBasic}}
\subfigure[Uitgebreide implementatie.]{
\begin{tikzpicture}[circuit logic US]
\def\dx{-1.75};
\def\dy{1.35};
\node[nand gate,inputs={normal,normal,normal}] (NA10) at (0,0.65) {};
\node[nand gate,inputs={normal,normal,normal}] (NA11) at (0,-0.65) {};
\draw (NA11.output -| 0.85,0) -- (0.85,-0.3) -- (-0.75,0.3) -- (NA10.input 3 -| -0.75,0) -- (NA10.input 3);
\draw (NA10.output -| 0.85,0) -- (0.85,0.3) -- (-0.75,-0.3) -- (NA11.input 1 -| -0.75,0) -- (NA11.input 1);
\draw (NA10.output) -- (NA10.output -| 1.1,0) node[anchor=west,scale=0.75]{$Q_n$};
\draw (NA11.output) -- (NA11.output -| 1.1,0) node[anchor=west,scale=0.75]{$Q$};
\pdot{NA10.output -| 0.85,0};
\pdot{NA11.output -| 0.85,0};
\node[nand gate,inputs={normal,normal,normal}] (NA01) at (NA10.input 2 -| \dx,0) {};
\node[nand gate,inputs={normal,normal,normal}] (NA02) at (NA11.input 2 -| \dx,0) {};
\draw (NA01.output) -- (NA10.input 2);
\draw (NA02.output) -- (NA11.input 2);
\node[nand gate,inputs={normal,normal,normal}] (NA00) at (\dx,2) {};
\node[nand gate,inputs={normal,normal,normal}] (NA03) at (\dx,-2) {};
\draw (NA01.output -| \dx+0.85,0) -- (\dx+0.85,\dy-0.3) -- (\dx-0.75,\dy+0.3) -- (NA00.input 3 -| \dx-0.75,0) -- (NA00.input 3);
\draw (NA00.output) -- (NA00.output -| \dx+0.85,0) -- (\dx+0.85,\dy+0.3) -- (\dx-0.75,\dy-0.3) -- (NA01.input 1 -| \dx-0.75,0) -- (NA01.input 1);
\draw (NA03.output) -- (NA03.output -| \dx+0.85,0) -- (\dx+0.85,-\dy-0.3) -- (\dx-0.75,-\dy+0.3) -- (NA02.input 3 -| \dx-0.75,0) -- (NA02.input 3);
\draw (NA02.output -| \dx+0.85,0) -- (\dx+0.85,-\dy+0.3) -- (\dx-0.75,-\dy-0.3) -- (NA03.input 1 -| \dx-0.75,0) -- (NA03.input 1);
\draw (NA01.output -| \dx+0.85,0) -- (\dx+0.85,0.4) -- (\dx-0.75,-0.4) -- (NA02.input 1 -| \dx-0.75,0) -- (NA02.input 1);
\pdot{NA01.output -| \dx+0.85,0};
\pdot{NA02.output -| \dx+0.85,0};
\pdot{NA02.input 3 -| \dx-0.75,0};
\draw (NA02.input 3 -| \dx-0.75,0) -- ++(-0.5,0) |- (NA00.input 2);
\draw (NA02.input 2) -- (NA02.input 2 -| \dx-0.75,0) -- ++(-0.25,0) |- (NA01.input 2);
\draw (\dx-1,0) -- ++(-0.75,0) node[anchor=east,scale=0.75]{Klok Clk};
\pdot{\dx-1,0};
\draw (NA03.input 2) -- (NA03.input 2 -| \dx-1.75,0) node[anchor=east,scale=0.75]{$D$};
\coordinate (CLRI) at (\dx-1.75,-2.5);
\coordinate (PRI) at (\dx-1.75,2.5);
\draw (PRI) node[anchor=east,scale=0.75]{Preset $\mbox{PR}^*$} -- (PRI -| -0.75,0) |- (NA10.input 1);
\draw (CLRI) node[anchor=east,scale=0.75]{Clear $\mbox{CLR}^*$} -- (CLRI -| -0.75,0) |- (NA11.input 3);
\draw (PRI -| \dx-0.75,0) |- (NA00.input 1);
\pdot{PRI -| \dx-0.75,0};
\draw (CLRI -| \dx-1.5,0) |- (NA03.input 3);
\pdot{CLRI -| \dx-1.5,0};
\draw (\dx-1.5,0 |- NA03.input 3) |- (NA01.input 3);
\pdot{\dx-1.5,0 |- NA03.input 3};
\end{tikzpicture}
\figlab{edgeTriggeredFlipflopExtended}}
\subfigure[Tijdsgrafiek]{\begin{tikzpicture}
\timebehav{5.00}{-1.00}{8.50}{8}{0.75}{0/$Qn$,1/$Q$,2/$S^*$,3/$R^*$,4/$A$,5/$B$,6/$D$,7/Clk}{0/0.00/1/1.55/0, 0/1.55/0/3.88/1, 0/3.88/1/6.55/0, 0/6.55/0/8.88/1, 0/8.88/1/11.48/1, 1/0.00/0/1.30/1, 1/1.30/1/4.13/0, 1/4.13/0/6.30/1, 1/6.30/1/9.13/0, 1/9.13/0/11.48/0, 2/0.00/1/1.05/0, 2/1.05/0/2.30/1, 2/2.30/1/6.05/0, 2/6.05/0/7.30/1, 2/7.30/1/11.48/1, 3/0.00/1/3.63/0, 3/3.63/0/4.88/1, 3/4.88/1/8.63/0, 3/8.63/0/9.88/1, 3/9.88/1/11.13/0, 3/11.13/0/11.48/0, 4/0.00/0/0.68/1, 4/0.68/1/3.18/0, 4/3.18/0/5.38/1, 4/5.38/1/7.55/0, 4/7.55/0/11.48/0, 5/0.00/1/0.43/0, 5/0.43/0/2.93/1, 5/2.93/1/5.13/0, 5/5.13/0/6.68/1, 5/6.68/1/11.48/1, 6/0.00/0/0.18/1, 6/0.18/1/2.68/0, 6/2.68/0/3.93/1, 6/3.93/1/6.43/0, 6/6.43/0/11.48/0, 7/0.00/0/0.80/1, 7/0.80/1/2.05/0, 7/2.05/0/3.30/1, 7/3.30/1/4.55/0, 7/4.55/0/5.80/1, 7/5.80/1/7.05/0, 7/7.05/0/8.30/1, 7/8.30/1/9.55/0, 7/9.55/0/10.80/1, 7/10.80/1/11.48/1}{1.30/0.25/1/0, 3.63/0.25/3/0, 6.30/0.25/1/0, 8.63/0.25/3/0, 1.05/0.25/2/1, 3.88/0.25/0/1, 6.05/0.25/2/1, 8.88/0.25/0/1, 0.80/0.25/7/2, 2.05/0.25/7/2, 5.80/0.25/7/2, 7.05/0.25/7/2, 3.30/0.32/7/3, 4.55/0.32/7/3, 8.30/0.32/7/3, 9.55/0.32/7/3, 10.80/0.32/7/3, 0.43/0.25/5/4, 2.93/0.25/5/4, 5.13/0.25/5/4, 7.30/0.25/2/4, 0.18/0.25/6/5, 2.68/0.25/6/5, 4.88/0.25/3/5, 6.43/0.25/6/5};
\end{tikzpicture}
}
\caption{Edge-triggered flipflop.}
\figlab{edgeTriggeredFlipflop}
\end{figure}
In de meer uitgebreide versie zijn er naast de data- en klokingang ook nog twee andere ingangen beschreven:
\begin{itemize}
 \item \termen{Preset $\mbox{PR}^*$}: Indien dit signaal laag wordt, slaan we asynchroon een 1 op in de flipflop.
 \item \termen{Clear $\mbox{CLR}^*$}: Indien dit signaal laag wordt, slaan we asynchroon een 0 op in de flipflop. 
\end{itemize}
Deze twee signalen worden ook wel de \termen{asynchrone set en reset} genoemd. Ze zijn asynchroon omdat ze onafhankelijk van de toestand van de klok een waarde in het geheugen kunnen inbrengen, deze eigenschap wordt bijvoorbeeld gebruikt om bij het opkomen van de stroom in de elektronica de flipflop in een gekende toestand te brengen\footnote{Bij het opkomen van de stroom zal de flipflop of latch immers een waarde aannemen afhankelijk van de ruis en minimale verschillen in poortvertragingen.}.
\subsubsection{Types flipflops}
\label{sss:typesFlipflops}
Er bestaan analoog aan de latches ook verschillende types flipflops. Hierbij is niet de klokaansturing variabel, maar de manier hoe data ingelezen wordt. Denk bijvoorbeeld aan het verschil tussen een SR-latch en D-latch. Elk type flipflop kunnen we karakteriseren aan de hand van twee tabellen:
\begin{itemize}
 \item De \termen{karakteristieke tabel}: deze tabel gebruikt men bij de implementatie van de flipflops. Het toont aan de linkerkant de ingangen, en aan de rechterkant geeft het weer hoe er op deze ingangen wordt ingespeeld. 
 \item De \termen{excitatietabel}: deze tabel beschrijft de verschillende vormen van gedrag van het component aan de linkerkant, en aan de rechterkant hoe we dit gedrag kunnen verwezenlijken met de ingangen.
\end{itemize}
Deze twee tabellen zijn niet strikt het omgekeerde omdat de excitatietabel een kolom voorziet voor $Q$ en $Q_{\mbox{\tiny{next}}}$, terwijl de karakteristieke tabel uitsluitend \'e\'en kolom aan de rechterkant voorziet. In de volgende paragrafen zullen we de verschillende types flipflops bespreken met hun karakteristieke- en excitatietabel.
\paragraph{SR-flipflop} De \termen{SR-flipflop} ofwel \termen{set-reset flipflop} werkt volledig analoog aan een SR-latch, alleen verandert de waarde uitsluitend op het moment dat de klok van een laag signaal naar een hoog signaal gaat. De karakteristieke tabel is dan ook volledig equivalent met deze van de SR-latch op figuur \ref{fig:setResetLatch}. \figref{setResetFlipflop} toont de interface en de karakteristieke- en excitatietabel van de SR-flipflop.
\begin{figure}[hbt]
\centering
\begin{tikzpicture}
\node[srff,anchor=west] (I) at (0,0) {};
\node (KT) at (8.5,0) {$\begin{array}{cc|c}
S&R&Q_{\mbox{\tiny{next}}}\\\hline
0&0&Q\\
0&1&0\\
1&0&1\\
1&1&\mbox{N/A}
\end{array}$};
\node[anchor=east] (ET) at (14,0) {$\begin{array}{cc|cc}
Q&Q_{\mbox{\tiny{next}}}&S&R\\\hline
0&0&0&-\\
0&1&1&0\\
1&0&0&1\\
1&1&-&0\\
\end{array}$};
\node[anchor=north] (IT) at (I.south) {Symbool};
\node[anchor=north] (KTT) at (KT.south |- IT.north) {Karakteristieke tabel};
\node[anchor=north] (ETT) at (ET.south |- IT.north) {Excitatietabel};
\end{tikzpicture}
\caption{Set-reset flipflop.}
\figlab{setResetFlipflop}
\end{figure}
\paragraph{D-flipflop} Ook de \termen{data-flipflop} of \termen{D-flipflop} is conceptueel equivalent aan zijn latch tegenhanger. Op het moment dat de klok van laag naar hoog gaat, zal het signaal dat aan de data-ingang $D$ staat in het geheugen worden geladen. Hierdoor is het bouwen van schakelingen met D-flipflops meestal zeer eenvoudig. Merk dus op dat elk signaal slechts \'e\'en klokperiode bewaard wordt. \figref{dataFlipflop} toont het symbool samen met de karakteristieke- en excitatietabel.
\begin{figure}[hbt]
\centering
\begin{tikzpicture}
\node[dff,anchor=west] (I) at (0,0) {};
\node (KT) at (8.5,0) {$\begin{array}{c|c}
D&Q_{\mbox{\tiny{next}}}\\\hline
0&0\\
1&1
\end{array}$};
\node[anchor=east] (ET) at (14,0) {$\begin{array}{cc|c}
Q&Q_{\mbox{\tiny{next}}}&D\\\hline
0&0&0\\
0&1&1\\
1&0&0\\
1&1&1\\
\end{array}$};
\node[anchor=north] (IT) at (I.south) {Symbool};
\node[anchor=north] (KTT) at (KT.south |- IT.north) {Karakteristieke tabel};
\node[anchor=north] (ETT) at (ET.south |- IT.north) {Excitatietabel};
\end{tikzpicture}
\caption{Data-flipflop.}
\figlab{dataFlipflop}
\end{figure}
\paragraph{T-flipflop} Een nieuwe variant is de zogenaamde \termen{toggle-flipflop} ofwel \termen{T-flipflop}. De toggle-flipflop heeft een `toggle'-ingang $T$. Indien deze ingang bij een stijgende klokflank hoog is, zal de flipflop het omgekeerde van zijn huidige waarde opslaan, de zogenaamde `\termen{toggle}'-operatie. Indien $T=0$, blijft opgeslagen toestand dezelfde. We kunnen een toggle-flipflop bouwen met behulp van een data-flipflop en een XOR-poort. \figref{toggleFlipflop} toont het symbool, een mogelijke implementatie en de karakteristieke- en excitatietabel van de T-flipflop.
\begin{figure}[hbt]
\centering
\begin{tikzpicture}[circuit logic US]
\node[tff,anchor=west] (I) at (0,0) {};
\begin{scope}[xshift=4.5cm]
\filldraw[draw=black,dashed,fill=black!20] (-1.55,-1.1) rectangle (1.55,1.1);
\node[dff,scale=0.7] (DFF) at (0.65,0) {};
\node[xor gate,scale=0.7] (X) at (DFF.D -| -0.65,0) {};
\draw (X.output) -- (DFF.D);
\draw (X.input 1) -- ++(-0.2,0) |- (1.35,0.9) |- (DFF.Q);
\pdot{DFF.Q -| 1.35,0};
\draw (DFF.Q -| 1.35,0) -- (DFF.Q -| 1.75,0) node[scale=0.75,anchor=west]{$Q$};
\draw (DFF.Qn) -- (DFF.Qn -| 1.75,0) node[scale=0.75,anchor=west]{$Q_n$};
\draw (DFF.Clk) -- (DFF.Clk -| -1.75,0) node[scale=0.75,anchor=east]{Clk};
\draw (X.input 2) -- (X.input 2 -| -1.75,0) node[scale=0.75,anchor=east]{$T$};
\end{scope}
\node (KT) at (8.5,0) {$\begin{array}{c|c}
T&Q_{\mbox{\tiny{next}}}\\\hline
0&Q\\
1&Q'\\
\end{array}$};
\node[anchor=east] (ET) at (14,0) {$\begin{array}{cc|c}
Q&Q_{\mbox{\tiny{next}}}&T\\\hline
0&0&0\\
0&1&1\\
1&0&1\\
1&1&0\\
\end{array}$};
\node[anchor=north] (IT) at (I.south) {Symbool};
\node[anchor=north] (IMT) at (4.5,0 |- IT.north) {Implementatie};
\node[anchor=north] (KTT) at (KT.south |- IT.north) {Karakteristieke tabel};
\node[anchor=north] (ETT) at (ET.south |- IT.north) {Excitatietabel};
\end{tikzpicture}
\caption{Toggle-flipflop.}
\figlab{toggleFlipflop}
\end{figure}
\paragraph{JK-flipflop} De \termen{JK-flipflop} of voluit \termen{Jack Kilby-flipflop}\footnote{Genoemd naar Jack Kilby (1923-2005), Amerikaans natuurkundige en nobelprijswinnaar.} is een combinatie van de set-reset flipflop en de toggle-flipflop. Bij een SR-flipflop komen we immers in een ongeldige toestand indien we aan de ingangen $\left(S,R\right)=\left(1,1\right)$. De JK-flipflop lost dit probleem op door in dat geval een toggle-operatie uit te voeren op het geheugenelement, zoals te zien op de karakteristieke tabel op figuur \ref{fig:jackKilbyFlipflop}. We kunnen een JK-flipflop realiseren met behulp van een SR-flipflop waarbij:
\begin{equation}
\left\{\begin{array}{l}
S=J\cdot Q'\\
R=K\cdot Q
\end{array}\right.
\end{equation}
JK-flipflops worden vooral gebruikt om goedkopere schakelingen te synthetiseren. Deze eigenschap kunnen we afleiden uit de vele don't cares in de excitatietabel.
\begin{figure}[hbt]
\centering
\begin{tikzpicture}[circuit logic US]
\node[jkff,anchor=west] (I) at (0,0) {};
\begin{scope}[xshift=4.5cm]
\filldraw[draw=black,dashed,fill=black!20] (-1.55,-1.1) rectangle (1.55,1.1);
\node[srff,scale=0.7] (SRFF) at (0.65,0) {};
\node[and gate,scale=0.7] (A0) at (SRFF.S -| -0.65,0) {};
\node[and gate,scale=0.7] (A1) at (SRFF.R -| -0.65,0) {};
\draw (A0.output) -- (SRFF.S);
\draw (A1.output) -- (SRFF.R);
\draw (A0.input 2) -- ++(-0.2,0) |- (1.35,-0.9) |- (SRFF.Qn);
\draw (A1.input 1) -- ++(-0.4,0) |- (1.35,0.9) |- (SRFF.Q);
\pdot{SRFF.Q -| 1.35,0};
\pdot{SRFF.Qn -| 1.35,0};
\draw (SRFF.Q -| 1.35,0) -- (SRFF.Q -| 1.75,0) node[scale=0.75,anchor=west]{$Q$};
\draw (SRFF.Qn) -- (SRFF.Qn -| 1.75,0) node[scale=0.75,anchor=west]{$Q_n$};
\draw (SRFF.Clk) -- (SRFF.Clk -| -1.75,0) node[scale=0.75,anchor=east]{Clk};
\draw (A0.input 1) -- (A0.input 1 -| -1.75,0) node[scale=0.75,anchor=east]{$J$};
\draw (A1.input 2) -- (A1.input 2 -| -1.75,0) node[scale=0.75,anchor=east]{$K$};
\end{scope}
\node (KT) at (8.5,0) {$\begin{array}{cc|c}
J&K&Q_{\mbox{\tiny{next}}}\\\hline
0&0&Q\\
0&1&0\\
1&0&1\\
1&1&Q'
\end{array}$};
\node[anchor=east] (ET) at (14,0) {$\begin{array}{cc|cc}
Q&Q_{\mbox{\tiny{next}}}&J&K\\\hline
0&0&0&-\\
0&1&1&-\\
1&0&-&1\\
1&1&-&0\\
\end{array}$};
\node[anchor=north] (IT) at (I.south) {Symbool};
\node[anchor=north] (IMT) at (4.5,0 |- IT.north) {Implementatie};
\node[anchor=north] (KTT) at (KT.south |- IT.north) {Karakteristieke tabel};
\node[anchor=north] (ETT) at (ET.south |- IT.north) {Excitatietabel};
\end{tikzpicture}
\caption{Jack-Kilby flipflop.}
\figlab{jackKilbyFlipflop}
\end{figure}
\paragraph{Overzicht} We bundelen vervolgens al de latches en flipflops in figuur \ref{fig:latchFlipflopOverview}. Indien een cel leeg is bij dit overzicht is er geen realisatie mogelijk. Zo kunnen we onmogelijk een toggle-latch bouwen: deze zou immers bij een actief kloksignaal continue inverteren, waardoor het uiteindelijke resultaat onvoorspelbaar is. We kunnen elke flipflop verrijken met de assynchrone preset en clear ingangen. Verder bestaan er ook nog varianten van deze componenten waarbij we eerst een negatie toepassen op de ingang. In dat geval plaatsen we een cirkel bij deze ingang. Indien we een negatie toepassen op de klokingang bij een flipflop zal de flipflop dus de waarde memoriseren bij een falling edge, in plaats van een rising edge.
\begin{figure}[hbt]
\centering
\begin{tikzpicture}[decoration={brace}]
\def\dx{3};
\def\dy{-2.5};
\def\ya{0.5*\dy+0.25};
\def\xa{0.5*\dx-0.25};
\node (T0) at (\dx,\ya) {\textbf{Set-Reset (SR)}};
\node (T1) at (2*\dx,\ya) {\textbf{Data (D)}};
\node (T2) at (3*\dx,\ya) {\textbf{Toggle (T)}};
\node (T3) at (4*\dx,\ya) {\textbf{Jack Kilby (JK)}};
\node[rotate=90] (C0) at (\xa,\dy) {\textbf{Ongeklokt}};
\node[rotate=90] (C1) at (\xa,2*\dy) {\textbf{Geklokt}};
\node[rotate=90] (C2) at (\xa,3*\dy) {\textbf{Flankgeklokt}};
\foreach \i in {1,2,3} {
  \draw[thin,dashed] (0.5*\dx+\i*\dx,0 |- T0.north) -- (0.5*\dx+\i*\dx,3.5*\dy);
}
\foreach \i in {1,2} {
  \draw[thin,dashed] (0,0.5*\dy+\i*\dy -| C0.north) -- (4.5*\dx,0.5*\dy+\i*\dy);
}
\draw[very thick] (T0.south -| C0.south) -- (T0.north -| C0.south) -- (4.5*\dx,0 |- T0.north) -- (4.5*\dx,3.5*\dy) -- (0,3.5*\dy -| C2.north) -- (T0.south -| C0.north) -- (T0.south -| C0.south);
\draw[very thick] (4.5*\dx,0 |- T0.south) -- (T0.south -| C0.south) -- (0,3.5*\dy -| C0.south);
\draw [decorate,thick] (4.5*\dx+0.25,0 |- T0.south) -- (4.5*\dx+0.25,2.5*\dy);
\node[rotate=-90,anchor=south] (G0) at (4.5*\dx+0.5,1.5*\dy) {\textbf{Latch}};
\draw [decorate,thick] (4.5*\dx+0.25,2.5*\dy) -- (4.5*\dx+0.25,3.5*\dy);
\node[rotate=-90,anchor=south] (G1) at (4.5*\dx+0.5,3*\dy) {\textbf{Flipflop}};
\foreach\i/\j/\dj/\t/\tx in {1/1/-1/srlatch/NOR,1/1/1/srlatchnand/NAND, 2/1/0/clsrlatch/,2/2/0/cldlatch/, 3/1/-1/srff/Basis,3/1/1/srffx/Uitgebreid,3/2/-1/dff/Basis,3/2/1/dffx/Uitgebreid,3/3/-1/tff/Basis,3/3/1/tffx/Uitgebreid,3/4/-1/jkff/Basis,3/4/1/jkffx/Uitgebreid} {
  \node[scale=0.75,\t] (P) at (\j*\dx+\dj*0.22*\dx,\i*\dy) {};
  \draw(P.south) node[anchor=north,scale=0.75]{\tx};
}
\end{tikzpicture}
\caption{Overzicht van de interfaces van geheugencomponenten.}
\figlab{latchFlipflopOverview}
\end{figure}
\subsection{Registers}
\label{ss:registers}
\paragraph{Register}Een flipflop kan \'e\'en bit opslaan voor \'e\'en of meerdere klokflanken. Meestal willen we echter meerdere bits opslaan om bijvoorbeeld een getal voor te stellen. Dit kunnen we doen door middel van verschillende flipflops die elk een individuele bit opslaan. Een \termen{register} is in dat opzicht eigenlijk ook een $n$-flipflop. Omdat we vaak meerdere bits opslaan defini\"eren we dus het registercomponent. Een register neemt enige functionaliteit weg van de flipflops: zo kunnen we niet beslissen dat een individuele flipflop een bit aan de ingang opslaat, en de andere flipflops niet. Desalniettemin maken ze schema's eenvoudiger en worden registers op chipniveau verkocht. Verder biedt een dergelijk chip naast geheugenopslag ook extra functies aan: bijvoorbeeld schuifregisters en tellers. Een register heeft tradioneel ingangen voor de klok $\mbox{Clk}$, data $D_1, D_2,\ldots, D_n$, load $\mbox{LD}$ en assynchrone preset $\mbox{Pr}^*$ en clear $\mbox{Clr}^*$. Als uitgangen heeft het minstens de data-uitgangen $Q_1, Q_2, \ldots, Q_n$. Enkel indien de load hoog is bij een rising edge zal de waarde die aan de data-ingangen staat opgeslagen worden. Indien dit niet zo is, wordt de vorige waarde behouden. \figref{register} toont de interface van een register, samen met een mogelijke implementatie. We werken hier met D-flipflops en multiplexers om te kiezen tussen het laden van een nieuwe waarde en een vorige waarde. Meestal wordt de clear en preset in negatieve logica ge\"implementeerd\footnote{Vandaar de asterisk (*) bij $\mbox{Clr}^*$ en $\mbox{Pr}^*$.}.
\begin{figure}[hbt]
\centering
\subfigure[Interface]{\begin{tikzpicture}
\node[regd,scale=0.85] (R) at (0,0) {};
\draw (R.Q0) -- ++(0,-0.4) node[scale=0.75,anchor=north]{$Q_0$};
\draw (R.Q1) -- ++(0,-0.4) node[scale=0.75,anchor=north]{$Q_1$};
\draw (R.Q2) -- ++(0,-0.4) node[scale=0.75,anchor=north]{$Q_2$};
\draw (R.Q3) -- ++(0,-0.4) node[scale=0.75,anchor=north]{$Q_3$};
\draw (R.D0) -- ++(0,0.4) node[scale=0.75,anchor=south]{$D_0$};
\draw (R.D1) -- ++(0,0.4) node[scale=0.75,anchor=south]{$D_1$};
\draw (R.D2) -- ++(0,0.4) node[scale=0.75,anchor=south]{$D_2$};
\draw (R.D3) -- ++(0,0.4) node[scale=0.75,anchor=south]{$D_3$};
\draw (R.PR) -- ++(0.4,0) node[scale=0.75,anchor=west]{$\mbox{Pr}^*$};
\draw (R.CLR) -- ++(0.4,0) node[scale=0.75,anchor=west]{$\mbox{Clr}^*$};
\draw (R.LD) -- ++(-0.4,0) node[scale=0.75,anchor=east]{LD};
\draw (R.Clk) -- ++(-0.4,0) node[scale=0.75,anchor=east]{Clk};
\end{tikzpicture}}
\subfigure[Implementatie]{\begin{tikzpicture}
\filldraw[dashed,fill=black!20] (-1.5,-1.25) rectangle (5.5,1.65);
\foreach \i/\j in {0/3,1/2,2/1,3/0} {
  \node[scale=0.75,mux2to1] (M\i) at (1.5*\i-0.6,1.2) {};
  \node[scale=0.5,dffxn] (D\i) at (1.5*\i,0) {};
  \draw (M\i.output) |- (D\i.D);
  \draw (M\i.data1) --++ (0,0.5) node[scale=0.75,anchor=south]{$D_\j$};
  \coordinate (outQ\i) at (D\i.Q -| 1.5*\i+0.6,0);
  \coordinate (clkQ\i) at (1.5*\i-0.6,-1);
  \draw (clkQ\i) |- (D\i.Clk);
  \pdot{outQ\i};
  \draw (D\i.Q) -| (1.5*\i+0.6,-1.45) node[scale=0.75,anchor=north]{$Q_\j$};
  \draw (D\i.PR) -- (D\i.PR |- 0,0.8);
  \draw (D\i.CLR) -- (D\i.CLR |- 0,-0.8);
}
\foreach \i/\ii in {0/1,1/2,2/3} {
  \draw (M\i.data0) -- ++(0,0.1) -| (1.5*\i+0.2,1) -| (outQ\i);
  \draw (M\i.selout0) -- (M\ii.selin0);
  \pdot{D\ii.CLR |- 0,-0.8};
  \pdot{D\ii.PR |- 0,0.8};
  \pdot{clkQ\i};
}
\draw (M3.data0) -- ++(0,0.1) -| (outQ3);
\draw (M0.selin0) -- (M0.selin0 -| -1.75,0) node[scale=0.75,anchor=east]{LD};
\draw (clkQ3) -- (clkQ3 -| -1.75,0) node[scale=0.75,anchor=east]{Clk};
\draw (D0.PR |- 0,0.8) -- (5.75,0.8) node[scale=0.75,anchor=west]{$\mbox{Pr}^*$};
\draw (D0.CLR |- 0,-0.8) -- (5.75,-0.8) node[scale=0.75,anchor=west]{$\mbox{Clr}^*$};
\end{tikzpicture}}
\caption{Interface en implementatie van een 4-bit register.}
\figlab{register}
\end{figure}
\paragraph{Schuifregister}We hebben reeds vermeld dat men de meeste registers uitrust met extra functionaliteit. Een concreet voorbeeld hiervan is het \termen{schuifregister}. Indien we met normale registers werken kunnen we per klokflank tussen twee acties kiezen: een nieuwe waarde inladen (``load''), of de oude waarde behouden. Een schuifregisters voegt daar een functionaliteit aan toe: de oude waarde \'e\'en plaats naar rechts schuiven, en deze waarde bij de rising edge inladen. Hiervoor bevat een schuifregister een extra ingang $\mbox{Shft}$ die indien actief, de waarde van het register opschuift. Sommige shuifregisters bevatten bovendien extra ingangen om te bepalen of er naar links of rechts geschoven moet worden. $\mbox{SerIn}$ is een andere ingang, deze bepaalt welke bit er op de vrijgekomen plaats komt te staan. \figref{shiftRegister} toont de implementatie van een schuifregister die de data naar rechts opschuift.
\begin{figure}[hbt]
\centering
\subfigure[Implementatie]{\begin{tikzpicture}[scale=1.25]
\def\sc{1.25};
\filldraw[dashed,fill=black!20] (-1.5,-1.25) rectangle (5.5,2.15);
\foreach \i/\j in {0/3,1/2,2/1,3/0} {
  \node[scale=0.75*\sc,mux2to1] (M\i) at (1.5*\i-0.6,1.2) {};
  \node[scale=0.75*\sc,mux2to1] (N\i) at (1.5*\i-0.2,1.7) {};
  \draw (N\i.output) -- (N\i.output |- 0,1.45) -| (M\i.data0);
  \node[scale=0.5*\sc,dffxn] (D\i) at (1.5*\i,0) {};
  \draw (M\i.output) |- (D\i.D);
  \draw (M\i.data1) --++ (0,1) node[scale=0.75*\sc,anchor=south]{$D_\j$};
  \coordinate (outQ\i) at (D\i.Q -| 1.5*\i+0.6,0);
  \coordinate (clkQ\i) at (1.5*\i-0.6,-1);
  \draw (clkQ\i) |- (D\i.Clk);
  \pdot{outQ\i};
  \draw (D\i.Q) -| (1.5*\i+0.6,-1.45) node[scale=0.75*\sc,anchor=north]{$Q_\j$};
  \draw (D\i.PR) -- (D\i.PR |- 0,0.8);
  \draw (D\i.CLR) -- (D\i.CLR |- 0,-0.8);
}
\foreach \i/\ii in {0/1,1/2,2/3} {
  \coordinate (datbN\i) at (1.5*\i+0.4,1.95);
  \pdot{datbN\i};
  \draw (N\i.data0) |- (datbN\i) -- (1.5*\i+0.4,1) -| (outQ\i);
  \draw (N\ii.data1) |- (datbN\i);
  \draw (M\i.selout0) -- (M\ii.selin0);
  \draw (N\i.selout0) -- (N\ii.selin0);
  \pdot{D\ii.CLR |- 0,-0.8};
  \pdot{D\ii.PR |- 0,0.8};
  \pdot{clkQ\i};
}
\draw (N3.data0) -- ++(0,0.1) -| (outQ3);
\draw (M0.selin0) -- (M0.selin0 -| -1.75,0) node[scale=0.75*\sc,anchor=east]{LD};
\draw (N0.selin0) -- (N0.selin0 -| -1.75,0) node[scale=0.75*\sc,anchor=east]{Shft};
\draw (N0.data1) |- (-1.75,1.95) node[scale=0.75*\sc,anchor=east]{SerIn};
\draw (clkQ3) -- (clkQ3 -| -1.75,0) node[scale=0.75*\sc,anchor=east]{Clk};
\draw (D0.PR |- 0,0.8) -- (5.75,0.8) node[scale=0.75*\sc,anchor=west]{$\mbox{Pr}^*$};
\draw (D0.CLR |- 0,-0.8) -- (5.75,-0.8) node[scale=0.75*\sc,anchor=west]{$\mbox{Clr}^*$};
\end{tikzpicture}}
\caption{Implementatie van een 4-bit schuifregister.}
\figlab{shiftRegister}
\end{figure}
Schuifregisters worden vooral gebruikt om data serieel te maken: er wordt een getal ingeladen in het register, en vervolgens kunnen we per klokflank de laatste bit doorsturen. Aangezien het getal telkens verder opschuift sturen we zo na $n$-klokcycli een $n$-bit getal door. Dit kan handig zijn indien de kostprijs van meerdere draden te hoog is.
\subsection{Tellers}
\label{ss:counters}
Een andere functionaliteit die men vaak combineert met registers is tellen. Een \termen{teller} maakt het mogelijk om de waarde tijdens een klokflank met \'e\'en op te hogen: \termen{increment}. De meeste tellers laten echter ook toe om naar beneden te tellen: \termen{decrement}. De meeste tellers laten ook toe om een waarde in te laden en deze dan vervolgens in de volgende klokcycli te verhogen of verlagen. Indien de teller op de maximale of minimale voor te stellen waarde komt, treedt na de volgende cyclus een ``\termen{wrap-around}'' op: het getal voor het kleinste getal is het grootste en vice versa. Modulo-tellers wachten niet op de maximaal voor te stellen waarde alvorens deze wrap-around toe te passen. Zo telt een BCD-teller enkel tussen 0 en 9. We zouden een teller kunnen implementeren aan de hand van een register en een opteller\footnote{Analoog aan het schuifregister die uit een schuifoperatie en register bestaat.}. Er bestaan echter goedkopere manieren om tellers te implementeren.
\paragraph{}
Een teller die enkel naar boven telt wordt een \termen{up-counter} genoemd. Analoog wordt een teller die enkel naar beneden telt een \termen{down-counter} genoemd. Een teller die in beide richtingen kan tellen is een \termen{bidirectionial counter} ofwel \termen{bidirectionele teller}. Tellers introduceren ook een aantal nieuwe ingangen:
\begin{itemize}
 \item \termen{Counter Enabled $\mbox{CE}_{\mbox{\tiny{in}}}$}: enkel indien dit signaal hoog is, wordt het tellen uitgevoerd. In het andere geval blijft de waarde uit de vorige klokcyclus behouden.
 \item \termen{Down-Up $D/U^*$}: deze ingang bepaald de richting waarin er geteld wordt. Uit het symbool kunnen we afleiden dat indien het signaal laag is we naar boven tellen, indien het signaal hoog is tellen we naar beneden.
\end{itemize}
Een uitgang die we regelmatig in een teller zien terugkomen in \termen{Counter Enabled $\mbox{CE}_{\mbox{\tiny{out}}}$}. Deze uitgang is hoog op de klokflank waarbij de teller een wrap-around uitvoert. Indien we dan een cascade van tellers bouwen zal de teller die erna geschakeld is bij een wrap-around opgehoogd worden. Op die manier kunnen we bijvoorbeeld met 3 4-bit tellers een 12-bit teller bouwen. Deze uitgang wordt soms ook de ``\termen{Ripple Carry Output (RCO)}'' genoemd.
\begin{figure}[hbt]
\centering
\begin{tikzpicture}[circuit logic US]
\def\sc{0.8};
\filldraw[dashed,fill=black!20] (-1.7*\sc,-2*\sc) rectangle (3*3*\sc+2.5*\sc,1.6*\sc);
\node[anchor=east,scale=\sc] (CE) at (-2*\sc,1.4*\sc) {$\mbox{CE}_{\mbox{\tiny{in}}}$};
\node[anchor=west,scale=\sc] (CEO) at (12*\sc,1.4*\sc) {$\mbox{CE}_{\mbox{\tiny{out}}}$};
\node[anchor=east,scale=\sc] (Clk) at (-2*\sc,-1.4*\sc) {Clk};
\node[anchor=east,scale=\sc] (Clr) at (-2*\sc,-1.7*\sc) {$\mbox{Clr}^*$};
\foreach\i in {0,1,2,3} {
  \node[tffxn,scale=\sc] (T\i) at (3*\sc*\i,0) {};
  \coordinate(Q\i) at (3*\sc*\i+\sc,-2.2*\sc);
  \coordinate(TT\i) at (3*\sc*\i-\sc,1.4*\sc);
  \coordinate (C\i) at (T\i.CLR |- Clr);
  \draw (TT\i) |- (T\i.T);
  \pdot{TT\i};
  \draw (T\i.CLR) -- (C\i);
  \draw (T\i.Q) -| (Q\i) node[anchor=north,scale=\sc]{$Q_\i$};
}
\foreach \i/\ii in {0/1,1/2,2/3} {
  \pdot{C\i};
  \draw (T\i.Qn) -- (T\ii.Clk);
}
\draw (CEO) -- (CE);
\draw (C3) -- (Clr);
\draw (Clk) -- (Clk -| TT0) |- (T0.Clk);
\end{tikzpicture}
\caption{Asynchrone 4-bit teller}
\figlab{asynchroneCounter}
\end{figure}
\paragraph{Asynchrone teller}
Bij een \termen{asynchrone teller}, \termen{asynchronous counter} ofwel \termen{ripple counter} wordt de verandering van de bittoestand van \'e\'en van de bits gebruikt als klokingang voor de volgende bit. Dit leidt tot minimaal hardwaregebruik, maar heeft hetzelfde probleem als een ripple adder: het signaal dient te propageren door de bits\footnote{Hiervan komt overigens de notie van asynchroniciteit: niet alle bits veranderen op hetzelfde moment van waarde.}, wat leidt tot grote vertragingen als het om veel bits gaat. Vooral indien we logica koppelen aan hoge bits kunnen deze vertragingen nefast zijn en leiden tot een lage performantie. We kunnen een down-counter realiseren door de $Q_n$-uitgang aan de toggle-ingang van de volgende flipflop te koppelen (en niet de $Q$-uitgang). Eventueel kunnen we deze schakeling dus zelfs uitbreiden door een multiplexer te plaatsen tussen de flipflops die selecteert tussen $Q$ en $Q_n$ en zo de gebruiker laat kiezen in welke richting de teller werkt. Het nadeel is dat ook deze multiplexer een vertraging induceert. Een eenvoudige implementatie van een asynchrone teller staat op figuur \ref{fig:asynchroneCounter}.
\begin{figure}[hbt]
\centering
\begin{tikzpicture}[circuit logic US]
\def\sc{0.8};
\filldraw[dashed,fill=black!20] (-1.7*\sc,-2*\sc) rectangle (3*3*\sc+2.5*\sc,1.6*\sc);
\node[anchor=east,scale=\sc] (CE) at (-2*\sc,1.4*\sc) {$\mbox{CE}_{\mbox{\tiny{in}}}$};
\node[anchor=west,scale=\sc] (CEO) at (12*\sc,1.4*\sc) {$\mbox{CE}_{\mbox{\tiny{out}}}$};
\node[anchor=east,scale=\sc] (Clk) at (-2*\sc,-1.4*\sc) {Clk};
\node[anchor=east,scale=\sc] (Clr) at (-2*\sc,-1.7*\sc) {$\mbox{Clr}^*$};
\foreach\i in {0,1,2,3} {
  \node[tffxn,scale=\sc] (T\i) at (3*\sc*\i,0) {};
  \coordinate(Q\i) at (3*\sc*\i+\sc,-2.2*\sc);
  \coordinate(TT\i) at (3*\sc*\i-\sc,1.4*\sc);
  \coordinate(Cl\i) at (3*\sc*\i-\sc,-1.4*\sc);
  \coordinate (C\i) at (T\i.CLR |- Clr);
  \draw (T\i.CLR) -- (C\i);
  \draw (T\i.Q) -| (Q\i) node[anchor=north,scale=\sc]{$Q_\i$};
  \draw (T\i.Clk) -| (Cl\i);
  \node[and gate,rotate=-90,scale=\sc] (A\i) at (3*\sc*\i+1.5*\sc,0.5*\sc) {};
  \draw (A\i.input 2) -- ++(0,0.2*\sc) -| (T\i.Q -| Q\i);
  \pdot{T\i.Q -| Q\i};
}
\draw (TT0) |- (T0.T);
\foreach \i/\ii in {0/1,1/2,2/3} {
  \draw (T\ii.T) -| (TT\ii |- 0,-0.2*\sc) -| (A\i.output);
  \draw (T\ii.T -| TT\ii) -- (TT\ii |- CE) -| (A\ii.input 1);
  \pdot{C\i};
  \pdot{Cl\i};
  \pdot{T\ii.T -| TT\ii};
}
\draw (CEO) -- ++(-1.5*\sc,0) |- (A3.output |- 0,-0.2*\sc) -- (A3.output);
\draw (A0.input 1) |- (CE);
\draw (C3) -- (Clr);
\draw (Clk) -- (Cl3);
\end{tikzpicture}
\caption{Synchrone 4-bit teller}
\figlab{synchroneCounter}
\end{figure}
\paragraph{Synchrone teller}
Een \termen{synchrone teller} laat alle bits wel tegelijk van waarde veranderen. Dit doen we door de volgende toestand reeds vooraf uit te rekenen en bij het kloksignaal deze toestand op te slaan in de flipflops. Hoe we deze toestand berekenen staat ons in principe vrij net als de keuze van het type flipflop. \figref{synchroneCounter} toont een implementatie met T-flipflops. Merk op dat ondanks het feit dat de nieuwe toestand berekend wordt voor de klokflank de teller daarom geen vertraging kan teweeg brengen: het berekenen van de volgende toestand van een teller met een grote woordlengte vraagt immers ook veel tijd (en kan dus het kritieke pad worden). Het voordeel van een synchrone teller is dat we deze berekening parallel doen met andere berekeningen die van de teller afhangen. Doordat de volgende toestand ook meteen beschikbaar is zullen berekeningen die afhangen van de teller ook onmiddellijk kunnen worden uitgerekend.
\begin{figure}[hbt]
\centering
\subfigure[Algemene implementatie]{\begin{tikzpicture}[circuit logic US]
\def\sc{0.8};
\def\ya{2.8*\sc};
\def\yb{\ya-0.6*\sc};
\filldraw[dashed,fill=black!20] (-1.7*\sc,-2*\sc) rectangle (3*3*\sc+2.5*\sc,3.6*\sc);
\node[anchor=east,scale=\sc] (CE) at (-2*\sc,2.7*\sc) {$\mbox{CE}_{\mbox{\tiny{in}}}$};
\node[anchor=west,scale=\sc] (CEO) at (12*\sc,2.7*\sc) {$\mbox{CE}_{\mbox{\tiny{out}}}$};
\node[anchor=east,scale=\sc] (DU) at (-2*\sc,3.4*\sc) {$\mbox{D}/\mbox{U}^*$};
\node[anchor=east,scale=\sc] (LD) at (-2*\sc,1.4*\sc) {LD};
\node[anchor=east,scale=\sc] (Clk) at (-2*\sc,-1.4*\sc) {Clk};
\node[anchor=east,scale=\sc] (Clr) at (-2*\sc,-1.7*\sc) {$\mbox{Clr}^*$};
\foreach\i in {0,1,2,3} {
  \node[dffxn,scale=\sc] (T\i) at (3*\sc*\i,0) {};
  \coordinate(TT\i) at (3*\sc*\i-\sc,1.4*\sc);
  \node[mux2to1,scale=\sc] (M\i) at (TT\i) {};
  \coordinate(Q\i) at (3*\sc*\i+\sc,-2.2*\sc);
  \coordinate(Cl\i) at (3*\sc*\i-\sc,-1.4*\sc);
  \coordinate (C\i) at (T\i.CLR |- Clr);
  \draw (T\i.CLR) -- (C\i);
  \draw (M\i.output) |- (T\i.D);
  \draw (M\i.data1) -- (M\i.data1 |- 0,3.9*\sc) node[anchor=south,scale=\sc]{$D_\i$};
  \draw (T\i.Q) -| (Q\i) node[anchor=north,scale=\sc]{$Q_\i$};
  \pdot{T\i.Q -| Q\i};
  \draw (T\i.Clk) -| (Cl\i);
  \node[counterdir,anchor=Ei,scale=\sc] (CD\i) at (CE -| 3*\sc*\i-0.5*\sc,0) {};
  \draw (CD\i.Di) -| (M\i.data0);
  \draw (T\i.Q -| Q\i) -- (Q\i |- M\i.north) -| (CD\i.Qi);
  \coordinate (DU\i) at (CD\i.Dir |- DU);
  \draw (CD\i.Dir) -- (DU\i);
  %\node[and gate,scale=0.75*\sc] (A\i) at (3*\sc*\i+2.5*\sc,\ya) {};
  %\node[xor gate,rotate=180,scale=0.75*\sc] (Xa\i) at (3*\sc*\i-0.35*\sc,\yb) {};
  %\node[xor gate,scale=0.75*\sc] (Xb\i) at (3*\sc*\i+\sc,\yb) {};
  %\coordinate (Xm\i) at (3*\sc*\i+0.325*\sc,0 |- Xb\i.input 2);
  %\coordinate (DU\i) at (3*\sc*\i+0.55*\sc,0 |- DU);
  %\coordinate (CE\i) at (3*\sc*\i+0.15*\sc,0 |- CE);
  %\pdot{Xm\i};
  %\pdot{CE\i};
  %\draw (Xm\i) -- (Xm\i |- M\i.data0) -| (T\i.Q -| Q\i);
  %\draw (Xa\i.output) -| (M\i.data0);
  %\draw (Xa\i.input 1) -- (Xb\i.input 2);
  %\draw (Xb\i.input 1) -| (DU\i);
  %\draw (Xa\i.input 2) -| (CE\i);
  %\draw (Xb\i.output) -- ++(0.2*\sc,0) |- (A\i.input 2);
}
\foreach \i/\ii in {0/1,1/2,2/3} {
  \draw (M\i.selout0) -- (M\ii.selin0);
  \draw (CD\i.Eii) -- (CD\ii.Ei);
  \pdot{DU\i};
  %\draw (A\i.output) -- (A\i.output -| 3*\sc*\ii+0.325*\sc,0) |- (A\ii.input 1);
}
%\draw (CE) -- (CE -| 0.325*\sc,0) |- (A0.input 1);
\draw (CE) -- (CD0.Ei);
\draw (CEO) -- (CD3.Eii);
\draw (M0.selin0) -- (LD);
\draw (DU3) -- (DU);
\draw (C3) -- (Clr);
\draw (Clk) -- (Cl3);
\end{tikzpicture}
\figlab{loadableSynchroneCounter}}
\subfigure[Interface]{\begin{tikzpicture}[circuit logic US]
\node[counterdbit] (C) at (0,0) {};
\draw (C.Q0) -- ++(0,-0.4) node[scale=0.75,anchor=north]{$Q_0$};
\draw (C.Q1) -- ++(0,-0.4) node[scale=0.75,anchor=north]{$Q_1$};
\draw (C.Q2) -- ++(0,-0.4) node[scale=0.75,anchor=north]{$Q_2$};
\draw (C.Q3) -- ++(0,-0.4) node[scale=0.75,anchor=north]{$Q_3$};
\draw (C.D0) -- ++(0,0.4) node[scale=0.75,anchor=south]{$D_0$};
\draw (C.D1) -- ++(0,0.4) node[scale=0.75,anchor=south]{$D_1$};
\draw (C.D2) -- ++(0,0.4) node[scale=0.75,anchor=south]{$D_2$};
\draw (C.D3) -- ++(0,0.4) node[scale=0.75,anchor=south]{$D_3$};
\draw (C.CEIN) -- (-2.025,0 |- C.CEIN) node[scale=0.75,anchor=east]{$\mbox{CE}_{\mbox{\tiny{in}}}$};
\draw (C.CEOUT) -- ++(0.4,0) node[scale=0.75,anchor=west]{$\mbox{CE}_{\mbox{\tiny{out}}}$};
\draw (C.CLR) -- (-2.125,0 |- C.CLR) node[scale=0.75,anchor=east]{$\mbox{Clr}^*$};
\draw (C.LD) -- (-2.125,0 |- C.LD) node[scale=0.75,anchor=east]{LD};
\draw (C.DU) -- (-2.125,0 |- C.DU) node[scale=0.75,anchor=east]{D/U$^*$};
\draw (C.Clk) -- (-2.125,0 |- C.Clk) node[scale=0.75,anchor=east]{Clk};
\end{tikzpicture}
\figlab{counterInterface}}
\subfigure[Hulpcomponent]{\begin{tikzpicture}[circuit logic US]
\def\sc{0.8};
\filldraw[dashed,fill=black!20] (-2.75*\sc,-1.25*\sc) rectangle (2.75*\sc,1.25*\sc);
\node[anchor=south,scale=\sc] (Dir) at (0,1.5*\sc) {Dir};
\node[anchor=north,scale=\sc] (Qi) at (0,-1.5*\sc) {$Q_i$};
\node[anchor=east,scale=\sc] (Ei) at (-3*\sc,0.5*\sc) {$E_i$};
\node[anchor=east,scale=\sc] (Di) at (-3*\sc,-0.5*\sc) {$D_i$};
\node[xor gate,scale=\sc,rotate=180] (Xa) at (-1.25*\sc,-0.5*\sc) {};
\node[xor gate,scale=\sc] (Xb) at (0.75*\sc,-0.5*\sc) {};
\node[and gate,scale=\sc,anchor=input 1] (A) at (1.75*\sc,0.5*\sc) {};
\node[anchor=west,scale=\sc] (Eii) at (3*\sc,0 |- A.output) {$E_{i+1}$};
\draw (Xa.input 1) -- (Xb.input 2);
\draw (Xb.input 1) -| (Dir);
\draw (Xa.output) -- (Di);
\draw (Ei) -- (A.input 1);
\draw (Xb.output) -- ++(0.25*\sc,0) |- (A.input 2);
\draw (Xa.input 2) -| (Ei -| -0.5*\sc,0);
\draw (A.output) -- (Eii);
\draw (Qi) -- (Qi |- Xa.input 1);
\pdot{Qi |- Xa.input 1};
\pdot{Ei -| -0.5*\sc,0};
\end{tikzpicture}
\figlab{loadableSynchroneCounterLambda}}
\caption{Parallel-laarbare bidirectionele 4-bit teller.}
\figlab{loadableSynchroneCounterTotal}
\end{figure}
\paragraph{Parallel-laadbare bidirectionele teller}
Een speciaal geval van een synchrone teller is een \termen{parallel-laadbare bidirectionele teller}. Een implementatie van zo'n teller staat op figuur \ref{fig:loadableSynchroneCounter}. Deze teller is \termen{parallel laadbaar} wanneer we de waarde van de teller kunnen zetten op een ingegeven waarde in \'e\'en klokflank\footnote{Het schuifregister of figuur \ref{fig:shiftRegister} kan bijvoorbeeld ook sequentieel geladen worden, waarbij we per klokflank een bit van het getal inschuiven.}. Bij het laden wordt de LD ingang hoog gezet. In het andere geval dienen we elke bit te berekenen. Hiervoor wordt op figuur \ref{fig:loadableSynchroneCounter} een naamloos component ingevoerd. De nieuwe waarde van een bit wordt ge\"inverteerd indien er geteld wordt op deze bit. In het andere geval blijft de bit onveranderd. Een bit wordt opgeteld indien de vorige bit geteld wordt en de bit \'e\'en is bij een telling naar boven (analoog aan een overdracht (``carry'') dus), of nul en een telling naar beneden (een lening (``borrow'') dus). Bij het berekenen van de eerste bit $Q_0$ gebruiken we logischerwijs de counter enabled CE ingang. \figref{loadableSynchroneCounterLambda} toont een implementatie voor een dergelijk component. De teller die we hiermee ontwikkelen verenigt de meeste functionaliteiten die we in tellers kunnen terugvinden. \figref{counterInterface} toont een interface die vaak gebruik wordt voor tellers. Indien een functionaliteit niet wordt aangeboden wordt de vermelding eenvoudigweg weggelaten.
\paragraph{Modulo-teller}Tot nu toe hebben we enkel teller geconstrueerd die tellen van 0 tot het maximaal voor te stellen getal. Bij een $n$-bit teller is dit dus $2^n-1$. In de praktijk komt het geregeld voor dat we tellen tot een bepaalde waarde om daarna terug bij 0 te beginnen. Stel bijvoorbeeld dat we een teller maken die het aantal minuten bijhoudt. Indien dit getal boven de 60 gaat komt het aantal minuten terug op 0, en dient het aantal uren verhoogd te worden. Ook controllers die elektronica aansturen dienen soms een beperkt aantal toestanden met de regelmaat van de klok te herhalen, zelden zijn dit aantal toestanden een macht van twee. Voor dergelijke problemen kunnen we een \termen{Modulo-teller} gebruiken. Een modulo-teller bestaat traditioneel uit een teller met daarrond extra logica die indien de maximale waarde wordt vastgesteld, in de volgende klokflank een 0 in de teller laadt. We kunnen deze teller veralgemenen door ook het tellen naar beneden toe te laten, in dat geval dient de maximale waarde in de klokcyclus na 0 ingeladen te worden. \figref{moduloCounter} toont een algemeen geval van een 4-bit teller.
\begin{figure}[hbt]
\centering
\subfigure[Algemeen geval]{
\begin{tikzpicture}[circuit logic US]
\node[counterdbit] (C) at (0,0) {};%1,392625
\draw (C.CLR) -- (-4,0 |- C.CLR) node[scale=0.75,anchor=east]{Clr$^*$};
\draw (C.Clk) -- (-4,0 |- C.Clk) node[scale=0.75,anchor=east]{Clk};
\draw (C.DU) -- (-4,0 |- C.DU) node[scale=0.75,anchor=east]{D/U$^*$};
\draw (C.CEIN) -- (-4,0 |- C.CEIN) node[scale=0.75,anchor=east]{$\mbox{CE}_{\mbox{\tiny{in}}}$};
\draw (C.Q0) -- ++(0,-3) node[scale=0.75,anchor=north]{$Q_0$};
\draw (C.Q1) -- ++(0,-3) node[scale=0.75,anchor=north]{$Q_1$};
\draw (C.Q2) -- ++(0,-3) node[scale=0.75,anchor=north]{$Q_2$};
\draw (C.Q3) -- ++(0,-3) node[scale=0.75,anchor=north]{$Q_3$};
\node[or gate,rotate=90,scale=0.75] (O) at (-3.5,-1.25) {};
\node[mux8to4,xscale=1.450651042,yscale=0.75] (M) at (0,2) {};%0.96
\node[and gate,rotate=180,scale=0.75,inputs={normal,normal,inverted,inverted,inverted,inverted}] (A1) at (-2.75,-2) {};
\draw (A1.output) -| (O.input 2);
\coordinate (DUT) at (C.DU -| -2.25,0);
\pdot{DUT};
\coordinate (CET) at (C.CEIN -| -2.5,0);
\pdot{CET};
\coordinate (LDT) at (C.LD -| -2,0);
\pdot{LDT};
\draw (LDT) |- (2,1.25) |- (C.CEOUT -| 2.3,0) node[scale=0.75,anchor=west]{$\mbox{CE}_{\mbox{\tiny{out}}}$};
\draw (DUT) |- (M.selin0);
\draw (A1.output |- 0,-2.75) rectangle (A1.input 1 |- 0,-3.75);
\draw (A1.output |- 0,-3.25) -| (O.input 1);
\foreach \x in {0,1,2,3} {
  \draw (M.output\x) -- (C.D\x);
  \draw (M.data0\x) -- ++(0,0.3) node[scale=0.75,anchor=south]{$0$};
  \draw (M.data1\x) -- ++(0,0.3);
}
\draw (-1.160520834,2.75) node[scale=0.75,anchor=south]{MAX};
\foreach\x/\y in {3/3,2/4,1/5,0/6} {
  \draw (A1.input \y) -- (A1.input \y -| C.Q\x);
  \draw (A1.input 1 |- 0,-3.75+\y/7) -- (0,-3.75+\y/7 -| C.Q\x);
  \pdot{0,-3.75+\y/7 -| C.Q\x};
  \pdot{A1.input \y -| C.Q\x};
}
\draw (A1 |- 0,-3.25) node[scale=0.75,text width=0.75 cm]{= MAX};
\draw (A1.input 2) -- (A1.input 2 -| -1.5,0);
\pdot{A1.input 2 -| -1.5,0};
\draw (A1.input 1) -- (A1.input 1 -| -1.25,0);
\pdot{A1.input 1 -| -1.25,0};
\draw (DUT) |- (-1.25,-1.25) |- (A1.input 1 |- 0,-3.75+1/7);
\draw (CET) |- (-1.5,-1.5) |- (A1.input 1 |- 0,-3.75+2/7);
\draw (O.output) |- (C.LD);
\end{tikzpicture}
\figlab{moduloCounter}}
\subfigure[BCD-teller]{
\begin{tikzpicture}[circuit logic US]
\node[counterdbit] (C) at (0,0) {};%1,392625
\draw (C.CLR) -- (-4,0 |- C.CLR) node[scale=0.75,anchor=east]{Clr$^*$};
\draw (C.Clk) -- (-4,0 |- C.Clk) node[scale=0.75,anchor=east]{Clk};
\draw (C.DU) -- (-4,0 |- C.DU) node[scale=0.75,anchor=east]{D/U$^*$};
\draw (C.CEIN) -- (-4,0 |- C.CEIN) node[scale=0.75,anchor=east]{$\mbox{CE}_{\mbox{\tiny{in}}}$};
\draw (C.Q0) -- ++(0,-3) node[scale=0.75,anchor=north]{$Q_0$};
\draw (C.Q1) -- ++(0,-3) node[scale=0.75,anchor=north]{$Q_1$};
\draw (C.Q2) -- ++(0,-3) node[scale=0.75,anchor=north]{$Q_2$};
\draw (C.Q3) -- ++(0,-3) node[scale=0.75,anchor=north]{$Q_3$};
\node[or gate,rotate=90,scale=0.75] (O) at (-3.5,-1.25) {};
\node[mux8to4,xscale=1.450651042,yscale=0.75] (M) at (0,2) {};%0.96
\node[and gate,rotate=180,scale=0.75,inputs={normal,normal,inverted,inverted,inverted,inverted}] (A1) at (-2.75,-2) {};
\node[and gate,rotate=180,scale=0.75,inputs={normal,normal,normal,inverted,inverted,normal}] (A2) at (-2.75,-3.25) {};
\draw (A1.output) -| (O.input 2);
\coordinate (DUT) at (C.DU -| -2.25,0);
\pdot{DUT};
\coordinate (CET) at (C.CEIN -| -2.5,0);
\pdot{CET};
\coordinate (LDT) at (C.LD -| -2,0);
\pdot{LDT};
\draw (LDT) |- (2,1.25) |- (C.CEOUT -| 2.3,0) node[scale=0.75,anchor=west]{$\mbox{CE}_{\mbox{\tiny{out}}}$};
\draw (DUT) |- (M.selin0);
\draw (A1.output |- 0,-3.25) -| (O.input 1);
\foreach \x/\y in {0/1,1/0,2/0,3/1} {
  \draw (M.output\x) -- (C.D\x);
  \draw (M.data0\x) -- ++(0,0.3) node[scale=0.75,anchor=south]{$0$};
  \draw (M.data1\x) -- ++(0,0.3) node[scale=0.75,anchor=south]{$\y$};
}
\foreach\x/\y in {3/3,2/4,1/5,0/6} {
  \draw (A1.input \y) -- (A1.input \y -| C.Q\x);
  \draw (A2.input \y) -- (A2.input \y -| C.Q\x);
  \pdot{A1.input \y -| C.Q\x};
  \pdot{A2.input \y -| C.Q\x};
}
\draw (A1.input 2) -- (A1.input 2 -| -1.5,0);
\pdot{A1.input 2 -| -1.5,0};
\draw (A1.input 1) -- (A1.input 1 -| -1.25,0);
\pdot{A1.input 1 -| -1.25,0};
\draw (DUT) |- (-1.25,-1.25) |- (A2.input 1);
\draw (CET) |- (-1.5,-1.5) |- (A2.input 2);
\draw (O.output) |- (C.LD);
\end{tikzpicture}
\figlab{bcdCounter}}
\caption{4-bit modulo-tellers.}
\end{figure}
We tellen hierbij van 0 tot en met MAX. Uiteraard moet MAX in dit geval wel voor te stellen zijn met 4-bit. Voor een $n$-bit modulo-counter geldt dus steeds: MAX$\leq 2^n$. Verder dienen we \'e\'en component afhankelijk van MAX te synthetiseren. Dit component vergelijkt de waarde die op de $Q_i$-uitgangen staat met MAX, indien deze aan elkaar gelijk zijn en $\mbox{D/U}^*$ en $\mbox{CE}_{\mbox{\tiny{in}}}$ hoog zijn, dient een 1 aan de uitgang van dit component te verschijnen, in alle andere gevallen een 0. Voor eender welke MAX is dit te realiseren met een AND-poort\footnote{Waarom?}. Verder zien we op de figuur ook een variant van een multiplexer. Deze multiplexer stelt eigenlijk 4 multiplexer voor die elk een bit uit het linkse en een bit uit het rechtse vak als invoer hebben, en die selecteren met dezelfde ingang, namelijk: $\mbox{D/U}^*$. Vanaf de volgende sectie zullen we vaak met registers met een groot aantal bits werken. In dat geval besparen ontwerpers zich het tekenen van de verschillende parallelle lijnen door \'e\'en lijn te tekenen. Verder wordt er dus gebruik gemaakt van de notatie van de multiplexers om aan te duiden dat elke bit van deze lijnen door een dergelijk component gaat.
\paragraph{BCD-teller}
Een speciaal geval van een modulo teller is een \termen{BCD-teller}. Deze teller telt van 0 tot en met 9. BCD ofwel Binary Coded Decimal werd reeds eerder besproken in subsectie \ref{ss:bcd}. BCD-tellers zijn populair bij het voorstellen van getallen die ook snel naar de gebruiker moeten worden gecommuniceerd. Dit komt omdat bij de omzetting van een getal naar het equivalent op bijvoorbeeld seven-segment displays, we cijfer per cijfer kunnen werken. Indien we het getal binair opslaan wordt deze omzetting veel complexer. Op figuur \ref{fig:bcdCounter} implementeren we dan ook een BCD-teller. Vermits 9 binair voorgesteld wordt door $1001_2$ kunnen we een AND-poort realiseren die in dat geval een 1 op de uitgang plaatst. Verder dienen we dus ook deze waarde in de multiplexer in te brengen. Een oplettende lezer zal misschien merken dat sommige van de multiplexers in dat geval zinloos worden, omdat we bijvoorbeeld moeten kiezen tussen een 0 en een 0.
\section{Synchrone schakelingen}
\label{s:synchroneSequence}
Nu we de bouwstenen hebben beschreven om een sequenti\"ele schakeling te bouwen zullen we een stappenplan ontwikkelen uit een set van specificaties een sequenti\"ele schakeling te ontwikkelen. In deze sectie gaan we ervan uit dat deze schakeling synchroon is. Dit betekent dat er sprake is van een klok die de schakeling aanstuurt. Het stappenplan kan opgedeeld worden in volgende stappen:
\begin{enumerate}
 \item Het opstellen van een toestandsdiagram uit de specificaties.
 \item Het minimaliseren van het aantal toestanden door F-gelijke toestanden te bepalen.
 \item Het implementeren van de toestanden in een geheugen
 \begin{enumerate}
  \item Het coderen van deze toestanden in geheugenelementen.
  \item Het kiezen van het type flipflop die de toestanden bijhoudt.
 \end{enumerate}
 \item Het implementeren van de combinatorische logica
 \begin{enumerate}
  \item Implementeren van logica die de volgende toestand berekent.
  \item Implementeren van logica die de uitgangen (uitvoer) berekent.
 \end{enumerate}
\end{enumerate}
\subsection{Leidende voorbeelden}
We zullen dit stappenplan doorlopen met behulp van twee voorbeelden. Het eerste is een Moore-FSM. Uit subsectie \ref{ss:classificationSequential} weten we nog dat de uitgang dus volledig bepaald wordt door de toestand, en een Mealy-FSM waarbij ook de invoer een rol speelt. Het stappenplan zelf verschilt meestal op enkele punten tussen het proces voor een Moore-machine en een Mealy-machine. In dat geval zullen op de afbeeldingen aan de linkerkant de Moore-machine staan, en aan de rechterkant de Mealy-machine. Indien dit niet relevant is voor het concept zullen we telkens gebruik maken van de Mealy-machine. Deze keuze is uiteraard louter arbitrair.
\paragraph{}
We implementeren een Moore-machine die 1 op de uitvoer aanlegt als de laatste drie klokcycli afwisselend een 0, 1 en 0 zijn. We implementeren een min of meer equivalente constructie voor een Mealy-machine. Alleen dienen hier de twee laatste klokcycli een 0 en 1 op de ingang aangelegd te worden, en dient er op het moment zelf een 0 op de ingang te staan. De Mealy-machine is dus de Moore-machine maar met \'e\'en klokcyclus verschoven.
\subsection{Stap 1: opstellen van het toestandsdiagram}
Het opstellen van het \termen{toestandsdiagram} is meestal de moeilijkste stap. Dit komt omdat men niet formeel kan uitdrukken hoe men uit de specificaties een toestandsdiagram opstelt. Het opstellen van een dergelijk diagram omvat echter meestal dezelfde vaardigheden als deze die bij bijvoorbeeld programmeren aan bod komen. Door middel van oefening kan men dan ook bekwaamheid verwerven.
\begin{figure}[hbt]
\centering
\subfigure[Moore-machine]{\begin{tikzpicture}[->,shorten >=1pt,auto,node distance=2cm,on grid,semithick,state/.style=state with output,every state/.style={draw=black!50,very thick,fill=black!20,scale=0.75}]
\node[state] (H) {$H$\nodepart{lower} $0$};
\node[state] (I) [right=of H] {$I$\nodepart{lower} $0$};
\node[state] (E) [above right=of I] {$E$\nodepart{lower} $0$};
\node[state] (M) [below right=of E] {$M$\nodepart{lower} $0$};
\node[state] (K) [right=of M] {$K$\nodepart{lower} $0$};
\node[state] (L) [below=of H] {$L$\nodepart{lower} $0$};
\node[state] (J) [below=of I] {$J$\nodepart{lower} $1$};
\node[state] (N) [below=of M] {$N$\nodepart{lower} $0$};
\node[state] (O) [below=of K] {$O$\nodepart{lower} $0$};
\node[state] (D) [left=of E] {$D$\nodepart{lower} $0$};
\node[state] (F) [below right=of J] {$F$\nodepart{lower} $0$};
\node[state] (G) [right=of F] {$G$\nodepart{lower} $0$};
\node[state] (B) [above=of H] {$B$\nodepart{lower} $0$};
\node[state] (C) [below=of L] {$C$\nodepart{lower} $0$};
\node[state,initial,initial text=RST, initial where=right] (A) [above=of B] {$A$\nodepart{lower} $0$};
\path (H) edge [loop above] node {0} (H)
          edge node {1} (I)
      (I) edge[swap] node {0} (J)
          edge[bend left] node {1} (K)
      (M) edge[bend left,swap] node {0} (J)
          edge node {1} (K)
      (K) edge node {0} (N)
          edge node {1} (O)
      (L) edge node {0} (H)
          edge node {1} (I)
      (J) edge node {0} (L)
          edge[bend left,swap] node {1} (M)
      (N) edge[bend left] node {0} (L)
          edge[swap] node {1} (M)
      (O) edge node {0} (N)
          edge [loop below] node {1} (O)
       (D) edge[swap] node {0} (H)
           edge node {1} (I)
       (E) edge node {0} (J)
           edge[bend left] node {1} (K)
       (F) edge[bend left] node {0} (L)
           edge node {1} (M)
       (G) edge node {0} (N)
           edge[swap] node {1} (O)
       (B) edge[bend left,swap] node {0} (D)
           edge[bend left] node {1} (E)
       (C) edge[bend right] node {0} (F)
           edge[bend right] node {1} (G)
       (A) edge node {0} (B)
           edge[bend right] node {1} (C);
\end{tikzpicture}}
\subfigure[Mealy-machine]{\begin{tikzpicture}[->,shorten >=1pt,auto,node distance=2cm,on grid,semithick,every state/.style={draw=black!50,very thick,fill=black!20,scale=0.75}]
\node[state] (D) {$D$};
\node[state] (B) [above right=of D] {$B$};
\node[state] (E) [below right=of B] {$E$};
\node[state,initial,initial text=RST, initial where=right] (A) [below right=of E]{$A$};
\node[state] (G) [below left=of A] {$G$};
\node[state] (C) [below left=of G] {$C$};
\node[state] (F) [above left=of C] {$F$};
\path (D) edge node[swap] {1/0} (E)
          edge [loop above] node {0/0} (D)
      (E) edge[bend left,swap] node {0/1} (F)
          edge node {1/0} (G)
      (F) edge node {0/0} (D)
          edge[bend left,swap] node {1/0} (E)
      (G) edge node[swap] {0/0} (F)
          edge [loop below] node {1/0} (G)
      (B) edge node {0/0} (D)
          edge node {1/0} (E)
      (C) edge node {0/0} (F)
          edge node {1/0} (G)
      (A) edge[bend right,swap] node {0/0} (B)
          edge[bend left] node {1/0} (C);
\end{tikzpicture}}
\caption{Toestandsdiagrammen van de leidende voorbeelden.}
\figlab{toestandsdiagram}
\end{figure}
\paragraph{Het toestandsdiagram} Alvorens we een toestandsdiagram kunnen opstellen zullen we eerst een toestandsdiagram formeel defini\"eren. Een toestandsdiagram bestaat uit een set van \termen{toestanden}. Een toestand duiden we aan met een ellips. Meestal benoemen we toestanden met een hoofdletter. Dit is niet verplicht maar maakt het makkelijk om naar een toestand te verwijzen. Verder zijn er ook \termen{transities}: overgangen van de ene toestand naar de andere\footnote{Een transitie kan ook naar dezelfde toestand gaan.} toestand onder een bepaalde \termen{ingangscombinatie}. Een transitie duiden we dan ook aan met behulp van een gerichte pijl tussen de twee toestanden. Voor elke mogelijke ingangscombinatie dienen we in elke toestand een transitie te voorzien. De ingangscombinatie waarbij de transitie van toepassing is noteren we bij de gerichte pijl. Het kloksignaal is in een synchrone sequenti\"ele schakeling geen onderdeel van de ingangscombinatie. Tot slot bevat een toestandsdiagram ook de \termen{initalisatie}. Een gerichte pijl die naar een bepaalde toestand wijst maar niet uit een toestand komt. Het is de eerste toestand waarin de schakeling zich bevindt. Verder keert de schakeling ook naar deze toestand terug als de gebruiker een reset-operatie op de schakeling uitvoert. De \termen{uitgangscombinatie} wordt ook weergegeven op het toestandsdiagram. Deze verschilt uiteraard tussen een Moore-machine en een Mealy-machine: bij een Moore-machine worden de uitgangen bij de toestanden genoteerd, dus in de ellips. Vermits de uitgangen bij een Mealy-machine afhangen van de ingangscombinatie wordt de uitgang bij de transitiepijlen gezet. We maken hierbij een onderscheid tussen de ingangscombinatie en de uitgang door hier een slash (``/'') tussen te plaatsen. Op figuur \ref{fig:toestandsdiagram} geven we de toestandsdiagrammen van de Moore- en Mealy-machine weer.
\begin{table}[hbt]
\centering
\subtable[Moore-machine]{\small{\begin{tabular}{c|cc|c}
Toestand&0&1&Uitgang\\\hline
$A$&$B$&$C$&0\\
$B$&$D$&$E$&0\\
$C$&$F$&$G$&0\\
$D$&$H$&$I$&0\\
$E$&$J$&$K$&0\\
$F$&$L$&$M$&0\\
$G$&$N$&$O$&0\\
$H$&$H$&$I$&0\\
$I$&$J$&$K$&0\\
$J$&$L$&$M$&1\\
$K$&$N$&$O$&0\\
$L$&$H$&$I$&0\\
$M$&$J$&$K$&0\\
$N$&$L$&$M$&0\\
$O$&$N$&$O$&0
\end{tabular}}}
\subtable[Mealy-machine]{\small{\begin{tabular}{c|cc}
Toestand&0&1\\\hline
$A$&$B/0$&$C/0$\\
$B$&$D/0$&$E/0$\\
$C$&$F/0$&$G/0$\\
$D$&$D/0$&$E/0$\\
$E$&$F/1$&$G/0$\\
$F$&$D/0$&$E/0$\\
$G$&$F/0$&$G/0$
\end{tabular}}}
\caption{Toestandstabellen van de leidende voorbeelden.}
\tbllab{toestandstabel}
\end{table}
\paragraph{Toestandstabel}Een toestandsdiagram is een grafische voorstelling. Bij een groot aantal toestanden of bij bijvoorbeeld invoer van een toestandsdiagram in een computer, maken we meestal gebruik van een alternatieve voorstelling: de \termen{toestandstabel}. De rijen in de toestandstabel stellen de verschillende toestanden voor, de kolommen stellen de invoercombinatie voor. In een cel $i,j$ voor toestand $A_i$ en invoer $I_j$ plaatsen we de volgende toestand. In het geval van een Moore-machine voorzien we een extra kolom die per rij de uitgangscombinatie van deze toestand weergeeft. Bij een Mealy-machine plaatsen we in elke cel naast de volgende toestand de uitgangscombinatie voor toestand $A_i$ en invoer $I_j$. Ook hier plaatsen we een slash om de volgende toestand van de uitvoer te onderscheiden. De toestandstabellen van de leidende voorbeelden staan in tabel \ref{tbl:toestandstabel}.
\paragraph{Opstellen van een toestandsdiagram en -tabel}Zoals reeds gezegd is het opstellen van een toestandsdiagram of -tabel een kunst. Een algemene techniek die gehanteerd kan worden is het aantal klokflanken te bepalen dat we geheugen moeten voorzien.  In het geval van de voorbeelden is dit drie klokflanken voor de Moore-machine en twee klokflanken voor de Mealy-machine. Vervolgens kunnen we een toestandstabel of -diagram opstellen die de overgang naar verschillende toestanden van het geheugen voorstelt. Zo komt $A$ in beide voorbeelden overeen met de toestand waarin we starten, er is bijgevolg nog geen sprake over inhoud in het geheugen. $B$ en $C$ betekent dat er na de eerste klokflank respectievelijk een 0 en een 1 in het geheugen opgeslagen is. $D$, $E$, $F$ en $G$ spreken over de respectievelijke geheugentoestanden $00$, $01$, $10$ en $11$. Voor de Mealy-machine is dit reeds voldoende. Bij de Moore-machine introduceren we nog een extra niveau. De overgangen op dit laatste niveau houden in dat we de oudste bit vergeten en de nieuwe bit memoriseren. Hierdoor gaan alle overgangen van een toestand op het laatste niveau enkel naar toestanden op hetzelfde niveau. We kunnen machinaal eenvoudig een tabel opstellen voor een arbitraire diepte. Een groot nadeel is dat indien we $n$ klokflanken willen memoriseren, we $2^{n+1}-1$ toestanden bekomen. Na het opstellen van deze tabel dienen we enkel de specificaties nog te vertalen in de uitgangen van de toestanden in het geval van een Moore-machine, en de overgangen bij een Mealy-machine. Meestal kan men echter door logisch te redeneren voorkomen dat we dergelijke grote toestandstabellen moeten opstellen. In de volgende subsectie reduceren we het aantal toestanden tot de theoretische ondergrens. Of we de initi\"ele tabel opstellen met de hier beschreven methode of door logisch te redeneren verandert niets aan het resultaat van de volgende stap.
\subsection{Stap 2: Minimaliseren van de toestanden}
\label{ss:minimizeFSMSeq}
Nu we een toestandsdiagram opgesteld hebben kunnen we dit diagram implementeren met behulp van logica. Het loont echter meestal de moeite om eerst het toestandsdiagram te minimaliseren. Minder toestanden impliceren minder geheugencomponenten om de toestand bij te houden. Bovendien kunnen we meestal ook de achterliggende logica die de volgende toestand en de uitgangen berekent minimaliseren. Hiertoe geven we een methode die gegarandeerd het kleinste toestandsdiagram vindt die de specificaties kan implementeren. Met minimaal bedoelen we hier het aantal toestanden. Dit betekent dus niet noodzakelijk dat de schakeling die we hiermee verwezenlijken ook minimaal is.
\subsubsection{Het minimalisatiealoritme}
\paragraph{Het algemeen idee}De methode die we implementeren gaat uit van een positief idee: alle toestanden zijn vertegenwoordigers van dezelfe toestand. Uiteraard is dit niet altijd het geval. Er zijn twee omstandigheden waarin twee toestanden niet gelijk aan elkaar zijn:
\begin{enumerate}
 \item De uitgang van de toestanden of overgangen uit een toestand zijn verschillend. In dat geval kunnen we immers onmogelijk de uitgangsfunctie implementeren. Een uitgang kan immers niet tegelijk 0 of 1 zijn.
 \item De uitgang na een willekeurig aantal willekeurige invoer (configuraties bij een klokflank) is verschillend. In dat geval betekent dit dat we dus in de toekomst op het vorige probleem zullen botsen.
\end{enumerate}
\paragraph{De eerste voorwaarde}We kunnen deze twee condities eenvoudig in een algoritme implementeren. Het algoritme werkt op basis van een partitionering van de aanvankelijke toestandsruimte. Als twee toestanden tot dezelfde partitite behoren. betekent dit dat ze eigenlijk hetzelfde zijn. Zoals eerder vermeld begint de methode met een positieve ingesteldheid: alle toestanden zijn gelijk. De initi\"ele configuratie bevat dus \'e\'en partitie waar alle toestanden in zitten. Vervolgens kunnen de eerste voorwaarde toepassen. Toestanden die een verschillende uitgang opleveren kunnen onmogelijk hetzelfde zijn. Voor een Moore-machine betekent dit dus de uitgang van de toestand. In het leidend voorbeeld is de uitgang 0 of 1. We partitioneren de configuratie dus in een partitie die 0 als uitgang geeft en een partitie met 1 als uitgang. De configuratie is dan:
\begin{equation}
\begin{array}{lr}
\mbox{partitie}_{\mbox{\small{Moore,0}}}=\left\{\left\{A,B,C,D,E,F,G,H,I,K,L,M,N,O\right\},\left\{J\right\}\right\}&\mbox{(Leidend voorbeeld)}
\end{array}
\end{equation}
Merk echter op dat een schakeling ook meerdere lijnen als uitgang kan hebben. Indien de uitgang $n$ bits telt levert dit ons een configuratie op van hoogstens $2^n$ partities. In het geval een Mealy-machine leidt een toestand niet rechtstreeks tot een uitgang. Twee toestanden zijn dan gelijk indien voor elke invoer-configuratie bij deze toestand, we dezelfde uitvoerconfiguratie bekomen. In het leidend voorbeeld beschouwen we een 1-bit ingang en een 1-bit uitgang. Dit leidt dus tot hoogstens 4 partities. In ons geval zijn er maar 2 partities:
\begin{equation}
\begin{array}{lr}
\mbox{partitie}_{\mbox{\small{Mealy,0}}}=\left\{\left\{A,B,C,D,F,G\right\},\left\{E\right\}\right\}&\mbox{(Leidend voorbeeld)}
\end{array}
\end{equation}
In het algemene geval met een $m$-bit ingang en een $n$-bit uitgang bekomen we hoogstens $2^{n+m}$ partities.
\paragraph{De tweede voorwaarde}De tweede voorwaarde kunnen we ook afdwingen door middel van iteratie. Hierbij itereren we over de lengte van de invoer. Het specifieke geval is uiteraard het geval waarbij de uitvoer na \'e\'en klokcyclus reeds verschilt. Dit kunnen we herformuleren tot het volgende: Indien twee toestanden $x_0$ en $y_0$ onder een willekeurige invoer $i_0$ naar twee toestanden $x_1$ en $y_1$ gaan, en deze twee toestanden behoren niet tot dezelfde partities, dan behoren $x_0$ en $y_0$ ook niet tot dezelfde partitie. Deze uitspraak is logisch, stel dat $x_0$ en $y_0$ tot dezelfde partitie zouden behoren, dan kan het gebeuren dat we in de toestand geraken die door de partitie waar $x_0$ en $y_0$ toe behoort. Indien we daarna de invoer $i_0$ dienen te verwerken komen we in toestand die zowel $x_1$ en $x_2$ dient te vertegenwoordigen. Vermits $x_1$ en $y_1$ echter tot een andere partitie behoren is dit onmogelijk. We gebruiken deze regel om partities vervolgens verder op te delen. We gebruiken de eerste configuratie van de minimale Moore-machine als voorbeeld. Hierbij gaan van de eerste partitie de toestanden $A$, $B$, $C$, $D$, $F$, $G$, $H$, $K$, $L$, $N$ en $O$ onder invoer van $0$ en $1$ naar de eerste partitie. De toestanden $E$, $I$ en $M$ gaan onder invoer van $0$ naar de tweede partitie ($J$) en onder invoer van $1$ naar de eerste partitie. Bijgevolg splitsen we de eerste partitie op in $\left\{\left\{A,B,C,D,F,G,H,K,L,N,O\right\},\left\{E,I,M\right\}\right\}$. Vermits de tweede partitie reeds uit \'e\'en element bestaat, valt deze niet verder op te delen. De totale partitie na \'e\'en iteratie is dan gelijk aan:
\begin{equation}
\begin{array}{lr}
\mbox{partitie}_{\mbox{\small{Moore,1}}}=\left\{\left\{A,B,C,D,F,G,H,K,L,N,O\right\},\left\{E,I,M\right\},\left\{J\right\}\right\}&\mbox{(Leidend voorbeeld)}
\end{array}
\end{equation}
Het enige wat we nu moeten doen is deze stap herhalen op de nieuwe partitie $\mbox{partitie}_{\mbox{\small{Moore,1}}}$. Op die manier passen we de regel toe bij een invoerlengte van 2. De enige vraag die open blijft is wanneer we mogen stoppen. Deze vraag is eenvoudig te beantwoorden: zolang er partities bijkomen kan een volgende stap de partities verder verdelen. Indien de uitvoering van een stap geen wijzigingen aan de partities aanbrengt, zal de volgende stap dit uiteraard ook niet meer doen, en alle stappen hierna ook niet meer. In dat geval is het veilig om te stoppen. Bij de minimalisatie van de Moore-machine zullen we dan volgende partities bekomen:
\begin{equation}
\small{
\begin{array}{lr}
\left\{\begin{array}{l}
\mbox{partitie}_{\mbox{\small{Moore,0}}}=\left\{\left\{A,B,C,D,E,F,G,H,I,K,L,M,N,O\right\},\left\{J\right\}\right\}\\
\mbox{partitie}_{\mbox{\small{Moore,1}}}=\left\{\left\{A,B,C,D,F,G,H,K,L,N,O\right\},\left\{E,I,M\right\},\left\{J\right\}\right\}\\
\mbox{partitie}_{\mbox{\small{Moore,2}}}=\left\{\left\{A,C,G,K,O\right\},\left\{B,D,F,H,L,N\right\},\left\{E,I,M\right\},\left\{J\right\}\right\}\\
\mbox{partitie}_{\mbox{\small{Moore,3}}}=\left\{\left\{A,C,G,K,O\right\},\left\{B,D,F,H,L,N\right\},\left\{E,I,M\right\},\left\{J\right\}\right\}
\end{array}\right.&\mbox{(Leidend voorbeeld)}
\end{array}}
\end{equation}
Het geval van de Mealy-machine is volledig analoog: we bekomen dan:
\begin{equation}
\small{
\begin{array}{lr}
\left\{\begin{array}{l}
\mbox{partitie}_{\mbox{\small{Mealy,0}}}=\left\{\left\{A,B,C,D,F,G\right\},\left\{E\right\}\right\}\\
\mbox{partitie}_{\mbox{\small{Mealy,1}}}=\left\{\left\{A,C,G\right\},\left\{B,D,F\right\},\left\{E\right\}\right\}\\
\mbox{partitie}_{\mbox{\small{Mealy,2}}}=\left\{\left\{A,C,G\right\},\left\{B,D,F\right\},\left\{E\right\}\right\}
\end{array}\right.&\mbox{(Leidend voorbeeld)}
\end{array}}
\end{equation}
\subsubsection{Omzetting naar een toestandsdiagram en -tabel}
Nadat we het aantal toestanden geminimaliseerd hebben, dienen we alleen nog een nieuwe Moore- of Mealy-machine te bouwen op basis van de partitie van de toestanden van de originele machine. Zoals we al enkele keren vermeld hebben staat een partitie voor de toestand van de minimale machine. We dienen dus voor elke partitie een nieuwe toestand te voorzien. Dit doet men meestal door de nieuwe toestand dezelfde naam te geven als de toestand in de partitie die alfabetisch het eerst voorkomt. Maar we kunnen uiteraard ook een arbitraire naam kiezen, of een naam die de letters van alle inwendige toestanden omvat. Verder dienen we in het geval van een Moore machine aan elke toestand een uitvoerconfiguratie toe te kennen. Omdat we de eerste voorwaarde hebben afgedwongen hebben alle toestanden in een partitie dezelfde uitvoerconfiguratie, bijgevolg nemen we de configuratie van \'e\'en van de toestanden in de partitie over. Hetzelfde geldt voor de Mealy-machine. De uitgang verbonden aan de transitie uit een bepaalde nieuwe toestand is dezelfde als de uitgang van dezelfde transitie uit \'e\'en van de toestanden in de partitie. De transitiefunctie kunnen we opstellen op basis van de tweede voorwaarde: alle toestanden in partitie zullen voor eenzelfde ingangscombinatie naar eenzelfde partitie lopen. Op die manier kunnen we dus een transitie-functie opstellen over de nieuwe toestanden. De initi\"ele toestand ten slotte is de toestand verbonden aan de partitie die de oorspronkelijke initi\"ele toestand bevat. Op figuur \ref{fig:minimaalToestandsdiagram} en tabel \ref{tbl:minimaalToestandstabel} staan de nieuwe machines na minimalisatie van de leidende voorbeelden. We kunnen eenvoudig vaststellen dat minimalisatie soms tot spectaculaire reducties van het aantal toestanden kan leiden.
\begin{figure}[hbt]
\centering
\subfigure[Moore-machine]{\begin{tikzpicture}[->,shorten >=1pt,auto,node distance=2cm,on grid,semithick,state/.style=state with output,every state/.style={draw=black!50,very thick,fill=black!20,scale=0.75}]
\node[state,initial,initial text=RST, initial where=left] (A) {$A$\nodepart{lower} $0$};
\node[state] (B) [right=of A] {$B$\nodepart{lower} $0$};
\node[state] (E) [below=of B] {$E$\nodepart{lower} $0$};
\node[state] (J) [left=of E] {$J$\nodepart{lower} $1$};
\path (A) edge node {0} (B)
          edge[loop above] node {1} (A)
      (B) edge[loop above] node {0} (B)
          edge node {1} (E)
      (E) edge[swap] node {0} (J)
          edge[bend right] node {1} (A)
      (J) edge[bend left,swap] node {0} (B)
          edge[bend right] node {1} (E);
\end{tikzpicture}
\figlab{minimaalToestandsdiagramMoore}}
\subfigure[Mealy-machine]{\begin{tikzpicture}[->,shorten >=1pt,auto,node distance=2cm,on grid,semithick,every state/.style={draw=black!50,very thick,fill=black!20,scale=0.75}]
\node[state,initial,initial text=RST, initial where=left] (A) {$A$};
\node[state] (B) [right=of A] {$B$};
\node[state] (E) [below=of B] {$E$};
\path (A) edge node {0/0} (B)
          edge[loop above] node {1/0} (A)
      (B) edge[loop above] node {0/0} (B)
          edge node[swap] {1/0} (E)
      (E) edge[bend right,swap] node {0/1} (B)
          edge node {1/0} (A);
\end{tikzpicture}
\figlab{minimaalToestandsdiagramMealy}}
\caption{Geminimaliseerde toestandsdiagrammen van de leidende voorbeelden.}
\figlab{minimaalToestandsdiagram}
\end{figure}
\begin{table}[hbt]
\centering
\subtable[Moore-machine]{\small{\begin{tabular}{c|cc|c}
Toestand&0&1&Uitgang\\\hline
$A$&$B$&$A$&0\\
$B$&$B$&$E$&0\\
$E$&$J$&$A$&0\\
$J$&$B$&$E$&1
\end{tabular}}}
\subtable[Mealy-machine]{\small{\begin{tabular}{c|cc}
Toestand&0&1\\\hline
$A$&$B/0$&$A/0$\\
$B$&$B/0$&$E/0$\\
$E$&$B/1$&$A/0$
\end{tabular}}}
\caption{Geminimaliseerde toestandstabellen van de leidende voorbeelden.}
\tbllab{minimaalToestandstabel}
\end{table}
\paragraph{Een formeel algoritme}
De cursus ``Automaten en Berekenbaarheid'' van prof. Demoen\cite{aenb10} bevat een formeel algoritme voor de minimalisatie van een deterministische eindige toestandsautomaat (DFA). Met minimale veranderingen kan dit algoritme omgevormd worden tot een algoritme die Moore- en Mealy-machines minimaliseert. We beschouwen dit in \algoref{alg:minimizeFSM}. ??%TODO: complete algorithm
\begin{algorithm}[hbt]
\caption{Minimaliseren van een toestandsdiagram.}\label{alg:minimizeFSM}
\begin{algorithmic}[1]
\Procedure{Minimize}{$S,I,O,\delta,f$}\Comment{Minimaliseer de machine}
\State $\mathcal{Q}\gets\Call{Minimize1}{S,I,O,f}$\Comment{Initi\"ele partitionering gebaseerd op uitvoer.}
\Repeat
\State $\mathcal{P}\gets\mathcal{Q}$
\State $\mathcal{Q}\gets\Call{Minimize2}{S,I,\delta,\mathcal{P}}$\Comment{Bereken de nieuwe partitionering op basis van de oude.}
\Until{$\mathcal{P}=\mathcal{Q}$}\Comment{Indien geen verandering is het algoritme ten einde.}
\State \textbf{return} $\mathcal{Q}$\Comment{De uiteindelijke partitie.}
\EndProcedure
\Function{Minimize1Moore}{$S,I,O,f$}
\State \textbf{return} $\left\{\left\{s|s\in S:f\left(s\right)=o\right\}|o\in O\right\}$\Comment{$I$ wordt genegeerd.}
\EndFunction
\Function{Minimize1Mealy}{$S,I,O,f$}
\State \textbf{return} $\left\{\left\{s|s\in S,\forall i\in I:f\left(s,i\right)=o\right\}|o\in O\right\}$
\EndFunction
\Function{Minimize2}{$S,I,\delta,\mathcal{P}$}
\State \textbf{return} $\left\{\left\{s|s\in P:f\left(s\right)=o\right\}|P\in\mathcal{P},\right\}$
\EndFunction
\end{algorithmic}
\end{algorithm}
Het algoritme bevat de algemene \procedureref{Minimize}-procedure. Deze procedure maakt gebruik van twee functies: \procedureref{Minimize1} en \procedureref{Minimize2}. De eerste functie wijkt af tussen een Moore- en Mealy-machine. Daarom wordt deze functie opgesplitst: voor de Moore-machine gebruiken we dus \procedureref{Minimize1Moore}, de Mealy-machine maakt gebruik van \procedureref{Minimize1Mealy}. We dienen ook een formele beschrijving te geven van de variabelen in het algoritme. Het algoritme heeft als invoer 3 verzamelingen:
\begin{itemize}
 \item $S$ is de verzameling van alle toestanden.
 \item $I$ is de verzameling van alle invoerconfiguraties.
 \item $O$ is de verzameling van alle uitvoerconfiguraties.
\end{itemize}
Daarnaast heeft het algoritme ook nood aan twee functies:
\begin{itemize}
 \item Een transitiefunctie $\delta:S\times I\rightarrow S$ die de volgende toestand beschouwd na een bepaalde invoer op de ingangen te hebben waargenomen bij de klokflank.
 \item Een uitvoerfunctie $f$. Bij de Moore-machine is deze functie van de signatuur $f:S\rightarrow O$, bij een Mealy-machine is dit $f:S\times I\rightarrow O$.
\end{itemize}

\subsection{Stap 3: Implementeren van de toestanden in het geheugen}
\subsubsection{Stap 3A: Coderen van de toestanden}
\label{term:minimalBitChange}
\label{term:grayCodeCounter}
Tot nu toe hebben we toestanden altijd voorgesteld met letters. Uiteraard dienen we deze toestanden op de een of andere manier voor te stellen in de geheugencomponenten in onze schakeling. Vermits letters niet opgeslagen kunnen worden in een flipflop of een register\footnote{Uiteraard kunnen we bijvoorbeeld wel het ASCII equivalent opslaan.} zullen we de toestanden moeten omzetten naar een binaire voorstelling. Dit wordt het \termen{coderen} van de toestanden genoemd. Doorgaans lijkt dit geen ingewikkeld probleem, we kunnen eenvoudigweg elke toestand een opeenvolgend nummer geven en deze toestanden dan binair coderen. Merk echter op dat de toestand geen impliciete ordening hebben, alleen al door opeenvolgende nummers te gebruiken zijn er $n!$ mogelijke coderingen mogelijk. Er bestaan echter technieken die ervoor zorgen dat we door een intelligente codering minder hardware zullen gebruiken. Dit leidt meestal tot goedkopere en snellere schakelingen. Algemeen zijn er drie technieken die we kunnen gebruiken:
\begin{itemize}
 \item ``\termen{Straightforward codering}'': soms is de codering triviaal.
 \item De ``\termen{one-hot codering}'': hierbij stellen we $n$ toestanden voor door $n$ bits, elke toestand krijgt z'n eigen bit. Indien de toestand actief is, is deze bit hoog, de andere bits zijn dan laag.
 \item De ``\termen{minimal-bit-change}'': we zoeken een codering die telkens een minimum aan bits verandert. Hierbij gebruiken we $\left\lceil\log_2 n\right\rceil$ bits om de $n$ toestanden voor te stellen.
\end{itemize}
Uiteraard dienen we ons niet te beperken tot \'e\'en van deze implementaties. Indien het aantal uitgangen bijvoorbeeld niet voldoet om een straightforward codering toe te passen kunnen we dit bijvoorbeeld aanvullen met extra bits die een minimal-bit change gebruiken. We bespreken nu de verschillende coderingstechnieken in detail.
\paragraph{Straightforward codering}Een codering kan triviaal zijn als de toestand een duidelijke betekenis heeft. Dit is doorgaans het geval bij bijvoorbeeld tellers. Ook indien we over een Moore-machine beschikken waarbij het elke toestand een andere uitgang heeft en de uitgang bevat minstens $\left\lceil\log_2 n\right\rceil$ bits kunnen we een straightforward codering toepassen. In dat geval gebruiken we de uitgang van een toestand ook als zijn codering. Merk op dat in dat geval we geen logica nodig hebben die de toestand naar de uitgang codeert. Deze codering is echter niet altijd ideaal. We reduceren dan de logica aan de uitgang, maar meestal resulteert dit in complexere logica om de transities te implementeren. Verder veranderen er meestal heel wat bits per overgang. We herinneren ons dat een CMOS-implementatie enkel energie verbruikt wanneer deze omschakelt. Indien we dus veel flipflops van waarde moeten laten veranderen betekent dit dus een groter vermogenverbruik. Een tweede aspect is dat niet elke bit tegelijk verandert. Vermits er meerdere bits zijn kan dit problemen opleveren: er bestaat een gevaar voor \termen{glitches}, een tijdelijk foute waarde op een lijn ten gevolge van asynchroon gedrag aan de ingang. Indien het resultaat van een teller dus nog in een andere schakeling gebruikt wordt leidt dit mogelijk tot problemen.??%TODO:extend
\paragraph{One-hotcodering}We voorzien een flipflop voor elke toestand, hierdoor is het aantal flipflops \bigoh{n} in tegenstelling tot de \bigoh{\log_2n}. Bij elke toestand is er \'e\'en flipflop hoog, alle andere flipflops zijn laag. Het spreekt dus voor zich dat we deze configuratie enkel kunnen gebruiken bij een klein aantal toestanden. One-hotcoderingen hebben enkele voordelen: het ontwerp van een dergelijke schakeling is vrij eenvoudig en is dus snel te realiseren. Verder is het ideaal voor een implementatie bij een FPGA. Een one-hotcodering verbruikt doorgaans weinig vermogen aangezien bij elke overgang tussen twee verschillende toestanden er twee flipflops omschakelen. Tot slot zijn ook de combinatorische schakelingen voor de uitgangen en de transities doorgaans vrij goedkoop. Het grootste nadeel van een one-hot codering is dan ook de kostprijs voor de flipflops.
\paragraph{Minimal-bit-change}Indien de kostprijs en vermogenverbruik een belangrijke factor zijn maken we meestal gebruik van de minimal-bit-change. Hierbij proberen we ervoor te zorgen dat de som van het aantal bits die veranderen van alle overgangen minimaal is. Dit probleem is echter niet triviaal en is NP-compleet. Bij een klein aantal toestanden kunnen we alle mogelijkheden uitproberen, bij een groter aantal maakt men gebruik van programma's zoals bijvoorbeeld MUSE, JEDI, MUSTANG,... Deze programma's werken op basis van heuristieken en geven meestal een benaderende oplossing. Soms worden er ook kansen bij de overgangen betrokken om het de kosten en het vermogenverbruik verder te minimaliseren. Een typisch voorbeeld is de \termen{Gray-code teller}\footnote{Vernoemd naar Frank Gray (1887-1969), fysicus bij Bell Labs.}. Gray ontwikkelde een teller die bij elke overgang slechts \'e\'en bit veranderde. \figref{grayCodeCounter} toont het toestandsdiagram van een 2-bit en 3-bit teller. Op de bogen plaatsen we hier het aantal bits die veranderen. Het feit dat we niet voor een straightforward implementatie voor een teller kiezen kan vreemd lijken. Deze teller dient dan ook meestal niet om de binaire waarde onmiddellijk uit te lezen, maar kan bijvoorbeeld gebruikt worden als een controller die een lus vormt over een vast aantal toestanden. Over het concreet oplossen van dit probleem gaan we niet verder in.
\begin{figure}[hbt]
\centering
\subfigure[Gray-code (minimal-bit-change)]{\begin{tikzpicture}[->,shorten >=1pt,auto,node distance=2cm,on grid,semithick,every state/.style={draw=black!50,very thick,fill=black!20,scale=0.75}]
\node[state,initial,initial text=RST, initial where=left] (A) {$00$};
\node[state] (B) [right=of A] {$01$};
\node[state] (C) [below=of B] {$11$};
\node[state] (D) [left=of C] {$10$};
\begin{scope}[node distance=1.414cm]
  \node[state,initial,initial text=RST, initial where=left,above left=of A] (AA) {$000$};
  \node[state] (CC) [above right=of B] {$011$};
  \node[state] (EE) [below right=of C] {$110$};
  \node[state] (GG) [below left=of D] {$101$};
\end{scope}
\node[state] (BB) [right=of AA] {$001$};
\node[state] (DD) [below=of CC] {$010$};
\node[state] (FF) [left=of EE] {$111$};
\node[state] (HH) [above=of GG] {$100$};
\path (A) edge node{1} (B)
      (B) edge node{1} (C)
      (C) edge node{1} (D)
      (D) edge node{1} (A);
\path (AA) edge node{1} (BB)
      (BB) edge node{1} (CC)
      (CC) edge node{1} (DD)
      (DD) edge node{1} (EE)
      (EE) edge node{1} (FF)
      (FF) edge node{1} (GG)
      (GG) edge node{1} (HH)
      (HH) edge node{1} (AA);
\end{tikzpicture}}
\subfigure[Straightforward]{\begin{tikzpicture}[->,shorten >=1pt,auto,node distance=2cm,on grid,semithick,every state/.style={draw=black!50,very thick,fill=black!20,scale=0.75}]
\node[state,initial,initial text=RST, initial where=left] (A) {$00$};
\node[state] (B) [right=of A] {$01$};
\node[state] (C) [below=of B] {$10$};
\node[state] (D) [left=of C] {$11$};
\begin{scope}[node distance=1.414cm]
  \node[state,initial,initial text=RST, initial where=left,above left=of A] (AA) {$000$};
  \node[state] (CC) [above right=of B] {$010$};
  \node[state] (EE) [below right=of C] {$100$};
  \node[state] (GG) [below left=of D] {$110$};
\end{scope}
\node[state] (BB) [right=of AA] {$001$};
\node[state] (DD) [below=of CC] {$011$};
\node[state] (FF) [left=of EE] {$101$};
\node[state] (HH) [above=of GG] {$111$};
\path (A) edge node{1} (B)
      (B) edge node{2} (C)
      (C) edge node{1} (D)
      (D) edge node{2} (A);
\path (AA) edge node{1} (BB)
      (BB) edge node{2} (CC)
      (CC) edge node{1} (DD)
      (DD) edge node{3} (EE)
      (EE) edge node{1} (FF)
      (FF) edge node{2} (GG)
      (GG) edge node{1} (HH)
      (HH) edge node{3} (AA);
\end{tikzpicture}}
\caption{Een 2-bit en 3-bit Gray-code teller en zijn straightforward equivalent.}
\figlab{grayCodeCounter}
\end{figure}
\paragraph{Leidende voorbeelden}Voor de Moore-machine kiezen we voor een minimal-bit-change benadering en voor de Mealy-machine een one-hot codering. Deze keuzes zijn louter didactief. De toegewezen bitvoorstellingen worden in een toestandstabel geschreven. Voor een one-hot codering is dit vrij triviaal. We wijzen elke toestand \'e\'en bit toe. Welke bit heeft geen invloed op de schakeling. Tabel \ref{tbl:mealyBitRepresentation} stelt de Mealy-machine uit de leidende voorbeelden voor met de one-hot codering.
\begin{table}[hbt]
\centering
\begin{tabular}{c|cc}
Toestand&0&1\\\hline
\texttt{001}&\texttt{010/0}&\texttt{001/0}\\
\texttt{010}&\texttt{010/0}&\texttt{100/0}\\
\texttt{100}&\texttt{010/1}&\texttt{001/0}\\
\end{tabular}
\caption{Codering van de Mealy-machine van het leidend voorbeeld.}
\tbllab{mealyBitRepresentation}
\end{table}
Bij het zoeken naar een minimal-bit-change voor de Moore-machine maken we gebruik van een greedy algoritme. We stellen eerst een tabel op die het aantal bindingen tussen twee toestanden bevat. Vervolgens wijzen we coderingen aan toestanden toe volgens het aantal bindingen tussen de toestanden. De bindingstabel staat beschreven in tabel \ref{tbl:mooreBitRepresentationBinding}. We zien dat $E$ en $J$ een dubbele binding bevat. We kennen hier de waarde $01$ toe aan $E$ en $11$ bij $J$. Verder is $B$ gelinkt aan $E$ en daarom geven we dit de waarde $00$. $A$ krijgt ten slotte de waarde $10$.
\begin{table}[hbt]
\centering
\subtable[Bindingstabel]{
\begin{tabular}{c|cccc}
&$A$&$B$&$E$&$J$\\\hline
$A$&1&1&1&0\\
$B$&-&1&1&1\\
$E$&-&-&0&2\\
$J$&-&-&-&0\\
\end{tabular}
\tbllab{mooreBitRepresentationBinding}
}
\subtable[Coderingstabel]{\begin{tabular}{c|cc|c}
Toestand&0&1&Uitgang\\\hline
\texttt{10}&\texttt{00}&\texttt{10}&\texttt{0}\\
\texttt{00}&\texttt{00}&\texttt{01}&\texttt{0}\\
\texttt{01}&\texttt{11}&\texttt{10}&\texttt{0}\\
\texttt{11}&\texttt{00}&\texttt{01}&\texttt{1}\\
\end{tabular}}
\caption{Codering van de Moore-machine van het leidend voorbeeld.}
\tbllab{mooreBitRepresentation}
\end{table}
\subsubsection{Stap 3B: De keuze van het type flipflop}
Nadat we elke toestand kunnen voorstellen met een sequentie aan bits, dienen we flipflops te voorzien om deze toestand bij te houden. Hierbij hebben we de keuze tussen de verschillende flipflops die we in \ref{sss:typesFlipflops} besproken hebben. Elk van deze types zal een specifieke combinatorische logica vereisen met specifieke kosten en vertraging. Om de meest optimale schakeling te realiseren zullen we altijd alle types moeten uitproberen. Verder kunnen we ook voor verschillende bits een verschillend type flipflop voorzien. Hierop gaan we echter niet in. De ervaring leert ons wel dat voor verschillende toepassingen verschillende soorten flipflops een zekere voorkeur genieten. De typische toepassingen zeg maar. Een cruciale factor is ook het aantal don't cares. In het algemeen is het zo dat hoe meer don't cares de combinatorische schakeling bevat, hoe eenvoudiger te implementeren. Sommige flipflops introduceren makkelijker don't cares dan andere flipflops.
\paragraph{De kosten van een flipflop}Als we de kostprijs van de verschillende types flipflops met elkaar vergelijken merken we dat de JK-flipflop opmerkelijk duurder\footnote{Ter illustratie een JK-flipflop ge\"integreerd circuit kost \$0.116, een gelijkaardige D-flipflop kost ongeveer \$0.049.} is. Dit betekent daarom niet dat de totale kostprijs van de schakeling groter is. Immers dienen we ook een combinatorisch gedeelte, de optelling van beide leidt tot de totale kostprijs.
\paragraph{Don't cares}Een JK-flipflop is dan duurder, maar door de typische realisatie leidt dit tot veel don't cares. Dit leidt tot een eenvoudige schakeling. Het is logisch dat een JK-flipflop nogal wat don't cares impliceert: er zijn immers enkele manieren om dezelfde waarde in een JK-flipflop te klokken. Ook een SR-flipflop laat ruimte voor don't cares in het combinatorische gedeelte. Een D-flipflop laat doorgaans weinig plaats voor don't cares. Dit komt omdat in elke klokflank een nieuwe waarde uit de ingang wordt gelezen. We dienen er dus telkens voor te zorgen dat op dit moment de juiste waarde aan de ingang staat. Een T-flipflop ten slotte heeft net hetzelfde probleem, meestal is het effect hier zelfs nog erger.
\paragraph{Eenvoud van het ontwerp}Indien we de schakeling zelf moeten realiseren moeten we ook rekening houden met het tijdsaspect en dus de eenvoud van het ontwerp. Doorgaans leidt een D-flipflop tot de eenvoudigste implementatie. Dit komt omdat we meestal geneigd zijn absoluut te denken. Verder heeft een D-flipflop ook slechts \'e\'en ingang, en is de component makkelijk te specifi\"eren: "wat we aan de ingang aanleggen staat de volgende klokflank in de flipflop". Het opstellen van een \termen{excitatietabel} is dan ook eenvoudig. Een T-flipflop is ook conceptueel eenvoudig: "indien de waarde moet omslaan, leg dan 1 aan op de ingang". Moeilijker zijn de SR- en JK-flipflop. Dit komt in de eerste plaats omdat beide componenten twee ingangen hebben. Merk dus op dat we een excitatietabel met twee ingangen moeten opstellen.
\paragraph{Toepassingen}Door jaren ervaring heeft men kennis opgebouwd welke toepassing welke flipflop vereist. In subsectie \ref{ss:counters} hebben we reeds teller gebruikt en hebben we vaak gebruik gemaakt van T- en D-flipflops. Tellers en frequentiedelers\footnote{Een frequentiedeler is een moduloteller die de frequentie waarmee er geteld wordt deelt door het modulo-getal. Enkel wanneer zich een overflow voordoet geven we het signaal door. Hierdoor kunnen we delen van de schakeling aan een lagere klokfrequentie laten werken.} worden typisch ge\"implementeerd met T-flipflops. Meestal zal dit tot ijle excitatietabellen leiden. Een D-flipflop wordt typisch gebruikt om een waarde zeer tijdelijk te onthouden, meestal slechts enkele klokflanken. Voor complexe toepassingen waarbij de waarde van de flipflop frequent een 0 of 1 wordt zijn SR- en JK-flipflops het meest geschikt.
\paragraph{Welke flipflop kiezen?}De vorige paragrafen proberen hints te geven welke flipflops het meest geschikt zijn, deze hints zijn echter niet absoluut. Om tot de goedkoopste schakeling te komen, moeten we alle mogelijkheden uitproberen. Meestal hebben we de tijd niet alle configuraties te proberen. In dat geval zijn D-flipflops meestal de beste keuze. Ook een FPGA volgt deze redenering en bevat enkel D-flipflops.
\paragraph{Samevatting}
We vatten de vorige paragrafen samen in tabel \ref{tbl:whichFlipflop}.
\begin{table}[hbt]
\centering
\small{\begin{tabular}{|*{5}{M}}
\hline
Flipflop&JK&SR&D&T\\\hline\hline
Kostprijs&\verb/--/&\verb/++/&\verb/++/&\verb/++/\\\hline
Don't cares&\verb/++/&\verb/+/&\verb/-/&\verb/--/\\\hline
Ontwerp&\verb/--/&\verb/-/&\verb/++/&\verb/+/\\\hline
Toepassingen&\multicolumn{2}{c|}{veel veranderingen}&tijdelijke geheugens&tellers, frequentiedelers\\\hline
\end{tabular}}
\caption{Keuze van het type flipflop.}
\tbllab{whichFlipflop}
\end{table}
\subsection{Stap 4: Implementeren van de combinatorische logica}
Met alle vorige stappen hebben we beslist hoe we de specificaties van de sequenti\"ele schakeling zullen implementeren. Het enige wat we nu nog moeten doen is de schakeling zelf implementeren. Deze implementatie is op te delen in het implementeren van de logica die de volgende toestand berekent, en de logica die de uitgang bepaalt. We bespreken elk van deze onderdelen apart.
\subsubsection{Stap 4A: Logica die de volgende toestand berekent}
\paragraph{Excitatietabellen van flipflops}Het deel van de logica die de volgende toestand van het circuit berekent is bij zowel een Moore- als een Mealy-machine identiek. Het is een combinatorische schakeling die als ingangen de invoer van het component en de toestand heeft, als uitvoer heeft het de ingangen van de flipflops die de toestand bijhouden. Het aantal uitgangen en de uitgangen zelf hangen dus af van het type flipflop die we gekozen hebben. We dienen dus een tabel op te stellen die op basis van de invoer van de schakeling en de toestand aangeeft welke ingangen van welke flipflops hoog of laag moeten zijn. Deze functies noemen we de \termen{excitatiefuncties} ze komen voort uit de \termen{excitatietabellen} van de verschillende flipflops. Deze tabellen werden reeds ge\"introduceert in \ref{sss:typesFlipflops}. In tabel \ref{tbl:excitationTablesFlipflops} geven we eerst opnieuw een kort overzicht van de excitatietabellen van de verschillende flipflops.
\begin{table}[hbt]
\centering
\subtable[JK]{
\begin{tabular}{c|c|cc}
$Q$&$Q_{\mbox{\small{next.}}}$&$J$&$K$\\\hline
0&0&0&-\\
0&1&1&-\\
1&0&-&1\\
1&1&-&0\\
\end{tabular}
\tbllab{excitationTablesFlipflopsJK}}
\subtable[SR]{
\begin{tabular}{c|c|cc}
$Q$&$Q_{\mbox{\small{next.}}}$&$S$&$R$\\\hline
0&0&0&-\\
0&1&1&0\\
1&0&0&1\\
1&1&-&0\\
\end{tabular}
\tbllab{excitationTablesFlipflopsSR}}
\subtable[D]{
\begin{tabular}{c|c|c}
$Q$&$Q_{\mbox{\small{next.}}}$&$D$\\\hline
0&0&0\\
0&1&1\\
1&0&0\\
1&1&1\\
\end{tabular}
\tbllab{excitationTablesFlipflopsD}}
\subtable[T]{
\begin{tabular}{c|c|c}
$Q$&$Q_{\mbox{\small{next.}}}$&$T$\\\hline
0&0&0\\
0&1&1\\
1&0&1\\
1&1&0\\
\end{tabular}
\tbllab{excitationTablesFlipflopsT}}
\caption{Excitatietabellen van de verschillende flipflops}
\tbllab{excitationTablesFlipflops}
\end{table}
Op basis van deze tabellen kunnen we nu de excitatiefuncties berekenen. Deze functies bevatten traditioneel veel don't cares. Deze don't cares hebben drie oorzaken:
\begin{itemize}
 \item De don't cares die we terug vinden bij de excitatietabellen. Dit zijn don't cares in de uitgang van de functie en komen enkel voor bij implementaties met JK- en SR-flipflops.
 \item Binaire coderingen van toestanden die niet bestaan. Een concreet geval is bijvoorbeeld $011$ bij een one-hot codering. Dergelijke invoer codeert nooit naar een uitvoer.
 \item Binaire coderingen van ingangscombinaties die niet mogelijk zijn. Stel bijvoorbeeld dat we een $2$-bit ingang beschouwen en enkel configuraties $00$, $01$ en $11$ kunnen voorkomen.
\end{itemize}
Deze laatste twee zijn een vorm van invoer die niet kan voorkomen. In een tabel kunnen we de rijen weglaten, of we kunnen de rij opvullen met don´t cares aan het uitganggedeelte.
\paragraph{De transitiefunctie}We stellen de transitiefunctie op door voor elke bit die de toestand voorstelt de ingangen van deze flipflop als uitgangen van onze schakeling te zien. Hiervoor kunnen we eerst het coderingstabel wijzigen. We lineariseren het diagram en laten de uitvoer voorlopig vallen. We illustreren dit concept met de Moore-machine uit het leidende voorbeeld. In tabel \ref{tbl:stateTableMooreComb} staat de toestandstabel van deze Moore-machine beschreven. Elke bit van de toestand $F_i$ is een ingang samen met de in dit geval 1-bit ingang $I_i$ van de schakeling. Als uitgang nemen we voorlopig de bits $D_i$ die de volgende toestand voorstellen. Deze linearisatie staat in tabel \ref{tbl:stateTableMooreCombLinD}.
\begin{table}[hbt]
\centering
\subtable[Coderingstabel]{\begin{tabular}{c|cc|c}
Toestand&0&1&Uitgang\\\hline
\texttt{10}&\texttt{00}&\texttt{10}&\texttt{0}\\
\texttt{00}&\texttt{00}&\texttt{01}&\texttt{0}\\
\texttt{01}&\texttt{11}&\texttt{10}&\texttt{0}\\
\texttt{11}&\texttt{00}&\texttt{01}&\texttt{1}\\
\end{tabular}
\tbllab{stateTableMooreComb}}
\subtable[D-flipflop]{\begin{tabular}{ccc|cc}
$F_0$&$F_1$&$I_0$&$D_0$&$D_1$\\\hline
0&0&0&0&0\\
0&0&1&0&1\\
0&1&0&1&1\\
0&1&1&1&0\\
1&0&0&0&0\\
1&0&1&1&0\\
1&1&0&0&0\\
1&1&1&0&1
\end{tabular}
\tbllab{stateTableMooreCombLinD}}
\subtable[T-flipflop]{\begin{tabular}{ccc|cc}
$F_0$&$F_1$&$I_0$&$T_0$&$T_1$\\\hline
0&0&0&0&0\\
0&0&1&0&1\\
0&1&0&1&0\\
0&1&1&1&1\\
1&0&0&1&0\\
1&0&1&0&0\\
1&1&0&1&1\\
1&1&1&1&0
\end{tabular}
\tbllab{stateTableMooreCombLinT}}
\subtable[JK-flipflop]{\begin{tabular}{ccc|cccc}
$F_0$&$F_1$&$I_0$&$J_0$&$K_0$&$J_1$&$K_1$\\\hline
0&0&0&0&-&0&-\\
0&0&1&0&-&1&-\\
0&1&0&1&-&-&1\\
0&1&1&1&-&-&0\\
1&0&0&-&1&0&-\\
1&0&1&-&0&0&-\\
1&1&0&-&1&-&0\\
1&1&1&-&1&-&1
\end{tabular}
\tbllab{stateTableMooreCombLinJK}}
\subtable[SR-flipflop]{\begin{tabular}{ccc|cccc}
$F_0$&$F_1$&$I_0$&$S_0$&$R_0$&$S_1$&$R_1$\\\hline
0&0&0&0&-&0&-\\
0&0&1&0&-&1&0\\
0&1&0&1&0&0&1\\
0&1&1&1&0&-&0\\
1&0&0&0&1&0&-\\
1&0&1&-&0&0&-\\
1&1&0&0&1&-&0\\
1&1&1&0&1&0&1
\end{tabular}
\tbllab{stateTableMooreCombLinSR}}
\caption{Voorstelling van de transitiefunctie van de Moore-machine.}
\tbllab{stateTableMooreCombLin}
\end{table}
Merk op dat deze reeds de implementatie zijn als we werken met D-flipflops. In het geval we niet met D-flipflops werken dienen we nog een extra stap te beschouwen: in dat geval beschouwen we voor elke flipflop het tupple $\left(F_i,D_i\right)$. Hierbij geldt $F_i=Q$ en $D_i=Q_{\mbox{\small{next.}}}$. We dienen dan nog enkel op te zoeken welke invoer het specifieke type flipflop nodig heeft in de excitatietabellen (zie tabel \ref{tbl:excitationTablesFlipflops}). In het geval van een SR- en JK-flipflop leidt dit dus tot een verdubbeling van het aantal uitgangen. Voorbeelden hiervan voor respectievelijk de T-, JK- en SR-flipflop staan in tabellen \ref{tbl:stateTableMooreCombLinT}, \ref{tbl:stateTableMooreCombLinJK} en \ref{tbl:stateTableMooreCombLinSR}. In principe hebben we nu alle elementen om de combinatorische schakeling te realiseren. We minimaliseren eerste de functies met behulp van Karnaugh-kaarten en implementeren dan vervolgens de logica. Merk op dat deze schakelingen meervoudige uitgangen hebben (1 per D- en T-flipflop en 2 per JK- en SR-flipflop). We kunnen dus ook de logica verder optimaliseren door implicanten over de verschillende Karnaugh-kaarten samen te nemen. Op figuur \ref{fig:mooreCombImpl} geven we voor elk type flipflop de Karnaugh-kaarten en een implementatie.
\begin{figure}[hbt]
\centering
\subfigure[D-flipflops]{
\begin{tikzpicture}[circuit logic US]
\def\ta{0.09};
\def\tb{0.03};
\kkaartcmarks[1]{0}{3}{/1/0/1/1/\ta,/3/1/3/1/\ta}{};
\kkaartc{0}{3}{$D_0$}{$F_0$/$F_1$/$I_0$}{0/0/1/1/0/1/0/0};
\kkaartcmarks[1]{0}{0}{/0/1/0/1/\ta,/1/0/1/0/\ta,/2/1/2/1/\ta}{};
\kkaartc{0}{0}{$D_1$}{$F_0$/$F_1$/$I_0$}{0/1/1/0/0/0/0/1};
\def\sc{0.75};
\def\sca{0.8};
\begin{scope}[xshift=6 cm,yshift=2.5 cm,scale=\sc]
\node[dffxn,scale=\sca] (FF0) at (0,2) {$F_0$};
\node[dffxn,scale=\sca] (FF1) at (0,-2) {$F_1$};
\coordinate (F0I) at (-3.5,3.8);
\coordinate (F1I) at (-3.7,4.0);
\coordinate (F2I) at (-3.9,4.0);
\coordinate (F3I) at (-4.1,4.0);
\coordinate (F4I) at (-4.3,4.0);
\coordinate (F0O) at (1,3.8);
\coordinate (F1O) at (1.2,4.0);

\coordinate (Clk0) at (F3I |- 0,1);
\coordinate (Clk1) at (F3I |- 0,-3);
\coordinate (CLR0) at (F4I |- 0,0.5);
\coordinate (CLR1) at (F4I |- 0,-3.5);
\coordinate (PR0) at (F4I |- 0,3.5);
\coordinate (PR1) at (F4I |- 0,-0.5);
\coordinate (MID) at (FF0.CLR |- 0,0);

\draw (FF0.Q) -| (F0O) -- (F0I);
\draw (FF1.Q) -| (F1O) -- (F1I);
\draw (F0I) -- (F0I |- 0,-4);
\draw (F1I) -- (F1I |- 0,-4);
\draw (F2I) -- (F2I |- 0,-4);
\draw (F3I) -- (F3I |- 0,-4);
\draw (F4I) -- (F4I |- 0,-4);
\coordinate (I0) at (-5,0);
\coordinate (I1) at (-5,-0.5);
\coordinate (I2) at (-5,-1);
\draw (I0) node[anchor=east]{$I_0$} -- (I0 -| F2I);
\draw (I1) node[anchor=east]{Clk} -- (I1 -| F3I);
\draw (I2) node[anchor=east]{Clr$^*$} -- (I2 -| F4I);
\coordinate (ORr) at (-1.5,0);
\coordinate (ANDr) at (-2.65,0);

\node[or gate] (O0) at (FF0.D -| ORr) {};
\node[or gate,inputs={normal,normal,normal}] (O1) at (FF1.D -| ORr) {};

\node[and gate,inputs={inverted,normal}] (A00) at ([yshift=0.5 cm] ANDr |- O0.output) {};
\node[and gate,inputs={normal,inverted,normal}] (A01) at ([yshift=-0.5 cm] ANDr |- O0.output) {};

\node[and gate,inputs={inverted,inverted,normal}] (A10) at ([yshift=1 cm] ANDr |- O1.output) {};
\node[and gate,inputs={inverted,normal,inverted}] (A11) at (ANDr |- O1.output) {};
\node[and gate,inputs={normal,normal,normal}] (A12) at ([yshift=-1 cm] ANDr |- O1.output) {};
\foreach \x in {0,1,2} {
  \coordinate (L\x) at (-3.5-0.2*\x,0);
}
\end{scope}
\foreach \x in {0,1} {
  \draw (O\x.output) -- (FF\x.D);
}
\foreach \x/\y/\z/\t in {0/0/1/0.2,0/1/2/0.2,1/0/1/0.2,1/1/2/0,1/2/3/0.2} {
  \draw (O\x.input \z) -- ++(-\t,0) |- (A\x\y.output);
}
\foreach \x/\y/\z/\t in {0/0/1/0,0/0/2/1,0/1/1/0,0/1/2/1,0/1/3/2,1/0/1/0,1/0/2/1,1/0/3/2,1/1/1/0,1/1/2/1,1/1/3/2,1/2/1/0,1/2/2/1,1/2/3/2} {
  \draw (A\x\y.input \z) -- (A\x\y.input \z -| L\t);
  \pdot{A\x\y.input \z -| L\t};
}
\foreach \x/\z/\t in {} {
  \draw (O\x.input \z) -- (O\x.input \z -| L\t);
  \pdot{O\x.input \z -| L\t};
}
\foreach \x/\y/\z in {FF0/Clk0/CLR0,FF1/Clk1/CLR1} {
  \draw (\x.Clk) -- ++(-0.35,0) |- (\y);
  \pdot{\y};
}
\draw (FF1.CLR) |- (CLR1);\pdot{CLR1};
\draw (FF0.PR) |- (PR0);\pdot{PR0};
\draw (FF0.CLR) -- (FF1.PR);\pdot{MID};\draw (MID) -- ++(-1,0) node[scale=0.75,anchor=east]{$1$};
\pdot{I0 -| F2I};
\pdot{I1 -| F3I};
\pdot{I2 -| F4I};
\end{tikzpicture}
\figlab{mooreTotalImplementationD}}
\subfigure[T-flipflops]{
\begin{tikzpicture}[circuit logic US]
\def\ta{0.09};
\def\tb{0.03};
\kkaartcmarks[1]{0}{3}{/1/0/2/1/\ta,/2/0/3/0/\tb}{};
\kkaartc{0}{3}{$T_0$}{$F_0$/$F_1$/$I_0$}{0/0/1/1/1/0/1/1};
\kkaartcmarks[1]{0}{0}{/0/1/1/1/\ta,/2/0/2/0/\ta}{};
\kkaartc{0}{0}{$T_1$}{$F_0$/$F_1$/$I_0$}{0/1/0/1/0/0/1/0};
\def\sc{0.75};
\def\sca{0.8};
\begin{scope}[xshift=6 cm,yshift=2.5 cm,scale=\sc]
\node[tffxn,scale=\sca] (FF0) at (0,2) {$F_0$};
\node[tffxn,scale=\sca] (FF1) at (0,-2) {$F_1$};
\coordinate (F0I) at (-3.5,3.8);
\coordinate (F1I) at (-3.7,4.0);
\coordinate (F2I) at (-3.9,4.0);
\coordinate (F3I) at (-4.1,4.0);
\coordinate (F4I) at (-4.3,4.0);
\coordinate (F0O) at (1,3.8);
\coordinate (F1O) at (1.2,4.0);

\coordinate (Clk0) at (F3I |- 0,1);
\coordinate (Clk1) at (F3I |- 0,-3);
\coordinate (CLR0) at (F4I |- 0,0.5);
\coordinate (CLR1) at (F4I |- 0,-3.5);
\coordinate (PR0) at (F4I |- 0,3.5);
\coordinate (PR1) at (F4I |- 0,-0.5);
\coordinate (MID) at (FF0.CLR |- 0,0);

\draw (FF0.Q) -| (F0O) -- (F0I);
\draw (FF1.Q) -| (F1O) -- (F1I);
\draw (F0I) -- (F0I |- 0,-4);
\draw (F1I) -- (F1I |- 0,-4);
\draw (F2I) -- (F2I |- 0,-4);
\draw (F3I) -- (F3I |- 0,-4);
\draw (F4I) -- (F4I |- 0,-4);
\coordinate (I0) at (-5,0);
\coordinate (I1) at (-5,-0.5);
\coordinate (I2) at (-5,-1);
\draw (I0) node[anchor=east]{$I_0$} -- (I0 -| F2I);
\draw (I1) node[anchor=east]{Clk} -- (I1 -| F3I);
\draw (I2) node[anchor=east]{Clr$^*$} -- (I2 -| F4I);
\coordinate (ORr) at (-1.5,0);
\coordinate (ANDr) at (-2.65,0);

\node[or gate] (O0) at (FF0.T -| ORr) {};
\node[or gate] (O1) at (FF1.T -| ORr) {};

\node[and gate,inputs={normal,inverted}] (A00) at ([yshift=0.5 cm] ANDr |- O0.output) {};

\node[and gate,inputs={inverted,normal}] (A10) at ([yshift=0.5 cm] ANDr |- O1.output) {};
\node[and gate,inputs={normal,normal,inverted}] (A11) at ([yshift=-0.5 cm] ANDr |- O1.output) {};
\foreach \x in {0,1,2} {
  \coordinate (L\x) at (-3.5-0.2*\x,0);
}
\end{scope}
\foreach \x in {0,1} {
  \draw (O\x.output) -- (FF\x.T);
}
\foreach \x/\y/\z/\t in {0/0/1/0.2,1/0/1/0.2,1/1/2/0.2} {
  \draw (O\x.input \z) -- ++(-\t,0) |- (A\x\y.output);
}
\foreach \x/\y/\z/\t in {0/0/1/0,0/0/2/2,1/0/1/0,1/0/2/2,1/1/1/0,1/1/2/1,1/1/3/2} {
  \draw (A\x\y.input \z) -- (A\x\y.input \z -| L\t);
  \pdot{A\x\y.input \z -| L\t};
}
\foreach \x/\z/\t in {0/2/1} {
  \draw (O\x.input \z) -- (O\x.input \z -| L\t);
  \pdot{O\x.input \z -| L\t};
}
\foreach \x/\y/\z in {FF0/Clk0/CLR0,FF1/Clk1/CLR1} {
  \draw (\x.Clk) -- ++(-0.35,0) |- (\y);
  \pdot{\y};
}
\draw (FF1.CLR) |- (CLR1);\pdot{CLR1};
\draw (FF0.PR) |- (PR0);\pdot{PR0};
\draw (FF0.CLR) -- (FF1.PR);\pdot{MID};\draw (MID) -- ++(-1,0) node[scale=0.75,anchor=east]{$1$};
\pdot{I0 -| F2I};
\pdot{I1 -| F3I};
\pdot{I2 -| F4I};
\end{tikzpicture}}
\subfigure[JK-flipflops]{
\begin{tikzpicture}[circuit logic US]
\def\ta{0.09};
\def\tb{0.03};
\kkaartcmarks[0.75]{0}{3.5}{/1/0/2/1/\ta}{};
\kkaartc[0.75]{0}{3.5}{$J_0$}{$F_0$/$F_1$/$I_0$}{0/0/1/1/-/-/-/-};
\kkaartcmarks[0.75]{1.5}{3.5}{/1/0/2/1/\ta,/0/0/3/0/\tb}{};
\kkaartc[0.75]{1.5}{3.5}{$K_0$}{$F_0$/$F_1$/$I_0$}{-/-/-/-/1/0/1/1};
\kkaartcmarks[0.75]{0}{-0.5}{/0/1/1/1/\ta}{};
\kkaartc[0.75]{0}{-0.5}{$J_1$}{$F_0$/$F_1$/$I_0$}{0/1/-/-/0/0/-/-};
\kkaartcmarks[0.75]{1.5}{-0.5}{/0/0/1/0/\ta,/2/1/3/1/\ta}{};
\kkaartc[0.75]{1.5}{-0.5}{$K_1$}{$F_0$/$F_1$/$I_0$}{-/-/1/0/-/-/0/1};
\def\sc{0.75};
\def\sca{0.8};
\begin{scope}[xshift=6 cm,yshift=2.5 cm,scale=\sc]
\node[jkffxn,scale=\sca] (FF0) at (0,2) {};
\node[jkffxn,scale=\sca] (FF1) at (0,-2) {};
\coordinate (F0I) at (-3.5,3.8);
\coordinate (F1I) at (-3.7,4.0);
\coordinate (F2I) at (-3.9,4.0);
\coordinate (F3I) at (-4.1,4.0);
\coordinate (F4I) at (-4.3,4.0);
\coordinate (F0O) at (1,3.8);
\coordinate (F1O) at (1.2,4.0);

\coordinate (Clk0) at (F3I |- FF0.Clk);
\coordinate (Clk1) at (F3I |- FF1.Clk);
\coordinate (CLR0) at (F4I |- 0,0.2);
\coordinate (CLR1) at (F4I |- 0,-3.8);
\coordinate (PR0) at (F4I |- 0,3.5);
\coordinate (PR1) at (F4I |- 0,-0.5);
\coordinate (MID) at (FF0.CLR |- 0,0);

\draw (FF0.Q) -| (F0O) -- (F0I);
\draw (FF1.Q) -| (F1O) -- (F1I);
\draw (F0I) -- (F0I |- 0,-4);
\draw (F1I) -- (F1I |- 0,-4);
\draw (F2I) -- (F2I |- 0,-4);
\draw (F3I) -- (F3I |- 0,-4);
\draw (F4I) -- (F4I |- 0,-4);
\coordinate (I0) at (-5,0);
\coordinate (I1) at (-5,-0.5);
\coordinate (I2) at (-5,-1);
\draw (I0) node[anchor=east]{$I_0$} -- (I0 -| F2I);
\draw (I1) node[anchor=east]{Clk} -- (I1 -| F3I);
\draw (I2) node[anchor=east]{Clr$^*$} -- (I2 -| F4I);
\coordinate (ORr) at (-1.5,0);
\coordinate (ANDr) at (-2.65,0);

\node[or gate,inputs={normal,inverted}] (O0) at (FF0.K -| ORr) {};
\node[or gate] (O1) at (FF1.K -| ORr) {};

%\node[and gate,inputs={normal,inverted}] (A00) at ([yshift=0.5 cm] ANDr |- O0.output) {};

\node[and gate,inputs={inverted,normal}] (A00) at (ANDr |- FF1.J) {};
\node[and gate,inputs={inverted,inverted}] (A10) at (ANDr |- O1.input 1) {};
\node[and gate,inputs={normal,normal}] (A11) at ([yshift=-0.8 cm] ANDr |- O1.input 1) {};
\foreach \x in {0,1,2} {
  \coordinate (L\x) at (-3.5-0.2*\x,0);
}
\end{scope}
\foreach \x in {0,1} {
  \draw (O\x.output) -- (FF\x.K);
}
\foreach \x/\y in {00/1} {
  \draw (A\x.output) -- (FF\y.J);
}
\foreach \x/\y/\z/\t in {1/0/1/0.2,1/1/2/0.2} {
  \draw (O\x.input \z) -- ++(-\t,0) |- (A\x\y.output);
}
\foreach \x/\y/\z/\t in {0/0/1/0,0/0/2/2,1/0/1/0,1/0/2/2,1/1/1/0,1/1/2/2} {
  \draw (A\x\y.input \z) -- (A\x\y.input \z -| L\t);
  \pdot{A\x\y.input \z -| L\t};
}
\foreach \x/\z/\t in {0/1/1,0/2/2} {
  \draw (O\x.input \z) -- (O\x.input \z -| L\t);
  \pdot{O\x.input \z -| L\t};
}
\foreach \x/\y/\t in {FF0/J/1} {
  \draw (\x.\y) -- (\x.\y -| L\t);
  \pdot{\x.\y -| L\t};
}
\foreach \x/\y/\z in {FF0/Clk0/CLR0,FF1/Clk1/CLR1} {
  \draw (\x.Clk) -- ++(-0.35,0) |- (\y);
  \pdot{\y};
}
\draw (FF1.CLR) |- (CLR1);\pdot{CLR1};
\draw (FF0.PR) |- (PR0);\pdot{PR0};
\draw (FF0.CLR) -- (FF1.PR);\pdot{MID};\draw (MID) -- ++(-1,0) node[scale=0.75,anchor=east]{$1$};
\pdot{I0 -| F2I};
\pdot{I1 -| F3I};
\pdot{I2 -| F4I};
\end{tikzpicture}}
\subfigure[SR-flipflops]{
\begin{tikzpicture}[circuit logic US]
\def\ta{0.09};
\def\tb{0.03};
\kkaartcmarks[0.75]{0}{3.5}{/1/0/1/1/\ta}{};
\kkaartc[0.75]{0}{3.5}{$S_0$}{$F_0$/$F_1$/$I_0$}{0/0/1/1/0/-/0/0};
\kkaartcmarks[0.75]{1.5}{3.5}{/2/0/2/1/\ta,/2/0/3/0/\tb}{};
\kkaartc[0.75]{1.5}{3.5}{$R_0$}{$F_0$/$F_1$/$I_0$}{-/-/0/0/1/0/1/1};
\kkaartcmarks[0.75]{0}{-0.5}{/0/1/1/1/\ta}{};
\kkaartc[0.75]{0}{-0.5}{$S_1$}{$F_0$/$F_1$/$I_0$}{0/1/0/-/0/0/-/0};
\kkaartcmarks[0.75]{1.5}{-0.5}{/0/0/1/0/\ta,/2/1/3/1/\ta}{};
\kkaartc[0.75]{1.5}{-0.5}{$R_1$}{$F_0$/$F_1$/$I_0$}{-/0/1/0/-/-/0/1};
\def\sc{0.75};
\def\sca{0.8};
\begin{scope}[xshift=6 cm,yshift=2.5 cm,scale=\sc]
\node[srffxn,scale=\sca] (FF0) at (0,2) {};
\node[srffxn,scale=\sca] (FF1) at (0,-2) {};
\coordinate (F0I) at (-3.5,3.8);
\coordinate (F1I) at (-3.7,4.0);
\coordinate (F2I) at (-3.9,4.0);
\coordinate (F3I) at (-4.1,4.0);
\coordinate (F4I) at (-4.3,4.0);
\coordinate (F0O) at (1,3.8);
\coordinate (F1O) at (1.2,4.0);

\coordinate (Clk0) at (F3I |- FF0.Clk);
\coordinate (Clk1) at (F3I |- FF1.Clk);
\coordinate (CLR0) at (F4I |- 0,0.2);
\coordinate (CLR1) at (F4I |- 0,-3.8);
\coordinate (PR0) at (F4I |- 0,3.5);
\coordinate (PR1) at (F4I |- 0,-0.5);
\coordinate (MID) at (FF0.CLR |- 0,0);

\draw (FF0.Q) -| (F0O) -- (F0I);
\draw (FF1.Q) -| (F1O) -- (F1I);
\draw (F0I) -- (F0I |- 0,-4);
\draw (F1I) -- (F1I |- 0,-4);
\draw (F2I) -- (F2I |- 0,-4);
\draw (F3I) -- (F3I |- 0,-4);
\draw (F4I) -- (F4I |- 0,-4);
\coordinate (I0) at (-5,0);
\coordinate (I1) at (-5,-0.5);
\coordinate (I2) at (-5,-1);
\draw (I0) node[anchor=east]{$I_0$} -- (I0 -| F2I);
\draw (I1) node[anchor=east]{Clk} -- (I1 -| F3I);
\draw (I2) node[anchor=east]{Clr$^*$} -- (I2 -| F4I);
\coordinate (ORr) at (-1.5,0);
\coordinate (ANDr) at (-2.65,0);

\node[or gate,inputs={normal,normal}] (O0) at (FF0.R -| ORr) {};
\node[or gate] (O1) at (FF1.R -| ORr) {};

%\node[and gate,inputs={normal,inverted}] (A00) at ([yshift=0.5 cm] ANDr |- O0.output) {};

\node[and gate,inputs={normal,normal}] (A00) at (ANDr |- O0.input 1) {};
\node[and gate,inputs={normal,inverted}] (A01) at ([yshift=-0.8 cm] ANDr |- O0.input 1) {};
\node[and gate,inputs={inverted,normal}] (A20) at (ANDr |- FF0.S) {};
\node[and gate,inputs={inverted,normal}] (A30) at (ANDr |- FF1.S) {};
\node[and gate,inputs={inverted,inverted}] (A10) at (ANDr |- O1.input 1) {};
\node[and gate,inputs={normal,normal}] (A11) at ([yshift=-0.8 cm] ANDr |- O1.input 1) {};
\foreach \x in {0,1,2} {
  \coordinate (L\x) at (-3.5-0.2*\x,0);
}
\end{scope}
\foreach \x in {0,1} {
  \draw (O\x.output) -- (FF\x.R);
}
\foreach \x/\y in {30/1,20/0} {
  \draw (A\x.output) -- (FF\y.S);
}
\foreach \x/\y/\z/\t in {1/0/1/0.2,1/1/2/0.2,0/0/1/0.2,0/1/2/0.2} {
  \draw (O\x.input \z) -- ++(-\t,0) |- (A\x\y.output);
}
\foreach \x/\y/\z/\t in {2/0/1/0,2/0/2/1,3/0/1/0,3/0/2/2,1/0/1/0,1/0/2/2,1/1/1/0,1/1/2/2,0/0/1/0,0/0/2/1,0/1/1/0,0/1/2/2} {
  \draw (A\x\y.input \z) -- (A\x\y.input \z -| L\t);
  \pdot{A\x\y.input \z -| L\t};
}
\foreach \x/\z/\t in {} {
  \draw (O\x.input \z) -- (O\x.input \z -| L\t);
  \pdot{O\x.input \z -| L\t};
}
\foreach \x/\y/\t in {} {
  \draw (\x.\y) -- (\x.\y -| L\t);
  \pdot{\x.\y -| L\t};
}
\foreach \x/\y/\z in {FF0/Clk0/CLR0,FF1/Clk1/CLR1} {
  \draw (\x.Clk) -- ++(-0.35,0) |- (\y);
  \pdot{\y};
}
\draw (FF1.CLR) |- (CLR1);\pdot{CLR1};
\draw (FF0.PR) |- (PR0);\pdot{PR0};
\draw (FF0.CLR) -- (FF1.PR);\pdot{MID};\draw (MID) -- ++(-1,0) node[scale=0.75,anchor=east]{$1$};
\pdot{I0 -| F2I};
\pdot{I1 -| F3I};
\pdot{I2 -| F4I};
\end{tikzpicture}}
\caption{Implementatie van de Moore-schakeling met verschillende soorten flipflops.}
\figlab{mooreCombImpl}
\end{figure}
Naast de logica voor de volgende toestand, verbinden we ook de klokingangen van de flipflops met het globale kloksignaal, en implementeren we de clear logica. De clear zorgt ervoor dat op het moment dat we de schakeling herzetten en dus het \clrsin{} signaal van de schakeling 0 wordt, de schakeling in de eerste toestand terechtkomt. In ons geval is dat 10. Daarom verbinden we de clear-ingang met de preset ingang van de eerste flipflop en de clear-ingang van de tweede flipflop. Op de clear-ingang van de eerste flipflop en de preset-ingang van de tweede wordt steeds een hoog signaal aangelegd. Merk verder ook op we in principe geen negatieve ingangen voor toestandssignalen moeten gebruiken. In plaats van een negatie aan de ingang van de AND-poort te plaatsen, kunnen we eenvoudig gebruik maken van de $\overline{Q}$-uitgang van de flipflop. Omdat we echter hiermee het aantal lijnen voor de flipflops bijna verdubbelen, maken we er in de figuur abstractie van. Dit principe kunnen we enkel toepassen voor signalen die uit de flipflops komen. De signalen die de ingangsconfiguratie bepalen hebben wel een expliciete NOT-poort nodig.
\paragraph{Het one-hot-codering geval}Naast het geval van de Moore-machine zullen we ook de transitiefunctie van de Mealy-machine implementeren. Over deze realisatie is het ook eenvoudiger om enkele eigenschappen te formaliseren. Deze eigenschappen gaan over de kosten van de transitiefunctie v\'o\'or dat we minimalisatie van de Karnaugh-kaarten toepassen. Verder hangen de eigenschappen af van het type flipflop die we gebruiken:
\begin{itemize}
 \item Bij een D-flipflop is het aantal AND-poorten voor een flipflop equivalent met het aantal transities naar de toestand die de flipflop voorstelt. Bij een transitie\footnote{Inclusief lussen.} naar toestand $x$ dient immer de overeenkomstige flipflop op 1 gezet te worden, in de andere gevallen is de waarde van de flipflop altijd 0. We kunnen dit verder formaliseren tot: ``Het aantal enen in de $D_i$ kolom is gelijk aan het aantal transities naar toestand $i$''.
 \item Bij een T-flipflop is het aantal T-poorten gelijk met het aantal transities die naar een de bijbehorende toestand gaan, of vanuit de toestand vertrekken, die geen lussen zijn. Of eenvoudiger: ``Het aantal enen in de $T_i$-kolom is gelijk aan het aantal transitites van of naar toestand $i$ die geen lussen zijn''.
 \item Bij een SR-flipflop komt is het aantal AND-poorten die naar de S-ingang gaan gelijk aan het aantal transities naar de toestand, die niet uit de toestand komen. Het aantal AND-poorten die voor de R-ingang staan zijn het aantal transities die uit de toestand vertrekken, en die geen lussen zijn. Deze logica kunnen we ook omzetten naar het aantal enen in de kolommen $S_i$ en $R_i$.
 \item Bij JK-flipflops gelden dezelfde regels als bij de SR-flipflops. Alleen dienen we hier S door J en R door K te vervangen.
\end{itemize}
We kunnen deze theorie testen met de praktijk in tabel \ref{tbl:stateTableMealyCombLin}. Hierbij hebben we de tabellen voor de verschillende types flipflops naast elkaar gezet, dit is louter om de voorstelling eenvoudig te houden. Merk op dat verschillende rijen uitsluitend don't cares bevatten. Dit is het gevolg van het feit dat heel wat binaire voorstellingen van toestanden bij de invoer geen overeenkomstige toestand hebben. Meestal worden deze rijen in een tabel weggelaten, wat de tweede tabel een stuk korter en leesbaarder maakt. Uiteraard dienen we dan wel op deze plaatsen in de Karnaugh-kaarten don't cares te plaatsen.
\begin{table}[hbt]
\centering
\subtable[D- en JK-flipflops]{\small{\begin{tabular}{cccc|ccc|cccccc}
$F_0$&$F_1$&$F_2$&$I_0$&$D_0$&$D_1$&$D_2$&$J_0$&$K_0$&$J_1$&$K_1$&$J_2$&$K_2$\\\hline
0&0&0&0&-&-&-&-&-&-&-&-&-\\
0&0&0&1&-&-&-&-&-&-&-&-&-\\
0&0&1&0&0&1&0&0&-&1&-&-&1\\
0&0&1&1&0&0&1&0&-&0&-&-&0\\
0&1&0&0&0&1&0&0&-&-&0&0&-\\
0&1&0&1&1&0&0&1&-&-&1&0&-\\
0&1&1&0&-&-&-&-&-&-&-&-&-\\
0&1&1&1&-&-&-&-&-&-&-&-&-\\
1&0&0&0&0&1&0&-&1&1&-&0&-\\
1&0&0&1&0&0&1&-&1&0&-&1&-\\
1&0&1&0&-&-&-&-&-&-&-&-&-\\
1&0&1&1&-&-&-&-&-&-&-&-&-\\
1&1&0&0&-&-&-&-&-&-&-&-&-\\
1&1&0&1&-&-&-&-&-&-&-&-&-\\
1&1&1&0&-&-&-&-&-&-&-&-&-\\
1&1&1&1&-&-&-&-&-&-&-&-&-
\end{tabular}}}
\subtable[T- en SR-flipflops]{\small{\begin{tabular}{cccc|ccc|cccccc}
$F_0$&$F_1$&$F_2$&$I_0$&$T_0$&$T_1$&$T_2$&$S_0$&$R_0$&$S_1$&$R_1$&$S_2$&$R_2$\\\hline
0&0&1&0&0&1&1&0&-&1&0&0&1\\
0&0&1&1&0&0&0&0&-&0&-&-&0\\
0&1&0&0&0&0&0&0&-&-&0&0&-\\
0&1&0&1&1&1&0&1&0&0&1&0&-\\
1&0&0&0&1&1&0&0&1&1&0&0&-\\
1&0&0&1&1&0&1&0&1&0&-&1&0
\end{tabular}}}
\caption{Implementatie van de Mealy-schakeling met verschillende soorten flipflops.}
\tbllab{stateTableMealyCombLin}
\end{table}
We zien dat $D_0$ \'e\'en 1 bevat, wat overeenkomt met \'e\'en transitite naar toestand $E$ op figuur \ref{fig:minimaalToestandsdiagramMealy}, verder bevatten de kolommen voor toestand $A$ en $B$ respectievelijk 2 en 3 enen. Bij de JK-flipflop heeft toestand $A$ 1 transitie naar $A$ en 1 transitie uit $A$. We zien dat dit ook overeenkomt met het aantal enen bij $J_2$ en $K_2$. We kunnen dus besluiten dat dergelijke eigenschappen handig kunnen zijn bij het controleren van onze tabel. Een andere eigenschap is dat bij een T-flipflop er altijd twee bits veranderen. Bij elke rij zijn er dus ofwel 0 enen ofwel 2. Op figuur \ref{fig:mealyCombImpl} implementeren we de D- en JK-flipflop variant van de Mealy-machine het implementeren van de T- en SR-flipflop variant wordt als een oefening voor de lezer overgelaten.
\begin{figure}[hbt]
\centering
\subfigure[D-flipflop]{\begin{tikzpicture}[circuit logic US]
\def\ta{0.09};
\kkaartdmarks[0.75]{0}{0}{/1/2/2/3/\ta}{}{};
\kkaartd[0.75]{0}{0}{$D_0$}{$F_0$/$F_1$/$F_2$/$I_0$}{-/-/0/0/0/1/-/-/0/0/-/-/-/-/-/-/-/-};
\kkaartdmarks[0.75]{0}{-2.5}{/0/0/3/1/\ta}{}{};
\kkaartd[0.75]{0}{-2.5}{$D_1$}{$F_0$/$F_1$/$F_2$/$I_0$}{-/-/1/0/1/0/-/-/1/0/-/-/-/-/-/-/-/-};
\kkaartdmarks[0.75]{0}{-5}{}{/2/3/1/\ta}{};
\kkaartd[0.75]{0}{-5}{$D_2$}{$F_0$/$F_1$/$F_2$/$I_0$}{-/-/0/1/0/0/-/-/0/1/-/-/-/-/-/-/-/-};
\def\sc{0.75};
\def\sca{0.8};
\begin{scope}[xshift=5 cm,yshift=-1.75 cm,scale=\sc]
\node[dffxn,scale=\sca] (FF0) at (0,3.333) {$FF_0$};
\node[dffxn,scale=\sca] (FF1) at (0,0) {$FF_1$};
\node[dffxn,scale=\sca] (FF2) at (0,-3.333) {$FF_2$};
\def\gx{-1.75};
\node[and gate] (G0) at (\gx,0 |- FF0.D) {};
\node[not gate] (G1) at (\gx,0 |- FF1.D) {};
\node[and gate,inputs={inverted,normal}] (G2) at (\gx,0 |- FF2.D) {};
\foreach \x in {0,...,2} {
  \coordinate (LI\x) at (1+0.2*\x,5+0.2*\x);
  \coordinate (LO\x) at (-2.5-0.2*\x,5+0.2*\x);
  \coordinate (LF\x) at (LO\x |- 0,-5);
  \draw (FF\x.Q) -| (LI\x) -- (LO\x) -- (LF\x);
  \draw (G\x.output) -- (FF\x.D);
}
\foreach \x in {3,4,5} {
  \coordinate (LO\x) at (-2.5-0.2*\x,5.4);
  \coordinate (LF\x) at (LO\x |- 0,-5);
  \draw (LO\x) -- (LF\x);
}
\coordinate (FFCLR1) at (LO5 |- 0,-1.667);
\coordinate (I0) at (-4.25,4.75);
\coordinate (I1) at (-4.25,1.5);
\coordinate (I2) at (-4.25,-1.667);
\draw (I0) node[anchor=east]{$I_0$} -- (I0 -| LO3);
\draw (I1) node[anchor=east]{Clk} -- (I1 -| LO4);
\draw (I2) node[anchor=east]{Clr$^*$} -- (I2 -| LO5);
\end{scope}
\pdot{I0 -| LO3};
\pdot{I1 -| LO4};
\foreach \x/\y/\z in {0/input 1/1,0/input 2/3,1/input/3,2/input 1/1,2/input 2/3} {
  \draw (G\x.\y) -- (G\x.\y -| LO\z);
  \pdot{G\x.\y -| LO\z};
}
\foreach \x/\y/\z in {0,1,2} {
  \draw (FF\x.Clk) -- (FF\x.Clk -| LO4);
  \pdot{FF\x.Clk -| LO4};
}
\draw (FF2.CLR) |- ++(-0.5,-0.11) node[anchor=east,scale=0.75]{$1$};
\draw (FF1.CLR) -- (FF2.PR);
\draw (FF0.PR) |- ++(-0.5,0.11) node[anchor=east,scale=0.75]{$1$};
\draw (FF1.PR) |- ++(-0.5,0.11) node[anchor=east,scale=0.75]{$1$};
\coordinate (FFCLR0) at ([yshift=-0.11 cm] FF0.CLR -| LO5);
\draw (FF0.CLR) |- (FFCLR0);
\draw (FFCLR1 -| FF1.CLR) |- (FFCLR1);
\pdot{FFCLR0};
\pdot{FFCLR1};
\pdot{FFCLR1 -| FF1.CLR};
\end{tikzpicture}}
\subfigure[JK-flipflop]{\begin{tikzpicture}[circuit logic US]
\def\ta{0.09};
\kkaartdmarks[0.75]{0}{0}{/1/2/2/3/\ta}{}{};
\kkaartd[0.75]{0}{0}{$J_0$}{$F_0$/$F_1$/$F_2$/$I_0$}{-/-/0/0/0/1/-/-/-/-/-/-/-/-/-/-/-/-};
\kkaartdmarks[0.75]{2.25}{0}{/0/0/3/3/\ta}{}{};
\kkaartd[0.75]{2.25}{0}{$K_0$}{$F_0$/$F_1$/$F_2$/$I_0$}{-/-/-/-/-/-/-/-/1/1/-/-/-/-/-/-/-/-};
\kkaartdmarks[0.75]{0}{-2.5}{/0/0/3/1/\ta}{}{};
\kkaartd[0.75]{0}{-2.5}{$J_1$}{$F_0$/$F_1$/$F_2$/$I_0$}{-/-/1/0/-/-/-/-/1/0/-/-/-/-/-/-/-/-};
\kkaartdmarks[0.75]{2.25}{-2.5}{/0/2/3/3/\ta}{}{};
\kkaartd[0.75]{2.25}{-2.5}{$K_1$}{$F_0$/$F_1$/$F_2$/$I_0$}{-/-/-/-/0/1/-/-/-/-/-/-/-/-/-/-/-/-};
\kkaartdmarks[0.75]{0}{-5}{/2/2/3/3/\ta}{}{};
\kkaartd[0.75]{0}{-5}{$J_2$}{$F_0$/$F_1$/$F_2$/$I_0$}{-/-/-/-/0/0/-/-/0/1/-/-/-/-/-/-/-/-};
\kkaartdmarks[0.75]{2.25}{-5}{/0/0/3/1/\ta}{}{};
\kkaartd[0.75]{2.25}{-5}{$K_2$}{$F_0$/$F_1$/$F_2$/$I_0$}{-/-/1/0/-/-/-/-/-/-/-/-/-/-/-/-/-/-};
\def\sc{0.75};
\def\sca{0.8};
\begin{scope}[xshift=7.5 cm,yshift=-1.75 cm,scale=\sc]
\node[jkffxn,scale=\sca] (FF0) at (0,3.333) {};
\node[jkffxn,scale=\sca] (FF1) at (0,0) {};
\node[jkffxn,scale=\sca] (FF2) at (0,-3.333) {};
\coordinate (SS) at (FF1.J -| -1.125,0);
\def\gx{-2};
\node[and gate] (GJ0) at (\gx,0 |- FF0.J) {};
\node[not gate] (GJ1) at (\gx,0 |- FF1.J) {};
\node[and gate] (GJ2) at (\gx,0 |- FF2.J) {};
\foreach \x in {0,...,2} {
  \coordinate (LI\x) at (1+0.2*\x,5+0.2*\x);
  \coordinate (LO\x) at (-2.7-0.2*\x,5+0.2*\x);
  \coordinate (LF\x) at (LO\x |- 0,-5);
  \draw (FF\x.Q) -| (LI\x) -- (LO\x) -- (LF\x);
}
\foreach \x in {0,1,2} {
  \draw (GJ\x.output) -- (FF\x.J);
}
\foreach \x in {} {
  \draw (GK\x.output) -- (FF\x.K);
}
\foreach \x in {3,4,5} {
  \coordinate (LO\x) at (-2.7-0.2*\x,5.4);
  \coordinate (LF\x) at (LO\x |- 0,-5);
  \draw (LO\x) -- (LF\x);
}
\coordinate (FFCLR1) at (LO5 |- 0,-1.667);

\coordinate (I0) at (-4.25,4.75);
\coordinate (I1) at (-4.25,1.5);
\coordinate (I2) at (-4.25,-1.667);
\draw (I0) node[anchor=east]{$I_0$} -- (I0 -| LO3);
\draw (I1) node[anchor=east]{Clk} -- (I1 -| LO4);
\draw (I2) node[anchor=east]{Clr$^*$} -- (I2 -| LO5);
\end{scope}
\pdot{I0 -| LO3};
\pdot{I1 -| LO4};
\draw (FF2.K) -| (SS);
\pdot{SS};
\foreach \x/\y/\z in {0/input 1/1,0/input 2/3,1/input/3,2/input 1/0,2/input 2/3} {
  \draw (GJ\x.\y) -- (GJ\x.\y -| LO\z);
  \pdot{GJ\x.\y -| LO\z};
}
\foreach \x/\y/\z in {0,1,2} {
  \draw (FF\x.Clk) -- (FF\x.Clk -| LO4);
  \pdot{FF\x.Clk -| LO4};
}
\draw (FF1.K) -- (FF1.K -| LO3);
\pdot{FF1.K -| LO3};
\draw (FF2.CLR) |- ++(-0.5,-0.11) node[anchor=east,scale=0.75]{$1$};
\draw (FF1.CLR) -- (FF2.PR);
\draw (FF0.PR) |- ++(-0.5,0.11) node[anchor=east,scale=0.75]{$1$};
\draw (FF1.PR) |- ++(-0.5,0.11) node[anchor=east,scale=0.75]{$1$};
\draw (FF0.K) -- ++(-0.5,0) node[anchor=east,scale=0.75]{$1$};
\coordinate (FFCLR0) at ([yshift=-0.11 cm] FF0.CLR -| LO5);
\draw (FF0.CLR) |- (FFCLR0);
\draw (FFCLR1 -| FF1.CLR) |- (FFCLR1);
\pdot{FFCLR0};
\pdot{FFCLR1};
\pdot{FFCLR1 -| FF1.CLR};
\end{tikzpicture}}
\caption{Implementatie van de Mealy-schakeling met verschillende soorten flipflops.}
\figlab{mealyCombImpl}
\end{figure}
Merk op dat we hier de verbindingen naar de \clrsin{} en \prsin{} ingangen anders geconfigureerd zijn dan bij de Moore-machine. Dit komt omdat de begintoestand van de Mealy-machine 001 is, terwijl dit 10 is bij de Moore-machine.
\subsubsection{Stap 4B: Logica die de uitgang toestand berekent}
Naast de logica die de volgende toestand berekent, dienen we ook de de logica te synthetiseren die de uitgangen bepaalt. Dit is eigenlijk niets anders dan het synthetiseren van een combinatorische schakeling. Hiervoor dienen een tabel op te stellen die vanuit de coderingstabel een tabel opstelt die de functie bepaalt. Vermits de uitgang bij een Moore-machine anders bepaald wordt dan bij een Mealy-machine, zal ook de techniek om deze tabel op te stellen licht verschillen. Bij een Moore-machine wordt de uitgang enkel bepaald door de toestand. Hierdoor kunnen we de tabel opstellen door de transitiekolommen weg te laten. Bij een Mealy-machine dienen we een linearisering toe te passen. Hierbij beschouwen we niet meer de codering van de volgende toestand zoals in de vorige stap, maar uiteraard de uitgang. Vanuit die tabel stellen we dan opnieuw een combinatorische schakeling op. Merk op dat deze schakeling niet afhangt van het type flipflop. Immers hebben alle types flipflop een $Q$-uitgang. De schakeling hangt wel af van de toestandscodering. Op tabel \ref{tbl:mooreMealyTables}
\begin{table}[hbt]
\centering
\subtable[Moore coderingstabel]{\begin{tabular}{c|cc|c}
Toestand&0&1&Uitgang\\\hline
\texttt{10}&\texttt{00}&\texttt{10}&\texttt{0}\\
\texttt{00}&\texttt{00}&\texttt{01}&\texttt{0}\\
\texttt{01}&\texttt{11}&\texttt{10}&\texttt{0}\\
\texttt{11}&\texttt{00}&\texttt{01}&\texttt{1}\\
\end{tabular}}
\subtable[Moore]{\begin{tabular}{cc|c}
$F_0$&$F_1$&$O_0$\\\hline
0&0&0\\
0&1&0\\
1&0&0\\
1&1&1\\
\end{tabular}}
\subtable[Mealy coderingstabel]{\begin{tabular}{c|cc}
Toestand&0&1\\\hline
\texttt{001}&\texttt{010/0}&\texttt{001/0}\\
\texttt{010}&\texttt{010/0}&\texttt{100/0}\\
\texttt{100}&\texttt{010/1}&\texttt{001/0}\\
\end{tabular}}
\subtable[Mealy]{\begin{tabular}{cccc|c}
$F_0$&$F_1$&$F_2$&$I_0$&$O_0$\\\hline
0&0&1&0&0\\
0&0&1&1&0\\
0&1&0&0&0\\
0&1&0&1&0\\
1&0&0&0&1\\
1&0&0&1&0
\end{tabular}}
\caption{Uitgangslogica van de Moore- en Mealy-machine naast hun coderingstabellen.}
\tbllab{mooreMealyTables}
\end{table}
toont de tabellen die we opstellen voor de Moore- en Mealy-machine naast hun coderingstabellen. Daarna is het alleen nog een kwestie van de schakeling te implementeren, dit wordt meestal opnieuw gedaan met behulp van Karnaugh-kaarten. We implementeren vervolgens op figuur \ref{fig:mealyTotalImplementation} een volledige sequenti\"ele schakeling gebaseerd op de Mealy-machine met een D-flipflop. Merk op dat we in deze figuur gebruik maken van de $\overline{Q}$-uitgangen. Hoewel we in de figuur de logica implementeren met AND-OR logica zal men in de praktijk altijd opteren voor NAND- en NOR-poorten bij de implementatie. Een andere interessante vaststelling dat het geheugen van de tweede flipflop niet gebruikt wordt. We kunnen bijgevolg deze flipflop en de logica er rond weglaten. Deze extra flipflop is het gevolg van een slecht gekozen toestandscodering. Dit drukt niet alleen de kosten, maar zorgt ook voor een kleinere vertraging waardoor we de klokfrequentie kunnen opdrijven. We gaan kort in op het tijdsgedrag in subsectie \ref{ss:timeBehaviorSeqSync}.
\begin{figure}[hbt]
\centering
\begin{tikzpicture}[circuit logic US]
\def\ta{0.09};
\kkaartdmarks{2.5}{-1}{/2/0/3/1/\ta}{}{};
\kkaartd{2.5}{-1}{$O_0$}{$F_0$/$F_1$/$F_2$/$I_0$}{-/-/0/0/0/0/-/-/1/0/-/-/-/-/-/-/-/-};
\def\sc{0.75};
\def\sca{0.8};
\def\xa{-2.75};
\def\xi{-4.25};
\def\xii{-4.75};
\def\xc{-2.25};
\def\xr{-1.75};
\def\xb{2};
\def\xs{-1.25};

\begin{scope}[scale=\sc]
\node[dffxn,scale=\sca] (FF0) at (0,4) {$FF_0$};
\node[dffxn,scale=\sca] (FF1) at (0,0) {$FF_1$};
\node[dffxn,scale=\sca] (FF2) at (0,-4) {$FF_2$};
\draw (FF0.PR) |- ++(0.25,0.15) node[anchor=west]{$1$};
\draw (FF1.PR) |- ++(0.25,0.15) node[anchor=west]{$1$};
\draw (FF2.CLR) |- ++(0.25,-0.15) node[anchor=west]{$1$};
\draw (FF2.PR) -- (FF1.CLR);
\coordinate (FFM) at (FF1.CLR |- 0,-2);
\node[and gate,anchor=output] (A0) at (FF0.D -| \xa,0) {};
\node[and gate,anchor=input 1] (A1) at (FF0.Q -| -\xc,0) {};
\node[not gate,anchor=output] (N0) at (FF1.D -| \xa,0) {};
\node[and gate,anchor=output] (A2) at (FF2.D -| \xa,0) {};
\draw (A0.output) -- (FF0.D);
\draw (N0.output) -- (FF1.D);
\draw (A2.output) -- (FF2.D);
\foreach \x in {0,1,2} {
  \coordinate (Cl\x) at (FF\x.Clk -| \xc,0);
  \draw (FF\x.Clk) -- (Cl\x);
}
\coordinate (In) at (FF1.D -| \xs,0);
\coordinate (Ina) at (-\xs,2);
\draw (FF0.Q) -- (A1.input 1);
\draw (A1.output) -- ++(0.5,0) node[anchor=west]{$O_0$};
\draw (N0.input) -- (N0.input -| \xii,0) node[anchor=east]{$I_0$};
\coordinate (R0) at (\xr,2.25);
\coordinate (R2) at (\xr,0 |- FFM);
\draw (FFM) -| (R0) -- (R0 -| \xii,0) node[anchor=east]{Clr$^*$};
\draw (FF0.CLR) |- (R0);
\draw (Cl0) -- (Cl2) -- (Cl2 -| \xii,0) node[anchor=east]{Clk};
\draw (In) |- (Ina) |- (A1.input 2);
\draw (N0.input -| \xi,0) |- (A0.input 2);
\draw (N0.input -| \xi,0) |- (A2.input 1);
\coordinate (I) at (N0.input -| \xi,0);
\draw (A2.input 2) -- (A2.input 2 -| \xi,0) |- (-\xr,-6) |- (FF1.Qn);
\draw (A0.input 1) -- (A0.input 1 -| \xi,0) |- (-\xr,6) |- (FF1.Q);
\end{scope}
\foreach \x in {1,2} {
  \pdot{Cl\x};
}
\pdot{R0};
\pdot{FFM};
\pdot{In};
\pdot{I};
\end{tikzpicture}
\caption{Volledige implementatie van de Mealy-machine met D-flipflops.}
\figlab{mealyTotalImplementation}
\end{figure}
\subsection{Tijdsgedrag}
\label{ss:timeBehaviorSeqSync}
Tot slot bepalen we de maximale klokfrequentie die we bij een implementatie van een sequenti\"ele schakeling kunnen hanteren. We herinneren ons uit het hoofdstuk over combinatorische schakelingen dat de vertraging in een systeem bepaald wordt door het kritische pad, de maximale tijd die een signaal nodig heeft om zich doorheen de schakeling voort te planten. In een sequenti\"ele schakeling betekent dit dat het signaal zich doorheen de combinatorische schakeling dient te propageren die de volgende toestand bepaald. Verder dient ook de set-tup-tijd van de geheugenmodules en de tijd die het kloksignaal nodig heeft om zich tot aan de $Q$- of $\overline{Q}$-uitgang te propageren in rekening te worden gebracht. Bij het bepalen van het kritische pad van de transitiefunctie moeten we alle paden vanuit alle ingangen (dus ook de flipflops die toestand bijhouden) beschouwen, naar alle flipflops. Of formeler:
\begin{equation}
\mbox{vertraging pad}=\mbox{logica transitie}+\mbox{set-up flipflop}+\mbox{Clk naar $Q$ of $\overline{Q}$}
\end{equation}
De vertragingen van de klok naar $Q$ en $\overline{Q}$ kunnen we berekenen vanuit de implementatie van de flipflop. De fabrikant van een flipflop zal bovendien steeds deze vertragingen in het datasheet\footnote{Elk ge\"integreerd circuit heeft een datasheet, een document opgesteld door de producent die de karakteristieken van het component formeel beschrijft.} van het component zetten. Indien de implementatie gegeven is, kunnen we de vertraging berekenen. Indien we een master-slave D-flipflop gebruiken dienen we enkel rekening te houden met de propagatietijd van de klok naar $Q$ of $\overline{Q}$ van een D-latch. We hebben deze vertragingen reeds uitgerekend in vergelijking (\ref{eqn:dLatchDelays}) op pagina \pageref{eqn:dLatchDelays}. Eenmaal we de vertraging van het kritisch pad hebben berekend kunnen we de maximale klokfrequentie bepalen. Dit doen we door het inverse te berekenen van deze vertraging:
\begin{equation}
f_{\mbox{\small{max.}}}=\displaystyle\frac{1}{\mbox{vertraging kritisch pad}}
\end{equation}
Merk wel op dat we bij het berekenen van de vertraging geen eenheid gegeven hebben. Deze vertragingen die we tot hier toe hebben berekend dienen uitsluitend om verschillende implementaties met elkaar te vergelijken. In de praktijk zal de technologie waarmee we de schakeling realiseren een tijds\'e\'enheid impliceren waarmee we die vertraging kunnen vermenigvuldigen.
\subsubsection{Leidend voorbeeld}
\paragraph{Moore-machine}
We berekenen de vertraging van de implementatie van de Moore-machine die beschreven staat op figuur \ref{fig:mooreTotalImplementationD}. In totaal kunnen we 10 verschillende paden onderscheiden: in de schakeling staan twee flipflops. Het signaal van deze flipflops vormt een ingang bij alle 5 de AND-poorten. Merk op dat we de inverter aan de AND-poort niet moeten meetellen in onze berekening. We dienen in dat geval de $\overline{Q}$-uitgang te beschouwen. In de onderstaande vergelijking berekenen we de vertraging van alle paden. Het subscript van de vertraging wijst op de bron- en doel-flipflop, en de AND-poort. Deze poorten nummeren we van 0 tot 4 van boven naar beneden:
\begin{equation}
\begin{array}{ll}
\left\{\begin{array}{l}
t_{\small{FF_0\rightarrow FF_0,0}}=\mbox{Clk}\rightarrow \overline{Q}+\mbox{set-up D}+\mbox{2-AND}+\mbox{2-OR}=4.2+1+2.4+2.4=10.0\\
t_{\small{FF_0\rightarrow FF_0,1}}=\mbox{Clk}\rightarrow Q+\mbox{set-up D}+\mbox{3-AND}+\mbox{2-OR}=4.2+1+2.8+2.4=10.4\\
t_{\small{FF_0\rightarrow FF_1,2}}=\mbox{Clk}\rightarrow \overline{Q}+\mbox{set-up D}+\mbox{3-AND}+\mbox{3-OR}=4.2+1+2.8+2.8=10.8\\
t_{\small{FF_0\rightarrow FF_1,3}}=\mbox{Clk}\rightarrow \overline{Q}+\mbox{set-up D}+\mbox{3-AND}+\mbox{3-OR}=4.2+1+2.8+2.8=10.8\\
t_{\small{FF_0\rightarrow FF_1,4}}=\mbox{Clk}\rightarrow Q+\mbox{set-up D}+\mbox{3-AND}+\mbox{3-OR}=4.2+1+2.8+2.8=10.8\\

t_{\small{FF_1\rightarrow FF_0,0}}=\mbox{Clk}\rightarrow Q+\mbox{set-up D}+\mbox{2-AND}+\mbox{2-OR}=4.2+1+2.4+2.4=10.0\\
t_{\small{FF_1\rightarrow FF_0,1}}=\mbox{Clk}\rightarrow \overline{Q}+\mbox{set-up D}+\mbox{3-AND}+\mbox{2-OR}=4.2+1+2.8+2.4=10.4\\
t_{\small{FF_1\rightarrow FF_1,2}}=\mbox{Clk}\rightarrow \overline{Q}+\mbox{set-up D}+\mbox{3-AND}+\mbox{3-OR}=4.2+1+2.8+2.8=10.8\\
t_{\small{FF_1\rightarrow FF_1,3}}=\mbox{Clk}\rightarrow Q+\mbox{set-up D}+\mbox{3-AND}+\mbox{3-OR}=4.2+1+2.8+2.8=10.8\\
t_{\small{FF_1\rightarrow FF_1,4}}=\mbox{Clk}\rightarrow Q+\mbox{set-up D}+\mbox{3-AND}+\mbox{3-OR}=4.2+1+2.8+2.8=10.8
\end{array}\right.&\mbox{(Voorbeeld)}
\end{array}
\end{equation}
Het maximale pad heeft dus een vertraging van $10.8$. Indien we een referentievertraging van $1\mbox{ ns}$ nemen, is de maximale frequentie $f_{\mbox{\small{max.}}}=1/10.8\mbox{ ns}\approx 93\mbox{ MHz}$. Indien we dezelfde schakeling met NAND-NAND logica zouden hebben gebouwd, zou de vertraging van elk pad gereduceerd worden met 2. In dat geval zou de maximale klokfrequentie $f_{\mbox{\small{max.}}}=1/8.8\mbox{ ns}\approx 114\mbox{ MHz}$.
\paragraph{Mealy-machine}
Voor de Mealy-machine gebruiken we de implementatie op figuur \ref{fig:mealyTotalImplementation}. In deze schakeling dienen we twee paden te beschouwen: het eerste loopt van $FF_1$ naar $FF_0$, vertrekt vanuit $Q$ en gaat door \'e\'en 2-AND-poort. Het andere pad ligt tussen $FF_1$ en $FF_2$. Ook hierbij passeren we een 2-AND-poort, merk echter op dat het signaal vertrekt vanuit $\overline{Q}$. De vertraging van dit pad is dus gelijk aan:
\begin{equation}
\begin{array}{ll}
\left\{\begin{array}{l}
t_{\small{FF_1\rightarrow FF_0}}=\mbox{Clk}\rightarrow Q+\mbox{set-up D}+\mbox{2-AND}=4.2+1+2.4=7.6\\
t_{\small{FF_1\rightarrow FF_2}}=\mbox{Clk}\rightarrow \overline{Q}+\mbox{set-up D}+\mbox{2-AND}=4.2+1+2.4=7.6
\end{array}\right.&\mbox{(Voorbeeld)}
\end{array}
\end{equation}
Beide paden hebben hier dus dezelfde vertraging. Als we een referentievertraging van $1\mbox{ ns}$ nemen, is de maximale frequentie $f_{\mbox{\small{max.}}}=1/7.6\mbox{ ns}\approx 131\mbox{ MHz}$.
\section{Asynchrone schakelingen}
\label{s:asynchroneSequence}
Naast schakelingen waar het ritme bepaald wordt door een klok, bestaan er ook schakelingen die een vorm van geheugen bezitten, maar van toestand veranderen doordat het ingangssignaal verandert. We hebben in dit hoofdstuk reeds dergelijke schakelingen ge\"implementeerd. In subsectie \ref{ss:flipflop} hebben we een latch geconstrueerd. Een latch heeft in principe geen klokingang, maar houdt wel een toestand bij. Op het moment dat we een ingang aan een SR-latch aanpasen, zal de latch soms in een andere toestand terechtkomen. Verder bezit het component ook duidelijk een geheugen vermits als de ingang $\left(S,R\right)=\left(0,0\right)$, de vorige toestand behouden blijft. De situatie van de latch illustreert hoe asynchrone schakelingen worden gerealiseerd: We hebben een aantal ingangssignalen, die door logica worden verwerkt in een soort geheugen. Dit geheugen vertraagt een terugkoppeling, en houdt de toestand bij. Het terugkoppelen betekent dat de uitgangen van een logische schakeling als een deel van de invoer fungeren die door die logica verwerkt wordt. In deze cursus zullen we uitsluitend asynchrone sequenti\"ele schakelingen beschouwen die voldoen aan een belangrijke voorwaarde: de \termen{fundamentele modus}, ofwel de ``\termen{fundamental mode restriction}''. Deze beperking bestaat uit twee delen:
\begin{itemize}
 \item Er mag hooguit \'e\'en ingangssignaal tegelijk veranderen. Een voorbeeld die dit concept duidelijk illustreert is de SR-latch: indien we $S$ en $R$ ingang tegelijk van 1 naar 0 omschakelen, komt het component in een oscillerend toestand komt.
 \item Een ingangsverandering mag slechts optreden als alle effecten van de vorige verandering uitgestorven zijn. Ook dit principe kennen we al impliciet. Zo dienen we de set-up tijd van een flipflop te respecteren om te voorkomen dat de flipflop in een onvoorspelbare toestand komt.
\end{itemize}
We zijn in principe niet verplicht om ons aan deze voorwaarden te houden. Alleen wordt het ontwerpen van schakelingen die deze restricties niet volgen zeer moeilijk.
\paragraph{}
Ook bij het ontwerpen van een asynchrone schakeling zullen we een stappenplan volgen. In grote lijnen lijkt deze procedure op het ontwikkelen van synchrone schakelingen. De stappen zelf wijken echter sterk af van hun synchrone variant. In de meeste stappen zullen we ook met de complexe problemen die eigen zijn aan een asynchrone schakeling moeten rekening houden. Het plan bestaat uit 4 stappen:
\begin{enumerate}
 \item Opstellen van een toestandstabel.
 \item Minimaliseren van het aantal toestanden.
 \item Coderen van de toestanden.
 \item Realisatie van de schakeling met digitale logica.
\end{enumerate}
\paragraph{Terminologie}
Bij synchrone schakelingen werkten we met een klokflank. Vaak gebruikt men bij een kloksignaal de termen ``stijgende klokflank'' en ``dalende klokflank''. Dit zijn immers de tijdstippen waarop gebeurtenissen plaatsvinden in de schakeling die relevant zijn. Een asynchrone schakeling heeft geen kloksignaal. Toch worden de termen ``\termen{stijgende flank}'' en ``\termen{dalende flank}'' ook in deze context gebruikt. De gebeurtenissen in een asynchrone schakeling zijn immers de omschakeling van \'e\'en van de ingangssignalen.
\subsection{Leidend voorbeeld}
Een oplettende lezer heeft misschien al opgemerkt dat de geheugencomponenten die we gebruiken in synchrone schakelingen -- latches en flipflops -- op hun beurt weer asynchrone schakelingen zijn. Daarom zullen we een nieuw type latch ontwerpen, die uiteraard louter fictief is. De latch wordt gebruikt als een interrupt-register. In de meeste computers voert een programma niet constant controles uit of een gebruiker bijvoorbeeld een toets aanslaat. De CPU bevat een component dat bijhoudt of er een bepaalde gebeurtenis plaatsvindt. Hardwarematig controleert de CPU frequent of zo'n gebeurtenis is opgetreden. In dat geval voert de processor een procedure uit gedefinieerd door het besturingssysteem. Op dat moment dient het component uiteraard gereset te worden. Verder kan een besturingssysteem ook een masker plaatsen op een interrupt. In dat geval zal de CPU niet reageren op deze gebeurtenis. Dit component heeft twee ingangen $I$ en $E$. $I$ zal een stijgende flank vertonen wanneer de gebruiker een toets aanslaat, maar zal na enige tijd weer dalen. $E$ bepaalt of de CPU luistert naar deze interrupt. Indien $E=0$ heeft het besturingssysteem een masker gezet. Indien we een laag signaal op $E$ aanleggen wordt een reset op de module uitgevoerd. Dit is het geval wanneer de processor de procedure van het besturingssysteem zal uitvoeren. Op het moment dat de procedure uitgevoerd is, zetten we $E$ dan terug hoog om te luisteren naar een mogelijke gebeurtenis. De uitgang $Q$ vertelt ons of er een gebeurtenis is opgetreden op het moment dat er geen masker op de interrupt.
\paragraph{Formele beschrijving}
De vorige paragraaf is relatief informeel beschreven. We zetten deze beschrijving om in een formeler equivalent. Dit doen we aan de hand van tabel \ref{tbl:formalDescriptionAsyncExample}. In de eerste kolom plaatsen we de gebeurtenis. Dit is dus een ingang die van waarde verandert. De tweede kolom bevat de voorwaarden die relevant zijn voor deze regel. Dit zijn testen in verband met ingangen en uitgangen. De derde kolom ten slotte bevat het effect. Een effect houdt in dat de uitgang verandert.
\begin{table}[hbt]
\centering
\begin{tabular}{c|c|c}
Gebeurtenis&Voorwaarde(n)&Effect(en)\\\hline
$I:0\rightarrow 1$&$E=1$&$Q\rightarrow 1$\\
$E:1\rightarrow 0$&&$Q\rightarrow 0$
\end{tabular}
\caption{Formele beschrijving van het leidend voorbeeld.}
\tbllab{formalDescriptionAsyncExample}
\end{table}
Merk verder op dat het onmogelijk is dat $Q$ een 1 bevat en $E$ een 0.
\subsection{Stap 1: Opstellen van een toestandstabel}
Net zoals bij de synchrone schakelingen stellen we eerst een toestandstabel op. Een eerste vraag die zich stelt is wat een toestand is. Een toestand is elke combinatie van in- en uitgangen waarin de schakeling kan terechtkomen. De schakeling die we zullen implementeren telt 3 verschillende signalen. Dit betekent dus dat we hooguit 8 toestanden kunnen bekomen. Aangezien configuraties met $\left(Q,E\right)=\left(1,0\right)$ onmogelijk kunnen voorkomen houden we 6 toestanden over. We stellen een configuratietabel op die voor elke mogelijke configuratie een toestand voorziet in tabel \ref{tbl:configTableAsync}.
\begin{table}[hbt]
\centering
\subtable[Configuratietabel]{\begin{tabular}{c|ccc}
Toestand&$Q$&$I$&$E$\\\hline
$a$&$0$&$0$&$0$\\
$b$&$0$&$0$&$1$\\
$c$&$0$&$1$&$0$\\
$d$&$0$&$1$&$1$\\
(onmogelijk)&$0$&$0$&$0$\\
$e$&$0$&$0$&$1$\\
(onmogelijk)&$0$&$1$&$0$\\
$f$&$0$&$1$&$1$
\end{tabular}
\tbllab{configTableAsync}}
\subtable[Toestandstabel]{\begin{tabular}{c|cccc|c}
\multirow{2}{*}{Toestand}&\multicolumn{4}{c|}{$I\ E$}&\multirow{2}{*}{$Q$}\\&00&01&11&10&\\\hline
$a$&$a$&$b$&$-$&$c$&$0$\\
$b$&$a$&$b$&$f$&$-$&$0$\\
$c$&$a$&$-$&$d$&$c$&$0$\\
$d$&$-$&$b$&$d$&$c$&$0$\\
$e$&$a$&$e$&$f$&$-$&$1$\\
$f$&$-$&$e$&$f$&$c$&$1$
\end{tabular}
\tbllab{stateTableAsync}
}
\caption{Configuratie- en toestandstabel van het leidend voorbeeld.}
\end{table}
Op basis van deze configuratietabel kunnen we nu een toestandstabel maken. Deze toestandtabel loopt erg gelijk met een toestandstabel van een Moore-machine. Waarbij de tabel uit drie delen bestaat: toestand, invoer en uitvoer. Een verschil met de toestandtabel van de Moore-machine is dat we niet alle kolommen invullen bij elke toestand. Een toestand is immers gekoppeld aan een bepaalde invoerconfiguratie\footnote{Beschreven in de configuratietabel.}, de fundamentele modus zorgt ervoor dat er slechts \'e\'en bit tegelijk kan omslaan, bijgevolg kunnen we kolommen die op meerdere bits afwijken niet invullen. Concreet betekent dit dat we voor een invoer van $n$-bits, $n+1$ van de $2^n$ kolommen kunnen invullen. In de overige kolommen schrijven we een liggende streepje (``-''). Veelal zal men ook de kolommen van de toestandtabel anders schikken. Men probeert configuraties die slechts \'e\'en bit verschillen naast elkaar te plaatsen. Dit is echter niet vereist.
\paragraph{}
We stellen de toestandstabel op met behulp van de configuratietabel en de tabel met mogelijke gebeurtenissen. Toestand $a$ betekent in dit geval bijvoorbeeld dat we $\left(I,E\right)=\left(0,0\right)$ aan de ingangen aanleggen. Het spreekt dus voor zich dat we bij de kolom met dezelde configuratie ook een $a$ schrijven. In dat geval is de toestand \termen{stabiel}. Bij de tweede kolom geldt $\left(I,E\right)=\left(0,1\right)$, dit betekent dus dat we een stijgende flank van $E$ beschouwen. Uit de gebeurtenissentabel leiden we af dat in dat geval $Q$ op 0 komt te staan. Dit is reeds het geval. we migreren dus naar de toestand met $\left(Q,I,E\right)=\left(0,0,1\right)$. Op de configuratietabel zien we dat dit toestand $b$ is. Bij de laatste kolom beschouwen we een stijgende flank van $I$, opnieuw gebeurt er niets vermits er niet aan de voorwaarden wordt voldaan in de gebeurtenissentabel. Op deze manier kunnen we de volledige toestandstabel opstellen zoals in tabel \ref{tbl:stateTableAsync}. We kunnen ook met behulp van een toestandsdiagram een grafische voorstelling van deze tabel geven. Het toestandsdiagram staat op figuur \ref{fig:stateDiagramAsync}.
\begin{figure}[hbt]
\centering
\begin{tikzpicture}[->,shorten >=1pt,auto,node distance=2cm,on grid,semithick,state/.style=state with output,every state/.style={draw=black!50,very thick,fill=black!20,scale=0.75}]
\clip (-1.5,-5.5) rectangle (4,1);
\node[state] (E) {$e$\nodepart{lower} $1$};
\node[state] (F) [right=of E] {$f$\nodepart{lower} $1$};
\node[state] (A) [below=of E] {$a$\nodepart{lower} $0$};
\node[state] (B) [below=of F] {$b$\nodepart{lower} $0$};
\node[state] (C) [below=of A] {$c$\nodepart{lower} $0$};
\node[state] (D) [below=of B] {$d$\nodepart{lower} $0$};
\path (A) edge[loop left] node {00} (A)
	  edge[bend left] node {01} (B)
	  edge[bend left] node {10} (C)
      (B) edge[bend left] node {00} (A)
	  edge[loop right] node {01} (B)
	  edge node {11} (F)
      (C) edge[bend left] node {00} (A)
	  edge[loop left] node {10} (C)
	  edge[bend left] node {11} (D)
      (D) edge node {00} (B)
	  edge[bend left] node {10} (C)
	  edge[loop right] node {11} (D)
      (E) edge node {00} (A)
	  edge[loop left] node {01} (E)
	  edge[bend left] node {11} (F)
      (F) edge[bend left] node {01} (E)
	  edge[loop right] node {11} (F);
	  %edge node {10} (C);
\draw (F) .. controls ([shift={(4cm,-4cm)}] F) and ([shift={(2cm,-3cm)}] C).. (C) node[midway] {10};
\end{tikzpicture}
\caption{Toestandsdiagram van het leidend voorbeeld.}
\figlab{stateDiagramAsync}
\end{figure}
\paragraph{}Een toestandtabel bij asynchrone schakelingen wordt ook een ``\termen{flow table}'' genoemd. Indien de tabel nog niet geminimaliseerd is, spreekt met van een ``\termen{primitive flow table}''.
\subsection{Stap 2: Minimaliseren van de toestanden}
Net als bij synchrone schakelingen loont het meestal de moeite om de toestandsruimte te minimaliseren. Dit leidt immers in de meeste gevallen tot een goedkopere schakeling. Vermits het probleem anders is, vertoont de minimalisatie van asynchrone schakelingen ook verschillen tegenover synchrone schakelingen. Een belangrijk verschil is dat we in de flow toestandstabel ook rekening moeten houden met don't cares. Elke toestand heeft er immers $2^n-n-1$ bij $n$-bit invoer. Daarnaast kunnen er uiteraard nog bijkomende don't cares optreden die probleemafhankelijk zijn. Meestal laten deze don't cares toe om op een flexibele manier de toestandsruimte te minimaliseren.
\paragraph{}
Er bestaan verschillende methodes om de toestandsruimte te minimaliseren. We opteren voor een methode die werkt met twee stappen. Allereerst is er de partitionering. Deze partitionering is volledig analoog aan de sequenti\"ele schakelingen (zie \ref{ss:minimizeFSMSeq} op pagina \pageref{ss:minimizeFSMSeq}).
%Het verschil tussen deze regels en de regels bij een sequenti\"ele machine, is dat we hier afdwingen dat beide toestanden naar dezelfde toestand $S_k$ gaan. Bij het minimaliseren van een sequenti\"ele machine mochten beide toestanden een andere volgende toestand hebben, zolang deze toestanden zich in dezelfde partitie bevinden. Een tweede verschil is dat we hier rekening houden met de don't cares. Deze regel zorgt er dan ook voor dat we meer toestanden kunnen tegenkomen die equivalent zijn met elkaar.
We dienen echter de voorwaarden van deze partitionering minimaal aan te passen. De nieuwe voorwaarde wordt dan:
\begin{enumerate}
 \item De toestanden hebben dezelfde uitgangscombinatie.
 \item De don't cares bevinden zich in dezelfde kolommen.
\end{enumerate}
Vervolgens passen we opnieuw de iteratiestap toe zoals bij synchrone schakelingen. De voorwaarden veranderen hierbij niet.
\paragraph{}
Nadat we de partitionering berekend hebben, kunnen we een eerste toestandsreductie toepassen. Bij de synchrone schakelingen betekent de toestandreductie dan ook meteen het einde van de minimalisatie. Bij synchrone schakelingen kunnen we verder reduceren. Hiertoe berekenen we \termen{compatibele toestanden}\footnote{Bemerkt het verschil met ``equivalente toestanden'' bij de partitionering.}. Twee toestanden $S_i$ en $S_j$ zijn equivalent indien:
\begin{enumerate}
 \item Dezelfde uitvoerconfiguratie hebben
 \item Voor elke ingangscombinatie geldt ofwel:
 \begin{enumerate}
  \item Zijn $S_i$ en $S_j$ beide stabiel
  \item De volgende toestand van minstens \'e\'en van de twee toestanden $S_i$ of $S_j$ is niet gespecificeerd, er staat dus een don't care in de kolom van minstens \'e\'en van de toestanden.
  \item Beide toestanden gaan naar dezelfde toestand $S_k$.
 \end{enumerate}
\end{enumerate}
Merk op dat de toestanden $S_i$, $S_j$ en $S_k$ niet de toestanden zijn uit de initi\"ele toestandstabel, maar uit de door partitionering reeds gereduceerde toestandstabel. Een andere mogelijke valkuil is dat beide toestanden naar dezelfde toestand moeten gaan, niet naar twee compatibele of equivalente toestanden. De rede hiervoor is dat de compatibiliteitsrelatie niet transitief is: Als toestand $A$ compatibel is met toestand $B$ en $B$ is compatibel met toestand $C$, is $A$ niet noodzakelijk compatibel met toestand $C$. De relatie is echter wel symmetrisch\footnote{Dit kunnen we eenvoudig aantonen door $S_i$ en $S_j$ om te wisselen. In dat geval zien we dat de voorwaarden niet veranderen.} en uiteraard reflexief.
\paragraph{}
Het feit dat er de compatibiliteitsrelatie niet transitief is introduceert ook een nieuw probleem: we willen verschillende compatibele toestanden samennemen in \'e\'en nieuwe toestand, een zogenaamde \termen{kliek $K_n$}. Vermits hiervoor elke twee toestanden compatibel met elkaar moeten zijn, is dit proces niet deterministisch. Men kan dit probleem op verschillende methodes oplossen. In de praktijk lost men dit vaak op met behulp van programma's die zoeken naar een zo goed mogelijke groepering. Nadat we een groepering hebben bepaald, minimaliseren we opnieuw de toestandentabel. Dit betekent echter nog niet dat het minimalisatiealgoritme ten einde is: door het groeperen van toestanden in een nieuwe toestand, kunnen deze nieuwe toestanden weer compatibiliteit vertonen. We dienen dus opnieuw een compatibiliteitsrelatie op te bouwen en eventueel te minimaliseren. De minimalisatie stopt op het moment dat geen enkele nieuwe toestand meer compatibel is met een andere. Schematisch geven we het hele proces weer in een flowchart op figuur \ref{fig:flowchartMinimizeAsynchrone}.
\begin{figure}[hbt]
\centering
\begin{tikzpicture}[node distance = 2cm, auto]
\node [block] (A) {Partitioneren van de toestanden};
\node [block, below of=A] (B) {Bepaal compatibele toestanden};
\node [block, below of=B] (C) {Groepeer compatibele toestanden};
\node [block, below of=C] (D) {Samenvoegen van de toestanden in een groep};
\node [decision,below of=D] (E) {Toestands\-tabel verandert?};
\node [block] (F) at (E -| 3,0) {Stop};
\node [block] (O) at (A -| -3,0) {Start};
\path [line] (O) -- (A);
\path [line] (A) -- (B);
\path [line] (B) -- (C);
\path [line] (C) -- (D);
\path [line] (D) -- (E);
\path [line] (E) -- node [near start,scale=0.75] {yes} ++(-3,0) |- (B);
\path [line] (E) -- node [near start,scale=0.75] {no} (F);
\end{tikzpicture}
\caption{Flowchart van het minimalisatieproces van asynchrone schakelingen.}
\figlab{flowchartMinimizeAsynchrone}
\end{figure}
\paragraph{Samenvoegen van toestanden in een nieuwe toestand}
Hoe bouwen we vanuit een groep toestanden een nieuwe toestand op? Dit kunnen we in principe al afleiden uit de definitie van compatibele toestanden. Uit de toestanden $\left\{s_1,s_2,\ldots,s_n\right\}$ construeren we een toestand $t$ indien alle toestanden met elkaar compatibel zijn. We dienen bij $t$ ook de overgangen te defini\"eren $s$ gaat onder een invoer $I$ over naar $\delta\left(s,I\right)$ deze functie noemen we ook de \termen{transitie-functie}. Verder voeren we ook een functie $\mathcal{G}\left(s\right)$ in, deze functie geeft de nieuwe toestand weer van de groep waar $s$ toe behoort. We kunnen dan volgende regels op $t$ defini\"eren\footnote{We voeren de regels in de volgorde van verschijnen uit. Het kan zijn dat een situatie aan twee criteria voldoet, in dat geval voeren we de eerste regel uit.}:
\begin{enumerate}
 \item $\delta\left(t,I\right)=-$ indien $\forall i:\delta\left(s_i,I\right)=-$. We plaatsen een don't care bij een transitie van $t$ met invoer $I$ als elke toestanden $s_i$ onder deze invoer ook een don't care bevat.
 \item $\delta\left(t,I\right)=t$ indien $\forall i:\delta\left(s_i,I\right)=-\vee\delta\left(s_i,I\right)\in\left\{s_1,s_2,\ldots,s_n\right\}$. Indien alle toestanden onder invoer $I$ binnen de groep blijven of een don't care bevatten, vormt $t$ onder invoer $I$ een stabiele toestand.
 \item $\delta\left(t,I\right)=u$ indien $\forall i:\delta\left(s_i,I\right)=-\vee\mathcal{G}\left(\delta\left(s_i,I\right)\right)=u$. Indien alle toestanden onder invoer $I$ naar dezelfde groep transformeren, transformeert $t$ onder deze invoer naar de toestand die deze groep voorstelt.
\end{enumerate}
% Door dit nieuwe criterium is de transitiviteit van de compatibiliteitsrelatie niet meer gewaarborgd. Bij de sequenti\"ele schakelingen geldt immers de eigenschap als $A$ compatibel is met $B$ en $B$ met $C$ is $A$ ook compatibel met $C$. De transitiviteit vervalt omdat we toelaten dat don't cares.
\subsubsection{Voorbeeld}
Om alle speciale gevallen in te sluiten gebruiken we een andere toestandstabel beschreven in tabel \ref{tbl:stateTableAsyncMiniOrig}.
\begin{table}[hbt]
\centering
\subtable[Initi\"ele tabel]{\begin{tabular}{c|cccc|c}
\multirow{2}{*}{Toestand}&\multicolumn{4}{c|}{$I\ E$}&\multirow{2}{*}{$Q$}\\&00&01&11&10&\\\hline
$a$&$a$&$f$&$-$&$c$&$0$\\
$b$&$-$&$b$&$h$&$-$&$1$\\
$c$&$-$&$c$&$g$&$j$&$0$\\
$d$&$-$&$f$&$d$&$-$&$1$\\
$e$&$g$&$-$&$d$&$e$&$1$\\
$f$&$-$&$f$&$k$&$-$&$0$\\
$g$&$l$&$g$&$j$&$-$&$0$\\
$h$&$-$&$l$&$h$&$e$&$1$\\
$i$&$i$&$e$&$-$&$f$&$1$\\
$j$&$b$&$-$&$-$&$j$&$0$\\
$k$&$-$&$b$&$k$&$e$&$1$\\
$l$&$-$&$l$&$k$&$-$&$1$\\
$m$&$m$&$l$&$-$&$e$&$1$
\end{tabular}
\tbllab{stateTableAsyncMiniOrig}}
\subtable[Tabel na partitionering]{\begin{tabular}{c|cccc|c}
\multirow{2}{*}{Toestand}&\multicolumn{4}{c|}{$I\ E$}&\multirow{2}{*}{$Q$}\\&00&01&11&10&\\\hline
$a$&$a$&$f$&$-$&$c$&$0$\\
$b$&$-$&$b$&$h$&$-$&$1$\\
$c$&$-$&$c$&$g$&$j$&$0$\\
$d$&$-$&$f$&$d$&$-$&$1$\\
$e$&$g$&$-$&$d$&$e$&$1$\\
$f$&$-$&$f$&$h$&$-$&$0$\\
$g$&$b$&$g$&$j$&$-$&$0$\\
$h$&$-$&$b$&$h$&$e$&$1$\\
$i$&$i$&$e$&$-$&$f$&$1$\\
$j$&$b$&$-$&$-$&$j$&$0$\\
$m$&$m$&$b$&$-$&$e$&$1$
\end{tabular}
\tbllab{stateTableAsyncMiniParty}}
\caption{Evolutie van de toestandstabel bij het minimaliseren voor en na partitioneren.}
\end{table}
Allereerst passen we de initialisatiestap van het partitiealgoritme toe. We dienen hierbij buiten een verschillende uitvoerconfiguratie ook rekening te houden met verschillende don't care configuraties. Dit leidt tot de volgende partitie:
\begin{equation}
\begin{array}{lr}
\mathcal{P}_0=\left\{\left\{a\right\},\left\{b,d,l\right\},\left\{c\right\},\left\{e\right\},\left\{f\right\},\left\{g\right\},\left\{h,k\right\},\left\{i,m\right\},\left\{j\right\}\right\}&\mbox{(voorbeeld)}
\end{array}
\end{equation}
Na het iteratieproces bij de partitionering die wel volledig behouden blijft bekomen we volgende partitie:
\begin{equation}
\begin{array}{lr}
\mathcal{P}_1=\left\{\left\{a\right\},\left\{b,l\right\},\left\{c\right\},\left\{d\right\},\left\{e\right\},\left\{f\right\},\left\{g\right\},\left\{h,k\right\},\left\{i\right\},\left\{j\right\},\left\{m\right\}\right\}&\mbox{(voorbeeld)}
\end{array}
\end{equation}
Op basis van deze partitionering kunnen we een nieuwe toestandstabel opstellen: tabel \ref{tbl:stateTableAsyncMiniParty}. We gebruiken hier telkens de alfabetisch eerste letter van de toestanden die in een bepaalde partitie zitten. Dit is echter een arbitraire keuze. Vervolgens minimaliseren we verder door een compatibiliteitsrelatie op te stellen. Zo is $a$ duidelijk niet compatibel met $b$ net als $a$ en $d$, ze hebben immers een verschillende uitvoer. Ook $\left(a,c\right)$ en $\left(a,e\right)$ zijn duidelijk geen elementen van de relatie: $a$ en $c$ gaan onder invoer $\left(0,1\right)$ naar een verschillende toestand, hetzelfde geldt voor $a$ en $e$ onder invoer $\left(1,0\right)$. $\left(a,f\right)$ behoort dan weer wel tot de relatie. Het zijn beide toestanden met dezelfde uitvoer, en in alle mogelijke invoercombinaties heeft ofwel minstens \'e\'en van de toestanden een don't care, of wijzen ze naar dezelfde toestand. Alle andere toestanden zijn niet compatibel met $a$ omwille van hiervoor opgesomde redenen. We kunnen zo verder alle toestanden tegen elkaar uitspelen, en defini\"eren zo de equivalentierelatie $\equiv_0$ als volgt\footnote{Bij de definitie hebben we de symmetrische redundantie weggelaten. Het spreekt echter voor zich dat als $x\equiv_0y$, dat ook geldt $y\equiv_0x$}:
\begin{equation}
\left\{\begin{array}{cc}
a\equiv_0f&b\equiv_0h\\
b\equiv_0m&c\equiv_0j\\
d\equiv_0e&f\equiv_0j\\
g\equiv_0j&h\equiv_0m
\end{array}\right.
\end{equation}
We kunnen deze relatie ook grafisch voorstellen, dit doen we met behulp van een \termen{Merger diagram}. Hierbij stellen we de toestanden voor als knopen, en indien twee toestanden compatibel zijn, tekenen we een ongerichte boog tussen de knopen die deze toestanden voorstellen. Een dergelijk Merger diagram staat op figuur \ref{fig:mergerDiagram0}.
\begin{figure}
\centering
\hfill{}
\subfigure[Iteratie 1]{\begin{tikzpicture}[shorten >=1pt,auto,node distance=1.3cm,on grid,semithick,every state/.style={draw=black!50,very thick,fill=black!20,scale=0.75}]
\node[state] (A) {$a$};
\node[state] (F) [below=of A] {$f$};
\node[state] (J) [right=of F] {$j$};
\node[state] (C) [below=of J] {$c$};
\node[state] (G) [above=of J] {$g$};
\node[state] (B) [right=of G] {$b$};
\node[state] (H) [right=of B] {$h$};
\node[state] (M) [below=of B] {$m$};
\node[state] (D) [right=of C] {$d$};
\node[state] (E) [right=of D] {$e$};
\node[state] (I) [left=of C] {$i$};
\path (A) edge (F)
      (F) edge (J)
      (J) edge (G)
      (J) edge (C);
\path (B) edge (H)
      (H) edge (M)
      (M) edge (B);
\path (D) edge (E);
\node[draw,dashed,fit=(A) (F)] {};
\node[draw,dashed,fit=(B) (H) (M)] {};
\node[draw,dashed,fit=(C) (J)] {};
\node[draw,dashed,fit=(D) (E)] {};
%\draw[rounded corners=3mm,dashed] (A.north west) -- (A.north east) -- (F.south east) -- (F.south west) -- cycle;
\end{tikzpicture}
\figlab{mergerDiagram0}}
\hfill{}
\subfigure[Iteratie 2]{\begin{tikzpicture}[shorten >=1pt,auto,node distance=1.3cm,on grid,semithick,every state/.style={draw=black!50,very thick,fill=black!20,scale=0.75}]
\node[state] (A) {$a$};
\node[state] (B) [right=of A] {$b$};
\node[state] (C) [right=of B] {$c$};
\node[state] (D) [below=of A] {$d$};
\node[state] (I) [right=of D] {$i$};
\node[state] (G) [right=of I] {$g$};
\path (C) edge (G);
\node[draw,dashed,fit=(C) (G)] {};
\end{tikzpicture}
\figlab{mergerDiagram1}}
\hfill{}
\subfigure[Iteratie 3]{\begin{tikzpicture}[shorten >=1pt,auto,node distance=1.3cm,on grid,semithick,every state/.style={draw=black!50,very thick,fill=black!20,scale=0.75}]
\node[state] (A) {$a$};
\node[state] (B) [right=of A] {$b$};
\node[state] (C) [right=of B] {$c$};
\node[state] (D) [below=of A] {$d$};
\node[state] (I) [right=of D] {$i$};
\end{tikzpicture}
\figlab{mergerDiagram2}}
\hfill{}
\caption{Merger-diagrammen van het leidend voorbeeld.}
\end{figure}
We kunnen vervolgens een keuze maken welke compatibele toestanden we samennemen. Merk op dat we enkel toestanden kunnen samennemen als elke twee toestanden compatibel met elkaar zijn. Zo kunnen we $\left\{a,f\right\}$ samennemen, maar $\left\{a,f,j\right\}$ is niet mogelijk omdat $a$ en $j$ niet compatibel met elkaar zijn. Op figuur \ref{fig:mergerDiagram0} duiden we met behulp van de stippelijnen aan welke toestanden we hebben samengenomen. Of formeler:\begin{equation}
\begin{array}{lr}
\mathcal{P}_2=\left\{\left\{a,f\right\},\left\{b,h,m\right\},\left\{c,j\right\},\left\{d,e\right\},\left\{g\right\},\left\{i\right\}\right\}&\mbox{(voorbeeld)}
\end{array}
\end{equation}
We hadden echter in plaats van $\left\{c,j\right\}$ ook voor $\left\{g,j\right\}$ kunnen opteren. Of $\left\{a,f\right\}$ en $\left\{c,j\right\}$ kunnen inwisselen voor $\left\{f,j\right\}$. Men kan argumenteren dat we met de laatste configuratie minder toestanden wegwerken. Merk echter op dat we eventueel na deze iteratie nog iteraties kunnen uitvoeren en toestanden wegwerken. Een strategie waarin we in elke iteratie het aantal toestanden maximaal reduceren zal niet altijd tot het beste resultaat leiden.
\paragraph{}
We genereren vervolgens voor elke voorgestelde groep een nieuwe toestand. Deze toestand stellen we met de alfabetisch laagste letter voor van de toestanden die de groep omvat. De toestandstabel die hieruit voortkomt staat in tabel \ref{tbl:stateTableIteration1}.
\begin{table}[hbt]
\centering
\subtable[Iteratie 1]{\small{\begin{tabular}{c|cccc|c}
\multirow{2}{*}{Toestand}&\multicolumn{4}{c|}{$I\ E$}&\multirow{2}{*}{$Q$}\\&00&01&11&10&\\\hline
$a$&$a$&$a$&$b$&$c$&$0$\\
$b$&$b$&$b$&$b$&$d$&$1$\\
$c$&$b$&$c$&$g$&$c$&$0$\\
$d$&$g$&$a$&$d$&$d$&$1$\\
$g$&$b$&$g$&$c$&$-$&$0$\\
$i$&$i$&$d$&$-$&$a$&$1$
\end{tabular}}
\tbllab{stateTableIteration1}}
\subtable[Iteratie 2 en 3]{\small{\begin{tabular}{c|cccc|c}
\multirow{2}{*}{Toestand}&\multicolumn{4}{c|}{$I\ E$}&\multirow{2}{*}{$Q$}\\&00&01&11&10&\\\hline
$a$&$a$&$a$&$b$&$c$&$0$\\
$b$&$b$&$b$&$b$&$d$&$1$\\
$c$&$b$&$c$&$c$&$c$&$0$\\
$d$&$c$&$a$&$d$&$d$&$1$\\
$i$&$i$&$d$&$-$&$a$&$1$
\end{tabular}}
\tbllab{stateTableIteration2}}
\caption{Evolutie van de toestandstabel bij het minimaliseren voor en na twee iteraties.}
\end{table}
Hierbij voegen we dus de toestanden samen. We zullen in deze tekst de groep $\left\{b,h,m\right\}$ volledig uitwerken. Deze groep hebben we voorgesteld door $b$. Bij de invoer $\left(0,0\right)$ zien we dat zowel $b$ als $h$ een don't care bevatten. Voor $m$ is dit echter een stabiele toestand bijgevolg is deze ingang ook een stabiele toestand. Bij de configuratie $\left(0,1\right)$ wijzen alle toestanden naar $b$, vermits $b$ in de groep zit die later $b$ zal worden, plaatsen we $b$ in de tabel. Indien $\left(1,1\right)$ op de ingang wordt aangelegd gaan $b$ en $h$ naar $h$, $m$ bevat een don't care. We kunnen dus zonder problemen ook $h$ invullen bij die don't care. $h$ is een onderdeel van de groep die later toestand $b$ zal worden. Bijgevolg kunnen we ook voor deze kolom $b$ invullen. In de laatste configuratie ten slotte -- $\left(1,0\right)$ -- gaan $h$ en $m$ naar $e$. We zien dat $e$ een onderdeel is van de groep $\left\{d,e\right\}$. Deze groep zal dus later voorgesteld worden door de toestand $d$. Bijgevolg vullen we $d$ in. Indien we deze procedure ook toepassen op de andere groepen resulteert dit in tabel \ref{tbl:stateTableIteration1}. Hiermee zijn we aan het einde gekomen van de eerste iteratie. Vermist we echter een andere toestandstabel hebben tegenover het begin van de iteratie, dienen we een nieuwe iteratie aan te vatten. Hierbij berekenen we opnieuw een compatibiliteitsrelatie: $\equiv_1$. We zullen de berekening van deze relatie achterwege laten, vermits we reeds de vorige relatie uitgebreid hebben beschreven. We defini\"eren $\equiv_1$ op symmetrie na als volgt:
\begin{equation}
c\equiv_1g
\end{equation}
Het Merger-diagram van deze relatie staat op figuur \ref{fig:mergerDiagram1}. Hier is de keuze van de groep wel deterministisch. De nieuwe partitie is dus:
\begin{equation}
\begin{array}{lr}
\mathcal{P}_3=\left\{\left\{a\right\},\left\{b\right\},\left\{c,g\right\},\left\{d\right\},\left\{i\right\}\right\}&\mbox{(voorbeeld)}
\end{array}
\end{equation}
Op basis van deze partitie ruilen we dus $\left\{c,g\right\}$ in voor een nieuwe toestand $c$. De vernieuwde toestandstabel is dan tabel \ref{tbl:stateTableIteration2}. Merk op dat we niet enkel de rij van toestand $c$ moeten aanpassen. Ook toestanden die naar $g$ kunnen springen, springen nu naar $c$, bijvoorbeeld toestand $d$. Opnieuw hebben we dus de toestandstabel aangepast. Dit impliceert dat we nogmaals een iteratie uitvoeren. De compatibiliteitsrelatie $\equiv_2$ blijkt echter leeg te zijn, geen enkele toestand is dus compatibel met een andere. Dit leidt tot een Merger-diagram zoals op figuur \ref{fig:mergerDiagram2}. We kunnen bijgevolg geen toestanden groeperen waardoor de tabel onveranderd blijft. We hebben dus het aantal toestanden geminimaliseerd tot 5 zoals in tabel \ref{tbl:stateTableIteration2}.
\paragraph{Alternatieve groepering}Bij de eerste iteratie bij het samenvoegen van de toestanden, konden we kiezen tussen twee vormen van groeperingen. Ofwel groeperen we $c$ en $j$ ofwel $g$ en $j$. Anderzijds hadden we ook $f$ en $j$ kunnen samenvoegen, maar deze configuratie is minder voordelig. We zullen bij wijze van extra voorbeeld ook het alternatieve scenario bespreken. Wanneer we dit alternatieve scenario uitwerken bekomen we na \'e\'en iteratie de waarden op \tblref{stateTableAltIteration1}.
\begin{table}[hbt]
\centering
\subtable[Iteratie 1]{\small{\begin{tabular}{c|cccc|c}
\multirow{2}{*}{Toestand}&\multicolumn{4}{c|}{$I\ E$}&\multirow{2}{*}{$Q$}\\&00&01&11&10&\\\hline
$a$&$a$&$a$&$b$&$c$&$0$\\
$b$&$b$&$b$&$b$&$d$&$1$\\
$c$&$-$&$c$&$g$&$g$&$0$\\
$d$&$g$&$a$&$d$&$d$&$1$\\
$g$&$b$&$g$&$g$&$g$&$0$\\
$i$&$i$&$d$&$-$&$a$&$1$
\end{tabular}}
\tbllab{stateTableAltIteration1}}
\subtable[Iteratie 2 en 3]{\small{\begin{tabular}{c|cccc|c}
\multirow{2}{*}{Toestand}&\multicolumn{4}{c|}{$I\ E$}&\multirow{2}{*}{$Q$}\\&00&01&11&10&\\\hline
$a$&$a$&$a$&$b$&$c$&$0$\\
$b$&$b$&$b$&$b$&$d$&$1$\\
$c$&$b$&$c$&$c$&$c$&$0$\\
$d$&$c$&$a$&$d$&$d$&$1$\\
$i$&$i$&$d$&$-$&$a$&$1$
\end{tabular}}
\tbllab{stateTableAltIteration2}}
\caption{Evolutie van de toestandstabel bij een alternatieve minimalisering voor en na twee iteraties.}
\end{table}
Op basis van deze tabel kunnen we verder bepalen welke toestanden compatibel zijn. Door de definitie van compatibele toestanden toe te passen zien we dat enkel $c$ en $g$ compatibel met elkaar zijn. Vermits er geen alternatieven zijn, voegen we beide toestanden samen en bekomen we \tblref{stateTableAltIteration2}. We kunnen opmerken dat deze tabel volledig identiek is aan de \tblref{stateTableIteration2}. Bijgevolg kunnen we deze tabel ook niet verder minimaliseren. We kunnen ook besluiten dat zelfs wanneer we een keuze kunnen maken tussen verschillende alternatieven, dit niet noodzakelijk betekent dat we een andere ofwel minder minimale configuratie zullen uitkomen. Soms kan men via verschillende keuzes toch hetzelfde of een even goedkope configuratie bekomen. Anderzijds is dit geen algemeen geldend principe: de keuze welke toestanden zullen worden samengevoegd kan wel degelijk verschillende eindconfiguraties opleveren.
\paragraph{Leidend voorbeeld} Bij wijzen van oefening kan de lezer de toestandstabel van het leidend voorbeeld minimaliseren. De oplossing staat in \sscref{asynchronousSequentialMinimalisation}.
\subsection{Stap 3: Codering van de toestanden}
Net als bij synchrone schakelingen moeten we elk van deze toestanden op een manier coderen in het geheugen. Het probleem is dat we bij het coderen van toestanden bij asynchrone schakelingen op nieuwe problemen stuiten.
\subsubsection{Terminologie}
\label{term:race}
Deze problemen brengen nieuwe terminologie met zich mee die we eerst zullen introduceren:
\begin{itemize}
 \item \termen{Race}: Een fenomeen dat optreedt wanneer door \'e\'en ingangsbit te veranderen er minstens twee toestandsvariabelen moeten veranderen.
 \item \termen{Critical race}: Indien een race tot een tijdelijk verkeerde toestand leidt (de codering van een andere toestand dus) spreken we van een critical race.
 \item \termen{Cycle}: Een sequentieel circuit werkt met terugkoppeling. Door deze terugkoppeling kunnen er oscillerende effecten optreden. Een race die in een oscillerend effect uitmondt heet een cycle.
\end{itemize}
Het optreden van deze een critical race of cycle hangt af van de vertragingskarakteristieken van de poorten. Het spreekt voor zich dat we trachten om deze fenomenen te voorkomen. Daarom zullen we ook technieken ontwikkelen die ons toelaten een goede toestandscodering te ontwikkelen waarbij we deze problemen reeds kunnen voorkomen.
\begin{figure}[hbt]
\centering
\importtikzsubfigure{asynchrone-trans}{Toestandstabel.}
\importtikzsubfigure{asynchrone-impl1}{Implementatie 1.}
\importtikzsubfigure{asynchrone-impl2}{Implementatie 2.}
\importtikzsubfigure{asynchrone-expected}{Verwacht gedrag.}
\importtikzsubfigure{asynchrone-cycle}{Cycle.}
\importtikzsubfigure{asynchrone-criticalrace}{Critical race.}
\caption{Voorbeelden van een cycle en critical race.}
\figlab{asynchrone-problems}
\end{figure}
\figref{asynchrone-problems} illustreert het principe van een critical race en cycle. Op \figref{asynchrone-impl1} en \figref{asynchrone-impl2} stellen we twee equivalente asynchrone schakelingen voor. Het enige wat we hebben aangepast is de twee NAND-poorten omzetten naar AND-poorten alsook een NAND-poort naar een OR-poort. In een combinatorische schakeling is dit een perfect geldige transformatie die tot equivalente resultaten leidt. In de implementatie bepalen de signalen $s_0$ en $s_1$ samen de toestand, $q$ bepaald de uitvoer en $i$ en $e$ de invoer. Bij wijze van voorbeeld beschouwen we het systeem in een toestand $\tupl{s_0,s_1}=\tupl{1,0}$ met aan de ingang een signaal $\tupl{i,e}=\tupl{0,1}$. We veranderen vervolgens het signaal van $i$ waardoor we volgens de toestandstabel op \figref{asynchrone-trans} in toestand $\tupl{s_0,s_1}=\tupl{1,0}$ zullen terechtkomen. Wanneer we dit doen volgens de eerste schakeling bekomen we het tijdsgedrag zoals op \figref{asynchrone-expected}. We zien dat beide toestandsignalen tegelijk veranderen. Zelfs wanneer er een klein tijdsverschil op de verandering zit zal dit echter niet tot grote problemen leiden.
\paragraph{}
Wanneer we echter werken met de tweede oplossing is het tijdsverschil veel groter. Hierdoor ontstaat er een terugkoppeling tussen het signaal $b'$ en $s_1$. Beide reageren telkens op de verandering van de ander waardoor we in een cycle terechtkomen. De toestandsverandering wordt dus $\tupl{s_0,s_1}=\tupl{0,1}\rightarrow\tupl{0,0}\rightarrow\tupl{0,1}\rightarrow\ldots$.
\paragraph{}

\paragraph{}Het aantal bits dat verandert tussen twee toestanden is een belangrijke eigenschap, deze wordt vaak ook de ``\termen{Hamming distance}'' ofwel ``\termen{Hammingafstand}'' genoemd, en is enkel gedefinieerd tussen twee bitreeksen van dezelfde lengte.
\subsubsection{Elimineren van critical races}
Men kan critical races elimineren door ervoor te zorgen dat er nooit twee of meer toestandssignalen tegelijk moeten veranderen. Overgangen zoals van $00$ naar $11$ zijn dus uit den boze. Er zijn grofweg drie methodes waarmee we dit kunnen realiseren. We zullen deze methodes ordenen volgens het vermogen om problemen op te lossen. Zo is de laatste methode in staat om alle problemen op te lossen, maar zal deze meestal een hoge kostprijs met zich meebrengen. We proberen dus de problemen op te lossen in de volgorde waarin de methodes worden voorgesteld. Verder zullen de methodes meestal in staat zijn een gedeelte van het probleem op te lossen, waarna de andere methodes de overige problemen kunnen oplossen. De methodes zijn:
\begin{enumerate}
 \item Het kiezen van een toestandscodering die elke overgang realiseert door hooguit \'e\'en bit te veranderen.
 \item Werken met zogezegde ``\termen{tussentoestanden}'': via een reeds bestaande toestand toch de finale toestand bereiken, bij de verschillende overgangen verandert dan telkens slechts \'e\'en bit.
 \item Het introduceren van ``\termen{overgangstoestanden}'': nieuw toestanden toevoegen die geen functionaliteit hebben buiten deze van het doorverwijzen naar een andere toestand.
\end{enumerate}
We zullen elk van deze methodes in de volgende subsubsecties bespreken en toepassen op twee voorbeelden. De toestandstabellen van beide voorbeelden zijn gegeven in \tblref{asynchrone-code-exa} en \tblref{asynchrone-code-exb}.
\begin{table}[hbt]
\centering
\importtabularsubtable{asynchrone-code-exa}{Voorbeeld 1.}
\importtabularsubtable{asynchrone-code-exb}{Voorbeeld 2.}
\caption{Toestandstabellen van de leidende voorbeelden bij de asynchrone toestandscodering.}
\end{table}
In de tabellen werden sommige overgangen geannoteerd met een subscript. Deze overgangen zijn stabiele configuraties: een set van toestand- en ingangbits die tot dezelfde toestand zullen leiden. In het eerste geval zijn er zeven van deze stabiele configuraties, in het tweede voorbeeld zes. Deze configuraties zijn:
\begin{equation}
T_1=\acclarray{
\tupl{a,\tupl{0,0}}_1\\
\tupl{a,\tupl{1,1}}_2\\
\tupl{b,\tupl{0,0}}_3\\
\tupl{b,\tupl{0,1}}_4\\
\tupl{c,\tupl{0,1}}_5\\
\tupl{c,\tupl{1,1}}_6\\
\tupl{d,\tupl{1,0}}_7
}\ \ \ 
T_2=\acclarray{
\tupl{a,\tupl{0,0}}_1\\
\tupl{a,\tupl{1,1}}_2\\
\tupl{a,\tupl{1,0}}_3\\
\tupl{b,\tupl{0,1}}_4\\
\tupl{c,\tupl{0,1}}_5\\
\tupl{c,\tupl{1,0}}_6
}
\end{equation}
Deze stabiele configuraties een belangrijke rol spelen in de verschillende methodes en de transitiediagram.
\subsubsection{Methode 1: Zoeken naar een goede codering}
Hierbij kunnen we terugdenken aan de ``minimal-bit-change'' en de ``Gray-code teller'' uit \ref{term:minimalBitChange}. Deze methode lost in de meeste gevallen reeds heel wat problemen op. Een methode die het beste resultaat oplevert is alle mogelijk toestandscoderingen afgaan en vervolgens het aantal overgangen met 1 veranderende bit tellen. Deze methode is wel niet effici\"ent vermits we \bigoh{n!} verschillende configuraties moeten analyseren. We kunnen uiteraard met behulp van heuristische methodes reeds tot een acceptabele configuratie komen. Bovendien is de kans groot dat niet elke overgang tot 1 veranderende bit is te herleiden, in dat geval moeten we de andere methodes gebruiken. Een optimale configuratie in methode 1 leidt niet noodzakelijk tot het beste eindresultaat.
\paragraph{Voorbeeld}
Bij wijze van voorbeeld zullen we een goede codering zoeken voor beide voorbeelden. In het eerste geval (\tblref{asynchrone-code-exa}) kunnen we oplossen met behulp van een heuristiek. Zo kennen we de toestand $d$ de codering $10$ toe. Omdat er slechts \'e\'en overgang is tussen $c$ en $d$, zullen we $c$ de encodering $01$ geven. $a$ krijgt de encodering $00$ omdat er zowel overgangen tussen $a$ en $c$, als tussen $a$ en $d$ zijn. $b$ krijgt ten slotte de overblijvende codering $11$. Het resultaat van deze codering staat in de coderingstabel in \tblref{asynchrone-code-exa-m1a}. We voeren ook wat nieuwe syntax in die het ons in de volgende stappen makkelijker zal maken: stabiele configuraties zullen we noteren tussen twee vertical bars (``$|$'') in de tabel. Cellen waarbij de toestand naar een andere toestand gaat waarbij de encodering minstens twee bits verandert worden onderlijnd. Deze configuraties zullen we nog trachtten aan te passen met de volgende methodes.
\begin{table}[hbt]
\centering
\importtabularsubtable{asynchrone-code-exa-m1a}{Voorbeeld 1, alternatief 1.}
\importtabularsubtable{asynchrone-code-exa-m1b}{Voorbeeld 1, alternatief 2.}
\importtabularsubtable{asynchrone-code-exb-m1}{Voorbeeld 2.}
\caption{Coderingstabellen van het voorbeeld na het toepassen van de eerste methode.}
\end{table}
\paragraph{}
In het geval van een klein aantal toestanden wel degelijk exhaustief zoeken. Dit wordt vergemakkelijkt omdat we symmetrie\"en kunnen uitbuiten. De concrete codering maakt immers niet zoveel uit, zolang de Hammingafstand maar dezelfde blijft. We introduceren hiervoor een \termen{transitiediagram}. Een transitiediagram is een grafe waarbij de knopen toestanden voorstellen. We plaatsen bogen tussen twee toestanden wanneer er transities tussen twee toestanden kunnen plaatsvinden. Deze bogen worden bevatten de annotaties naar dewelke stabiele configuratie deze transitie uiteindelijk zal migreren. Deze annotaties worden opgedeeld in twee types. Wanneer we na de overgang meteen in een stabiele configuratie terecht komen zetten we de bijbehorende annotatie op de boog zonder deze te onderlijnen. Wanneer we echter na deze toestand niet in een stabiele configuratie terechtkomen, noteren we de annotatie van de stabiele toestand waar we uiteindelijk in zullen terecht komen en wordt deze onderlijnd.
\paragraph{}
Het eerste leidende voorbeeld telt vier toestanden. We kunnen bijgevolg een grafe beschouwen met vier toestanden. Rotaties en spiegelingen bij deze grafes dienen we niet te beschouwen. De Hammingafstand blijft immers onder deze transformaties gelijk. Bijgevolg zijn er slechts twee mogelijke configuraties voorgesteld op \figref{asynchrone-code-exa-m1a} en \figref{asynchrone-code-exa-m1b}.
\begin{figure}[hbt]
\centering
\importtikzsubfigure{asynchrone-code-exa-m1a}{Voorbeeld 1, alternatief 1.}
\importtikzsubfigure{asynchrone-code-exa-m1b}{Voorbeeld 1, alternatief 2.}
\importtikzsubfigure{asynchrone-code-exb-m1}{Voorbeeld 2.}
\caption{Transitiediagramma van het voorbeeld na het toepassen van de eerste methode.}
\end{figure}
Zoals we kunnen vaststellen komt \figref{asynchrone-code-exa-m1a}. We kunnen door twee horizontale of verticale buren om te wisselen de andere versie bekomen. Indien we dit doen -- bijvoorbeeld met $B$ en $C$ -- bekomen we het alternatief op \figref{asynchrone-code-exa-m1b}. We hebben dit alternatief ook in tabelvorm geformaliseerd in \tblref{asynchrone-code-exa-m1b}. Een belangrijk aspect bij deze diagrammen is dat we het aantal overgangen waarbij twee of meer toestand-bits willen minimaliseren. Dit komt dus neer op de diagonale transities. We kunnen opmerken dat het eerste alternatief bijgevolg beter is dan het tweede.
\paragraph{}
Ook het tweede voorbeeld minimaliseren we met behulp van een transitiediagram zoals op \figref{asynchrone-code-exb-m1}. Men kan opnieuw stellen dat dit systeem equivalent is onder spiegeling. Onder rotatie is dit echter niet het geval. Hoe we immers de toestanden ook alloceren, er zal steeds een boog zijn met twee veranderen bits. We kunnen echter kiezen welke twee toestanden er verbonden zijn met deze boog. In het geval van $b$ en $c$ is er slechts sprake van \'e\'en transitie. Bijgevolg alloceren we de toestanden zoals weergegeven op \figref{asynchrone-code-exb-m1} en op \tblref{asynchrone-code-exb-m1}.
\subsubsection{Methode 2: Gebruik maken van een tussentoestand}
Vermits er geen kloksignaal is en we dus te maken hebben met schakeling die blijft zoeken naar een stabiele toestand, hoeven we niet noodzakelijk meteen de uiteindelijke toestand in een cel in te vullen. We kunnen ook een tijdelijke transitie naar een andere toestand ondernemen - en waarbij slechts \'e\'en toestandsbit verandert - waarna we met een reeks tussentoestanden in de uiteindelijke toestand belanden. Dit principe is dus niet beperkt tot \'e\'en tussentoestand. Een extra hulpmiddel die we hierbij kunnen hanteren zijn de don't cares die nog in de tabel staan. Vermits deze configuraties toch niet kunnen voorkomen, kunnen we een transitie invullen bij de bijbehorende don't cares. In het andere geval moeten we op zoek gaan naar een toestand die slechts \'e\'en bit verschilt van de huidige toestand en die dezelfde stabiele eindtoestand voor de invoer-bits stelt.
\paragraph{}
In het eerste voorbeeld is er sprake van twee transities die we moeten aanpassen: wanneer we ons in toestand $\tupl{0,0}_a$ met invoer $\tupl{0,1}$ en in toestand $\tupl{0,1}_c$ met invoer $\tupl{1,0}$. Om deze problemen op te lossen dien we de relevante kolommen van de invoer te inspecteren. In het eerste geval zoeken we naar een toestand met een Hammingafstand van 1 bit die ons onder invoer $\tupl{0,1}$ naar de toestand $\tupl{1,1}_b$ zal brengen. In het eerste geval lijkt zo'n toestand niet te bestaan. We kunnen echter toestand $\tupl{1,0}_d$ opmerken. Deze toestand verschilt slechts \'e\'en bit en bevat in de relevante kolom een don't care. Vermits de ingang toch niet kan voorkomen in deze toestand kunnen we deze gebruiken om de overgang te bewerkstelligen. We vullen dus $\tupl{1,1}_b$ in in de rij van toestand $\tupl{1,0}_d$ met invoer $\tupl{0,1}$. Verder passen we de rij van de originele toestand aan: de cel van toestand $\tupl{0,0}_a$ met invoer $\tupl{0,1}$ overschrijven we met de waarde $\tupl{1,0}_d$. Concreet betekent dit dus wanneer we ons in toestand $b$ bevinden en we leggen de relevante invoer aan, we eerst naar toestand $d$ zullen springen. Toestand $d$ is echter niet stabiel onder deze invoer waardoor we meteen naar toestand $b$ migreren. De oorspronkelijk bedoelde toestand.
\paragraph{}
We proberen ook de tweede problematische transitie van het eerste voorbeeld op te lossen: invoer $\tupl{1,0}$ in toestand $\tupl{0,1}_c$. We gaan opnieuw op zoek naar een toestand die in dezelfde kolom ook tot toestand $\tupl{1,0}_d$ leidt. Er bestaat hiervoor \'e\'en toestand: $\tupl{1,1}_b$ heeft een Hamming-afstand van $1$ en we zien dat in de kolom voor invoer $\tupl{1,0}$ deze ook een transitie naar $\tupl{1,0}_d$ leidt. Bijgevolg modificeren we de eerste cel zodat deze naar $\tupl{1,1}_b$ leidt. Na deze stappen bekomen we de coderingstabel in \tblref{asynchrone-code-exa-m2} en het transitiediagram in \figref{asynchrone-code-exa-m2}. De schuingedrukte cellen zijn cellen die we hebben aangepast. Zoals we kunnen zien zijn er geen transities meer die meer dan \'e\'en bit aanpassen. De toestandscodering is dus volledig aangepast.
\begin{table}[hbt]
\centering
\importtabularsubtable{asynchrone-code-exa-m2}{Voorbeeld 1.}
\importtabularsubtable{asynchrone-code-exa-m2c}{Voorbeeld 1 met compressie.}
\importtabularsubtable{asynchrone-code-exb-m2}{Voorbeeld 2.}
\caption{Coderingstabellen van het voorbeeld na het toepassen van de tweede methode.}
\end{table}
\begin{figure}[hbt]
\centering
\importtikzsubfigure{asynchrone-code-exa-m2}{Voorbeeld 1.}
\importtikzsubfigure{asynchrone-code-exb-m2}{Voorbeeld 2.}
\caption{Transitiediagramma van het voorbeeld na het toepassen van de tweede methode.}
\end{figure}
\paragraph{}
We kunnen in deze methode ook een ander aspect bewerkstelligen: compressie. Compressie probeert de sequentie van toestanden in te korten die de schakeling zal overlopen wanneer de ingang verandert. Stel dat de schakeling zich in toestand $\tupl{0,0}_a$ bevinden en de ingang vanuit een stabiele configuratie naar de invoer-bits $\tupl{1,0}$ verandert. In dat geval doorloopt de schakeling de volgende toestanden: $\tupl{0,0}_a\rightarrow\tupl{0,1}_c\rightarrow\tupl{1,1}_b\rightarrow\tupl{1,0}_d$. We doorlopen dus vier toestanden alvorens we in de uiteindelijke stabiele configuratie terecht komen. Dit terwijl de Hammingafstand tussen $a$ en $d$ slechts \'e\'en bit bedraagt. Dit betekent dus dat we rechtstreeks naar $d$ kunnen migreren. Hiervoor dienen we dus enkel de transitie van toestand $\tupl{0,0}_a$ met invoer $\tupl{1,0}$ aan te passen naar $\tupl{1,0}_d$. We bekomen dus de coderingstabel in \tblref{asynchrone-code-exa-m2c}.
\paragraph{}
We zullen ook proberen deze methode toe te passen op voorbeeld $2$. Meer bepaald wanneer we ons in toestand $\tupl{0,1}_b$ bevinden en we veranderen de ingang-bits naar $\tupl{1,1}$ beschouwen we momenteel een transitie waarbij twee bits veranderen. De enige toestand met een Hammingafstand van $1$ is echter $\tupl{0,0}_a$. Vermits in deze toestand we met ingang-bits $\tupl{1,1}$ in een stabiele configuratie zitten, kunnen we geen transitie bewerkstelligen naar $\tupl{1,0}_c$. Het probleem kan dus niet opgelost worden met deze methode. We zullen dus methode $3$ hiervoor moeten aanwenden.
\subsubsection{Methode 3: Invoeren van extra overgangstoestand}
In de laatste methode introduceren we om de problematische transities op te lossen een nieuwe toestand. Deze toestand mag -- net als in de vorige methode -- slechts een Hammingafstand van $1$ optekenen met de toestanded waartussen de problematisch codering zich bevindt. Dit is niet altijd mogelijk. Daarom zal men soms zelfs een pad van verschillende extra overgangstoestanden moeten ontwerpen die telkens \'e\'en bit van elkaar verschillen.
\paragraph{}
Omdat alle coderingen soms al in gebruik zijn of om aan de voorwaarde van de Hamming-afstand te voldoen is het daarom niet altijd eenvoudig of zelfs mogelijk om een codering met hetzelfde aantal bits te voorzien. In dat geval moeten we soms de toestandcodering uitbreiden naar meerdere bits. Wanneer we extra bits toevoegen betekent dit traditioneel dat we de volledige codering herbekijken wat veel werk met zich meebrengt. Door echter bits vooraan toe te voegen kunnen we de oude codering behouden (met leidende $0$-bits bijvoorbeeld) een maken we tegelijk ruimte om meer toestanden voor te stellen.
\paragraph{}
Eenmaal we een nieuwe codering hebben beschouwd is de realisatie eenvoudig: in de kolom van de relevante invoer-configuratie plaatsen we het pad door elke extra overgangstoestand te laten wijzen naar de volgende overgangstoestand in het pad. Ook passen we de rij van de oorspronkelijke toestand aan zodat deze wijst naar de eerste overgangstoestand in het pad. De overige kolommen vullen we op met don't cares alsook de uitvoer in de overgangstoestanden.
\paragraph{}
We hoeven niet voor elke problematische transitie meteen extra overgangstoestanden te voorzien. Soms kan men bijvoorbeeld eerst \'e\'en overgang oplossen en vervolgens met methode $2$ bijvoorbeeld proberen de nieuwe toestand als tussentoestand te beschouwen bij het oplossen van een andere problematische transitie.
\importtabulartable{asynchrone-code-exb-m3}{Coderingstabel van het voorbeeld na het toepassen van de derde methode.}
\importtikzfigure{asynchrone-code-exb-m3}{Transitiediagram van het voorbeeld na het toepassen van de derde methode.}
\subsubsection{Initi\"ele toestand}
\subsection{Stap 4: Realisatie met digitale logica}
\part{Processoren}
\chapter{Niet-Programmeerbare Processoren}
\chapterquote{We accepteren nu het feit dat leren een levenslang proces is om op de hoogte te blijven van veranderingen. En de meest urgente taak is mensen te leren hoe te leren.}{Peter F. Drucker, Amerikaans management consultant en auteur (1909-)}
\begin{chapterintro}
In de twee vorige hoofdstukken hebben we componenten gebouwd met een beperkte functionaliteit. De combinatorische schakelingen laten ons toe om schakelingen te ontwerpen die een rekenkundige operatie uitvoeren, maar we hebben geen geheugen beschikbaar om tussenresultaten in op te slaan. Het hoofdstuk over sequenti\"ele schakelingen maakt het mogelijk om schakelingen te ontwerpen met een geheugen. De meeste problemen hebben echter zeer grote toestandsruimtes (een 32-bit getal heeft meer dan vier miljard toestanden). Daarom volstaan de methodes uit dit hoofdstuk niet om een component te ontwikkelen die iets functioneel doet. Daarvoor zullen we methodes op een hoger niveau introduceren, dat van een niet-programmeerbare processor. Een niet programmeerbare processor voert een algoritme uit die op voorhand gekend is. Hierdoor kunnen we optimaal gebruik maken van de hardware en zoveel mogelijk instructies tegelijk uitvoeren. Het nadeel is dat eenmaal de processor geproduceerd is, we geen andere problemen met het component kunnen uitvoeren.
\end{chapterintro}
\minitoc[n]
\section{De Niet-Programmeerbare Processor}
Alvorens we de bouw van zo'n processor verder uitwerken, dienen we eerst enkele concepten te formaliseren. Allereerst ontleden we in deze sectie uit welke delen zo'n processor is opgebouwd. Vervolgens zullen we in sectie \ref{s:descriptionFSMD} een methode ontwikkelen om een algoritme formeel weer te geven. Deze beschrijving zal toelaten het algoritme later om te zetten naar een processor. In sectie \ref{s:memoryFSMD} ten slotte zullen we extra geheugencomponenten introduceren die we nodig zullen hebben bij de bouw van een processor.
\subsection{Algemene Structuur}
Een \termen{Niet-programmeerbare processor}, ofwel \termen{Finite State Machine with Data path (FSMD)} bestaat grofweg uit twee delen:
\begin{itemize}
 \item Een \termen{datapad}: een component die bewerkingen (rekenkundig, arithmetisch,...) uitvoert en de resultaten opslaat in tijdelijk geheugen.
 \item Een \termen{controller}: een component die het datapad aanstuurt. Het zegt welke actie op welk moment moet ondernomen worden.
\end{itemize}
In dit hoofdstuk is de controller niet programmeerbaar. Dat wil zeggen dat de controller telkens hetzelfde programma uitvoert. Dit betekent echter niet dat er een vaste cyclus in de controller zit. De controller kan afhankelijk van de waarden die in de geheugens van het datapad zitten, of van ingangen van de processor beslissen om andere acties te ondernemen. Een controller is dus een sequenti\"ele schakeling ofwel finite state machine. De synthese van een finite state machine werd in het hoofdstuk \ref{ch:SeqComp} reeds besproken. Uiteraard zullen we de karakteristieken die eigen zijn aan controllers in dit hoofdstuk bespreken.
\paragraph{}
Het spreekt voor zich dat de controller en het datapad continu data met elkaar uitwisselen. Enerzijds geeft de controller instructies aan het datapad. De groep signalen waarmee een controller een datapad aanstuurt noemen we het ``\termen{instructiewoord}''. Anderzijds zullen de instructies vaak afhangen van de toestand van variabelen opgeslagen in het datapad. De verzameling van signalen die het datapad over zijn variabelen doorstuurt naar de controller noemen we ``\termen{statussignalen}''.
\subsection{Het Datapad}
\section{Formeel Beschrijven van een Algoritme}
\label{s:descriptionFSMD}
\section{Geheugencomponenten}
\label{s:memoryFSMD}
\section{Synthese van een Niet-Programmeerbare Processor}
\label{s:syntheseFSMD}
\section{Tijdsgedrag}
\label{s:timeFSMD}
%\chapter{Programmeerbare Processoren}
\chapterquote{Mensen hebben met computers gemeen dat ze ook niets bereiken zonder goede programmeurs.}{Toon Verhoeven, Nederlands aforist (??-??)}
\begin{chapterintro}
In dit hoofdstuk bespreken we programmeerbare processoren. Deze processoren verschillen van de niet-programmeren processoren omdat een gebruiker een programma -- een reeks van instructies -- kan uitvoeren op de processor en bijgevolg in grote mate het algoritme die de processor uitvoert zelf kan bepalen. Centrale processoreenheden (CPU's) zijn hiervan slechts een subset van de programmeerbare processoren. We bespreken eerst de hoe een instructie eruitziet en hoe we een schakeling kunnen ontwerpen om deze instructies uit toe voeren. Cruciaal hierbij zijn de verschillende adresseermodi. Vervolgens beschouwen we de twee grote instructie-families: RISC en CISC. We zullen voor beide een instructieset ontwerpen en de verschillende aspecten die hierbij komen kijken bespreken.
\end{chapterintro}
\minitoc[n]
\section{De Programmeerbare Processor}
We zullen in deze sectie het verschil bespreken tussen een niet-programmeerbare processor (zie \chpref{nonprogramming}) en een programmeerbare processor. Algemeen is deze grens eerder vaag. Ook bij een niet programmeerbare processor zullen de controller en het datapad meestal be\"invloed worden door signalen van buiten de schakeling. Sommige technici kunnen door deze invoer te manipuleren de processor een algoritme laten uitvoeren waarvoor de schakeling niet ontworpen was.
\paragraph{}
Een duidelijke grens kunnen we trekken bij de controller. Bij conventie stellen we dat een processor niet-programmeerbaar is wanneer de schakeling wordt aangestuurd met een vaste controller. In het geval we dus een ander algoritme willen uitvoeren zullen we een ander controller moeten implementeren om het uit te voeren. Bij een programmeerbare processor dient men de controller niet aan te passen. De controller beschikt immers over een geheugen waarin het programma geladen kan worden. Door het geheugen aan te passen zal de controller het datapad anders aansturen waardoor een ander algoritme kan worden uitgevoerd. We kunnen dus stellen dat de eindige toestandsautomaat die in de controller werd ge\"implementeerd vast staat, onafhankelijk van het ingeladen programma. Het geheugen waaruit zo'n controller leest noemen we het \termen{programmageheugen}. De gegevens die van de controller naar dit geheugen stuurt worden het \termen{adres}, de \termen{programmateller} of de \termen{program counter (PC)} genoemd. De data van het programmageheugen naar de controller noemen we de \termen{instructie}.
\paragraph{}
Merk op dat de definitie hierbij geen concrete uitspraak doet over hoe een programma of instructie er precies dient uit te zien. De definitie impliceert bijvoorbeeld niet dat de processor elk te beschrijven algoritme moet kunnen uitvoeren. Hierbij kunnen we bijvoorbeeld denken aan een ``Graphical Processing Unit (GPU)''. Een GPU is een programmeerbare processor die gespecialiseerd is in grafische taken. De instructieset is dan ook eerder beperkt tot grafische operaties. Hoewel men dergelijk processoren meestal niet kan programmeren om bijvoorbeeld Dijkstra's algoritme uit te voeren, is de processor wel programmeerbaar.
\paragraph{}
\importtikzfigure{processor-programming}{De structuur van een programmeerbare processor.}
\figref{processor-programming} toont hoe een programmeerbare processor er in grote lijnen uitziet. Merk op dat het enige verschil tussen deze figuur en \figrefpag{processorInformationStreams} de introductie van een programmageheugen is.
\section{Programma als een Reeks Instructies}
Nu we de structuur van een programmeerbare processor hebben voorgesteld, zullen we in deze sectie de verschillende aspecten en terminologie van een programma bespreken.
\subsection{Programma}
Een algemeen aanvaarde definitie voor een \termen{programma} is een sequentie van instructies. De instructies zijn op zo'n manier bepaald en geordend dat ze samen een complexe (en nuttige) taak uitvoeren. In dit opzicht bevat een sequentie van instructies dus dezelfde informatie als de eindige toestandsmachine van de controller bij een niet-programmeerbare processor.
\subsection{Instructie}
Een instructie is een reeks van bits die de informatie van \'e\'en toestand in het ASM-schema of \'e\'en toestand in de eindige toestandsautomaat van de controller voorstellen. Door pipelining kan een instructie of multicycling kan een instructie echter meerdere klokcycli duren. Dit kan men implementeren door bijvoorbeeld een andere volgende instructie te kiezen. Om een toestand voor te stellen zijn drie types informatie vereist:
\begin{itemize}
 \item De aansturing van het datapad: het bepalen van de stuursignalen naar de verschillende functionele eenheden, registers, tri-state buffers en multiplexers betrokken in het datapad.
 \item \termen{Data-uitwisseling}: sommige instructies bepalen welke informatie er ingeladen of weggeschreven wordt naar (externe) geheugens.
 \item De volgende instructie: de meeste algoritmes bevatten lussen en voorwaardelijke gedeeltes. Deze controle wordt ge\"implementeerd doordat de instructies (impliciet) bepalen wat de volgende instructie zal zijn.
\end{itemize}
Een instructie moet niet elk type informatie expliciet specificeren. We kunnen bijvoorbeeld denken aan het bepalen van de volgende toestand: meestal wordt een programma zo gestructureerd dat de volgende instructie bijna altijd op het volgende adres staat. In dat geval zullen enkel instructies die afwijken van deze regel dit moeten specificeren.
\paragraph{Notatie van een instructie}
Om instructies uit te drukken bestaan er twee typische notaties: de \termen{mnemonische notatie} en de \termen{actie-notatie}. In de mnemonische notatie specificeert men eerst de operatie gevolgd door het doel\footnote{De plaats waar het resultaat zal worden opgeslagen.} en de operanden. Verder worden zowel het doel en de operanden gespecificeerd aan de hand van adressen. Een typische instructie is bijgevolg \verb+add A B C+. Dit voorbeeld is een instructie uit de \verb+80x86+ instructieset. Mnemonische notatie wordt dan ook vaak gebruikt in assembleertalen (\verb+80x86+, ). Actie-notatie specificeert daarentegen eerst het doel, meestal gevolgd door bijvoorbeeld een pijl met daarna de functie en de operand. Een concreet voorbeeld van een instructie volgens deze notatie is \verb/Mem[A] <- Mem[B]+Mem[C]/. Actie-notatie is populair bij hardwarespecificatietalen.
\subsection{Instructieformaat}
De mnemonische notatie en de actie-notatie zijn manieren om instructies voor te stellen zodat ze leesbaar zijn voor mensen. In digitale logica wordt een instructie enkel voorgesteld door een sequentie aan bits. De ontwerper van een processor dient dan ook een \termen{instructieformaat} te specificeren: een beschrijving hoe een sequentie bits een instructie bepaald. Men kan dit formaat natuurlijk vrij bepalen, maar meestal beoogt men een structurele opbouw: de instructie wordt onderverdeeld in \termen{velden}: groepen van bits waar een betekenis of een subtaak aan wordt toegekend. In de meeste instructieformaten komen volgende velden voor:
\begin{itemize}
 \item \termen{Instructietype}: een groep bits die de klasse van de instructie aangeeft (bijvoorbeeld een sprongbevel, een bewerking, een geprivilegieerde instructie,...)
 \item \termen{Opcode} ofwel \termen{operation code}: een groep bits die de bewerking voorstellen (bijvoorbeeld een optelling, vermenigvuldiging,...)
 \item Adres: een groep bits die de locatie van een operand of een resultaat specificeert. Een adres hoeft echter niet beperkt te zijn tot de locatie in een (extern) geheugen: ook registers en registerbanken kunnen soms worden geadresseerd.
 \item \termen{Adresseermode}: een adresseermode bepaald hoe het adres gespecificeerd in bits wordt omgezet in een fysisch adres. Zo kan de adresseermode bijvoorbeeld bepalen dat het adres een register uit de registerbank specificeert of dat de adressen indirect\footnote{Bij indirecte adressering leest men de waarde van het geheugen uit op de gegeven locatie. Die waarde bepaald dan de locatie van de effectieve waarde.} moeten worden berekend.
 \item Constante: bewerkingen zoals een optelling tellen soms een constante op bij een register. In dat geval moet de constante in de instructie worden ingebed.
\end{itemize}
Men dient op te merken dat een veld niet noodzakelijk een vaste lengte heeft of slechts \'e\'enmaal voorkomt. Stel bijvoorbeeld dat de adresseermode bepaald dat de gegevens uit een registerbank moeten uitgelezen worden, verwachten we dat het adres korter zal zijn dan wanneer we het adres uit een RAM-geheugen halen. Verder zal men bij sommige operaties twee of meer operanden moeten selecteren. In dat geval is het dus mogelijk dat het adresveld meerdere keren voorkomt. Sommige processoren kunnen voorzien ook een instructieformaat waarbij men twee of meer bewerkingen kan specificeren die dan parallel worden uitgevoerd. In dat geval komt de opcode dus twee of meer keer voor.
\subsection{Generische Instructiecyclus}
Een processor voert een instructie doorgaans uit in vijf stappen. Deze stappen noemt men de generische instructiecyclus. De stappen zijn:
\begin{enumerate}
 \item Lees de instructie in: het programmageheugen wordt uitgelezen en de instructie wordt in het \termen{instructieregister (IR)} opgeslagen. De programmateller wordt verhoogt.
 \item Bereken de adressen: Op basis van de adresseermodi en de adressen in de instructie worden de echte adressen bepaald. Een adres uit een registerbank zal er dus anders uitzien dan een adres uit het RAM-geheugen.
 \item Lees de operanden: de adressen van de operanden worden uitgelezen en weggeschreven in tijdelijke registers.
 \item Voer de bewerking uit: op basis van de data in de operand-registers en de opcode kan men de relevante bewerking op de relevante data uitvoeren. Het resultaat wordt in een tijdelijk register geplaatst.
 \item Schrijf het resultaat weg: het resultaat wordt weggeschreven in een registerbank of RAM-geheugen (indien dit relevant is voor de bewerking).
\end{enumerate}
Deze cyclus wordt eindeloos herhaalt en leent zich meestal erg goed tot pipelining: enkel wanneer de volgende instructie een operand moet inlezen die bepaald wordt door een instructie die kort ervoor is uitgevoerd moet de pipeline worden onderbroken. Men kan een sprongbevel uitvoeren door in de instructie de programmateller aan te passen.
\subsection{Uitvoeringssnelheid}
De uitvoeringssnelheid van een instructie hangt in grote mate af van twee factoren:
\begin{itemize}
 \item De snelheid van het datapad
 \item Het aantal toegangen tot extern geheugen.
\end{itemize}
We kunnen de snelheid van het datapad doorgaans opdrijven door meer hardware te voorzien die bijvoorbeeld bewerkingen in parallel uitvoeren. Het aantal toegangen tot extern geheugen kunnen we dan weer verlagen door kleine en simpele instructies te voorzien waardoor met elk een beperkt aantal operaties op het geheugen.
\paragraph{}
De grootte van een instructie is echter een trade-off. Wanneer we grote instructies voorzien met een groot aantal bits laten we de programmeur toe om een complexe taak in zo'n instructie te specificeren. Bijgevolg verwachten we dat een programma uit een klein aantal van dergelijke instructies zal bestaan. Omdat de instructies echter complex zijn, verwachten we een traag datapad en veel geheugentoegangen. Wanneer de processor enkel simpele instructies aanbied kan het datapad deze snel uitvoeren, maar een programma zal een groot aantal instructies bevatten. De complexiteit van een instructieset wordt dan ook soms uitgedrukt in het aantal adresvelden.
\subsection{Adresvelden}
In deze subsectie zullen we enkele instructiesets bespreken volgens het aantal adresvelden. Bij de verschillende instructiesets zullen we aantal geheugentoegangen berekenen die nodig zijn om de volgende $\brak{a+b}\times\brak{a-b}$ uit te rekenen. Deze operatie wordt doorgaans niet als \'e\'en instructie aangeboden (tenzij bij processoren die taken uitvoeren waarbij deze bewerking zeer regelmatig zou voorkomen). We zullen dan ook uitgaan van een algemene instructieset die de optelling (\termen{Add-instructie}), aftrekking (\termen{Sub-instructie}) en vermenigvuldiging (\termen{Mul-instructie}) voorziet.
\paragraph{}
Alvorens we de instructiesets met elkaar kunnen vergelijken, zullen we eerst enkele aannames moeten maken over hoe gegevens en instructies kunnen worden ingelezen. We zullen uitgaan van een woordlengte\footnote{De woordlengte is het aantal bits die in een geheugen onder \'e\'en adres worden opgeslagen.} $w$. We maken de assumptie dat een instructie zonder geheugenadressen in \'e\'en woord\footnote{Een woord is een sequentie van $w$ bits met $w$ de woordlengte.} kan worden opgeslagen. Het geheugen omvat $2^w$ adressen, bijgevolg telt het geheugen $w\cdot 2^w$ bits en is elk adres voor te stellen met een woord. Een instructie met $k$ geheugenadressen kan dus worden opgeslagen in $k+1$ woorden. We vergelijken de instructiesets op basis van geheugentoegangen. Dit zijn dus het aantal toegangen om de instructie uit te lezen samen met het uitlezen en wegschrijven van gegevens die in de instructie worden gespecificeerd.
\subsubsection{Instructies met 3 adresvelden}
Een instructieset met drie adresvelden bepaald meestal \'e\'en adres voor het resultaat en twee adressen voor de operanden. Bijvoorbeeld de \verb+Add a b c+ instructie berekend de optelling van de gegeven die op de geheugenplaatsen $b$ en $c$ staan en plaatst het resultaat dus in adres $a$. Om $\brak{a+b}\times\brak{a-b}$ dus uit te rekenen zullen we volgend programma uitvoeren:
\begin{verbatim}
Add c a b
Sub x a b
Mul c c x
\end{verbatim}
Per instructie voorzien we dus $4+3$ geheugentoegangen. $4$ instructietoegangen per instructie $2$ leesoperaties voor de operanden en $1$ schrijfoperatie. Omdat we $3$ instructies uitvoeren vereist het programma dus $21$ geheugentoegangen. We kunnen echter opmerken dat in de laatste instructies we tweemaal hetzelfde adres vermelden. Dit soort instructies vormen dan ook de argumentatie om soms instructies met twee adresvelden te gebruiken.
\subsubsection{Instructies met 2 adresvelden}
Instructiesets met twee adresvelden zijn vrij populair. \verb+80x86+ is een voorbeeld van zo'n instructieset. In zo'n instructieset vertegenwoordigen het eerste en tweede adresveld de adressen van de operanden en het eerste veld ook het adres van het resultaat. Het eerste adres wordt bijgevolg overschreven.
\paragraph{}
Een probleem bij dit mechanisme is dat we soms na de bewerking de operanden willen kunnen hergebruiken om andere operaties uit te voeren. Zo willen we na het berekenen van $a+b$ nog over $a$ en $b$ kunnen beschikken om $a-b$ uit te rekenen. We kunnen de waarde kopi\"eren om dit probleem te vermijden. We kunnen de waarde van $x$ kopi\"eren naar adres $y$ met behulp van twee instructies: \verb+Sub y y+ en \verb+Add y x+. De meeste processoren voorzien echter een kopieer instructie: de \termen{Mov-instructie}. Het voordeel van deze instructie is dat naast het ophalen van de instructie slechts twee geheugentoegangen vereist zijn.
\paragraph{}
We kunnen het algoritme dan ook als volgt implementeren:
\begin{verbatim}
Mov c a
Add c b
Mov x a
Sub x b
Mul c x
\end{verbatim}
De Mov-instructie vereist in totaal $5$ geheugentoegangen, de overige instructies vereisen $6$ geheugentoegangen. In totaal vereist het uitvoeren van het algoritme dus $28$ geheugentoegangen. We dienen echter wel op te merken dat de instructies sneller zullen worden uitgevoerd.
\subsubsection{Instructies met 1 adresvelden}
We kunnen de instructieset verder reduceren tot \'e\'en adresveld per instructie. Dit doen we met behulp van een \termen{accumulator (ACC)}. Een accumulator is een speciaal register die dienst doet als zowel de eerste operand en het register waar het resultaat in wordt geplaatst. Men kan deze instructieset dus vergelijken met de eerste instructieset, maar waarbij de eerste operator altijd een vast adres voorstelt. Het voordeel van een accumulator is de implementatie door middel van een register: de accumulator inlezen of resultaten wegschrijven vereist bijgevolg geen geheugentoegang.
\paragraph{}
Ook wanneer we een accumulator gebruiken zullen we soms tussenresultaten tijdelijk in een andere variabele willen opslaan om de waarde later in te lezen. We kunnen hier geen gebruik maken van de Mov-instructie omdat het resultaat altijd vast staat: de accumulator. Daarom worden twee nieuwe instructies ge\"introduceerd: de \termen{Load-instructie} leest het adres uit en plaatst de waarde in de accumulator, de \termen{Store-instructie} schrijft de waarde van de accumulator weg in het opgegeven adres. De Store-instructie is bijgevolg een instructie waar de accumulator geen dienst doet als de ontvanger van het resultaat.
\paragraph{}
We kunnen het algoritme realiseren met volgende code:
\begin{verbatim}
Load a
Add b
Store x
Load a
Sub b
Mul x
Store c
\end{verbatim}
Elke instructie vereist telkens $3$ geheugentoegangen: $2$ om de instructie uit te lezen en $1$ geheugentoegang om het adres uit te lezen of de resultaten weg te schrijven. Omdat het algoritme in $7$ instructies kan worden ge\"implementeerd, zijn er in totaal $21$ geheugentoegangen vereist.
\subsubsection{Instructies met 0 adresvelden}
\section{Processorontwerp}
\subsection{Complex Instruction Set Computer (CISC)}
\subsection{Reduced Instruction Set Computer (RISC)}
%\part{Very High Speed Integrated Circuit Hardware Description Language}
%\chapter{VHDL}
\chapterquote{In redelijke taal weerklinkt wat in werkelijkheid gebeurt.}{Gerard Bolland, Nederlands taalkundige en filosoof (1854-1922)}
\minitoc[n]
\section{Elementen van de VHDL-taal}
In deze sectie zullen we eerst de woordenschat in de VHDL taal bespreken, waarna we de belangrijkste types zullen bespreken samen met het 
\subsection{Lexicale elementen (woordenschat)}
Elk code-bestand\footnote{Onafhankelijk van de taal.} bestaat uit een sequentie van lexicale elementen die men \termen{tokens} noemt. We zullen eerst de verschillende lexicale elementen die in VHDL-code kunnen voorkomen beschrijven. Deze kunnen we dan als terminologie gebruiken. De tokens kunnen verder worden onderverdeeld in: \termen{literals} (deze vertegenwoordigen bepaalde waardes zoals getallen, tekst,...), \termen{identifiers} (dit zijn namen van variabelen, functies, componenten,... gebruikt om naar te refereren en defini\"eren), \termen{scheidingstekens}\footnote{Engels: \emph{delimiters}.} (haakjes en andere symbolen die waardes en identifiers combineren) en \termen{commentaar}. We beschrijven de verschillende soorten tokens verder in de volgende subsecties.

\subsection{Literals}
We onderscheiden vier soorten literals: getallen, karakters, karakterreeksen en bitreeksen. We overlopen de verschillende types. In VHDL is er ook nog sprake van \vhdltermen{enumeration literals} en van de \vhdltermen{NULL literal}, deze literals worden niet beschouwd.

\paragraph{Getal}Een getal (ook wel ``\vhdltermen{abstract literal}'' genoemd) is een numerieke waarde. Net als bij programmeertalen onderscheid men verschillende soorten numerieke waarden:
\begin{itemize}
  \item \vhdltermen{universal\_integer}: dit zijn numerieke literals die geen decimale komma (\vhdltermen{.}) bevatten, bijvoorbeeld \verb+1425+. Deze getallen zijn vergelijkbaar met het \texttt{int}-type in \texttt{Java}\footnote{De vergelijking gaat echter niet geheel op: een \texttt{int} is immers beperkt tot $32$ bit en heeft een specifieke representatie.}.
  \item \vhdltermen{universal\_real}: dit zijn numerieke getallen die wel een decimale komma bevatten. Bijvoorbeeld \verb+1425.1917+. Merk op dat indien er enkel nullen achter de komma staan dat het getal dezelfde waarde heeft als een \vhdltermen{universal\_integer}: \verb+1425.0+, maar daarom niet dezelfde representatie. In \texttt{Java} zouden we dit type voorstellen met bijvoorbeeld een \texttt{float} of \texttt{double}.
  \item \vhdltermen{physical types}: dit zijn fysische waarden. Fysische waarden bestaan uit een getal gevolgd door een fysische eenheid. Er wordt een spatie tussen het getal en de eenheid gezet. Bijvoorbeeld: \verb+1818 ns+. Dit type heeft geen echt equivalent in de meeste programmeertalen. Immers redeneren programmeertalen meestal niet over grootheden.
\end{itemize}
Getallen worden standaard in decimale notatie uitgedrukt. Men kan ze ook in exponenti\"ele vorm schrijven door het toevoegen van een '\vhdltermen{E}' of `\vhdltermen{e}' op het einde van het getal gevolgd door de exponent, bijvoorbeeld: \verb+14e2+. Om een andere basis te gebruiken, wordt volgend formaat gebruikt: \vhdltermen{base\#literal\#exp}, met de basis tussen 2 en 16. Zo kunnen we $1425$ schrijven als \verb+16#591#+, of $4864$ als \verb+16#13#E2+. Tot slot mag men ook vrij \emph{underscores} (\vhdltermen{\_}) toevoegen in een getal om de code leesbaar te houden. Bijvoorbeeld: \verb+19_171_425+.


\paragraph{Karakter} Een karakter (ofwel ``\vhdltermen{character literal}'') slaat een \termen{karakter} (ofwel \termen{character}) op: de eenheid van tekst. Deze literals worden tussen enkele aanhalingstekens (\vhdltermen{'}) geplaatst, bijvoorbeeld \verb+'K'+. In \texttt{Java} zou het equivalent van dit type een \texttt{char} zijn.

\paragraph{Karakterreeks}
Een \termen{karakterreeks} (ofwel ``\vhdltermen{string literal}'') slaat een sequentie karakters op, dus een tekstfragment. In \texttt{Java} noemt men een dergelijk type een \texttt{String}. Deze reeks wordt tussen dubbele aanhalingstekens (\vhdltermen{"}) geplaatst, bijvoorbeeld \texttt{"Hello World!"}. Een probleem doet zich voor wanneer we de dubbele aanhalingstekens (\texttt{"}) zelf in het tekstfragment willen gebruiken. In dat geval dienen we de aanhalingstekens dubbel te plaatsen, bijvoorbeeld:
\begin{quote}\verb+"""The answer?"" said Deep Thought. ""The answer to what?"""+\cite[\S25]{Adams81BOOK54}\end{quote}
\paragraph{Bitreeksen}
Een \vhdltermen{bitreeks} (ofwel ``\vhdltermen{bit string literal}'') is een sequentie aan bits (dit is een eenheid van informatie die ofwel \texttt{0} ofwel \texttt{1} is). Een bitreeks wordt geschreven als een sequentie van enen en nullen tussen dubbele aanhalingstekens (\vhdltermen{"}). Zoals we al vroeger in deze cursus hebben opgemerkt is zo'n sequentie niet effici\"ent, leesbaar en praktisch. Daarom laat men ook octale  en hexadecimale notatie toe. Hiertoe wordt er een \vhdltermen{B} (binair), \vhdltermen{O} (octaal) of \vhdltermen{X} (hexadecimaal) voor de dubbele aanhalingstekens geplaatst om de bitreeks in een bepaald formaat te lezen. Bij hexadecimale getallen gebruikt men \vhdltermen{a}, \vhdltermen{b}, \vhdltermen{c}, \vhdltermen{d}, \vhdltermen{e} en \vhdltermen{f} om respectievelijk $10$ tot $15$ voor te stellen. Net als bij getallen kan de gebruiker \termen{underscores} (\vhdltermen{\_}) toevoegen om tot meer leesbare code te komen.
\subsection{Identifiers}
Een \vhdltermen{identifier} is een referentie naar een variabele, functie, component,... Dit is vergelijkbaar met de naam van een variabele, methode, klasse,... in \texttt{Java}. Identifiers beginnen met een letter en mogen letters en cijfers bevatten. Ook de underscore (\vhdltermen{\_}) teken is toegelaten, indien dit niet het eerste of laatste karakter van de identifier vormt. Identifiers zijn niet hoofdletter gevoelig. In VHDL'93 wordt het begrip van een identifier verder uitgebreid. Deze uitbreiding wordt hier niet beschouwd. Voor meer details \cite[p. 4]{hardi00}.
\paragraph{Gereserveerde woorden}Niet alle namen die aan de hierboven beschreven regels mag men zelf gebruiken. Sommige identifiers zijn immers \termen{sleutelwoorden} die een reeds ingebouwde functie vevullen in VHDL: ze verwijzen naar functies die in de VHDL compiler zijn ingebakken. De woorden in \tblref{vhdl-reserved-words} mogen niet gebruikt worden als identifiers.

\importtabulartable{vhdl-reserved-words}{Gereserveerde woorden in VHDL.}

We onderscheiden verschillende soorten sleutelwoorden: enerzijds zijn er compiler-directieven zoals bijvoorbeeld \vhdltermen{case} en \vhdltermen{downto}, daarnaast zijn er basis-operaties zoals \vhdltermen{nand} en \vhdltermen{sla}, tot slot zijn er ook type-directieven zoals \vhdltermen{array} en \vhdltermen{signal}. Sommige gereserveerde worden werden pas ge\"introduceerd in VHDL'93, we gaan hier niet verder op in.

\subsection{scheidingstekens}
Een \vhdltermen{scheidingsteken} (ofwel ``\vhdltermen{delimiter}'') tenslotte wordt gebruikt om operaties op gegevens uit te voeren, bijvoorbeeld een optelling maar ook een index. De
scheidingstekens van VHDL worden weergegeven in \tblref{vhdl-delim}.

\importtabulartable{vhdl-delim}{Scheidingstekens in VHDL.}

We onderscheiden opnieuw verschillende soorten scheidingstekens: tekens die operaties voorstellen (zoals \texttt{+} en \texttt{*}), tekens die voorwaarden voorstellen (zoals \texttt{<>}), tekens die tekst afbakenen (zoals \texttt{(} en \texttt{[}) en tekens die aangeven dat twee fragmenten los van elkaar staan (zoals \texttt{,}).

\subsection{Commentaar}
Commentaar zijn delen van de code die genegeerd worden door de VHDL-compiler, maar die nuttig zijn voor programmeurs: ze bevatten gegevens over het project en richtlijnen die een belangrijke rol kunnen hebben tijdens het project. Commentaar plaatst men na twee horizontale strepen \verb+--+ en reikt tot het einde van de lijn. Dit is vergelijkbaar met de dubbele slash (\verb+//+) in \texttt{Java}.

\section{Typesysteem}
VHDL is getypeerd: een variabele slaat data op volgens een bepaald type, dit type geeft een interpretatie aan zowel de data en de operatoren die erop gedefinieerd zijn. De types kunnen bovendien door de programmeur uitgebreid worden. Men vertrekt echter steeds van basistypes.

\subsection{Voorgedefinieerde types}
Ook in de VHDL-secties werden sommige van deze types reeds gebruikt. \tblref{vhdl-type} geeft een overzicht van de meest populaire basistypes.
\begin{table}[hbt]
\centering
\importtabularsubtable{vhdl-type}{Overzicht van belangrijke types in VHDL.}
\importtabularsubtable{vhdl-type-deriv}{Overzicht van belangrijke afgeleide types in VHDL.}
\caption{Overzicht van belangrijke types en afgeleide types in VHDL.}
\end{table}

\paragraph{}
Naast deze reeks van basistypes bevat VHDL ook standaard enkele afgeleide types. Het zijn types die een gereduceerd bereik uit de basistypes vertegenwoordigen. Deze subtypes staan in \tblref{vhdl-type-deriv} samen met hun gereduceerd bereik.

\subsection{Zelf types defini\"eren}
Hoe defini\"eren we nu zelf types? Afhankelijk van het soort type dat we willen bouwen zijn er verschillende mogelijkheden:
\begin{itemize}
 \item Defini\"eren door opsomming;
 \item Defini\"eren door subtypering (beperk het bereik);
 \item Defini\"eren van fysische types;
 \item Afgeleide types met matrices en vectoren.
\end{itemize}

We bespreken de verschillende methodes nu in meer detail.

\subsubsection{Defini\"eren met opsomming}
We kunnen een type specificeren door alle mogelijke toestanden op te sommen. Deze methode wordt toegepast voor bijvoorbeeld de types \vhdltermen{bit} en \vhdltermen{byte}. De namen of karakters die de toestand bepalen worden tussen haakjes opgesomd na het sleutelwoord \vhdltermen{is}. Dit doen we na het sleutelwoord \vhdltermen{type} en de identifier voor het type. \vhdlref{type-enum} toont mogelijke definities van \vhdltermen{bit} en \vhdltermen{byte}. We defini\"eren ook het type \texttt{StateMachine}, dit type bevat drie elementen: \texttt{start}, \texttt{processing} en \texttt{wait} die de "toestanden" weergeven van een schakeling die we bijvoorbeeld zouden kunnen willen bouwen.

\importvhdl{type-enum}{Defini\"eren van types door opsomming.}


\subsubsection{Defini\"eren met subtypering}

Soms wensen we een type te specificeren die een deelverzameling omvat van een reeds eerder gespecificeerd type. Zo zal een \vhdltermen{integer} alle waarden omvatten die met $32$ bit voor te stellen zijn, het is echter mogelijk dat we bijvoorbeeld een type \texttt{byte} willen specificeren die uitsluitend numerieke waarden bevat die we op $8$ bit kunnen voorstellen. In dat geval gebruiken we het sleutelwoord \vhdltermen{subtype} gevolgd door een identifier die de naam van het type aanduidt, daarna volgt opnieuw het sleutelwoord \vhdltermen{is} en het "supertype". VHDL kent twee soorten beperkingen die hierop kunnen volgen: de "range beperking" en de "index beperking".

\paragraph{}
Een "range beperking" omvat twee literals die het begin en het einde markeren van een bereik van waarden. Vermits alle basistypes in VHDL een inherente orde-relatie hebben\footnote{Voor karakters maakt men gebruik van de ASCII-code.}

\begin{vhdlcode}[hbt]
\begin{lstlisting}
subtype byte is integer range 0 to 255;
subtype lowercase is character range 'a' to 'z';
\end{lstlisting}
\caption{Defini\"eren van types door subtypering.}
\vhdllab{constante}
\end{vhdlcode}

\begin{vhdlcode}[hbt]
\begin{lstlisting}
type bit is ('0','1');
type boolean is (false,true);
\end{lstlisting}
\caption{Defini\"eren van fysische types.}
\vhdllab{constante}
\end{vhdlcode}
\subsubsection{Data-objecten}
Een object in VHDL is een benoemd item met een specifieke waarde. We onderscheiden 3 soorten objecten: \vhdltermen{constante}, \vhdltermen{variabele} en \vhdltermen{signaal}.
\paragraph{Constante}Een constante is een object die slechts \'e\'enmaal een waarde kan toegekend worden. Ze geven een interpretatie aan de waarde, en maken de VHDL-code daarom beter verstaanbaar en aanpasbaar. We gebruiken hiervoor het sleutelwoord \vhdltermen{constant}. \vhdlref{constante} toont de declaratie van verschillende constanten. Merk op dat VHDL niet hoofdlettergevoelig is in programma-syntax. Alleen de waarde die in karakters en karakterreeksen wordt opgeslagen zijn hoofdlettergevoelig.
\begin{vhdlcode}[hbt]
\begin{lstlisting}
constant pi : real := 3.14159265;
constant byte_length : natural := 8;
constant word_length : natural := 4*byte_length;
\end{lstlisting}
\caption{Werken met constanten.}
\vhdllab{constante}
\end{vhdlcode}
Het algemene formaat is dus \verb+constant <identifier> : <type> := <waarde>;+.
\paragraph{Variabele}De syntax van een variabele is ongeveer dezelfde, alleen wordt het sleutelwoord \vhdltermen{variable} gebruikt. Verder is het onmiddellijk toekennen van een waarde optioneel. \vhdlref{variabele} toont het gebruik van variabelen in VHDL.
\begin{vhdlcode}[hbt]
\begin{lstlisting}
variable index : integer;
index := 0;
index := index + 1;
variable andere_index : natural := 12;
variable antwoord : natural := 4*andere_index-6*index;
\end{lstlisting}
\caption{Werken met variabelen.}
\vhdllab{variabele}
\end{vhdlcode}
\paragraph{Signaal}Een signaal ten slotte is het VHDL-equivalent van een fysische verbinding of een groep van verbindingen in de hardware. Een signaal wordt geconstrueerd met behulp van het sleutelwoord \vhdltermen{signal}. Toewijzingen aan een signaal gebeuren met de \vhdltermen{<=} operator. De toewijzing bij een signaal werkt anders dan bij een variabele en constante. Variabelen en constanten worden onmiddellijk toegewezen. Dit betekent dat expressies als \verb/index := index + 1;/ mogelijk zijn. Signalen veranderen echter wanneer elementen uit hun toewijzing veranderen. Beschouw het voorbeeld uit \vhdlref{signal}.
\begin{vhdlcode}[hbt]
\begin{lstlisting}
signal a : bit;
signal b : bit;
signal y : bit;
a <= '1';
b <= '0', '1' after 100 ns;
y <= a and b;
\end{lstlisting}
\caption{Werken met signalen.}
\vhdllab{signal}
\end{vhdlcode}
We zien drie signalen \verb+a+, \verb+b+ en \verb+y+. \verb+b+ zal na $100\mbox{ ns}$ een 1 aanleggen. Dit heeft als implicatie dat ook \verb+y+ van waarde zal veranderen. Indien we dit met variabelen en constanten zouden hebben berekend zou variabele \verb+y+ na de toewijzing een waarde toegewezen krijgen, en deze tot een volgende toewijzing behouden. Merk verder op dat we hier ook de betekenis van het codewoord \vhdltermen{after} tonen. Een signaal kan ook een initi\"ele waarde krijgen, hiervoor gebruiken we de \vhdltermen{:=} operator.
\subsubsection{Bibliotheken}
\subsubsection{Bewerkingen}
\section{Combinatorische schakelingen in VHDL}
\label{s:combinatorischVHDL}
Nu we verschillende schakelingen hebben gebouwd zullen we deze proberen te beschrijven met VHDL. In deze sectie zullen we eerst een formeel overzicht geven van de VHDL-syntax. Vervolgens zullen we in subsectie \ref{ss:combinatorischVHDLHardware} een methode ontwikkelen om combinatorische elementen te beschrijven met deze syntax. In elk hoofdstuk zullen we de methodologie uitbreiden zodat we de nieuwe componenten ook kunnen beschrijven.
\subsection{Hardwarebeschrijving met VHDL}
\subsubsection{Structurele beschrijving}
\begin{vhdlcode}[hbt]
\centering
\begin{lstlisting}
-- 2-naar-1 Multiplexer
--
architecture Struct of MUX2 is
  signal U,V,W : bit;
  component AND2 is
    port (X,Y: in bit;
          Z: out bit);
  end component AND2;
  component OR2 is
    port (X,Y: in bit;
          Z: out bit);
  end component OR2;
  component INV is
    port (X: in bit;
          Z: out bit);
    end component INV;
begin
  Gate1: component INV  port map (X=>S,Z=>U);
  Gate2: component AND2 port map (X=>A,Y=>S,Z=>W);
  Gate3: component AND2 port map (U,B,V);
  Gate4: component OR2  port map (X=>W,Y=>V,Z=>Y);
end Struct;
\end{lstlisting}
\caption{2-naar-1-multiplexer.}
\label{vhdl:bToAMultiplexer}
\end{vhdlcode}
\subsubsection{Gedragsbeschrijving}
\begin{vhdlcode}[hbt]
\centering
\begin{lstlisting}
-- Opteller
--
library ieee;
use ieee.std_logic_signed.all;

entity adder is
  generic(n : positive := 4);
  port(Cin : in std_logic;
       X,Y : in std_logic_vector(n-1 downto 0);
       S   : in std_logic_vector(n-1 downto 0);
       Cout, Overflow : out std_logic);
end adder;

architecture behav of adder is
  signal Sum : std_logic_vector(n downto 0);
begin
  Sum <= (X(n-1) & X) + Y + Cin;
  S <= Sum(n-1 downto 0);
  Cout <= Sum(n);
  Overflow <= Sum(n) xor X(n-1) xor Y(n-1) xor Sum(n-1);
end behav;
\end{lstlisting}
\caption{$n$-bit Opteller.}
\label{vhdl:adder}
\end{vhdlcode}
\subsubsection{Repetitieve structuren}
\label{ss:combinatorischVHDLHardware}
%\appendix
%\part{Appendices}
%\input{appendix_poorten_en_componenten}
%\chapter{Softwarepakketten}
\applab{software}
\chapterquote{Mensen die bezig zijn met software, zouden hun eigen hardware moeten bouwen.}{Alan Kay, Amerikaans informaticus (1940-)}
\begin{chapterintro}
Bij het schrijven van de cursus Digitale Elektronica en Processoren werden enkele softwarepakketten geschreven. Deze software laat toe aan de lezer om zelf oefeningen te maken, oplossingen te controleren en op een geautomatiseerde manier kleine combinatorische en sequenti\"ele te realiseren alsook een processor te bouwen. In dit hoofdstuk geven we een kort overzicht van deze software. Voor het \emph{Linux} besturingssysteem bestaat ook software die helpt bij het concreet realiseren van een schakeling op bijvoorbeeld een printplaat of het simuleren van een schakeling. Deze software wordt kort besproken in enkele secties.
\end{chapterintro}
\minitoc[n]
\section{Geschreven software}
\subsection{De software installeren}
De software is geschreven in Haskell en kan worden gedownload op volgend adres: \texttt{http://goo.gl/tkyilf}. De code staat onder git-subversiebeheer. Bijdragen aan de software wordt aangemoedigd. Men kan de software gebruiken door de \texttt{Makefile} te draaien. Dit doet men door in de desbetreffende map het commando \texttt{make} te typen. Standaard wordt de software niet ge\"installeerd op het systeem: de commando's kunnen alleen in de map zelf uitgevoerd worden. Indien men de programma's in om het even welke map wenst uit te voeren, kan men \texttt{make install} draaien.
\subsection{Invoer en Uitvoer}
Het programma neemt tekst als invoer en geeft tekst of diagrammen als uitvoer. We overlopen in de volgende subsubsecties de verschillende vormen van invoer.
\subsubsection{Bit}
Een bit is een logische waarde waarmee doorheen het volledige programma wordt gerekend. Een bit kent drie toestanden: waar (\texttt{1}), onwaar (\texttt{0}) en don't care (\texttt{-}). Het programma is echter in staat op verschillende manieren een bit te lezen. Zo stellen \texttt{t}, \texttt{T} en \texttt{1} alle drie waar voor; \texttt{f}, \texttt{F} en \texttt{0} duiden onwaar aan; en \texttt{d}, \texttt{D}, \texttt{x}, \texttt{X}, \texttt{-} betekenen allemaal don't care.
\subsubsection{Bitstring}
Meestal beperkt een toestand, invoer en uitvoer zich niet tot \'e\'en bit. Een bitstring is een sequentie van nul of meer bits. Men schrijft een bitstring eenvoudigweg aan de hand van een sequentie van voorstellingen voor bits zonder spaties of andere tekens. Een voorbeeld is \texttt{t-TX1xfDFd0} een geldige representatie van een bitstring.
\subsubsection{Tabel}
Een tabel is een twee dimensionale structuur. Een tabel is onderverdeeld in cellen. Een horizontale groep cellen noemt men een rij, een verticale groep een kolom. De bovenste rij noemt men doorgaans de hoofding en verklaart meestal de inhoud van de cellen eronder.
\paragraph{}
Cellen worden verticaal opgedeeld aan de hand van een verticaal streepje (\texttt{|}), horizontaal worden ze van elkaar onderscheiden door een nieuwe lijn. Optioneel kan men tussen twee lijnen ook een reeks streepjes (\texttt{-}) zetten, optioneel aangevuld met plus (\texttt{+}) en asterisk (\texttt{*}), een dergelijke lijn wordt eenvoudigweg genegeerd. De verticale streepjes hoeven niet op elkaar uitgelijnd te zijn: men kan bijvoorbeeld in de ene rij het eerste verticale streepje op positie $2$ zetten terwijl dit in de lijn erna op positie $20$ staat, maar het wordt toch aangeraden consistent te zijn.
\paragraph{Voorbeeld}
Een voorbeeld van een tabel is de volgende code:
\begin{verbatim}
dit|   is     | een | 001001-DXXD
-----------------------------------
geldige| 01111 | Tabel | a/0010 | Ook
  al     | is  | de    | tabel  | 111
---------+-----+-------+--------+----
moeilijk|  leesbaar |  |        |
\end{verbatim}
\paragraph{Soorten tabellen}
Een tabel bevat data die door het programma verwerkt kan worden. Om een tabel te kunnen verwerken dient deze niet alleen syntax-matig in orde te zijn. Het is ook belangrijk wat er waar in de tabellen staat. Hieronder geven we een kort overzicht van de verschillende types tabellen:
\begin{enumerate}
 \item \emph{Toestandstabel}: In de eerste kolom staan vanaf de tweede rij toestanden (tekst die geen bitstring voorstelt). Toestandstabellen komen verder in twee gedaantes voor: de \emph{Moore-toestandstabel} en de \emph{Mealy-toestandstabel}.
 \item \emph{Moore-toestandstabel}: Dit is een toestandstabel waarbij vanaf de tweede rij, de tweede tot en met de voorlaatste uit toestanden (tekst bestaat). In deze tabel staan in de eerste rij vanaf de $2$-de tot en met de voorlaatste kolom bitstrings die invoer weergeven. De toestanden moeten ook gedefinieerd zijn ergens in een rij in de eerste kolom. In de laatste kolom staan vanaf de tweede rij bitstrings.
 \item \emph{Mealy-toestandstabel}: Dit is een toestandstabel waarbij vanaf de tweede rij, de tweede tot en met de laatste uit toestanden (tekst bestaat) gepaard met bitstrings. De toestand en de bitstring worden van elkaar onderscheiden door middel van een slash (\texttt{/}). De toestanden moeten ook gedefinieerd zijn ergens in een rij in de eerste kolom.
 \item \emph{Coderingstabel}: In de eerste kolom staan vanaf de tweede rij bitstrings. Coderingstabellen komen verder in twee gedaantes voor: de \emph{Moore-coderingstabel} en de \emph{Mealy-coderingstabel}.
\end{enumerate}
\subsubsection{Schakeling}
Een schakeling beschrijft op poort-niveau, en soms op hoger niveau hoe verschillende componenten gegevens uitwisselen. Om het programma eenvoudig te houden werd niet geopteerd voor een grafische schil, maar voor tekstuele invoer. Een schakeling bestaat daarom uit twee delen:
\begin{enumerate}
 \item 
\end{enumerate}
\subsubsection{Karnaugh-kaart}
Een Karnaugh-kaart is een grafische voorstelling van een booleaanse-functie. Karnaugh-kaarten worden enkel geproduceerd als uitvoer.
\subsubsection{Assembler code}
Assembler-code is een datastructuur die is afgeleid uit het hoofdstuk rond programmeerbare processoren (zie \chpref{programmableprocessors}).
\subsection{Ondersteunde functies}
Volgende instructies worden ondersteund. Het is de bedoeling om het programma \texttt{dep} op te roepen met de instructie, bijvoorbeeld \texttt{dep showKarnaugh}. Men kan elke instructie ook aanroepen met de \texttt{-help} parameter om een overzicht te krijgen van wat de functie precies doet. Typische parameters zijn \texttt{-html}, \texttt{-latex} en \texttt{-svg} om de uitvoer niet aan de hand van ASCII-tekens op de \texttt{stdout} te regelen.
\begin{enumerate}
 \item \texttt{showKarnaugh}: toont \'e\'en of meerdere Karnaugh-kaarten aan de hand van een ingegeven tabel. De kaarten worden standaard op de \texttt{stdout} geschreven. Men kan ook gebruik maken van de \texttt{-svg} optie om een grafische uitvoer te genereren.
\end{enumerate}

%\chapter{Oplossingen van de Oefeningen}
\section{Hoofdstuk 1}
\section{Hoofdstuk 2}
\section{Hoofdstuk 3}
\subsection{Karnaugh-kaarten}
\subsubsection{Invullen van Karnaugh-kaarten}
\begin{figure}[hbt]
\centering
\begin{tikzpicture}
\foreach \x in {-1,0,1} {
  \foreach \y in {-1,0,1} {
    \draw[thick] (-1+3*\x,-1+3*\y) rectangle ++(2,2);
    \draw[thick] (3*\x-0.5,3*\y-1) -- ++(0,2);
    \draw[thick] (3*\x,3*\y-1) -- ++(0,2);
    \draw[thick] (3*\x+0.5,3*\y-1) -- ++(0,2);
    \draw[thick] (3*\x-1,3*\y-0.5) -- ++(2,0);
    \draw[thick] (3*\x-1,3*\y) -- ++(2,0);
    \draw[thick] (3*\x-1,3*\y+0.5) -- ++(2,0);
  }
  \draw[thick] (-4.125,3*\x) to node[midway,below,sloped]{$w$} (-4.125,3*\x-1);
  \draw[thick] (4.125,3*\x+0.5) to node[midway,above,sloped]{$x$} (4.125,3*\x-0.5);
  \draw[thick] (3*\x-0.5,4.125) to node[midway,above,sloped]{$y$} (3*\x+0.5,4.125);
  \draw[thick] (3*\x,-4.125) to node[midway,below,sloped]{$z$} (3*\x+1,-4.125);
}
\foreach \t/\x/\y/\va/\vb/\vc/\vd/\ve/\vf/\vg/\vh/\vi/\vj/\vk/\vl/\vm/\vn/\vo/\vp in {a/-1/-1/0/0/0/0/1/1/1/1/1/1/1/1/1/1/1/1, b/0/-1/0/1/1/1/0/0/0/0/1/1/1/1/0/1/1/1, c/1/-1/1/1/1/0/0/1/1/1/-/-/-/-/1/-/-/1, d/-1/0/1/1/1/1/1/0/1/0/-/-/-/-/1/-/-/1, e/0/0/1/0/1/1/1/1/1/1/-/-/-/-/1/-/-/1, f/1/0/1/1/1/0/0/1/0/1/-/-/-/-/1/-/-/1, g/-1/1/1/1/0/0/0/1/0/0/-/-/-/-/1/-/-/0, h/0/1/1/0/0/0/1/1/0/1/-/-/-/-/1/-/-/1, i/1/1/0/1/1/0/1/1/0/1/-/-/-/-/1/-/-/1} {
  \draw[thick] (3*\x-1,-3*\y+1) -- ++(-0.25,0.25) node[anchor=south east]{$\t$};
  \draw (3*\x-0.75,-3*\y+0.75) node{\va};
  \draw (3*\x-0.25,-3*\y+0.75) node{\vb};
  \draw (3*\x+0.25,-3*\y+0.75) node{\vc};
  \draw (3*\x+0.75,-3*\y+0.75) node{\vd};
  \draw (3*\x-0.75,-3*\y+0.25) node{\ve};
  \draw (3*\x-0.25,-3*\y+0.25) node{\vf};
  \draw (3*\x+0.25,-3*\y+0.25) node{\vg};
  \draw (3*\x+0.75,-3*\y+0.25) node{\vh};
  \draw (3*\x-0.75,-3*\y-0.25) node{\vi};
  \draw (3*\x-0.25,-3*\y-0.25) node{\vj};
  \draw (3*\x+0.25,-3*\y-0.25) node{\vk};
  \draw (3*\x+0.75,-3*\y-0.25) node{\vl};
  \draw (3*\x-0.75,-3*\y-0.75) node{\vm};
  \draw (3*\x-0.25,-3*\y-0.75) node{\vn};
  \draw (3*\x+0.25,-3*\y-0.75) node{\vo};
  \draw (3*\x+0.75,-3*\y-0.75) node{\vp};
}
\end{tikzpicture}
\caption{Oplossingen van het invullen van de Karnaugh-kaarten.}
\label{fig:apxKKaartenFill}
\end{figure}

\backmatter
%\printbibliography
\listoftables
\listoffigures
\listof{vhdlcode}{Lijst van VHDL-Codes}
\begin{twocolumn}
\nocite{*}
\bibliographystyle{alpha}
\bibliography{bibliography}
\label{reference}
\end{twocolumn}
\label{glos}
\printglossaries
\label{idx}
\printindex
% \chapter*{``And Now For Something Completely Different''}
% \begin{figure}[H]
% \centering
% \begin{tikzpicture}[scale=15]
% \def\h{1};
% \def\t{0.02};
% \def\ts{10};
% \fill (0,0) rectangle ++(-\t,\h);
% \fill (-0.5*\t,0.5*\h-0.5*\t) rectangle ++(0.5*\h+1.5*\t,\t);
% \fill (\h,0) arc (270:90:0.5*\h) -- (\h,\h-\t) arc (90:270:0.5*\h-\t) -- (\h,0);
% \draw (0.75*\h,0.5*\h) node[scale=\ts]{$\mathfrak{K}$};
% \foreach\s/\n/\tx in {0.5/1/S,0.25/3/O,0.125/7/O,0.0625/15/M,0.03125/31/F,0.015625/63/M,0.0078125/127/T,0.00390625/255/U} {
%   \foreach \y in {0,...,\n} {
%     \begin{scope}[yshift=\y*\s*\h cm,scale=\s]
%     \fill (-0.5*\t,0.5*\h-0.5*\t) rectangle ++(0.5*\h+1.5*\t,\t);
%     \fill (\h,0) arc (270:90:0.5*\h) -- (\h,\h-\t) arc (90:270:0.5*\h-\t) -- (\h,0);
%     \draw (0.75*\h,0.5*\h) node[scale=\ts*\s]{$\mathfrak{\tx}$};
%     \end{scope}
%   }
% }
% \end{tikzpicture}
% \end{figure}
\end{document}
%Start 19/03/2011
%Chapter 1: 28/03/2011
%Chapter 2:
%Chapter 3:
%Chapter 4:
%Chapter 5:
%Chapter 6: