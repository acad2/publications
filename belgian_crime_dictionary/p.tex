\individual{pardaensgina}{Gina Pardaens}{-mei 1998}{Sociaal maatschappelijk werker. Deze vrouw steunde slachtoffers van een kinderprostitutie-netwerk. Ze overleed na een ongeluk waarbij ze s'nachts met 80 kilometer per uur tegen de railing van een brug knalde op weg naar huis. Het openbaar ministerie stelde een ongeluk vast. Sommige vrienden (zoals \indref{vervloessemmarcel} en \annref{2167}) stellen echter dat er sabotage in het spel is. Enkele dagen voor haar dood schreef ze een brief. Hierin verklaart ze dat ze in het kader van haar werk doodsbedreigingen kreeg en voor haar leven vreesde. Ze vertelde de politie dat ze haar bedreigd hadden met een auto-ongeluk. Vlak voor haar dood vertelde Pardaens enkele vrienden over een videoband van een seksfeest waarop meisjes werden vermoord. Ze herkende \'e\'en van de deelnemers als \indref{nihoulmichel}.\cite{zdf20011028}}
\individual{priomichel}{Michel Piro}{×}{Piro was een bordeel- en nachtclub-bezitter. Drie maanden na de arrestatie van \indref{dutrouxmarc} wordt Piro koelbloedig op een parkeerplaats neergeschoten. Piro is een bekende in de onderwereld van \location{Charleroi}. Kort voor zijn dood neemt hij contact op met de ouders van \indref{lejeunejulie} en \indref{russomelissa} met een verzoek hen te ontmoeten. Hij wilde een avond organiseren om iets te onthullen maar waarover het precies ging werd niet onthult.}
\individual{ponceletguy}{Guy Poncelet}{×}{Oud hoofd van justitie. Vader van \indref{ponceletsimon}. Sinds de dood van zijn zoon probeert hij de \caseref{dutroux} zelf op te lossen. Poncelet is bezorgd dat veel getuigen onder verdachte omstandigheden zijn omgekomen en anderen reeds oud geworden zijn. Hierdoor zijn de getuigenissen minder scherp. Daarom vindt Guy Poncelet dat het proces sneller moest worden gevoerd in de \caseref{dutroux}.\cite{zdf20011028}}
\individual{ponceletsimon}{Simon Poncelet}{1955-1996}{Poncelet was een politieagent die onderzoek deed in de kringen van \indref{dutrouxmarc}. Hij werd tijdens een nachtdienst in zijn bureau neergeschoten. De moord is tot op vandaag niet opgehelderd.\cite{zdf20011028}}
\individual{purshhenriette}{Henriette Push}{×}{Getuige in de \caseref{dutroux}. De getuige was voor de aanvang van het proces reeds overleden.\cite{zdf20011028}}
\individual{pinonvirginie}{Virginie Pinon}{×}{Getuige in de \caseref{dutroux}. De getuige was voor de aanvang van het proces reeds overleden.\cite{zdf20011028}}
