\individual{dutrouxmarc}{Marc Dutroux}{kinderverkrachter en moordenaar}{Wordt in augustus 1996 gearresteerd samen met \indref{lelievremichel} en \indref{martinmichel}. Na twee dagen te zijn verhoord leidt Dutroux de speurders naar een kelder. In deze kelder bevindt zich achter een rek een ruimte van twee vierkante meter groot. In deze ruimte worden \indref{delhezletitia} en \indref{dardennesabine} levend aangetroffen. Sabine bracht 80 dagen in de kelder door. In de volgende dagen worden alle huizen van Dutroux doorzocht. Men vindt de lijken van \indref{lejeunejulie} en \indref{russomelissa}. Er werd reeds 14 maanden naar deze meisjes gezocht. Waarschijnlijk zijn ze in de kelder omgekomen van honger. Ook het huis van een vermoorde medeplichtige van Dutroux wordt onderzocht: hier worden de lijken van \indref{marshalan} en \indref{lambreckxeefje} gevonden. In een ander huis van Dutroux ontdekte men bouwmateriaal en een ondergronds tunnelcomplex.\cite{zdf20011028}}.
\individual{dardennesabine}{Sabine Dardenne}{}{Slachtoffer van \indref{dutrouxmarc}.}
\individual{delhezletitia}{Letitia Delhez}{}{Slachtoffer van \indref{dutrouxmarc}.}
\individual{debaetspatriek}{Patriek De Baets}{×}{Rijkswachtadjudant en ondervrager bij de eerste verhoren van \indref{loufregina}.\cite{xdossiers}}
\individual{deulinjeanine}{Jeanin Deulin}{×}{Moeder van \indref{taminiaujeanpaul}. Deulin getuigt het volgende: \begin{quote}Die vriend die hem een nog een dag voor zijn verdwijning heeft gezien kwam een aantal weken later bij mij en vertelde: ``Voor mij is hij niet dood: hij verstopt zich. Hij is bang voor iets. De politie maakt het hem moeilijk, hij heeft angst.\end{quote} Deulin denkt dat indien er geen priv\'eredenen zijn om haar zoon om te brengen, haar zoon vermoord is door de mensen die \indref{dutrouxmarc} proberen te beschermen.\cite{zdf20011028}}
\individual{dinantmichel}{Michel Dinant}{×}{Getuige in de \caseref{dutroux}. De getuige was voor de aanvang van het proces reeds overleden.\cite{zdf20011028}}
\individual{derijckepierrepol}{Pierre-Pol de Rijcke}{×}{Getuige in de \caseref{dutroux}. De getuige was voor de aanvang van het proces reeds overleden.\cite{zdf20011028}}
\individual{deleuzephilippe}{Philippe Deleuze}{×}{Getuige in de \caseref{dutroux}. De getuige was voor de aanvang van het proces reeds overleden.\cite{zdf20011028}}
\individual{dekaisedaniel}{Daniel Dekaise}{Wapenhandelaar}{Werd op 30 september 1983 overvallen in zijn zaak in \location{Waver}. Vermoedelijk was dit door de \groupref{bendevannijvel}.\cite{panoramabvn95}}
\individual{decrooherman}{Herman Decroo}{×}{Overleefde een bomaanslag die vermoedelijk het werd georganiseerd door de \groupref{cellullescommunistecombatantes}.\cite{dezaakdejachtopdeccc}}
\individual{dislairelucien}{Lucien Dislaire}{paracommando}{\begin{quote}Ik kom uit het noorden van de provincie Luxemburg. Op dat moment was ik manager van een bank en ex-paracommando. Op een dag kwamen enkele mensen bij mij thuis, en vroegen hulp bij het uitvoeren van enkele speciale manoeuvres in co\"ordinatie met Amerikaanse speciale eenheden. De Belgische commando's werd gevraagd om Amerikaanse paratroepers terug te halen die waren geland in het bos. Na deze operatie moesten ze zich naar vooraf bepaalde plaatsen begeven, en de barakken aanvallen van de rijkswacht.\\Ik had de voorzieningen, de wapens en de radio's bij mij, om het allemaal te co\"ordineren.\\Een paar dagen voorheen waren er problemen: de Amerikanen waren te ver gegaan.\end{quote}\cite{timewatchoperationgladiothefootsoldier}}
\individual{dejongherudy}{Rudy De Jonghe}{B.O.B. Leuven}{De Jonghe was een lid van de B.O.B. \location{Leuven} die onder meer onderzoek deed naar de \groupref{haemers}. De Jonghe getuigde over de situatie waarin \indref{haemerspatrick} opgroeide:\begin{quote}Ze hadden toen twintig kledingwinkels (...) in het Brusselse.\end{quote}. Over de groepsverkrachting van Haemers in de jaren '70 stelt De Jonghe dat Haemers onschuldig pleitte.\cite{bendehaemersdejeugdjaren}}
\individual{deraedtfernandemotte}{Fernande Motte de Raedt}{advocate}{de Raedt was de advocate van \indref{haemerspatrick}. Over Haemers stelt ze het volgende:\begin{quote}Het is echt zonde. Voor mij heeft hij zijn leven vergooid. Hij was in de wieg gelegd om een gelukkig leven te leiden, zeg maar. En hij heeft dat leven vergooid. Dat is duidelijk.\end{quote}Over de veroordeling van Haemers voor de groepsverkrachting eind jaren '70 zegt de Raedt het volgende:\begin{quote}Ze hadden een prostituee opgezocht en nadien geweigerd te betalen voor haar diensten. Het liep fout en hij werd beschuldigd van groepsverkrachting. Hij is effectief veroordeeld en de rechter heeft hem schuldig bevonden.\end{quote}\cite{bendehaemersdejeugdjaren}}
\individual{dewitmichele}{Mich\`ele Dewit}{×}{Bijgenaamd chouchou. Ze had een relatie met \indref{haemerspatrick}. Ze is de dochter van de peetvader van het \location{Brussel}se misdaadmilieu. Over Haemers stelt ze het volgende:\begin{quote}Hij was lang en blond en hij ging uit in een discotheek die in die tijd erg bekend was en \institution{Le Vaudeville} heette. Ik had het niet zo op hem begrepen. (...) Geen idee, hij droeg altijd een ribfluwelen broek en daar ben ik allergisch aan. Hij ging dus nogal boers gekleed. En ik was veeleer punk. (...) Op een dag hielp ik een vriendin een handje. Ze werkte in een bistro, of taverne, tegenover \institution{Le Vaudeville}. In die tijd was er een bekende bloemenverkoper met maar \'e\'en hand. Hij verkocht bloemen, hij kwam binnen en gaf me veertig, vijftig ruikers. En hij zei: ``Die zijn van die man op het terras''. Ik ben buiten gaan kijken wie die man was. En dat was Haemers. (...) Nee hij kende men niet. Hij \emph{vousvoyeerde} me. Ik heb hem bedankt en op de mond gekust, vraag me niet waarom. Ik heb hem gekust en ik ben weer naar binnen gegaan. Ik was best verlegen. Hij vroeg me tot hoe laat ik werkte. Nog tien minuten, zei ik. Daarna in \institution{Le Vaudeville} vroeg hij me of we konden samenwonen. Die eerste avond al en ik heb ja gezegd. (...) We hebben het niet besproken. (...) We kenden elkaar tien minuten. Hij zei me dat hij in de gevangenis had gezeten. In het begin vroeg ik hem waarvoor. Niet dat ik nieuwsgierig was, ik wou het gewoon weten. Uiteindelijk zei hij dat het voor verkrachting was. Het interesseerde me niet zo. Maar hij zei: ``Voor verkrachting''.\end{quote}Ze woont nu in \location{Perpinion}. Ze is altijd contact blijven houden met familieleden van Haemers.\cite{bendehaemersdejeugdjaren}}
\caseentry{dutroux}{Dutroux}{}
