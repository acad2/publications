\annon{6209}{}{Crimineel in de buurt van \location{Charleroi}. Leerde \indref{dutrouxmarc} in de gevangenis kennen. Hij stelt dat vermits de kinderen niet kortstondig na hun verdwijning werden teruggevonden, Dutroux geen ordinaire verkrachter is. Hij stelt dat er nooit losgeld is gevraagd aan de ouders van de vermoorde kinderen. Nochtans bezat Dutroux vele huizen. Hij stelt dan ook dat Dutroux aan geld kwam door de meisjes uit te lenen.\cite{zdf20011028}}
\annon{2167}{}{Vriendin van \indref{pardaensgina}. De getuige stelt dat Pardaens teveel met de dood is bedreigd om dergelijk ongeval nog geloofwaardig te laten lijken. Voor de getuige is een ongeval dan ook uitgesloten.\cite{zdf20011028}}
\annon{7840}{getuige overval \institution{Delhaize} in \location{Overijse}}{\begin{quote}Ik stond met twee Engelstalige vriendinnen te wachten aan de kassa toen we schoten hoorden. Pas toen de flessen naast mij aan stukken vlogen realiseerde ik mij dat het om een overval ging. Iedereen ging namelijk op de grond liggen. Ze liepen onmiddellijk door naar het kantoor van de zaakvoerder. Toen ze buitenkwamen bleef \'e\'en van de overvallers staan bij onze kassa. Hij wilde het geld en zei tegen de kassierster: ``Ouvre ta casse, o\`u tu va cr\^eve''. Omdat ik bang was dat hij zou schieten, hielp ik haar recht achter haar kassa, maar ze kreeg ze niet open. De overvaller bleef aandringen en zei nogmaals: ``Ouvre ta casse, o\`u tu va cr\^eve''. Hij sprak Frans met een Brussels accent. Al die tijd, zeker meer dan \'e\'en minuut, zagen we vanop de grond hoe de overvaller, heel koelbloedig, bleef wachten op dat geld. Hij droeg een oudemannenmasker, een lange mantel en daaronder legerlaarzen. Zijn wapen droeg hij heel de tijd links, wellicht was hij dus linkshandig.
