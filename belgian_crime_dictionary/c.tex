\individual{coenraadschristiaan}{Christiaan Coenraads}{gevangene}{Zou in de Brusselse gevangenis worden gehoord over zijn relatie met \indref{dutrouxmarc} en medeverdachten. Een dag voor dit verhoor ontsnapt hij uit de gevangenis. Een maand later wordt hij in een voorstad van Brussel vermoord teruggevonden.\cite{zdf20011028}}
\individual{claeyssandra}{Sandra Claeys}{×}{Getuige in de \caseref{dutroux}. De getuige was voor de aanvang van het proces reeds overleden.\cite{zdf20011028}}
\individual{carettepierre}{Pierre Carette}{×}{Carette was lid van de \groupref{cellullescommunistecombatantes}.\cite{dezaakdejachtopdeccc}}
\groupentry{cellullescommunistecombatantes}{Cellules Communiste Combatantes (CCC)}{De CCC wordt verdacht en is veroordeeld voor volgende feiten:\begin{enumerate}\item Diefstal van wapens en explosieven uit een Waalse steengroeve en legerkazerne, \eventdate{Voorjaar, 1984}.\item Drie bomaanslagen op bedrijven in \location{Brussel}, \eventdate{begin oktober 1984}.\item Twee bomaanslagen op politieke gebouwen. \indref{decrooherman} en \indref{goljean} overleven de aanslag.\location{Elsene}, \eventdate{15 oktober 1984}.\item Bomaanslag bij het hoofdkantoor van de \groupref{christelijkevolkspartij} in \location{Gent}, \eventdate{17 oktober 1984}.\item Opblazen van de \institution{NAVO} pijpleidingen, \eventdate{11 december 1984}.\end{enumerate}\cite{dezaakdejachtopdeccc}}
\groupentry{christelijkevolkspartij}{Christelijke Volkspartij (CVP)}{×}
