\clearpage\lettergroup{@}\dictchar{Appendix}
\appendix
\section{Mathematics}
In this section we will describe mathematical structures used in the corpus of this dictionary.
\begin{description}
 \item [Bag] A collection of items such that each item can occur zero, one or more times. A bag has no inherent order: there is no $i$-th item in a bag. A bag is sometimes called a \nmn{multiset}.
 \item [Set] A collection of items. Each item can belong or not belong to a set. Multiple occurrences are not allowed. A set has no inherent order.
 \item [List] A collection of items such that each item can occur zero, one or more times. A list has an inherent order: given that a list $L$ contains $n$ items, an item $l_i$ is called the $i$-th item of the list. A list is sometimes called a \nmn{vector} or \nmn{tuple}.
\end{description}
\subsection{Standard sets}
Although one can describe a set, for instance by enumerating the elements or describing the properties the objects must satisfy, some sets are that popular a symbol is dedicated to them. We list these lists together with the symbol.
\begin{description}
 \item [$\BBB$] Set of booleans, we represent booleans using $0$ and $1$: $\BBB\equiv\accl{0,1}$.
 \item [$\NNN$] Set of natural numbers up to (but not including) infinity: $\NNN\equiv\accl{0,1,2,\ldots}$. The natural numbers without zero are denoted with $\NNNz\equiv\accl{1,2,3,\ldots}$
 \item [$\ZZZ$] Set of integers (both positive and negative) up to (but not including) infinity: $\ZZZ\equiv\accl{0,1,-1,2,-2,\ldots}$
 \item [$\QQQ$] Set of rational numbers: $\QQQ\equiv\condset{\dfrac{a}{b}}{a\in\ZZZ,b\in\NNNz}$.
 \item [$\RRR$] Set of real numbers.
\end{description}
The sets are hierarchically ordered: all elements of a set are contained in the next sets:
\[
\BBB\subsetneq\NNN\subsetneq\ZZZ\subsetneq\QQQ\subsetneq\RRR
\]
\section{Graph theory}