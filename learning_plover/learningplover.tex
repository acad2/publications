\documentclass[titlepage]{book}
\newcommand{\plover}[0]{\textsc{Plover}}
\newcommand{\github}[0]{\textsc{GitHub}}
\newcommand{\python}[0]{\textsc{Python}}
\title{Learning \plover{}}
\author{Willem M. A. Van Onsem}
\begin{document}
\begin{titlepage}
\maketitle
\end{titlepage}
\chapter{Introduction}
\paragraph{}The software is free of charge, however one needs to buy a gaming keyboard (around \$45.00) and practice.
\paragraph{}This book is mainly for qwerty-users who aim to speed up typing.
\section{Installing \plover{}}
\plover{} is available on \github{}. One can obtain a copy using the following command.
\begin{verbatim}
wget https://github.com/openstenoproject/plover/archive/v2.5.7.tar.gz
\end{verbatim}
After the archive is downloaded, one can decompress it, and install it using the following commands:
\begin{verbatim}
tar -zxf v2.5.7.tar.gz
cd plover-2.5.7
sudo python setup.py install
\end{verbatim}
The programming will copy the executable to \verb+/usr/local/bin/plover+. One can then remove the installation files:
\begin{verbatim}
cd ..
rm -rf plover-2.5.7
\end{verbatim}
The program depends on several \python{} libraries. One can use the following commands to install these libraries:
\begin{verbatim}
sudo apt-get install python-xlib python-wxgtk2.8 wmctrl python-dev python-pip
sudo pip install -U appdirs simplejson pyserial
\end{verbatim}
\chapter{The \plover{} system}

\end{document}