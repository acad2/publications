\subsection{Implementatie}
\begin{frame}[fragile]{\dsarraylist{}: Implementatie}
Een \dsarraylist{} is een klasse in Java die werkt op basis van een \dsarray{} en een teller.
\begin{figure}
\centering
\begin{tikzpicture}
\genobj{alo}{elementen,aantal/3}{eenArrayList}
\begin{scope}[xshift=4 cm,yshift=0.5 cm]
\tikzarrayptr{4}{aloarray}
\pointrtlebridge{aloptrelementen}{aloarray.west}
\end{scope}
\begin{scope}[xshift=5.5 cm,yshift=2 cm]
\strobj{elem0}{Samson}
\end{scope}
\begin{scope}[xshift=6.5 cm,yshift=1.25 cm]
\strobj{elem2}{Octaaf}
\end{scope}
\begin{scope}[xshift=6 cm,yshift=-0.25 cm]
\strobj{elem3}{Alberto}
\end{scope}
\pointarraytopleft{aloarray1}{elem0}
\pointarraytopleft{aloarray3}{elem2}
\pointarraybotleft{aloarray2}{elem3}
\end{tikzpicture}
\caption{Structuur van een \dsarraylist{}-object}
\end{figure}
Telkens wanneer de \dsarray{} vol zit en we willen een element toevoegen, maken we een nieuwe (grotere) \dsarray{} aan en kopi\"eren we de oude elementen naar de nieuwe \dsarray{}.
\end{frame}
\begin{frame}{\dsarraylist{}: Implementatie Overzicht}
We zullen de implementatie van volgende methodes bespreken:
\begin{itemize}
 \item \texttt{setCapaciteit (int capaciteit)}: een methode die de capaciteit (lengte van de \dsarray{}) aanpast. In het geval we deze moeten aanpassen worden de elementen gekopieerd.
 \item \texttt{voegToe (Object object)}
 \item \texttt{voegToe (int index, Object object)}
 \item \texttt{verwijder (int index)}
 \item \texttt{verwijder (int index, Object object)}
\end{itemize}
\end{frame}
\subsubsection{\texttt{setCapaciteit(int capaciteit)}}
\begin{frame}[fragile]{\dsarraylist{}: \texttt{setCapaciteit (int capaciteit)}}
\begin{methodexample}[\texttt{setCapaciteit (int capaciteit)}]
\begin{center}
\begin{animateinline}{1}
\methodexec{arraylistsetcapaciteit}{0}{this,cap/6,nieuw,temp,i/2}{
  \begin{scope}[xshift=6 cm,yshift=-1cm,transform shape,yscale=-1,scale=0.75]
  \genobj{alo}{elementen,aantal/3}{eenArrayList}
  \end{scope}
  \begin{scope}[xshift=6 cm,yshift=-0.75 cm,transform shape,yscale=-1,scale=0.75]
    \tikzarrayptr{4}{aloarray}
  \end{scope}
  \begin{scope}[xshift=5 cm,yshift=1.75 cm,transform shape,yscale=-1,scale=0.75]
    \tikzarrayptr{6}{newarray}
  \end{scope}
  \begin{scope}[xshift=5.5 cm,yshift=0 cm,transform shape,yscale=-1,scale=0.75]
    \strobj{elem0}{Samson}
  \end{scope}
  \begin{scope}[xshift=6 cm,yshift=0.4 cm,transform shape,yscale=-1,scale=0.75]
    \strobj{elem3}{Alberto}
  \end{scope}
  \begin{scope}[xshift=6.5 cm,yshift=0.8 cm,transform shape,yscale=-1,scale=0.75]
    \strobj{elem2}{Octaaf}
  \end{scope}
  
  \pointleftbridge{aloptrelementen}{aloarray.west}
  \pointleftbridge[3cm]{aloptrelementen}{newarray.west}
  
  \pointarraytopright{aloarray1}{elem0}
  \pointarraytopright{aloarray3}{elem2}
  \pointarraytopright{aloarray2}{elem3}
  
  \pointleftbridge{ptrnieuw}{newarray.west}
  
  \pointarraybotleft{newarray1}{elem0}
  \pointarraybotleft{newarray3}{elem2}
  \pointarraybotleft{newarray2}{elem3}
  
  \begin{scope}[transform shape,yscale=-1,scale=0.75]
  \tikzarraybounds{aloarray}{3/5}{-1/2}
  \end{scope}
%\end{scope}
  \pointlertbridge{ptrthis}{alo.east}
  \pointlertbridge[0.5 cm]{ptrtemp}{elem2.east}
}
\end{animateinline}
\end{center}
\end{methodexample}
\end{frame}