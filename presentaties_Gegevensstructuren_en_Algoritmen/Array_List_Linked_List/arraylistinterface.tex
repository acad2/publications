\subsection{Definitie}
\begin{frame}{\dsarraylist{}: definitie}
\begin{definition}[\dsarraylist{}]
Een \term{\dsarraylist{}} is een gegevensstructuur die bestaat uit een opeenvolging van elementen. Elk element in een \dsarraylist{} heeft een unieke \term{index} en is toegankelijk in constante tijd (dit wordt ook wel \term{Random-Access} genoemd). Een \dsarraylist{} heeft een variabele \term{lengte}: we hoeven geen lengte op te geven bij de constructie van een \dsarraylist{} en in principe kunnen we eindeloos elementen toevoegen.
\end{definition}
Visueel zullen we een \dsarraylist{} in deze presentatie voorstellen als een opeenvolging van hokjes, maar met een open einde. Zoals hier:
\begin{figure}
\begin{tikzpicture}
\tikzarraylist{10}{a,b,c,d,e,f,g,h,i,j}
\end{tikzpicture}
\caption{Voorstelling van een \dsarraylist{}}
\end{figure}
\end{frame}
\subsection{In Java}
\subsubsection{Declaratie}
\begin{frame}[fragile]{\dsarraylist{} in Java: declaratie}
Een \dsarraylist{} is een klasse die uit de Java-bibliotheek komt. Deze klasse is \texttt{java.util.ArrayList}. Daarom kunnen we een \dsarraylist{} aanmaken zoals andere objecten. Een \dsarraylist{} is een \term{generische klasse}: we kunnen een type-parameter tussen scheve haken (\texttt{<>}).
\examplecode{Declaratie \dsarraylist{}}{arraylistdeclaration0}
\end{frame}
\begin{frame}[fragile]{\dsarraylist{} in Java: Primitieve types bij generische klasses}
\begin{letop}[Primitieve types bij generische klasses]
\small{Men kan een \dsarraylist{} declareren met elk \term{klasse-type}, primitieve types zijn niet toegelaten! Hiervoor heeft Java schaduwklasses ingevoerd:}
\begin{table}
\centering
\small{\begin{tabular}{l|l}
Primitief type&Klasse-type\\\hline
\texttt{byte}&\texttt{Byte}\\
\texttt{short}&\texttt{Short}\\
\texttt{int}&\texttt{Integer}\\
\texttt{long}&\texttt{Long}\\
\texttt{float}&\texttt{Float}\\
\texttt{double}&\texttt{Double}\\
\texttt{boolean}&\texttt{Boolean}\\
\texttt{char}&\texttt{Char}\\
\end{tabular}}
\caption{Omzetten van primitieve types naar klasse-types}
\end{table}
\end{letop}
\end{frame}
\subsubsection{Toevoegen en verwijderen}
\begin{frame}[fragile]{\dsarraylist{} in Java: toevoegen/verwijderen van elementen}
\dsarraylist{} biedt methodes aan om elementen toe te voegen en te verwijderen:
\small{\begin{itemize}
 \item \texttt{public boolean add (E element)}: voegt een element toe op het einde van de \dsarraylist{}
 \item \texttt{public void add (int index, E element)}: voegt een element toe op plaats \texttt{index}, de overige elementen worden naar rechts opgeschoven
 \item \texttt{public E remove (int index)}: verwijdert het element op \texttt{index}. De elementen erna worden naar links opgeschoven. Het element die oorspronkelijk op deze plaats stond, wordt teruggegeven
 \item \texttt{public boolean remove (Object o)}: het opgegeven element wordt uit de \dsarraylist{} verwijdert. Geeft \texttt{true} terug indien dit element in de \dsarraylist{} aanwezig was.
\end{itemize}}
\end{frame}
\begin{frame}[fragile]{\dsarraylist{} in Java: \texttt{add (E element)}}
\begin{methodexample}[\texttt{add (E element)}]
\begin{center}
\begin{animateinline}{1}
\methodexec{arraylistadd0me}{1}{}{
}\newframe

\methodexec{arraylistadd0me}{2}{namen}{
\begin{scope}[xshift=7 cm]
\tikzarraylistptr{0}{namenlist}
\pointleftbridge{ptrnamen}{offsetnamenlist}
\end{scope}
}\newframe
\methodexec{arraylistadd0me}{3}{namen}{
\begin{scope}[xshift=3 cm,yshift=1 cm]
\strobj{elem0}{Samson}
\end{scope}
\begin{scope}[xshift=7 cm]
\tikzarraylistptr{1}{namenlist}
\pointleftbridge{ptrnamen}{offsetnamenlist}
\pointarraytopright{namenlist1}{elem0}
\end{scope}
}\newframe

\methodexec{arraylistadd0me}{4}{namen}{
\begin{scope}[xshift=3 cm,yshift=1 cm]
\strobj{elem0}{Samson}
\end{scope}
\begin{scope}[xshift=4 cm,yshift=-0.5 cm]
\strobj{elem1}{Gert}
\end{scope}
\begin{scope}[xshift=7 cm]
\tikzarraylistptr{2}{namenlist}
\pointleftbridge{ptrnamen}{offsetnamenlist}
\pointarraytopright{namenlist1}{elem0}
\pointarraybotright{namenlist2}{elem1}
\end{scope}
}\newframe

\methodexec{arraylistadd0me}{5}{namen}{
\begin{scope}[xshift=3 cm,yshift=1 cm]
\strobj{elem0}{Samson}
\end{scope}
\begin{scope}[xshift=4 cm,yshift=-0.5 cm]
\strobj{elem1}{Gert}
\end{scope}
\begin{scope}[xshift=5 cm,yshift=1.5 cm]
\strobj{elem2}{Octaaf}
\end{scope}
\begin{scope}[xshift=7 cm]
\tikzarraylistptr{3}{namenlist}
\pointleftbridge{ptrnamen}{offsetnamenlist}
\pointarraytopright{namenlist1}{elem0}
\pointarraybotright{namenlist2}{elem1}
\pointarraytopright{namenlist3}{elem2}
\end{scope}
}
\end{animateinline}
\end{center}
\end{methodexample}
\end{frame}
\begin{frame}[fragile]{\dsarraylist{} in Java: \texttt{add (int index, E element)}}
\begin{methodexample}[\texttt{add (int index, E element)}]
\begin{center}
\begin{animateinline}{1}
\methodexec{arraylistadd1me}{5}{namen}{
\begin{scope}[xshift=3 cm,yshift=1 cm]
\strobj{elem0}{Samson}
\end{scope}
\begin{scope}[xshift=4 cm,yshift=-0.5 cm]
\strobj{elem1}{Gert}
\end{scope}
\begin{scope}[xshift=5 cm,yshift=1.5 cm]
\strobj{elem2}{Octaaf}
\end{scope}
\begin{scope}[xshift=7 cm]
\tikzarraylistptr{3}{namenlist}
\pointleftbridge{ptrnamen}{offsetnamenlist}
\pointarraytopright{namenlist1}{elem0}
\pointarraybotright{namenlist2}{elem1}
\pointarraytopright{namenlist3}{elem2}
\end{scope}
}\newframe

\methodexec{arraylistadd1me}{6}{namen}{
\begin{scope}[xshift=3 cm,yshift=1 cm]
\strobj{elem0}{Samson}
\end{scope}
\begin{scope}[xshift=4 cm,yshift=-0.5 cm]
\strobj{elem1}{Gert}
\end{scope}
\begin{scope}[xshift=5 cm,yshift=1.5 cm]
\strobj{elem2}{Octaaf}
\end{scope}
\begin{scope}[xshift=6 cm,yshift=-1 cm]
\strobj{elem3}{Alberto}
\end{scope}
\begin{scope}[xshift=7 cm]
\tikzarraylistptr{4}{namenlist}
\pointleftbridge{ptrnamen}{offsetnamenlist}
\pointarraytopright{namenlist1}{elem0}
\pointarraybotright{namenlist2}{elem1}
\pointarraytopright{namenlist4}{elem2}
\pointarraybotright{namenlist3}{elem3}
\end{scope}
}
\end{animateinline}
\end{center}
\end{methodexample}
\end{frame}
\begin{frame}[fragile]{\dsarraylist{} in Java: \texttt{remove (int index)}}
\begin{methodexample}[\texttt{remove (int index)}]
\begin{center}
\begin{animateinline}{1}
\methodexec{arraylistremove0me}{6}{namen}{
\begin{scope}[xshift=3 cm,yshift=1 cm]
\strobj{elem0}{Samson}
\end{scope}
\begin{scope}[xshift=4 cm,yshift=-0.5 cm]
\strobj{elem1}{Gert}
\end{scope}
\begin{scope}[xshift=5 cm,yshift=1.5 cm]
\strobj{elem2}{Octaaf}
\end{scope}
\begin{scope}[xshift=6 cm,yshift=-1 cm]
\strobj{elem3}{Alberto}
\end{scope}
\begin{scope}[xshift=7 cm]
\tikzarraylistptr{4}{namenlist}
\pointleftbridge{ptrnamen}{offsetnamenlist}
\pointarraytopright{namenlist1}{elem0}
\pointarraybotright{namenlist2}{elem1}
\pointarraytopright{namenlist4}{elem2}
\pointarraybotright{namenlist3}{elem3}
\end{scope}
}\newframe

\methodexec{arraylistremove0me}{7}{namen,verw}{
\begin{scope}[xshift=3 cm,yshift=1 cm]
\strobj{elem0}{Samson}
\end{scope}
\begin{scope}[xshift=4 cm,yshift=-0.5 cm]
\strobj{elem1}{Gert}
\end{scope}
\begin{scope}[xshift=5 cm,yshift=1.5 cm]
\strobj{elem2}{Octaaf}
\end{scope}
\begin{scope}[xshift=6 cm,yshift=-1 cm]
\strobj{elem3}{Alberto}
\end{scope}
\begin{scope}[xshift=7 cm]
\tikzarraylistptr{3}{namenlist}
\pointleftbridge{ptrnamen}{offsetnamenlist}
\pointarraytopright{namenlist1}{elem0}
\pointarraytopright{namenlist3}{elem2}
\pointarraybotright{namenlist2}{elem3}
\pointlertbridge{ptrverw}{elem1}
\end{scope}
}\newframe

\methodexec{arraylistremove0me}{8}{namen,verw}{
\begin{scope}[xshift=3 cm,yshift=1 cm]
\strobj{elem0}{Samson}
\end{scope}
% \begin{scope}[xshift=4 cm,yshift=-0.5 cm]
% \strobj{elem1}{Gert}
% \end{scope}
\begin{scope}[xshift=5 cm,yshift=1.5 cm]
\strobj{elem2}{Octaaf}
\end{scope}
\begin{scope}[xshift=6 cm,yshift=-1 cm]
\strobj{elem3}{Alberto}
\end{scope}
\begin{scope}[xshift=7 cm]
\tikzarraylistptr{2}{namenlist}
\pointleftbridge{ptrnamen}{offsetnamenlist}
\pointarraytopright{namenlist1}{elem0}
\pointarraybotright{namenlist2}{elem3}
\pointlertbridge{ptrverw}{elem2}
\end{scope}
}
\end{animateinline}
\end{center}
\end{methodexample}
\end{frame}
\begin{frame}[fragile]{\dsarraylist{} in Java: \texttt{remove (Object o)}}
\begin{methodexample}[\texttt{remove (Object o)}]
\begin{center}
\begin{animateinline}{1}
\methodexec{arraylistremove1me}{8}{namen,verw}{
\begin{scope}[xshift=3 cm,yshift=1 cm]
\strobj{elem0}{Samson}
\end{scope}
% \begin{scope}[xshift=4 cm,yshift=-0.5 cm]
% \strobj{elem1}{Gert}
% \end{scope}
\begin{scope}[xshift=5 cm,yshift=1.5 cm]
\strobj{elem2}{Octaaf}
\end{scope}
\begin{scope}[xshift=6 cm,yshift=-1 cm]
\strobj{elem3}{Alberto}
\end{scope}
\begin{scope}[xshift=7 cm]
\tikzarraylistptr{2}{namenlist}
\pointleftbridge{ptrnamen}{offsetnamenlist}
\pointarraytopright{namenlist1}{elem0}
\pointarraybotright{namenlist2}{elem3}
\pointlertbridge{ptrverw}{elem2}
\end{scope}
}\newframe

\methodexec{arraylistremove1me}{9}{namen,verw,wasin/false}{
\begin{scope}[xshift=3 cm,yshift=1 cm]
\strobj{elem0}{Samson}
\end{scope}
% \begin{scope}[xshift=4 cm,yshift=-0.5 cm]
% \strobj{elem1}{Gert}
% \end{scope}
\begin{scope}[xshift=5 cm,yshift=1.5 cm]
\strobj{elem2}{Octaaf}
\end{scope}
\begin{scope}[xshift=6 cm,yshift=-1 cm]
\strobj{elem3}{Alberto}
\end{scope}
\begin{scope}[xshift=7 cm]
\tikzarraylistptr{2}{namenlist}
\pointleftbridge{ptrnamen}{offsetnamenlist}
\pointarraytopright{namenlist1}{elem0}
\pointarraybotright{namenlist2}{elem3}
\pointlertbridge{ptrverw}{elem2}
\end{scope}
}\newframe

\methodexec{arraylistremove1me}{10}{namen,verw,wasin/true}{
\begin{scope}[xshift=3 cm,yshift=1 cm]
\strobj{elem0}{Samson}
\end{scope}
% \begin{scope}[xshift=4 cm,yshift=-0.5 cm]
% \strobj{elem1}{Gert}
% \end{scope}
\begin{scope}[xshift=5 cm,yshift=1.5 cm]
\strobj{elem2}{Octaaf}
\end{scope}
% \begin{scope}[xshift=6 cm,yshift=-1 cm]
% \strobj{elem3}{Alberto}
% \end{scope}
\begin{scope}[xshift=7 cm]
\tikzarraylistptr{1}{namenlist}
\pointleftbridge{ptrnamen}{offsetnamenlist}
\pointarraytopright{namenlist1}{elem0}
%\pointarraybotright{namenlist2}{elem3}
\pointlertbridge{ptrverw}{elem2}
\end{scope}
}
\end{animateinline}
\end{center}
\end{methodexample}
\end{frame}
% \subsubsection{Toegang tot elementen}
% \begin{frame}[fragile]{\dsarraylist{} in Java: toegang tot elementen}
% Omdat \dsarraylist{} een gewone klasse is, krijgt met toegang tot de elementen via methodes:
% \begin{itemize}
%  \item \texttt{public E get (int index)}
%  \item \texttt{public E set (int index, E element)}
% \end{itemize}
% \end{frame}
% \subsection{Lengte}
% \begin{frame}[fragile]{\dsarraylist{} in Java: lengte (aantal elementen)}
% We kunnen het aantal elementen in de \dsarraylist{} opvragen met behulp van de \texttt{size()}-methode.
% \begin{example}[Lengte van een \dsarraylist{}]
% \begin{lstlisting}
% ArrayList<String> namen = new ArrayList<String>();
% namen.add("Samson");
% namen.add("Gert");
% namen.add("Alberto");
% namen.add("Octaaf");
% System.out.println(namen.size()); //4
% \end{lstlisting}
% \end{example}
% \begin{letop}[\texttt{length} versus \texttt{size()}]
% Bij een \dsarray{} vragen we de lengte op met behulp van \texttt{.length}. Bij een \dsarraylist{} is dat met de \texttt{size()}-methode. Omgekeerd werkt dit niet!
% \end{letop}
% \end{frame}
% \subsubsection{Andere operaties}