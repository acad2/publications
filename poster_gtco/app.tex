legtap/legen/3/12.5/5 cm/Application/App/{An \emph{application} is a scenario in which theory is useful.},
rnaseq/move/20/8/7 cm/{RNA sequencing}/{RNA}/{An \emph{RNA sequence} is a sequence consisting out of adenine, cytosine, guanine and uracil. The \emph{G-enzym} breaks an RNA sequence after each guanine character, the \emph{UC-enzym} after each cytosine and uracil character. RNA sequencing consists of given an unordered bag of subsequences after the G-enzym cuts and an unordered bag of subsequences after the UC-enzym cuts, to reconstruct the original RNA sequence. An efficient strategy is to construct an Eulerian digraph and then find an Eulerian path that represents the sequence.},
regal/grcl/42/1.5/9.5 cm/{Register allocation}/{\texttt{aex}$\gets$}/{\emph{Register allocation} is a step in a compiler where an arbitrary number of variables is allocated to a limited number of processor registers. Variables that are alive simultaneously cannot be allocated to the same register (unless with \emph{spilling}). Register allocation aims to search heuristically to a coloring of the variables such that the number of colors does not exceed the number of registers.},
freqas/grcl/42/6/9.5 cm/{Frequency assignment}/{Hz}/{A number of \emph{transmitters} send information using a specific frequency band. Two transmitters cannot use the same frequency band if their broadcast areas overlap.},
schedl/grcl/42/9.25/9.5 cm/{Scheduling}/{\clock}/{In a \emph{scheduling problem} one aims to schedule a number of tasks that require resources. Some resources cannot be used simultaneously (rooms, students,...). One colors the tasks with non-overlapping time slots and tasks sharing a resource cannot be assigned the same color.}