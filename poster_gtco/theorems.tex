legthm/legen/3/5.5/5 cm/Theorem/Thm/{A \emph{theorem} is a statement that has been proven on the basis of previously established statements, such as other theorems.},
eulerthm/graph/31/8/9 cm/{Euler's theorem}/$\mbox{even}\ \Sigma{\funm{deg}{v_i}}$/{A graph contains an Eulerian tour if the sum of the degrees of the vertices of a graph is twice the number of edges.},
chartree/tree/12/1.5/7.5 cm/{Characterization of a tree}/{$T\equiv T$}/{Let $T$ be an $n$-vertex graph, then the following statements are equivalent:\begin{enumerate}[noitemsep,nolistsep]\item $T$ is a tree; \item $T$ contains no cycles and has $n-1$ edges; \item $T$ is connected and has $n-1$ edges; \item $T$ is connected and every edge is a cut-edge; \item Any two vertices of $T$ are connected by exactly one path; \item $T$ contains no cycles, and for every new edge $e$, the graph $T+e$ has exactly one cycle.\end{enumerate}.},
ceuler/move/32.5/1.75/9.5 cm/{Characterization of Eulerian graphs}/{Euler $G$}/{The following statements are equivalent for a connected graph $G$: (1) $G$ is Eulerian; (2) the degree of every vertex in $G$ is even; and (3) there is a set of cycles in $G$ whose edge-sets partition $E_G$.},
greed/cop/13.25/7.25/7 cm/{Optimal greedy algorithm}/{$\tupl{E,\calI}\leftrightarrow$\textdollaroldstyle$\nearrow$}/{Let $S=\tupl{E,\calI}$ be a hereditary subset system. The greedy algorithm solves every instance of the maximum-weight problem associated with $S$ optimal if and only if $S$ is a matroid.},
mincut/maflw/30.75/6.75/7 cm/{Weak/Strong duality}/{Dual \watertap}/{For every feasible flow $f$ for a given network and for every capacity $c$ of every flow cut, it holds that $f\leq c$. Furthermore the maximum flow $f^{\star}$ equals the minimum capacity over all possible flow cuts.},
treecv/tree/50.25/2/7 cm/{Number of vertices in a tree}/{$\#V_T$}/{The number of vertices in a complete $m$-ary tree equals\[\dfrac{m^{h+1}-1}{m-1}.\] A complete binary tree has $2^h+1$ vertices.},
treemh/tree/50.25/6.5/7 cm/{Height $m$-ary tree}/{$\fun{h}{m\mbox{-}T}$}/{The height $h$ of an $m$-ary tree with $n$ vertices, then \[h+1\leq n\leq\dfrac{m^{h+1}-1}{m-1}.\]},
chrrel/grcl/11.75/7.5/7 cm/{Bounds on the chromatic number}/{$\fun{\omega}{G}\sim\fun{\chi}{G}\sim\fun{\alpha}{G}$}/{For any graph $G$, it holds that \[\fun{\max}{\fun{\omega}{G},\ceil{\dfrac{\abs{V_G}}{\fun{\alpha}{G}}}}\leq\fun{\chi}{G}.\] The chromatic number of a join $G+H$ is: \[\fun{\chi}{G+H}=\fun{\chi}{G}+\fun{\chi}{H}.\]}