legdef/legen/3/2.75/5 cm/Definition/Def/{A \emph{definition} is a statement of the meaning of a term.},%
grph/graph/4/2.25/7 cm/Graph/{$\fun{G}{V_G,E_G}$}/{A \emph{graph} $\fun{G}{V_G,E_G}$ consists out of a set of \emph{vertices} $V_G$ and a set of \emph{edges} $E_G$ to each edge is associated a set of one or two vertices called its \emph{endpoints}. A \emph{trivial graph} consists out of one vertex and no edges. The \emph{formal specification} of a graph $\fun{G}{V,E}$ consists of an \emph{incidence function} $\mbox{endpts}$, which maps each edge $e$ to a one- or two-elements subset of $V$ called $\funm{endpts}{e}$.},
simg/graph/4/8/7 cm/{Simple (di)graph}/$G^{-}$/{A \emph{simple} (di)graph is a (di)graph that does not contain self-loops and multi-edges (or multi-arcs).},
dirg/graph/12/2/7 cm/{Directed graph}/$G^{\rightarrow}$/{A \emph{directed graph} (or \emph{digraph}) is a graph where each of the edges is directed. The \emph{underlying graph} of a digraph is the graph generated by removing all the edge directions from the digraph. A directed edge is called an \emph{arc}. The endpoints of an arc are named \emph{head} and \emph{tail}.},
slfl/graph/19/2/5 cm/{self-loop, proper edge}/{$e=\tupl{v,w}$}/{A \emph{self-loop} is an edge that joins a single endpoint to itself. A \emph{proper edge} is an edge that is not a self-loop.},
degv/graph/25/2/5 cm/{Degree, degree sequence, indegree, outdegree}/{$\funm{deg}{v}$}/{The \emph{degree} of a vertex $v$ is the number of proper edges incident on $v$ plus twice the number of self-loops on $v$. The degree sequence is a non-increasing arrangement of the degrees of the vertices of a graph. The degree is denoted by $\funm{deg}{v}$ or $\fun{\delta}{v}$. The \emph{indegree} of a vertex $v$ of a digraph is the number of arcs with head the given vertex. The \emph{outdegree} of a vertex $v$ of a digraph is the number of arcs with tail the given vertex. $\fun{\submax{\delta}}{G}$ denotes the maximum degree of the vertices of the graph $G$.},
mule/graph/19/5.75/5 cm/{Multi-edge, multi-arc}/$v_i\rightrightarrows v_j$/{A \emph{multi-edge} is a set of two or more edges with identical endpoints. The \emph{edge-multiplicity} is the number of edges between vertices or the number of self-loops of a vertex. A \emph{multi-arc} is a set of two or more arcs with identical endpoints.},
weighg/graph/12/7/7 cm/{weighted graph}/{\textlinb{\BPtalent} $G$}/{A weighted graph is a graph where each edge is assigned a number called its \emph{edge-weight}.},
subgr/graph/50/2/7 cm/{(Proper) subgraph}/{$\subseteq G$}/{A graph $H$ is a \emph{subgraph} of a graph $G$ (and $G$ is said a \emph{supergraph} of $H$) if $V_H\subseteq V_G$ and $E_H\subseteq G$. A graph $H$ is a \emph{proper subgraph} of a graph $G$ (and $G$ is said a \emph{proper supergraph} of $H$) if $V_H\subsetneq V_G$ and $E_H\subsetneq G$.},
spansubg/graph/50/6.25/7 cm/{Spanning subgraph}/{\spanning{}$ G$}/{A graph $H$ is a spanning subgraph of $G$ if $H$ is a subgraph of $G$ and $V_H=V_G$.},
cliq/graph/50/9.5/7 cm/{Clique, clique number}/{$\fun{\calC}{G}, \fun{\omega}{G}$}/{A \emph{clique} of a graph $G$ is a subset $S\subseteq V_G$ such that for every two vertices $v_i,v_j\in S$, there exists an edge between $v_i$ and $v_j$ and no superset $S'\supsetneq S$ has this property. The \emph{clique number} $\fun{\omega}{G}$ of a graph $G$ is the number of vertices in the largest clique of $G$.},
adjv/graph/2.75/12/4 cm/{Adjacent vertices}/{$v_i\sim v_j$}/{Two vertices are \emph{adjacent} or \emph{neighbors} if the two vertices are joined by an edge.},
adje/graph/7.5/12/4 cm/{Adjacent edges}/{$e_i\sim e_j$}/{Two edges are \emph{adjacent} if the edges have an endpoint in common.},
ince/graph/12.25/12/4 cm/{Incident}/{$v_i\sim e_j$}/{An edge $e$ is \emph{incident} to a vertex $v$ if $v$ is an endpoint of $e$.},
vdsg/graph/33.5/2/7 cm/{Vertex-deletion subgraph}/{$G-v$}/{The subgraph induced on the vertex set $V_G\setminus\accl{v}$. That is $V_{G-v}=V_G\setminus\accl{v}$ and $E_{G-v}=\condset{e\in E_G}{v\notin\funm{endpts}{e}}$.},%Grouped
edsg/graph/33.5/5.5/7 cm/{Edge-deletion subgraph}/{$G-e$}/{The subgraph induced on the edge set $E_G\setminus\accl{e}$. That is $V_{G-e}=V_G$ and $E_{G-e}=E_G\setminus\accl{e}$.},%Grouped
gruni/graph/33.5/8.5/7 cm/{Graph union}/{$G\cup H$}/{The \emph{graph union} is the graph whose vertex- and edge-sets are $V_{G\cup H}=V_G\cup V_H$ and $E_{G\cup H}=E_G\cup E_H$.},%Grouped
grjoin/graph/33.5/11.5/7 cm/{Graph join}/{$G+H$}/{The \emph{graph join} is the graph obtained from $G\cup H$ by adding edges between every vertex of $G$ and every vertex of $H$.},%Grouped
sbgiv/graph/41.5/2/7 cm/{Subgraph induced on a vertex subset}/{$\fun{G}{V}$}/{The subgraph induced on a vertex subset $V\subseteq V_G$ is the subgraph of $G$ with $V_{\fun{G}{V}}=V$ and $E_{\fun{G}{V}}=\condset{e\in E_G}{\funm{endpts}{e}\subseteq V}$.},
sbgie/graph/41.5/5.5/7 cm/{Subgraph induced on an edge subset}/{$\fun{G}{E}$}/{The subgraph induced on a vertex subset $E\subseteq E_G$ is the subgraph of $G$ with $E_{\fun{G}{E}}=E$ and $V_{\fun{G}{E}}=\condset{v\in V_G}{\exists e\in E:v\in\funm{endpts}{e}}$.},
linegr/graph/41.5/8.75/7 cm/{Line graph}/{$\funsig{L}{G}{G}$}/{The \emph{line graph} of a digraph $G$ is the digraph $\fun{L}{G}$ whose vertex-set corresponds to the arc-set of $G$, and an arc-set of $G$, and an arc in $\fun{L}{G}$ is directed from vertex $v$ to $w$ if, in $G$, $\funm{head}{v}=\funm{tail}{w}$.},%Grouped
nregg/grfa/4/2.5/7 cm/{$n$-regular graph}/{$n$-reg}/{An $n$-regular graph is a graph where every vertex has degree $n$.},%Grouped
compgr/grfa/12/1.5/7 cm/{Complete graph}/{$K_n$}/{A complete graph on $n$ vertices is a simple graph where for every two different vertices, there is an edge.\importtikzcenter{compgr}},%Grouped
bipagr/grfa/20/1.5/7 cm/{Bipartite graph}/{$K_{n,m}$}/{A bipartite graph is a graph such that the vertex set $V$ can be partitioned in two disjunctive subsets $V_1$ and $V_2$ such that every edge has one endpoint in $V_1$ and one endpoint in $V_2$. $V_1$ and $V_2$ are called the \emph{bipartition subsets}. A \emph{complete} bipartite graph is a bipartite graph where every possible edge exists.\importtikzcenter{bipagr}},%Grouped
pathgr/grfa/4/5/7 cm/{Path graph}/{$P_n$}/{A path graph is a simple connected graph such that $\abs{V}=\abs{E}+1$ and can be drawn such that all of the vertices and edges lie on a straight line.\importtikzcenter{pathgr}},%Grouped
cyclgr/grfa/12/6/7 cm/{Cycle graph}/{$C_n$}/{A cycle graph is a simple connected graph such that $\abs{V}=\abs{E}$ and can be draw such that all of its vertices and edges lie on a circle.\importtikzcenter{cyclegr}},%Grouped
nwheel/grfa/12/10.75/7 cm/{$n$-wheel}/{$W_n$}/{The \emph{$n$-wheel} is the join $K_1+C_n$ of a single vertex $K_1$ and an $n$-cycle $C_n$.\importtikzcenter{wheelgr}},%Grouped
petersen/grfa/4/9/7 cm/{Petersen graph}/{Petersen}/{The \emph{Petersen graph} is a graph with $10$ vertices and $15$ edges as depicted in the figure below:\importtikzcenter{petersen}},
hypergr/grfa/28/2/7 cm/{Hypercube graph}/{$Q_n$}/{The hypercube graph $Q_n$ is an $n$-regular graph whose vertex-set is the set of all bitstrings of length $n$, and such that there is an edge between two vertices if and only if they differ exactly one bit.\importtikzcenter{hypergr}},%Grouped
graycod/grfa/28/8.75/7 cm/{Gray code}/{$1\times \brak{0\leftrightarrow 1}$}/{A \emph{gray code} of order $n$ is an ordering of $2^n$ bitstrings of length $n$ such that consecutive bitstrings differ precisely one bit position.},%Grouped
walk/move/4/2/7 cm/{Walk}/{$\rightsquigarrow$}/{A walk is an alternating sequence of vertices and edges $\tupl{v_0,e_1,v_1,\ldots,e_n,v_n}$ such that $\funm{endpts}{e_i}=\tupl{v_{i-1},v_i}$ for every $e_i$. The \emph{length} of a walk is the number of edge steps in the sequence. A \emph{trivial walk} is a walk with one vertex and no edges. A trivial walk has length $0$.},%Grouped
diwalk/move/4/6.5/7 cm/{Directed walk}/{$\rightsquigarrow^{\rightarrow}$}/{A directed walk is a walk defined on a directed graph such that for every edge $\funm{tail}{e_i}=v_{i-1}$ and $\funm{head}{e_i}=v_i$.},
clwalk/move/4/9.5/7 cm/{Open/closed walk}/{$v\rightsquigarrow v$}/{A closed walk is a walk such that the first and the last vertex of the sequence are the same. An \emph{open walk} is a walk that is not a closed walk.},
dstnc/move/12/2.25/7 cm/{Distance}/{$\fun{d}{v_i,v_j}$}/{The distance between two vertices $v_i$ and $v_j$ is the length of the shortest walk from $v_i$ to $v_j$.},
trl/move/12/5/7 cm/{Trail}/{$\looparrowright$}/{A trail is a walk where every edge is visited at most once.},%Grouped
pth/move/12/7.5/7 cm/{Path}/{$\rightarrowtail$}/{A path is a trail with no repeating vertices (except the initial and final vertices).},%Grouped
cycl/move/12/10.25/7 cm/{Cycle}/{$\circlearrowleft$}/{A cycle is a nontrivial closed walk that is a path. An edge $e$ that is contained in a cycle is called a \emph{cycle-edge}.},%Grouped
cncwl/move/20/1.75/7 cm/{Concatenation}/{$\rightsquigarrow\circ\rightsquigarrow$}/{The concatenation of two walks $W_1=\tupl{v_0,e_1,\ldots,e_m,v_m}$ and $W_2=\tupl{v_m,e_{m+1},v_{m+1},\ldots,e_n,v_n}$ is the walk $W_1\circ W_2=\tupl{v_0,e_1,v_1,\ldots,e_n,v_n}$.},
subwl/move/20/5.5/7 cm/{Subwalk}/{$\subseteq\rightsquigarrow$}/{A subsequence of the entries of a walk $W$ that is a walk itself.},
eulert/move/28.25/2/6 cm/{Eulerian trail}/{Euler $\rightsquigarrow$}/{An \emph{Eulerian trail} of a graph $G$ is a trail that contains every edge $e\in E_G$ of the graph. An \emph{Eulerian tour} a closed Eulerian trail. An \emph{Eulerian graph} is a graph that contains an Eulerian tour.},%Grouped
posttour/move/28.25/6.25/6 cm/{Postman tour}/{Post $\rightsquigarrow$}/{A \emph{postman tour} in a graph $G$ is a closed walk that uses each edge of $G$ at least once. In a weighted graph, the \emph{optimal postman tour} is a postman tour whose total edge-weight is a minimum. Finding such tour is known as the \emph{Chinese Postman Problem}.},%Grouped
hamilcyc/move/28.25/11.25/6 cm/{Hamiltonian cycle}/{Hamilton $\rightsquigarrow$}/{A \emph{Hamiltonian cycle} in a graph is a cycle that contains all the vertices of the graph. A graph is a \emph{Hamiltonian graph} if it contains a Hamiltonian cycle.},%Grouped
reach/move/42/2/7 cm/{Reachable}/{$v_i\rightsquigarrow^{\star} v_j$}/{A vertex $v_i$ is reachable from a vertex $v_j$ in a graph (or digraph) if there exists a walk (or directed walk) from $v_i$ to $v_j$.},
conn/move/42/5/7 cm/{Connected graph}/{$V\rightsquigarrow^{\star}V$}/{A graph is \emph{connected} if for all vertices $v_i,v_j\in V$, $v_i$ is reachable from $v_j$.},
wkconn/move/50/2/7 cm/{Weakly/strongly connected digraph}/{$\vec{V}\rightsquigarrow^{\star}\vec{V}$}/{A digraph is \emph{weakly connected} if the underlying graph is connected. A digraph is \emph{strongly connected} if for all vertices $v_i,v_j\in V$, $v_i$ is reachable from $v_j$ and $v_j$ is reachable from $v_i$. A graph is \emph{strongly orientable} if there exists an assignment of directions to the edges of the graph such that the resulting graph is strongly connected.},
cpn/move/42/8/7 cm/{Component}/{$\fun{C_G}{v}$}/{A component of a graph $G$ is a maximal connected subgraph of $G$. For a given vertex $v$ there is exactly one component containing $v$ denoted by $\fun{C_G}{v}$.},
vercut/move/42/11.25/7 cm/{Vertex-cut, cut-vertex}/{\Rightscissors$v$}/{A vertex-cut in a graph $G$ is a set $V\subseteq V_G$ of vertices such that $G-V$ has more components than $G$. A vertex-cut is minimal if no proper subset of $V$ is an vertex-cut. A \emph{cut-vertex} (or \emph{cutpoint}) is a vertex-cut consisting of a single vertex.},
edgcut/move/50/11.25/7 cm/{Edge-cut,cut-edge}/{\Rightscissors$e$}/{An edge-cut in a graph $G$ is a set $E\subseteq E_G$ of edges such that $G-E$ has more components than $G$. An edge-cut is minimal if no proper subset of $E$ is an edge-cut. A \emph{cut-edge} (or \emph{bridge}) is an edge-cut consisting of a single edge.},
tsp/move/34.5/8/5 cm/{Traveling salesman problem}/{TSP}/{The \emph{traveling salesman problem} is an optimization problem where one aims to find the minimum Hamiltonian cycle for a given weighted graph. The traveling salesman problem is known to be \ccnph{}. A popular approximation algorithm is the \emph{DoubleTheTree} algorithm.},
herdss/cop/5/3/8 cm/{Hereditary subset system, maximum-weight problem, matroid, augmentation property}/{$S=\tupl{E,\calI}$}/{A \emph{hereditary subset system} $S=\tupl{E,\calI}$ is a finite set $E$ together with a collection $\calI\subseteq\powset{E}$ of subsets of $E$ closed under \emph{inclusion}: for every $I\in\calI$ and for every $I'\subseteq I$, then $I'\in\calI$. The subsets of $E$ in $\calI$ are called the \emph{independent subsets} whereas the subsets of $E$ not in $\calI$ are called \emph{dependent subsets}. The \emph{maximum-weight problem} associated with a hereditary subset system $S=\tupl{E,\calI}$ associates a \emph{weight function} $\funsig{w}{E}{\RRR^+_0}$. One aims to find a \emph{maximum-weight} independent subset $I\in\calI$ such that the total weight $\smlsumdomain[e]{I}{\fun{w}{e}}$ is maximum. A \emph{matroid} is a hereditary system $S=\tupl{E,\calI}$ where the \emph{augmentation property} holds: if $I,J\in\calI$ and $\abs{I}<\abs{J}$, then there exists an element $e\in J\setminus I$ such that $I\cup\accl{e}\in\calI$.},
greed/cop/13.25/3/7 cm/{Greedy algorithm}/{\textdollaroldstyle$\nearrow$}/{A \emph{greedy algorithm} is an algorithm that aims to solve a maximum-weight problem by iteratively adding an element $e\in E$ to the set given the new set is still in $\calI$ such that the weight of the element is maximum.},%Grouped
indepset/cop/22/2.5/7 cm/{Independent set, independent number}/{$\fun{\alpha}{G}$}/{A subset $S\subset V_G$ of a graph $G$ is an independent set if every pair of vertices in $S$ is not adjacent. The \emph{(vertex) independent number} $\fun{\alpha}{G}$ of a graph $G$ is the size of the largest independent set of vertices.\importtikzcenter{independent}},
matching/cop/22/8.25/7 cm/{Matching}/{$M$}/{A matching of a graph $G$ is a subset $S$ of $E_G$ such that every pair of edges is not adjacent. A matching that involves all vertices in a set $V$ is called a \emph{$V$-saturated matching}. A \emph{perfect matching} (or \emph{1-factor}) is a matching such that every vertex in $G$ is an endpoint of one of the edges.\importtikzcenter{matching}},
proj/cop/31.5/2/9 cm/{Scheduling problem}/{\clock}/{A \emph{scheduling problem} considers a number of \emph{activities} where certain activities cannot begin until other activities are completed. Such activity $A_i$ that must be completed before one can work on an activity $A_j$ is said to be a \emph{predecessor}. The aim is to schedule the activities such that the \emph{overall completion time} is minimized and to identify the \emph{critical activities}: activities that - if delayed - will increase the overall completion time.},
aoan/cop/40.5/2/7 cm/{Activity-on-arc network}/{AOA}/{In an \emph{activity-on-arc network}, one uses a weighted digraph to represent precedence of activities in a project. Each activity is represented using an arc where the weight of the arc represents the duration of the activity. The vertices represent \emph{events}: the completion of one or more activities and the start of activities that depend on the previous tasks. Since an activity can be involved in multiple subsets of precedence activities, one sometimes introduces a \emph{dummy activity} with weight $0$.\importtikzcenter{aoa}},
trees/tree/4/2.25/7 cm/Tree/$T$/{A \emph{tree} is a connected graph that contains no cycles.},%Grouped
assg/cop/48.5/2/7 cm/{Assignment problem}/{$w_j\leftarrow t_i;$}/{An assignment problem aims to assign \emph{workers} to \emph{tasks} such that every worker works on one task, every task is done by one worker and the total cost of the assignment is minimized. The assignment problem can be solved to optimality with the \emph{Hungarian algorithm}.\frml{\fcop[-]{$\sumieqb[i,j]{1}{n}{c_{i\,j}\cdot x_{i\,j}}$}{$\sumieqb[i]{1}{n}{x_{i\,j}}=1&j=1,\ldots,n$\\&$\sumieqb[j]{1}{n}{x_{i\,j}}=1&i=1,\ldots,n$\\&$x_{i\,j}\in\accl{0,1}&i,j=1,\ldots,n$}}},
sdr/cop/56.25/6.5/7 cm/{System of distinct representatives}/{SDR}/{Given a tuple $\calF=\tupl{F_1,F_2,\ldots,F_n}$ of sets, a tuple $T=\tupl{t_1,t_2,\ldots,t_n}$ is called a \emph{system of representatives} for $\calF$ if for all $i$, $t_i\in F_i$. If all $t_i$ are distinct as well, $T$ is called a \emph{system of distinct representatives} for $\calF$. The problem can be formulated as a bipartite graph where each \emph{representative} $a_i$ is a node in the first bipartition subset, each set $F_i$ is a node in the second bipartition subset and there exists an edge between a representative $t_i$ and a set $F_j$ if $a_i\in F_j$.\importtikzcenter{repr}},
mafl/maflw/4.75/2.25/8.5 cm/{Maximum flow}/{Max-flow}/{Given a weighted graph where the weights represent \emph{capacities} $c_{v\,w}$ over the edges $\tupl{v,w}$. The \emph{maximum flow} problem considers two special vertices in $V_G$ called the \emph{source} $s$ and the \emph{sink} $t$. The aim is to find an assignment of \emph{flows} on to the edges $\funsig{f}{E_G}{\RRR}$ such that \emph{Kirchoff's laws} hold for every vertex that is not a source or sink, each flow is less than the capacity of the corresponding edge, and the total flow that originates from the source is maximized.\frml{\fcop[+]{$\sumdomain[\tupl{s,v}]{E_G}{f_{s\,v}}$}{$\sumdomain[\tupl{w,v}]{E}{f_{w\,v}}=\sumdomain[\tupl{v,w}]{E}{f_{v\,w}}$&$v\in V_G\setminus\accl{s,t}$\\&$-c_{e}\leq f_{e}\leq c_{e}&e\in E_G$}}\importtikzcenter{maxflow}},
dirtree/tree/4/4.75/7 cm/{Directed tree}/{$T^{\rightarrow}$}/{A digraph whose underlying graph is a tree.},%Grouped
rttree/tree/4/7.5/7 cm/{Rooted tree}/{$T^{\rightarrow}_r$}/{A directed tree with a distinguished vertex $r$ called the \emph{root} such that for every other vertex $v$, thee is a directed path from $r$ to $v$.},%Grouped
dptree/tree/4/11.25/7 cm/{Depth, height}/{$\fun{d}{v},\fun{h}{T}$}/{The \emph{depth} or \emph{level} of a vertex $v$ in a rooted tree is the distance from its root to $v$ denoted as $\fun{d}{v}$. The \emph{height} of a rooted tree is the greatest depth of that tree denoted $\fun{h}{T}$.},%Grouped
stdrawtree/tree/21.5/1.75/4 cm/{Standard plane drawing}/{\smallpencil $T$}/{A \emph{standard plane drawing} of a rooted tree is a plane drawing of the tree such that the root is at the top, and the vertices at each level are horizontally aligned.},%Grouped
prnttree/tree/12/8/7 cm/{Parent,child}/{$v_i ? v_j$}/{If a vertex $v$ immediately precedes vertex $w$ on the unique path from the root to $w$ of a rooted tree, then $v$ is the \emph{parent} of $w$ and $w$ the \emph{child} of $v$. Two vertices $v$ and $w$ are \emph{siblings} if they have the same parent in a rooted tree. A vertex that has no children is an \emph{leaf}. A vertex in a rooted tree that is not a \emph{leaf} is an \emph{internal vertex}.},%Grouped
dscasctree/tree/20/8/7 cm/{Descendant,ancestor}/{$v_i ?^{\star} v_j$}/{Vertex $w$ is a \emph{descendant} of vertex $v$ and vertex $v$ is an \emph{ancestor} of $w$ if $v$ is on the unique path from the root to $w$. If in addition $w\neq v$, then $w$ is a \emph{proper descendant} of $v$ and $v$ is a \emph{proper ancestor} of $w$. The two vertices are \emph{related} if one is a descendant of the other.},%Grouped
marytree/tree/44.5/1.75/7 cm/{$m$-ary tree}/{$m$-$T$}/{An \emph{$m$-ary tree} is a tree where every vertex has $m$ or fewer children. A \emph{complete $m$-ary tree} is a tree where every internal vertex has exactly $m$ children and all leaves have the same depth.},%Grouped
ordtree/tree/44.5/5.5/7 cm/{Ordered tree}/{$T^{\leq}$}/{An \emph{ordered tree} is a rooted tree in which the children of each vertex are assigned a fixed ordering.},%Grouped
bintree/tree/44.5/8.5/7 cm/{binary tree}/{bin-$T$}/{A \emph{binary tree} is an ordered $2$-ary tree in which each child is designated either a \emph{left-child} or a \emph{right child}. The \emph{left subtree} of a vertex $v$ in a binary tree is the binary
 subtree spanning the \emph{left-child} and its descendants, analogue we define the \emph{right subtree}.},%Grouped
fronted/tree/28.25/1.75/7 cm/{Frontier edge/arc}/{$T\rightarrow e\rightarrow G$}/{For a given tree $T$ in a graph $G$, a \emph{frontier edge} is a non-tree edge with one endpoint in $T$. A \emph{frontier arc} for a rooted tree in a digraph is an arc whose tail is in $T$ and whose head is not in $T$.},
treegrow/tree/28.25/5.5/7 cm/{Tree-growing algorithm}/{$T\nearrow$}/{A \emph{tree-growing algorithm} is an algorithm that takes as input a graph $G$ and starts with an empty tree $T_0$. In each iteration the $\funm{nextEdge}{G,S}$ and $\funm{updateFrontier}{G,S}$ selects and updates the tree for a given set of frontier edges $S$. The algorithm terminates if all vertices are added to the tree. The \emph{discovery order} is the order in which the vertices of the graph are added to the tree and the \emph{discovery number} is the position of a vertex in the discovery-order list starting with $0$. The edges not added to the tree are either \emph{skip-edges} if the endpoints are not related with respect to the tree; or \emph{cross-edges} if the endpoints are related.},
stree/tree/36/1.75/7 cm/{Spanning tree/forest}/{\spanning{} $T$}/{A spanning tree of a graph $G$ is a tree that contains all vertices of $G$. A \emph{spanning forest} of a graph $G$ is a graph that consists out of one or more trees such that each vertex belongs to one of the trees. A \emph{full spanning forest} of a graph $G$ is a spanning forest consisting of a collection of trees, such hat each tree is a spanning tree of a different component of $G$. A \emph{minimum spanning tree} of a weighted graph is a spanning tree such that the sum of the weight of the edges in the tree is minimum. The minimum spanning tree can be computed in polynomial time with \emph{Kruskal}'s algorithm.},%Grouped
spath/tree/36/9.75/7 cm/{Shortest path}/{$v_i\rightarrowtail^{-} v_j$}/{The shortest path between two vertices $v_i$ and $v_j$ in a weighted graph is a path between $v_i$ and $v_j$ such that the sum of the weight of the edges in the path is minimum.},%Grouped
vexcol/grcl/4/2.5/7 cm/{vertex-coloring}/{$\funsig{f}{V_G}{C}$}/{A \emph{vertex-coloring} for a graph $G$ is a mapping $\funsig{f}{V_G}{C}$ from its vertex-set to a set $C$ whose elements are called \emph{colors}. Typically, the colors are represented by positive integers. An assignment of colors to a proper subset of $V_G$ is called a \emph{partial coloring} and the \emph{color degree} of a vertex $v\in V_G$ is the number of different colors used by the neighbors of $v$ that have been assigned a color. If $C=\accl{1,2,\ldots,k}$, the mapping is called a \emph{$k$-coloring}. An vertex-coloring is \emph{proper} if two adjacent vertices are assigned different colors. If there exists a proper $k$-coloring for a graph $G$, the graph is said to be \emph{(vertex) $k$-colorable}.},
ngbrh/grcl/4/10.25/7 cm/{Open and closed neighborhood}/{$\funf{N_G}{v},\fun{N_G}{v}$}/{Given a graph $G$ with a vertex $v\in V_G$, then the \emph{closed neighborhood} of $v$ denoted $\funf{N_G}{v}$ is the subset $V\subseteq V_G$ of vertices consisting out of $v$ and all its neighbors. The \emph{open neighborhood} denoted $\fun{N_G}{v}$ is the subset consisting of all the neighbors of $v$. The subgraphs induced by $\funf{N_G}{v}$ and $\fun{N_G}{v}$ are called the \emph{closed neighborhood subgraph} and the \emph{open neighborhood subgraph}.},
clrcls/grcl/12/1.5/7 cm/{color class}/{$\calC_i$}/{Given a vertex-coloring of $G$ the subset $V_c\subseteq V_G$ containing all vertices for a given color $c$ is called a \emph{color class}.},
chrmnum/grcl/12/4.75/7 cm/{(Vertex) chromatic number}/{$\fun{\chi}{G}$}/{The \emph{(vertex) chromatic number} of a graph $G$ is the smallest $k$ for which $G$ is $k$-colorable, we say a graph is \emph{$\fun{\chi}{G}$-chromatic}. A graph is \emph{(chromatically) $k$-critical} if $\fun{\chi}{G}=k$ and the \emph{edge-deletion subgraph} $G-e$ is $k-1$-colorable for all $e\in E_G$.},
obstrgr/grcl/12/9/7 cm/{obstruction to $k$-chromaticity}/{$O_k$}/{A graph that forces every graph that contains it to have a chromatic number greater than $k$.},
bincod/cseq/4/2.5/7 cm/{binary code, codeword, prefix code}/{$\funsig{f}{\Sigma}{\accl{0,1}^{\star}}$}/{A \emph{binary code} is an mapping of \emph{symbols} to a set of bitstrings. Each bitstring is referred to as a \emph{codeword}. A \emph{prefix code} is a binary code such that no codeword is an initial substring of any other codeword. One can use a binary tree to represent an encoding by adding the symbols that correspond to a binary path in the leaves. The \emph{average weighted length} (or \emph{expected length}) of the binary code is the sum of the products of the length of each codeword times the relative frequency of the symbol it encodes.\importtikzcenter{binenc}},
bruyn/cseq/20/2.5/7 cm/{deBruijn sequence}/{deBruijn, $D_{2,n}$}/{A \emph{$\tupl{2,n}$-deBruijn sequence} is a bitstring of $2^n$ such that each bitstring of length $n$ occurs exactly once as a substring in the sequence. Wraparound is taken into account. The \emph{$\tupl{2,n}$-deBruijn digraph} denoted $D_{2,n}$ is a digraph with a vertex for each of the $2^{n-1}$ bitstrings and $2^n$ arcs labeled by bitstrings of length $n$. For each bitstring $b_1b_2\ldots b_{n-1}$ there is an arc to bitstring $b_2b_3\ldots b_n$ labeled $b_1b_2\ldots b_n$. Since each vertex has indegree $2$ and outdegree $2$, we can construct a deBruijn sequence by constructing an Eulerian tour.\importtikzcenter{debruijn}}