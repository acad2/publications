legdef/legen/4/2.5/4.75 cm/Definition/Def/{A \emph{definition} is a statement of the meaning of a term.},
grph/graph/4/2.5/7 cm/Graph/{$\fun{G}{V,E}$}/{A \emph{graph} $\fun{G}{V,E}$ consists out of a set of \emph{vertices} $V$ and a set of \emph{edges} $E$ to each edge is associated a set of one or two vertices called its \emph{endpoints}.},
trvg/graph/10/2.5/7 cm/{Trivial graph}/$G_0$/{A \emph{trivial graph} consists out of one vertex and no edges.},
slfl/graph/10/2.5/7 cm/{self-loop}/$e=\tupl{v,v}$/{A \emph{self-loop} is an edge that joins a single endpoint to itself.},
prpe/graph/10/2.5/7 cm/{Proper edge}/$e=\tupl{u,v}$/{A \emph{proper edge} is an edge that is not a self-loop.},
mule/graph/10/2.5/7 cm/{Multi-edge}/$\accl{\tupl{u,v},\tupl{u,v},\ldots}$/{A \emph{multi-edge} is a set of two or more edges with identical endpoints. The \emph{edge-multiplicity} is the number of edges between vertices or the number of self-loops of a vertex.},
simg/graph/10/2.5/7 cm/{Simple graph}/$G$/{A \emph{simple graph} is a graph with no self-loops nor multi-edges.},
fmsp/graph/10/2.5/7 cm/{Formal specification}/$??$/{A \emph{formal specification} of a graph $\fun{G}{V,E}$ consists of an \emph{incidence function} $\mbox{endpts}$, which maps each edge $e$ to a one- or two-elements subset of $V$ called $\funm{endpts}{e}$.},
dirg/graph/10/2.5/7 cm/{Directed graph}/$G^{\rightarrow}$/{A \emph{directed graph} (or \emph{digraph}) is a graph where each of the edges is directed.},
arc/graph/10/2.5/7 cm/{Arc}/$u\rightarrow v$/{An \emph{arc} (or \emph{directed edge}) is an edge from one of the endpoints (the \emph{tail}) to its other endpoint (the \emph{head}).},
mula/graph/10/2.5/7 cm/{Multi-arc}/$\accl{u\rightarrow v,u\rightarrow v,\ldots}$/{A \emph{multi-arc} is a set of two or more arcs with identical endpoints.},
simd/graph/10/2.5/7 cm/{Simple digraph}/$\vec{G}$/{A \emph{simple digraph} is a directed graph with no self-loops nor multi-arcs.},
undg/graph/10/2.5/7 cm/{Underlying graph}/$\underline{G}$/{The \emph{underlying graph} of a digraph is the graph generated by removing all the edge directions from the digraph.},
adjv/graph/10/2.5/7 cm/{Adjacent vertices}/{$v_i\sim v_j$}/{Two vertices joined by an edge. Adjacent vertices are said to be neighbors.},
adje/graph/10/2.5/7 cm/{Adjacent edges}/{$e_i\sim e_j$}/{Two edges with an endpoint in common.},
ince/graph/10/2.5/7 cm/{Incident}/{$v_i\sim e_j$}/{Edge $e$ is incident on vertex $v$ if vertex $v$ is an endpoint of $e$.},
degv/graph/10/2.5/7 cm/{Degree of a vertex}/{$\funm{deg}{v}$}/{The number of proper edges incident on $v$ plus twice the number of self-loops on $v$. The degree is denoted by $\funm{deg}{v}$ or $\fun{\delta}{v}$. $\fun{\submax{\delta}}{G}$ denotes the maximum degree of the vertices of the graph $G$.},
indeg/graph/10/2.5/7 cm/{Indegree}/{$\funm{ideg}{v}$}/{The indegree of a vertex of a graph is the number of arcs with head the given vertex.},
outdeg/graph/10/2.5/7 cm/{Outdegree}/{$\funm{odeg}{v}$}/{The outdegree of a vertex of a graph is the number of arcs with tail the given vertex.},
degreeseq/graph/10/2.5/7 cm/{Degree-sequence}/{$\fun{\vec{\mbox{deg}}}{G}$}/{The degree sequence of a graph is a non-increasing arrangement of the degrees of the vertices of that graph.},
nregg/graph/10/2.5/7 cm/{$n$-regular graph}/{$n$-regular}/{An $n$-regular graph is a graph where every vertex has degree $n$.},
compgr/graph/10/2.5/7 cm/{$K_n$}/{Complete graph}/{A complete graph on $n$ vertices is a simple graph where for every two different vertices, there is an edge.},
bipagr/graph/10/2.5/7 cm/{$\subseteq K_{n,m}$}/{Bipartite graph}/{A bipartite graph is a graph such that the vertex set $V$ can be partitioned in two disjunctive subsets $V_1$ and $V_2$ such that every edge has one endpoint in $V_1$ and one endpoint in $V_2$. $V_1$ and $V_2$ are called the \emph{bipartition subsets}.},
cbipgr/graph/10/2.5/7 cm/{$K_{n,m}$}/{Complete bipartite graph}/{A bipartite graph where every possible edge exists.},
pathgr/graph/10/2.5/7 cm/{$P_n$}/{Path graph}/{A path graph is a simple connected graph such that $\abs{V}=\abs{E}+1$ and can be drawn such that all of the vertices and edges lie on a straight line.},
cyclgr/graph/10/2.5/7 cm/{$C_n$}/{Cycle graph}/{A cycle graph is a simple connected graph such that $\abs{V}=\abs{E}$ and can be draw such that all of its vertices and edges lie on a circle.},
hypergr/graph/10/2.5/7 cm/{$Q_n$}/{Hypercube graph}/{The hypercube graph $Q_n$ is an $n$-regular graph whose vertex-set is the set of all bitstrings of length $n$, and such that there is an edge between two vertices if and only if they differ exactly one bit.},
graycod/graph/10/2.5/7 cm/{}/{Gray code}/{A \emph{gray code} of order $n$ is an ordering of $2^n$ bitstrings of length $n$ such that consecutive bitstrings differ precisely one bit position.},
pvc/graph/10/2.5/7 cm/{}/{Proper vertex coloring}/{A proper vertex coloring of a graph is an assignment of colors to the vertices such that adjacent vertices are assigned different colors.},
walk/graph/10/2.5/7 cm/{$\tupl{v_0,e_1,v_1,\ldots,e_n,v_n}$}/{Walk}/{A walk is an alternating sequence of vertices and edges $\tupl{v_0,e_1,v_1,\ldots,e_n,v_n}$ such that $\funm{endpts}{e_i}=\tupl{v_{i-1},v_i}$ for every $e_i$.},
diwalk/graph/10/2.5/7 cm/{$\tupl{v_0,\vec{e_1},v_1,\ldots,\vec{e_n},v_n}$}/{Directed walk}/{A directed walk is a walk defined on a directed graph such that for every edge $\funm{tail}{e_i}=v_{i-1}$ and $\funm{head}{e_i}=v_i$.},
lnwalk/graph/10/2.5/7 cm/{$\abs{\tupl{v_0,e_1,v_1,\ldots,e_n,v_n}}$}/{Length of a walk}/{The length of a walk is the number of edge steps in the sequence.},
clwalk/graph/10/2.5/7 cm/{$\tupl{v_0,e_1,v_1,\ldots,e_n,v_n=v_0}$}/{Closed walk}/{A closed walk is a walk such that the first and the last vertex of the sequence are the same.},
opwalk/graph/10/2.5/7 cm/{$\tupl{v_0,e_1,v_1,\ldots,e_n,v_n\neq v_0}$}/{Open walk}/{An open walk is a walk such that the first and the last vertex of the sequence or not the same.},
trvwlk/graph/10/2.5/7 cm/{$\tupl{v_0}$}/{Trivial walk}/{A trivial walk is a walk with one vertex and no edges. A trivial walk has length $0$.},
dstnc/graph/10/2.5/7 cm/{$\fun{d}{v_i,v_j}$}/{Distance}/{The distance between two vertices $v_i$ and $v_j$ is the length of the shortest walk from $v_i$ to $v_j$.},
cncwl/graph/10/2.5/7 cm/{$W_1\circ W_2$}/{Concatenation}/{The concatenation of two walks $W_1=\tupl{v_0,e_1,\ldots,e_m,v_m}$ and $W_2=\tupl{v_m,e_{m+1},v_{m+1},\ldots,e_n,v_n}$ is the walk $W_1\circ W_2=\tupl{v_0,e_1,v_1,\ldots,e_n,v_n}$.},
subwl/graph/10/2.5/7 cm/{}/{Subwalk}/{A subsequence of the entries of a walk $W$ that is a walk itself.},
reach/graph/10/2.5/7 cm/{$v_i\rightsquigarrow v_j$}/{Reachable}/{A vertex $v_i$ is reachable from a vertex $v_j$ in a graph (or digraph) if there exists a walk (or directed walk) from $v_i$ to $v_j$.},
conn/graph/10/2.5/7 cm/{$V\rightsquigarrow V$}/{Connected graph}/{A graph is \emph{connected} if for all vertices $v_i,v_j\in V$, $v_i$ is reachable from $v_j$.},
wkconn/graph/10/2.5/7 cm/{$\vec{V}\rightsquigarrow\vec{V}$}/{Weakly connected digraph}/{A digraph is weakly connected if the underlying graph is connected.},
srconn/graph/10/2.5/7 cm/{}/{Strongly connected digraph}/{A digraph is strongly connected if for all vertices $v_i,v_j\in V$, $v_i$ is reachable from $v_j$ and $v_j$ is reachable from $v_i$.},
sror/graph/10/2.5/7 cm/{}/{Strongly orientable}/{A graph is strongly orientable if there exists an assignment of directions to the edges of the graph such that the resulting graph is strongly connected.},
subgr/graph/10/2.5/7 cm/{$\subseteq G$}/{Subgraph}/{A graph $H$ is a subgraph of a graph $G$ (and $G$ is said a \emph{supergraph} of $H$) if $V_H\subseteq V_G$ and $E_H\subseteq G$.},
psubgr/graph/10/2.5/7 cm/{$\subsetneq G$}/{Proper subgraph}/{A graph $H$ is a proper subgraph of a graph $G$ (and $G$ is said a \emph{proper supergraph} of $H$) if $V_H\subsetneq V_G$ and $E_H\subsetenq G$.},
spansubg/graph/10/2.5/7 cm/{}/{Spanning subgraph}/{A graph $H$ is a spanning subgraph of $G$ if $H$ is a subgraph of $G$ and $V_H=V_G$.},
sbgiv/graph/10/2.5/7 cm/{$\fun{G}{V}$}/{Subgraph induced on a vertex subset}/{The subgraph induced on a vertex subset $V\subseteq V_G$ is the subgraph of $G$ with $V_{\fun{G}{V}}=V$ and $E_{\fun{G}{V}}=\condset{e\in E_G}{\funm{endpts}{e}\subseteq V}$.},
sbgie/graph/10/2.5/7 cm/{$\fun{G}{E}$}/{Subgraph induced on an edge subset}/{The subgraph induced on a vertex subset $E\subseteq E_G$ is the subgraph of $G$ with $E_{\fun{G}{E}}=E$ and $V_{\fun{G}{E}}=\condset{v\in V_G}{\exists e\in E:v\in\funm{endpts}{e}}$.},
cliq/graph/10/2.5/7 cm/{}/{Clique}/{A clique of a graph $G$ is a subset $S\subseteq V_G$ such that for every two vertices $v_i,v_j\in S$, there exists an edge between $v_i$ and $v_j$ and no superset $S'\supsetneq S$ has this property.},
vdsg/graph/10/2.5/7 cm/{Vertex-deletion subgraph}/{$G-v$}/{The subgraph induced on the vertex set $V_G\setminus\accl{v}$. That is $V_{G-v}=V_G\setminus\accl{v}$ and $E_{G-v}=\condset{e\in E_G}{v\notin\funm{endpts}{e}}$.},
cliqn/graph/10/2.5/7 cm/{}/{Clique number}/{The clique number $\fun{\omega}{G}$ of a graph $G$ is the number of vertices in the largest clique of $G$.},
indepset/graph/10/2.5/7 cm/{}/{Independent set}/{A subset $S\subset V_G$ of a graph $G$ is an independent set if every pair of vertices in $S$ is not adjacent.},
indepnum/graph/10/2.5/7 cm/{}/{Independent number}/{The (vertex) independent number $\fun{\alpha}{G}$ of a graph $G$ is the size of the largest independent set of vertices.},
matching/graph/10/2.5/7 cm/{}/{Matching}/{A matching of a graph $G$ is a subset $S$ of $E_G$ such that every pair of edges is not adjacent. A \emph{perfect matching} is a matching such that every vertex in $G$ is an endpoint of one of the edges.},
cpn/graph/10/2.5/7 cm/{$\fun{C_G}{v}$}/{Component}/{A component of a graph $G$ is a maximal connected subgraph of $G$. For a given vertex $v$ there is exactly one component containing $v$ denoted by $\fun{C_G}{v}$.},
trl/graph/10/2.5/7 cm/{Trail}/{Trail}/{A trail is a walk with no repeating edges.},
pth/graph/10/2.5/7 cm/{Path}/{Path}/{A path is a trail with no repeating vertices (except the initial and final vertices).},
cycl/graph/10/2.5/7 cm/{Cycle}/{Cycle}/{A cycle is a nontrivial closed path.},
trees/tree/4/2.5/7 cm/Tree/$T$/{A \emph{tree} is a connected graph with no cycles.},
stree/tree/4/2.5/7 cm/{Spanning tree}/$ST$/{A spanning tree of a graph $G$ is a tree that contains all vertices of $G$. A \emph{spanning forest} of a graph $G$ is a graph that consists out of one or more trees such that each vertex belongs to one of the trees. A \emph{full spanning forest} of a graph $G$ is a spanning forest consisting of a collection of trees, such hat each tree is a spanning tree of a different component of $G$.},
edsg/graph/10/2.5/7 cm/{Edge-deletion subgraph}/{$G-e$}/{The subgraph induced on the edge set $E_G\setminus\accl{e}$. That is $V_{G-e}=V_G$ and $E_{G-e}=E_G\setminus\accl{e}$.},
vercut/graph/10/2.5/7 cm/{Vertex-cut}/{}/{A vertex-cut in a graph $G$ is a set $V\subseteq V_G$ of vertices such that $G-V$ has more components than $G$. A vertex-cut is minimal if no proper subset of $V$ is an vertex-cut.},
cutver/graph/10/2.5/7 cm/{Cut-vertex}/{}/{A cut-vertex (or cutpoint) is a vertex-cut consisting of a single vertex.},
edgcut/graph/10/2.5/7 cm/{Edge-cut}/{}/{An edge-cut in a graph $G$ is a set $E\subseteq E_G$ of edges such that $G-E$ has more components than $G$. An edge-cut is minimal if no proper subset of $E$ is an edge-cut.},
cutedg/graph/10/2.5/7 cm/{Cut-edge}/{}/{A cut-edge (or bridge) is an edge-cut consisting of a single edge.},
cycedg/graph/10/2.5/7 cm/{cycle-edge}/{}/{An edge $e$ is called a cycle-edge if $e$ lies in some cycle of that graph.},
weighg/graph/10/2.5/7 cm/{weighted graph}/{}/{A weighted graph is a graph where each edge is assigned a number called its \emph{edge-weight}.},
eulert/graph/10/2.5/7 cm/{Eulerian trail}/{}/{a trail that contains every edge of the graph.},
eulertr/graph/10/2.5/7 cm/{Eulerian tour}/{}/{a closed Eulerian trail.},
eulerg/graph/10/2.5/7 cm/{Eulerian graph}/{}/{a graph that contains an Eulerian tour.},
posttour/graph/10/2.5/7 cm/{Postman tour}/{}/{A \emph{postman tour} in a graph $G$ is a closed walk that uses each edge of $G$ at least once. In a weighted graph, the \emph{optimal postman tour} is a postman tour whose total edge-weight is a minimum. Finding such tour is known as the \emph{Chinese Postman Problem}.},
hamilcyc//graph/10/2.5/7 cm/{Hamiltonian cycle}/{}/{A \emph{Hamiltonian cycle} in a graph is a cycle that contains all the vertices of the graph. A graph is a \emph{Hamiltonian graph} if it contains a Hamiltonian cycle.},
dirtree/tree/10/2.5/7 cm/{Directed tree}/{}/{A digraph whose underlying graph is a tree.},
rttree/tree/10/2.5/7 cm/{Rooted tree}/{$\tupl{T,r}$}/{A directed tree with a distinguished vertex $r$ called the \emph{root} such that for every other vertex $v$, thee is a directed path from $r$ to $v$.},
dptree/tree/10/2.5/7 cm/{Depth}/{}/{The \emph{depth} or \emph{level} of a vertex $v$ in a rooted tree is the distance from its root to $v$.},
hgtree/tree/10/2.5/7 cm/{Height}/{}/{The \emph{height} of a rooted tree is the greatest depth of that tree.},
prnttree/tree/10/2.5/7 cm/{Parent and child}/{}/{If a vertex $v$ immediately precedes vertex $w$ on the unique path from the root to $w$ of a rooted tree, then $v$ is the \emph{parent} of $w$ and $w$ the \emph{child} of $v$.},
slbtree/tree/10/2.5/7 cm/{Siblings}/{}/{Two vertices $v$ and $w$ are \emph{siblings} if they have the same parent in a rooted tree.},
leaftree/tree/10/2.5/7 cm/{Leaf}/{}/{A vertex that has no children is an \emph{leaf}.},
intvtree/tree/10/2.5/7 cm/{Internal vertex}/{}/{A vertex in a rooted tree that is not a \emph{leaf} is an \emph{internal vertex}.},
dscasctree/tree/10/2.5/7 cm/{Descendant and ancestor}/{}/{Vertex $w$ is a \emph{descendant} of vertex $v$ and vertex $v$ is an \emph{ancestor} of $w$ if $v$ is on the unique path from the root to $w$. If in addition $w\neq v$, then $w$ is a \emph{proper descendant} of $v$ and $v$ is a \emph{proper ancestor} of $w$. The two vertices are \emph{related} if one is a descendant of the other.},
stdrawtree/tree/10/2.5/7 cm/{Standard plane drawing}/{}/{A \emph{standard plane drawing} of a rooted tree is a plane drawing of the tree such that the root is at the top, and the vertices at each level are horizontally aligned.},
marytree/tree/10/2.5/7 cm/{$m$-ary tree}/{}/{A tree where every vertex has $m$ or fewer children. A \emph{complete $m$-ary tree} is a tree where every internal vertex has exactly $m$ children and all leaves have the same depth.},
ordtree/tree/10/2.5/7 cm/{Ordered tree}/{}/{A rooted tree in which the children of each vertex are assigned a fixed ordering.},
bintree/tree/10/2.5/7 cm/{binary tree}/{}/{An ordered $2$-ary tree in which each child is designated either a \emph{left-child} or a \emph{right child}. The \emph{left subtree} of a vertex $v$ in a binary tree is the binary subtree spanning the \emph{left-child} and its descendants, analogue we define the \emph{right subtree}.},
vexcol/graph/10/2.5/7 cm/{vertex-coloring}/{}/{A \emph{vertex-coloring} for a graph $G$ is a mapping $\funsig{f}{V_G}{C}$ from its vertex-set to a set $C$ whose elements are called \emph{colors}. Typically, the colors are represented by positive integers. An assignment of colors to a proper subset of $V_G$ is called a \emph{partial coloring} and the \emph{color degree} of a vertex $v\in V_G$ is the number of different colors used by the neighbors of $v$ that have been assigned a color. If $C=\accl{1,2,\ldots,k}$, the mapping is called a \emph{$k$-coloring}. An vertex-coloring is \emph{proper} if two adjacent vertices are assigned different colors. If there exists a proper $k$-coloring for a graph $G$, the graph is said to be \emph{(vertex) $k$-colorable}.},
clrcls/graph/10/2.5/7 cm/{color class}/{}/{Given a vertex-coloring of $G$ the subset $V_c\subseteq V_G$ containing all vertices for a given color $c$ is called a \emph{color class}.},
chrmnum/graph/10/2.5/7 cm/{(Vertex) chromatic number}/{$\fun{\chi}{G}$}/{The \emph{(vertex) chromatic number} of a graph $G$ is the smallest $k$ for which $G$ is $k$-colorable, we say a graph is \emph{$\fun{\chi}{G}$-chromatic}. A graph is \emph{(chromatically) $k$-critical} if $\fun{\chi}{G}=k$ and the \emph{edge-deletion subgraph} $G-e$ is $k-1$-colorable for all $e\in E_G$.},
gruni/graph/10/2.5/7 cm/{Graph union}/{$G\cup H$}/{The \emph{graph union} is the graph whose vertex- and edge-sets are $V_{G\cup H}=V_G\cup V_H$ and $E_{G\cup H}=E_G\cup E_H$.},
grjoin/graph/10/2.5/7 cm/{Graph join}/{$G+H$}/{The \emph{graph join} is the graph obtained from $G\cup H$ by adding edges between every vertex of $G$ and every vertex of $H$.},
nwheel/graph/10/2.5/7 cm/{$n$-wheel}/{$W_n$}/{The \emph{$n$-wheel} is the join $K_1+C_n$ of a single vertex and an $n$-cycle.},
obstrgr/graph/10/2.5/7 cm/{obstruction to $k$-chromaticity}/{}/{A graph that forces every graph that contains it to have a chromatic number greater than $k$.},
ngbrh/graph/10/2.5/7 cm/{Open and closed neighborhood}/{$\funf{N_G}{v},\fun{N_G}{v}$}/{Given a graph $G$ with a vertex $v\in V_G$, then the \emph{closed neighborhood} of $v$ denoted $\funf{N_G}{v}$ is the subset $V\subseteq V_G$ of vertices consisting out of $v$ and all its neighbors. The \emph{open neighborhood} denoted $\fun{N_G}{v}$ is the subset consisting of all the neighbors of $v$. The subgraphs induced by $\funf{N_G}{v}$ and $\fun{N_G}{v}$ are called the \emph{closed neighborhood subgraph} and the \emph{open neighborhood subgraph}.},
fronted/tree/10/2.5/7 cm/{Frontier edge}/{$T\rightarrow e\rightarrow G$}/{For a given tree $T$ in a graph $G$, a \emph{frontier edge} is a non-tree edge with one endpoint in $T$. A \emph{frontier arc} for a rooted tree in a digraph is an arc whose tail is in $T$ and whose head is not in $T$.},
treegrow/tree/10/2.5/7 cm/{Tree-growing algorithm}/{}/{A \emph{tree-growing algorithm} is an algorithm that takes as input a graph $G$ and starts with an empty tree $T_0$. In each iteration the $\funm{nextEdge}{G,S}$ and $\funm{updateFrontier}{G,S}$ selects and updates the tree for a given set of frontier edges $S$. The algorithm terminates if all vertices are added to the tree. The \emph{discovery order} is the order in which the vertices of the graph are added to the tree and the \emph{discovery number} is the position of a vertex in the discovery-order list starting with $0$. The edges not added to the tree are either \emph{skip-edges} if the endpoints are not related with respect to the tree; or \emph{cross-edges} if the endpoints are related.},
linegr/graph/10/2.5/7 cm/{Line graph}/{$\funsig{L}{G}{G}$}/{The \emph{line graph} of a digraph $G$ is the digraph $\fun{L}{G}$ whose vertex-set corresponds to the arc-set of $G$, and an arc-set of $G$, and an arc in $\fun{L}{G}$ is directed from vertex $v$ to $w$ if, in $G$, $\funm{head}{v}=\funm{tail}{w}$.},
foo/graph/10/2.5/7 cm/{Foo}/{Bar}/{Foobar}