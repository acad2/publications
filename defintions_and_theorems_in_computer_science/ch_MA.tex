\chapter{Multiagent Systems}

\begin{defi}[Sequence of decision elements, decision element, participation, positive set, neutral set, negative set, rate of return, degree of secure of rate $Z$, date of knowledge]
The structure decision $P$ of finite set of decision elements $E=\accol{e_1,e_2,\ldots,e_Y}$ is called as \term{sequence of decision elements}: $P=\tuple{\accol{EW^+},\accol{EW^{\pm}},\accol{EW^-},Z,SP,DT}$, Where:
\begin{enumerate}
 \item $EW^+=\tuple{e_0,pe_0},\tuple{e_q,pe_q},\ldots,\tuple{e_p,pe_p}$; couple $\tuple{e_x,pe_x}$, where $e_x\in E$ and $pe_x\in\ccinterval{0}{1}$. Denote a \term{decision element} and \term{participation} this element in set $EW^+$; decision element $e_x\in EW^+$ will be denoted as $e_x^+$; The set $EW^+$ is called \term{positive set}, in other words it is a set of decision elements, about which the agent knows that these elements are in the environment.
 \item $EW^{\pm}=\tuple{e_r,pe_r},\tuple{e_s,pe_s},\ldots,\tuple{e_t,pe_t}$; couple $\tuple{e_x,pe_x}$, where $e_x\in E$ and $pe_x\in\ccinterval{0}{1}$. Denote a decision element and participation this element in set $EW^+$; decision element $e_x\in EW^+$ will be denoted as $e_x^{\pm}$; The set $EW^{\pm}$ is called \term{neutral set}, in other words it is a set of decision elements, about which the agent does not know that these elements are in the environment.
 \item $EW^-=\tuple{e_r,pe_r},\tuple{e_s,pe_s},\ldots,\tuple{e_t,pe_t}$; couple $\tuple{e_x,pe_x}$, where $e_x\in E$ and $pe_x\in\ccinterval{0}{1}$. Denote a decision element and participation this element in set $EW^-$; decision element $e_x\in EW^+$ will be denoted as $e_x^-$; The set $EW^{\pm}$ is called a \term{negative set}, in other words it is a set of decision elements, about which the agent knows that these elements are not in the environment.
 \item $Z\in\ccinterval{0}{1}$ -- \term{rate of return}
 \item $SP\in\ccinterval{0}{1}$ -- \term{degree of secure of rate $Z$}
 \item $DT$ -- \term{date of knowledge}.
\end{enumerate}
\cite{conf/fedcsis/Sobieska-KarpinskaH12}
\end{defi}

\begin{defi}[Profile]
Set of decision elements $E=\accol{e_1,e_2,\ldots,e_Y}$ is given. A \term{profile} $A=\accol{A^{(1)},A^{(2)},\ldots,A^{(M)}}$ is called set of $M$ decisions of finite set of decision elements $E$, such that:
\begin{equation}
\group{
A^{(1)}=\tuple{\accol{EW^+}^{(1)},\accol{EW^{\pm}}^{(1)},\accol{EW^-}^{(1)},Z^{(1)},SP^{(1)},DT^{(1)}}\\\\
A^{(2)}=\tuple{\accol{EW^+}^{(2)},\accol{EW^{\pm}}^{(2)},\accol{EW^-}^{(2)},Z^{(2)},SP^{(2)},DT^{(2)}}\\\\
\vdots\\\\
A^{(M)}=\tuple{\accol{EW^+}^{(M)},\accol{EW^{\pm}}^{(M)},\accol{EW^-}^{(M)},Z^{(M)},SP^{(M)},DT^{(M)}}
}
\end{equation}
\cite{conf/fedcsis/Sobieska-KarpinskaH12}
\end{defi}

\section{Negotiation and Task Sharing}

\begin{defi}[Deal]
A \term{deal} is a division of $L_A\cup L_B$ to two disjoint subsets, \tuple{D_A,D_B} such that $D_A\cup D_B=L_A\cup L_B$ and $D_A\cap D_B=\emptyset$. This deal means that each agent $i$ agrees to deliver all the letters in $D_i$.
\cite{conf/ijcai/ZlotkinR89}
\end{defi}

\begin{defi}[Utility for an agent]
If \tuple{D_A,D_B} is a deal, then
\begin{equation}
\fun{\mbox{Utility}_i}{D_A,D_B}=\funm{Cost}{L_i}-\funm{Cost}{D_i}
\end{equation}
In other words, the \term[Utility for an agent]{utility for agent $i$} is the difference between the cost of achieving the goal alone and the cost of his part of the deal.
\cite{conf/ijcai/ZlotkinR89}
\end{defi}

\begin{defi}[Negotiation strategy]
A \term{negotiation strategy} is a function from the history of the negotiation to the current message (offer) that is consistent with the negotiation protocol.
\cite{conf/ijcai/ZlotkinR89}
\end{defi}

\begin{defi}[Rational negotiation strategy]
Agent $A$ will be said to be using a \term{rational negotiation strategy} if at any step $t+1$ that $A$ sticks to his last offer $\funm{Risk}{A,t}>\funm{Risk}{B,t}$
\cite{conf/ijcai/ZlotkinR89}
\end{defi}

\begin{defi}[Sufficient concessions, Minimal sufficient concession]
\funm{SC}{A,t} is the set of all the \term{sufficient concessions} of $A$ at step $t$. That is, if agent $A$ offers the ``next step'' a deal from \funm{SC}{A,t}, then if $B$ does not make a concession in the same ``next step'', $B$ will have to do so in the step after that (assuming that $B$ is using a rational strategy). $\delta^*$ will be a \term{minimal sufficient concession} of $A$ in step $t$ if $\fun{\mbox{Utility}_B}{\delta^*}=\displaystyle\min_{\delta\in\funm{SC}{A,t}}\fun{\mbox{Utility}_B}{\delta}$.
\cite{conf/ijcai/ZlotkinR89}
\end{defi}

\begin{defi}[Negotiation strategy in equilibrium]
A negotiation strategy $s$ will be in \term[Negotiation strategy in equilibrium]{equilibrium} if the following condition holds: under the assumption that $A$ uses $s$, $B$ prefers $s$ to any other strategy.
\cite{conf/ijcai/ZlotkinR89}
\end{defi}

\begin{defi}[Extended Zeuthen strategy]
The \term{extended Zeuthen strategy} will be the Zeuthen strategy, plus the ``last step equilibrium strategy'' in last step situations.
\cite{conf/ijcai/ZlotkinR89}
\end{defi}

\begin{defi}[Mixed deal]
If \tuple{D_A,D_B} is a deal and $0\leq p\leq 1$, $p\in\RRR$, then \flatbrak{\tuple{D_A,D_B}:p} will be a \term{mixed deal}. The meaning of such a deal is that the agents will perform \tuple{D_A,D_B} with probability $p$, or \tuple{D_B,D_A} with probability $1-p$.
\cite{conf/ijcai/ZlotkinR89}
\end{defi}

\begin{defi}[Cost for an agent with a mixed deal, Utility for an agent with a mixed deal]
If \flatbrak{\tuple{D_A,D_B}:p} is a mixed deal then \term[Cost for an agent with a mixed deal]{$\fun{\mbox{Cost}_i}{\flatbrak{\tuple{D_A,D_B}:p}}=p\cdot\funm{Cost}{D_i}+\brak{1-p}\cdot\funm{Cost}{D_j}$}. If \flatbrak{\delta:p} is a mixed deal then \term[Utility for an agent with a mixed deal]{$\fun{\mbox{Utility}_i}{\flatbrak{\delta:p}}=\funm{Cost}{L_i}-\fun{\mbox{Cost}_i}{\flatbrak{\delta:p}}$}.
\cite{conf/ijcai/ZlotkinR89}
\end{defi}