\documentclass[titlepage]{book}

\usepackage{fullpage,importsreferences,amsmath,amsthm,amsfonts}

\title{Optimization: Special Topics}
\author{Frits Spieksma\and Edited by: Jo Devriendt \and Willem Van Onsem}
\date{February, 2013\\Edited: February 2014}
\begin{document}
\begin{titlepage}
\maketitle
\end{titlepage}
\tableofcontents
\chapter*{Preface}
These notes are lecture notes used for the course ``Optimization: special topics''. Its subject is mainly combinatorial optimization with an emphasis on modeling issues and solution strategies. Its audience is students, more specifically: master students in engineering or management or economic programs.
\paragraph{}
The course assumes that the reader has had a first introduction to operations research and has some elementary knowledge of mathematical modeling, linear programming, and graph theory.
\paragraph{}
There are some excellent textbooks on combinatorial optimization. We mention Schrijver's trilogy: ``Combinatorial Optimization. Polyhedra and Efficiency'' \cite{schrijver-book}. Another example is: ``Combinatorial optimization'' by Cook, Cunningham, Pulleyblank, and Schrijver \cite{Cook:98}. Other related textbooks are: Nemhauser and Wolsey \cite{citeulike:2212037}, Wolsey \cite{wolseyip}.
\paragraph{}
Please notice that Chapter 3 is written by Prof. Yves Crama.

\chapter{Formulations in combinatorial optimization}
In this first chapter, we will give an informal notion of combinatorial optimization problems. The defining
characteristic of a combinatorial optimization problem is that it has a finite number of feasible solutions.
We now give formulations of a number of basic combinatorial optimization problems.

\section{Formulations of basic combinatorial optimization problems}

Below, we present mathematical formulations of some well-known combinatorial optimization problems.
Some of these problems can be formulated in a straightforward way. Others, however, have more complicated formulations. Among the latter ones are problems that can be solved very easily by a greedy
algorithm, like the spanning tree problem. Other problems are notoriously hard to solve. This already
shows that a formulation need not give insight into the problem's difficulty. Moreover, a problem may
have several correct formulations. This leads us to a very important question: which formulations are
good and which are bad? Clearly, an answer to this question depends on one's goal; we will adopt here
a solver's point of view. Thus, a formulation is better than another formulation when it leads to better
solutions, or when it produces solutions faster.

Almost all formulations will use binary or integral variables. This shows that integer programming
is closely related to (or, as a matter of fact, is itself a problem from the domain of) combinatorial
optimization.

\subsection{The Matching Problem}

The Matching Problem (also known as Edge Packing) is one of the fundamental problems in Combinatorial
Optimization. It is described as follows. We are given an arbitrary graph G = (V, E). A subset of the
edges E ′ ⊆ E is called a matching (or an edge packing) if each vertex of V is incident to at most one edge
of E ′ . In other words, E ′ is a matching if no two edges of E ′ have a vertex in common. The problem is
to find a matching in G consisting of as many edges as possible (a maximum cardinality matching).

Figure 1.1: The red edges form a matching.

A matching is called maximum when no matching of larger cardinality exists. A matching is called
maximal when it cannot be enlarged.
The matching problem can be formulated as an integer program as follows, where we use the symbol
δ(v) to denote the set of edges that are incident to vertex v ∈ V . For instance, in Figure 1.1, δ(7) =
{{3, 7}, {4, 7}, {7, 8}}. We define a 0-1 variable for each edge e ∈ E as follows:

%TODO

if edge e is selected in the matching;
otherwise.

And here is the integer program:

%TODO

The weighted matching problem is a generalization of the cardinality matching problem above by assuming that there is a given weight function w defined on the edges. Then, the objective is to find a
maximum weight matching, and the objective is changed accordingly to max

%TODO

we xe . 

(Obviously, it is a generalization, since the cardinality matching problem arises when we = 1 for each e ∈ E). In
the perfect matching problem the set of feasible solutions is restricted to perfect matchings. These are
matchings such that each vertex is incident to precisely one edge in the matching; then, constraints (1.2)
become equalities. Notice that a perfect matching may not exist (consider e.g. a triangle), whereas there
is always a feasible solution to (1.1)-(1.3). %TODO:REF
We emphasize that we distinguish between, on the one hand, the problem itself, and, on the other hand,
its formulation as an integer program. Indeed, these two are not the same! In fact, in some cases
it is appropriate to show the correctness of a formulation. Such an argument is usually based on a
correspondence between feasible solutions to the problem, and vectors of decision variables satisfying the
constraints.
Let us illustrate this matter for the matching problem. Notice that there is a 1-1 correspondence between
subsets of the set of edges E and the 0-1 vectors defined by the variables, which are indexed by the edges.
To prove that this formulation is correct we must show that there is a 1-1 correspondence between the
subsets of the edges that define matchings and the feasible solutions of the above formulation. Moreover,
we must show that the matching and its corresponding vector have the same value. This can be done as
follows. First, consider an arbitrary matching M in a graph. By definition, this implies that each node
v ∈ V of the graph is incident with at most one edge of M . Let us now construct a solution vector xM in
a straightforward manner: we put a “1” in xM when the corresponding edge is in M , and otherwise we

put a “0” in the vector xM . Clearly, xM is a 0-1 vector, and obviously satisfies (1.3). Also, the solution
vector xM corresponding to M satisfies the constraints (1.2) since at most one of the variables in the
left-hand side has value 1. Thus, a matching M corresponds to a feasible solution of the integer program.
Second, consider a subset of the edges E ′ that is NOT a matching. Then there is a vertex, say v, that
is incident to at least two edges in E ′ . But then, at least two of the variables in the left-hand side of
(1.2) have value 1 for this particular vertex. Thus, the vector corresponding to E ′ is not feasible in the
formulation. Finally, we observe that the value of each set E ′ ⊆ E is equal to its number of edges, i.e.,

%TODO

In general, it may not be trivial to prove the 1-1 correspondence of feasible solutions of a combinatorial
problem to solutions of its formulation, i.e., solutions satisfying the constraints. For many problems,
however, a correctness proof is omitted because the problem is an (extension of) a well-known problem,
and the correctness of a formulation is evident.

\subsection{The Independent Set Problem}

Another basic problem within the field of combinatorial optimization is the Independent Set problem
(also known as Stable Set, or as Node Packing). It can be described as follows.
Consider an arbitrary graph G = (V, E). A subset of the vertices V ′ ⊆ V is called an independent set
(or a stable set, or a node packing) if each edge of E is incident to at most one vertex of V ′ . In other
words, V ′ is an independent set if no two vertices of an edge are selected both. The problem is to find an
independent set in G containing as many vertices as possible (a maximum cardinality independent set).
The problem can be formulated as an integer program as follows. We define a 0-1 variable for each edge
v ∈ V as follows:

And here is the integer program:

%TODO

Figure 1.2: A stable set.

%TODO

Spanning forest

Consider an undirected graph G = (V, E). A subset of the edges E ′ ⊆ E is called a forest if the subgraph
of G induced by E ′ is acyclic, see Figure 1.3 for an example. In case a forest consists of |V | − 1 edges it
is called a tree. The problem is to find a maximum weight forest in G.
The problem to find a maximum weight forest can be formulated as an integer program as follows. We
define a 0-1 variable for each edge e ∈ E as follows:

%TODO

Figure 1.3: The edges in red form a forest

%TODO

The constraints (1.8) limit the number of chosen edges with both endvertices in any given subset V ′ of
the nodes to at most |V ′ | − 1. Since a cycle contains as many edges as vertices, the constraints (1.8)
prevent the graph (V, E ′ ) from containing cycles.
Notice that the number of subsets of V of size 2 or bigger is 2|V | −|V |−1. Thus, the number of constraints
(1.8) is exponential in the size of the problem. Thus, the size of the formulation is exponential in the size
of the input, although the problem is trivially solvable by a greedy algorithm.
The spanning tree problem is a variant of the spanning forest problem, where the set of feasible solutions is
restricted to those acyclic subgraphs that are connected. It is well-known that for an acyclic subgraph the
requirement of connectedness is equivalent to the requirement of having |V | − 1 edges, i.e., |E ′ | = |V | − 1.
Thus, by adding the constraint

e

xe = n − 1 to (1.7)-(1.9), a correct formulation of the minimum weight

spanning tree problem arises.

\subsection{The Knapsack Problem}

We are given a set of n items, each with a weight aj and a value cj for j = 1, . . . , n. Feasible solutions
are the subsets of the set of items with cumulative weight at most b. The objective is to find a feasible
solution of maximum value. The problem formulation contains binary variables xj which indicate whether
element j with j = 1, . . . , n is in the knapsack:

%TODO

And here is the integer program:

%TODO

In this text, we will come back extensively to the knapsack problem. There is a book devoted to this
problem, see Kellerer et al. (5).

\section{Formulations and difficulty}

Does the formulation of a problem tell us anything about the problem's difficulty? The answer is no,
it doesn't. Consider for instance the matching problem (Section 1.1) and the stable set problem (Section 1.2). These problems look similar, since the roles of the edges and vertices are interchanged. However,
there is a striking difference between them. The matching problem can be solved in polynomial time, but
the node packing problem is NP-hard (see Chapter 3). In other words, whereas, for the matching problem, fast and efficient algorithms exist, and have been designed, no such algorithms have been found for
the node packing problem. Indeed, this does not rule out the possibility that fast algorithms could exist
for node packing, however, no one has ever found such an algorithm. In fact, it is widely suspected that
such algorithms do not exist, but a proof of this hypothesis is lacking. All this boils down to the famous,
1 million-dollar worth, P = NP question. In practice, this means that we can find an optimal solution

to a matching problem on a graph with, say 10.000 nodes and 50.000 edges within seconds, while there
is no algorithm known that would return an optimal solution with one hour for the stable set instance
on the same graph. Therefore: a formulation does not give an indication of the solvability of a problem.
Also, the number of variables and/or constraints is no clue concerning the difficulty of a problem. For
instance, the “natural” formulation of the minimum spanning tree problem (see Section 1.1.3) has an
exponential number of constraints, while the problem is easily solvable by a greedy algorithm.

\section{Multiple formulations of a combinatorial optimization problem}

\subsection{Traveling Salesman Problem}

Probably the most well-known problem in Combinatorial Optimization is the Traveling Salesman Problem
(TSP). The TSP is the prototype combinatorial optimization problem. No other problem has received as
much attention as the TSP, and no other problem has captured the imagination as an easy-to-describe,
yet hard-to-solve problem. A description of the problem is as follows. Given is a set of n “cities” and a
distance ci,j between each pair of them. The goal is to find a tour of minimum length, that is to start in
some city, visit each other city once, and to return to the city where the tour was started. More formally,
given an n×n matrix C = cij , find a permutation π of {1, 2, . . . , n} such that

n−1
i=1 cπ(i),π(i+1) +cπ(n),π(1)

is minimum.
Different formulations of the TSP exist. Here is a conventional one, using binary variables xij indicating
whether city j is visited directly after city i:

%TODO

Constraints (1.16) are called the subtour elimination constraints. An equivalent way of writing them is:
xij ≥ 1 for all nonempty S ⊂ V.
i∈S j ∈S
/

Notice that this formulation has a polynomial number of variables (n2 ), and an exponential number of
constraints (O(2n )). The latter fact might be considered a disadvantage of formulation (1.13) - (1.17).
There is, however, an alternative for this formulation which uses additional real variables ui , one such
variable for each city i. The interpretation of this variable is the position of city i in the tour, while
putting the position of city 1 first. Next, by replacing constraints (1.16) by the following constraints:

%TODO

we arrive at a formulation that is called the Miller-Tucker-Zemlin (MTZ) formulation of the TSP. It is
an interesting exercise to verify the correctness of the MTZ-formulation. More concrete: why is it that
(1.18) exclude subtours? The answer lies in noticing that (i) any solution satisfying (1.14), (1.15), and
(1.17) consists of a collection of subtours (or a feasible solution, that is, a single tour). In case there are
subtours, then there is a subtour that does not contain city 1. Let us now, for each pair of consecutive
cities i and j in this subtour, sum the corresponding inequalities (1.18). The u-variables will cancel out,
and the resulting lefthand side will be larger than the resulting righthand side, which means that this
subtour will be forbidden by (1.18). Thus, any subtour not containing city 1 will be forbidden, and hence,
the only possible solution is a single tour.
There is more than one book devoted to the TSP. We mention: Applegate et al. (1) and Lawler et al. (6).
We will end this section with a problem for which we have two natural formulations. Both formulations
have the same sets of variables, but they have different sets of constraints. Later we will see different
formulations of problems where the sets of variables differ.

\subsection{Uncapacitated Facility Location}

We are given a set of m facilities and n clients. Let us call the set of facilities M ≡ {1, 2, . . . , m}, and
let us call the set of clients N ≡ {1, 2, . . . , n}. Each of the facilities can (but need not be) be opened to
11

serve clients. Each client must be served by a facility. The cost for opening facility i is fi , i ∈ M ; the
cost for serving client j by facility i is cij , i ∈ M, j ∈ N . This problem can be formulated in two ways
with the following sets of variables, defined for each i ∈ M, j ∈ N :

%TODO

The first formulation makes use of the fact that there is an upper bound on the number of clients that
are served by a facility, namely the total number of clients n.

%TODO

In the second formulation, each of the constraints (1.22) is disaggregated into n new constraints, leading
to constraints (1.26).

%TODO

Which of these formulations is preferable is not a matter of comparing the number of constraints or
variables. In fact, as will be shown in the sequel, large formulations with many constraints and/or
variables are usually better from a solver's point of view. This depends on the techniques that are

used to solve the problem. Since these techniques very often rely heavily on linear programming, the
quality of the formulation depends almost always on the accuracy of the linear programming relaxation,
i.e., the problem which results when the integrality constraints are removed. For the uncapacitated
facility location problem formulation (UFL2) is better than (UFL1), since the constraints (1.26) imply
the constraints (1.22).

\section{Combinatorial Optimization: a general formulation}

In this section we (informally) argue that each combinatorial optimization problem can be formulated
as an integer program. In a combinatorial optimization problem a finite ground set E is given. To each
element e ∈ E a weight we is attached. A family S of subsets of E is identified as the set of feasible
solutions. This family depends on the particular problem. The weight of a set E ′ ⊆ E is the cumulative
weight of its elements, i.e., w(E ′ ) =

%TODO

we . The associated optimization problem is to find the

maximum (or minimum) weight feasible solution E ′ ∈ S, i.e.,
max {w(E ′ )}

E ′ ∈S

The set of feasible solutions S is usually given implicitly. It is described by the properties of feasible
solutions; it may be very large. For instance, in case of the matching problem, the ground set equals
the set of edges, and the set S is the collection of edge-sets that are matchings.In case of the knapsack
problem, the ground set equals the set of items, and the set S equals the collection of item-sets that can
be put together in the knapsack.
Many examples of problems that fit in the above formulation are found in graph theory (see the first
section). Among them are well-known problems like the shortest path problem, the minimum spanning
tree problem, and the traveling salesman problem. The shortest path problem is defined as follows. In a
graph G = (V, E) the feasible solutions are the subsets of the edges that form paths between two specified
vertices s and t. Among the paths between s and t we want to find the one with a minimum number of
edges, or if a length function is given on the edges, we want to find a path of minimum total length. In
the minimum spanning tree problem a graph G = (V, E) is given together with a weight function on the
edges. A feasible solution is a set of edges that forms a tree. Among the trees we want to find one with
minimum weight. This problem is easily solvable by a greedy algorithm, as is well known. However, if

we restrict the set of feasible solutions to trees that form paths, the problem becomes the Hamiltonian
path problem, which is highly intractable. Thus, in general, problems do not become simpler when the
set of feasible solutions is reduced.
To formulate a combinatorial optimization problem in mathematical terms, we introduce decision variables for all elements of the ground set E. Each decision variable denotes a choice, namely whether the
corresponding element is chosen or not. So, a variable can have two values, which are usually taken from
{0, 1}. A set E ′ ⊆ E can be described by a binary vector xE ′ = (xe )e∈E with n = |E| components as
follows:

%TODO

if element e is in E ′ ;
otherwise.

The finite set of vectors X ⊆ IRn corresponding to feasible solutions from S can then be described by
means of constraints. In many cases, these constraints are linear and involve binary variables. The
objective function w is usually a linear function of the components of x ∈ X. The problem is then

max{wx | Ax ≤ b , x ∈ {0, 1}n }
where A and b depend on the problem at hand. This type of formulation is often called an Integer (or
Binary) Linear Program, (ILP).

\section{General Combinatorial Optimization Problems}

In the first section, we introduced binary variables to model decisions of the yes-no type, more specifically,
to decide whether an element is in a set or not. These variables were used further to describe the
constraints that determine the feasible solutions. In all the examples these constraints could be written
as linear functions with a bound imposed on them. Similarly, the objective could be formulated as a linear
function of the variables. In this section we will generalize the formulations of combinatorial problems,
with respect to all three items, i.e., the variables, the constraints, and the objective.
The decisions in combinatorial problems may be more complicated than simple yes/no decisions. One
may have to introduce integer variables or even real variables, like in linear programming, to model

certain decisions. And, of course, combinations of these types of variables are possible in a single problem
formulation. The combinatorial nature (finite or countable number of feasible solutions) may not be so
evident in these problems. However, the number of “interesting” feasible solutions is usually still finite
in such problems. For instance, in linear programming, the interesting feasible solutions are the vertices
of a polyhedron. The number of vertices is usually finite.
One may consider any function of the variables with a bound imposed on it as a constraint. Moreover,
logical compositions of constraints, like implications and disjunctions (logical ”or”) can be used to model
the restrictions of a problem. In abstracto, any relation on the variables that restricts the set of feasible
solutions can be used as a constraint.
The objective of a problem can be any function of the variables, thus it is not restricted to linear functions.
The above observations lead to the following abstract formulation of optimization problems. We distinguish between the real variables, denoted by the vector x, and the integral variables, denoted by the
vector y. A formulation of a combinatorial optimization problem contains the following items:
- a vector of n real decision variables:
x = (x1 , . . . , xn )
- a vector of m integral decision variables:
y = (y1 , . . . , ym )
- a set C of k constraints:
C = {C1 , . . . , Ck }
- an objective function on the variables:
f (x; y).
The set of feasible solutions consists of vectors (x; y) that satisfy all the constraints. Each variable has a
domain D, usually the set of real numbers IR, or the set of integer ZZ. The solution space of the problem
is the Cartesian product of the domains of the variables, i.e., IRn × ZZ m .
For many solution techniques, especially the ones that we are going to discuss, it is of eminent importance
to restrict the domains of the variables as much as possible. Some constraints imply lower and upper

bounds on the value of a variable, directly or indirectly. For instance, binary variables have an explicit
lower and upper bound. If this is the case, we usually take this into account in the problem formulation
explicitly, i.e., if a real variable xi has a lower bound li and an upper bound ui , then we describe its
domain as [li , ui ]; if yj is an integral variable, with a lower bound li and an upper bound ui , then we
describe its domain as {lj , . . . , uj }.

\section*{Exercises}
Exercise 1
Consider the stable set problem on an undirected graph G = (V, E). Show that the formulation (1.4)-(1.6)
is a correct formulation of the stable set problem.
Exercise 2
Consider the matching depicted in Figure 1.1. Is it maximal? Is it maximum?
Exercise 3
Consider the stable set depicted in Figure 1.2. Is it maximal? Is it maximum?
Exercise 4
Consider an undirected graph G = (V, E). A edge cover E ′ is a subset of the edges such that each node is
incident to at least one edge in E ′ . Formulate the problem of finding a minimum cardinality edge cover
as an integer linear program.
Exercise 5
Consider an undirected graph G = (V, E). A node cover V ′ is a subset of the nodes such that each edge
is incident to at least one node in V ′ . Formulate the problem of finding a minimum cardinality node
cover as an integer linear program.
Exercise 6
A clique partitioning of an undirected, complete graph G is a partitioning of the vertices into subsets
V1 , V2 , . . . , Vk such that the subgraph induced by each Vi is a complete graph itself (i = 1, . . . , k). Consider
now a complete graph G and an arbitrary weight wij for each edge of the graph (notice that wij can be
negative; in fact, the problem is only interesting when there are both positive and negative edge weights).
The objective is to find a clique partitioning of maximal weight, that is, a clique partitioning such that
the cumulative weight of the edges that have both vertices in one and the same component is maximum.
Formulate this problem as an integer linear programming problem. (Hint: use a variable for each edge).
Exercise 7
Consider the TSP.

• When the distance matrix C is known to be symmetric, can you simplify formulation (1.13) - (1.17)

by “merging” constraints (1.14) and (1.15)?
• Which of the two formulations given in Section 1.3.1 is stronger?
Exercise 8
Given is a set of axis-aligned rectangles in the plane. Each of these rectangles needs to be stabbed, either
by a horizontal line or by a vertical line. The problem is to find a set horizontal and vertical lines, stabbing
each rectangle at least once, with minimum cardinality. We call this the rectangle stabbing problem.
Formulate this problem as an integer linear programming problem. (Hint: first, give a formulation for
the instance depicted in Figure 1.4).
Exercise 9
Given an instance of the rectangle stabbing problem as depicted in Figure 1.4, what fractional solution
is the optimal linear programming relaxation (of the integer program that you just wrote down) for this
instance?

Figure 1.4: A rectangle stabbing instance

\bibliographystyle{alpha}
\bibliography{biblio}

\end{document}