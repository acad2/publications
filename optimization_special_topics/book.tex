\documentclass[titlepage]{book}

\usepackage{fullpage,importsreferences-en,brackets-en,amsmath,amsthm,amssymb,amsfonts,tikz,makeidx,glossaries,hyperref,enumitem}
\usetikzlibrary{shapes}

\makeindex
\makeglossary

\newcommand{\conceptsee}[2]{\toindex{#1}\indexlayout{#1}}
\newcommand{\concept}[2][]{\ifthenelse{\equal{#1}{}}{\toindex{#2}\indexlayout{#2}}{\toindex{#1}\indexlayout{#2}}}
\newcommand{\concepts}[1]{\concept[#1]{#1s}}

\newcommand{\indexlayout}[1]{\emph{#1}}

\makeatletter
\newcommand\toindex{\@ifstar{\@dblarg{\@toindexs}}{\@toindex}}
\def\@toindexs[#1]#2{\index{#1@#2}}
\newcommand\@toindex[2][]{%
  \if\relax\detokenize{#1}\relax
    %
    \begingroup
    \@splitword#2\@nil%
    \uppercase\expandafter{%
      \expandafter\def\expandafter\@initial\expandafter{\@first}}%
    \toks0=\expandafter{\@initial}%
    \toks2=\expandafter{\@rest}%
    \edef\x{\endgroup\noexpand\index{\the\toks0 \the\toks2 }}\x
  \else
    \index{#1}
  \fi
}
\def\@splitword#1#2\@nil{\def\@first{#1}\def\@rest{#2}}
\makeatother

\title{Optimization: Special Topics}
\author{Frits Spieksma\\ \\Edited by:\\Jo Devriendt\\Willem Van Onsem}
\date{February, 2013\\Edited: February 2014 - June 2014}

\theoremstyle{plain}
\newtheorem{theorem}{Theorem}[chapter]
\newtheorem{corollary}[theorem]{Corollary}
\newtheorem{conjecture}[theorem]{Conjecture}
\theoremstyle{definition}
\newtheorem{exercise}{Exercise}[chapter]
\newtheorem{definition}[theorem]{Definition}
\newtheorem{lemma}[theorem]{Lemma}
\newtheorem{proposition}[theorem]{Proposition}
\theoremstyle{remark}
\newtheorem{example}{Example}
\newtheorem{application}{Application}
\newtheorem{answer}{Answer}
\newtheorem{hint}{Hint}
\newtheorem{note}{Note}

\begin{document}
\begin{titlepage}
\maketitle
\end{titlepage}
\tableofcontents

\chapter*{Preface}
These notes are lecture notes used for the course ``Optimization: special topics''. Its subject is mainly \concept{combinatorial optimization} with an emphasis on modeling issues and solution strategies. Its audience is students, more specifically: master students in engineering or management or economic programs.

\paragraph{}
The course assumes that the reader has had a first introduction to operations research and has some elementary knowledge of mathematical modeling, linear programming, and graph theory.

\paragraph{}
There are some excellent textbooks on combinatorial optimization. We mention Schrijver's trilogy: ``Combinatorial Optimization. Polyhedra and Efficiency'' \cite{schrijver-book}. Another example is: ``Combinatorial optimization'' by Cook, Cunningham, Pulleyblank, and Schrijver \cite{Cook:98}. Other related textbooks are: Nemhauser and Wolsey \cite{citeulike:2212037}, Wolsey \cite{wolseyip}.

\paragraph{}
Please notice that \chpref{complexity} is written by Prof. Yves Crama.

\chapter{Formulations in combinatorial optimization}
\chplab{formulations}
In this first chapter, we will give an informal notion of \concepts{combinatorial optimization problem}. The defining characteristic of a \concept{combinatorial optimization problem} is that it has a finite number of \concepts{feasible solution}. We now give \concepts{formulation} of a number of basic \concepts{combinatorial optimization problem}.

\section{Formulations of Basic Combinatorial Optimization Problems}
\seclab{formulations}
Below, we present mathematical formulations of some well-known \concepts{combinatorial optimization problem}. Some of these problems can be formulated in a straightforward way. Others, however, have more complicated formulations. Among the latter ones are problems that can be solved very easily by a \concept{greedy algorithm}, like the \concept{spanning tree} problem. Other problems are notoriously hard to solve. This already shows that a \concept{formulation} need not give insight into the problem's difficulty. Moreover, a problem may have several correct \concepts{formulation}. This leads us to a very important question: which \concepts{formulation} are good and which are bad? Clearly, an answer to this question depends on one's goal; we will adopt here a \concept{solver}'s point of view. Thus, a \concept{formulation} is better than another \concept{formulation} when it leads to better \concepts{solution}, or when it produces \concepts{solution} faster.

\paragraph{}
Almost all \concepts{formulation} will use \concept[binary variable]{binary} or \concepts{integral variable}. This shows that \concept{integer programming} is closely related to (or, as a matter of fact, is itself a problem from the domain of) \concept{combinatorial optimization}.

\subsection{The Matching Problem}
\ssclab{matching}
The \concept{Matching} problem (also known as the \conceptsee{Edge packing}{Matching} problem) is one of the fundamental problems in Combinatorial Optimization. It is described as follows.

\begin{definition}[Matching problem]
We are given an arbitrary \concept{graph} $G=\tupl{V,E}$. A subset of the \concepts{edge} $E'\subseteq E$ is called a \concept{matching} (or an \concept{edge packing}) if each \concept{vertex} of $V$ is \concept{incident} to at most one \concept{edge} of $E'$. In other words, $E'$ is a \concept{matching} if no two edges of $E'$ have a \concept{vertex} in common. The problem is to find a \concept{matching} in $G$ consisting of as many \concepts{edge} as possible (a \concept{maximum cardinality matching}).
\end{definition}

\importtikzfigure{matching-red}{The red edges form a matching.}

\paragraph{}
A matching is called \concept[maximum solution]{maximum} when no matching of larger cardinality exists. A matching is called \concept[maximal solution]{maximal} when it cannot be enlarged.

\paragraph{}
The \concept{matching problem} can be formulated as an \concept{integer program} as follows, where we use the symbol $\fun{\delta}{v}$ to denote the set of edges that are incident to vertex $v\in V$. For instance, in \figref{matching-red}, $\fun{\delta}{v}=\accl{\tupl{3,7},\tupl{4,7},\tupl{7,8}}$. We define a \concept{0-1 variable} for each edge $e\in E$ as follows:
\begin{equation}
\semboolvar{x_e}{if edge $e$ is selected in the matching;}{otherwise.}
\end{equation}
And here is the integer program:
\begin{eqnarray}
\mbox{maximize}&\sumdomain[e]{E}{x_e}\eqnlab{matching-m}\\
\mbox{subject to}&\forall v\in V:\sumdomain[e]{\fun{\delta}{v}}{x_e}\leq 1\eqnlab{matching-c1}\\
&\forall e\in E:x_e\in\accl{0,1}\eqnlab{matching-c2}
\end{eqnarray}

\paragraph{}
The \concept{weighted matching} problem is a generalization of the \concept{cardinality matching problem} above by assuming that there is a given \concept{weight function} $\funsig{w}{E}{\RRR}$ defined on the \concepts{edge}. Then, the \concept{objective} is to find a \concept{maximum weight matching}, and the \concept{objective} is changed accordingly to $\max\isumdomain[e]{E}{w_e\cdot x_e}$ (Obviously, it is a generalization, since the \concept{cardinality matching problem} arises when the \concept{weight} $w_e$ for each $e\in E$ is the same positive number). In the \concept{perfect matching} problem the set of \concepts{feasible solution} is restricted to \concepts{perfect matching}. These are \concepts{matching} such that each \concept{vertex} is \concept{incident} to precisely one \concept{edge} in the \concept{matching}; then, constraints \eqnnref{matching-c1} become \concept[equality]{equalities}. Notice that a \concept{perfect matching} may not exist (consider e.g. a \concept{triangle}), whereas there is always a \concept{feasible solution} to \eqnrefr{matching-m}{matching-c2}.

\paragraph{}
We emphasize that we distinguish between, on the one hand, the \concept{problem} itself, and, on the other hand, its \concept{formulation} as an \concept{integer program}. Indeed, these two are not the same! In fact, in some cases it is appropriate to show the \concept{correctness} of a \concept{formulation}. Such an \concept[correctness argument]{argument} is usually based on a correspondence between \concepts{feasible solution} to the problem, and vectors of \concepts{decision variable} satisfying the \concepts{constraint}.

\paragraph{}
Let us illustrate this matter for the \concept{matching problem}. Notice that there is a 1-1 correspondence between \concepts{subset} of the set of \concepts{edge} $E$ and the 0-1 \concepts{vector} defined by the \concepts{variable}, which are indexed by the \concepts{edge}. To prove that this \concept{formulation} is correct we must show that there is a 1-1 correspondence between the \concepts{subset} of the \concepts{edge} that define \concepts{matching} and the \concepts{feasible solution} of the above \concept{formulation}. Moreover, we must show that the \concept{matching} and its corresponding \concept{vector} have the same value. This can be done as follows. First, consider an arbitrary \concept{matching} $M$ in a \concept{graph}. By definition, this implies that each \concept{node} $v\in V$ of the \concept{graph} is \concept{incident} with at most one \concept{edge} of $M$ . Let us now construct a \concept{solution vector} $\vec{x}_M$ in a straightforward manner: we put a ``$1$'' in $\vec{x}_M$ when the corresponding \concept{edge} is in $M$, and otherwise we put a ``$0$'' in the \concept{vector} $\vec{x}_M$. Clearly, $\vec{x}_M$ is a \concept{0-1 vector}, and obviously satisfies \eqnref{matching-c2}. Also, the \concept{solution vector} $\vec{x}_M$ corresponding to $M$ satisfies the \concepts{constraint} \eqnref{matching-c1} since at most one of the \concepts{variable} in the left-hand side has value $1$. Thus, a \concept{matching} $M$ corresponds to a \concept{feasible solution} of the \concept{integer program}. Second, consider a \concept{subset} of the \concepts{edge} $E'$ that is \textbf{not} a \concept{matching}. Then there is a \concept{vertex}, say $v$, that is \concept{incident} to at least two \concepts{edge} in $E'$. But then, at least two of the \concepts{variable} in the left-hand side of \eqnref{matching-c1} have value $1$ for this particular \concept{vertex}. Thus, the \concept{vector} corresponding to $E'$ is not \concept{feasible} in the \concept{formulation}. Finally, we observe that the value of each \concept{set} $E'\subseteq E$ is equal to its number of \concepts{edge}, i.e., $\abs{E'}=\isumdomain[e]{E'}{1}=\isumdomain[e]{E}{x_e}$.

\paragraph{}
In general, it may not be trivial to prove the 1-1 correspondence of \concepts{feasible solution} of a \concept{combinatorial problem} to \concepts{solution} of its \concept{formulation}, i.e., solutions satisfying the \concepts{constraint}. For many problems, however, a \concept{correctness proof} is omitted because the problem is an (extension of) a well-known problem, and the correctness of a formulation is evident.

\subsection{The Independent Set Problem}
\ssclab{independentset}

Another basic problem within the field of combinatorial optimization is the \concept{Independent set} problem (also known as \conceptsee{Stable set}{independent set}, or as \conceptsee{Node packing}{Independent set}). It can be described as follows.

\begin{definition}[Independent set problem]
Consider an arbitrary \concept{graph} $G=\tupl{V,E}$. A \concept{subset} of the \concept[vertex]{vertices} $V'\subseteq V$ is called an \concept{independent set} (or a \concept{stable set}, or a \concept{node packing}) if each \concept{edge} of $E$ is \concept{incident} to at most one \concept{vertex} of $V'$. In other words, $V'$ is an \concept{independent set} if no two \concept[vertex]{vertices} of an \concept{edge} are selected both. The problem is to find an \concept{independent set} in $G$ containing as many \concept[vertex]{vertices} as possible (a \concept{maximum cardinality independent set}).
\end{definition}

The problem can be formulated as an \concept{integer program} as follows. We define a 0-1 \concept{variable} for each \concept{vertex} $v\in V$ as follows:
\begin{equation}
\semboolvar{x_v}{if vertex $v$ is selected in the independent set;}{otherwise.}
\end{equation}
And here is the \concept{integer program}:
\begin{eqnarray}
\mbox{maximize}&\sumdomain[v]{V}{x_v}\eqnlab{stableset-m}\\
\mbox{subject to}&\forall\tupl{v_1,v_2}\in E:x_{v_1}+x_{v_2}\leq 1\eqnlab{stableset-c1}\\
&\forall v\in V:x_v\in\accl{0,1}\eqnlab{stableset-c2}
\end{eqnarray}

\importtikzfigure{stableset-red}{A stable set.}

\subsection{The Spanning Forest Problem}
\ssclab{spanningforest}

We start with a definition for the \concept{spanning forest problem}:

\begin{definition}[Spanning forest problem]
Consider an \concept{undirected graph} $G=\tupl{V,E}$. A \concept{subset} of the \concepts{edge} $E'\subseteq E$ is called a \concept{forest} if the \concept{subgraph} of $G$ induced by $E'$ is \concept[acyclic graph]{acyclic}, see \figref{forest-red} for an example. In case a \concept{forest} consists of $\abs{V}-1$ edges it is called a \concept{tree}. The problem is to find a \concept{maximum weight forest} in $G$.
\end{definition}

The problem to find a \concept{maximum weight forest} can be formulated as an \concept{integer program} as follows. We define a 0-1 \concept{variable} for each \concept{edge} $e\in E$ as follows:
\begin{equation}
\semboolvar{x_e}{if edge $e$ is selected in the forest;}{otherwise.}
\end{equation}
And here is the \concept{integer program}:
\begin{eqnarray}
\mbox{maximize}&\sumdomain[e]{E}{w_e\cdot x_e}\eqnlab{forest-m}\\
\mbox{subject to}&\forall V'\subseteq V:2\leq\abs{V'}\leq\abs{V}:\sumdomain[e]{\fun{\delta}{v_1}\cap\fun{\delta}{v_2},v_1,v_2\in V}{x_e}\leq\abs{V'}-1\eqnlab{forest-c1}\\
&\forall e\in E:x_e\in\accl{0,1}\eqnlab{forest-c2}
\end{eqnarray}

\importtikzfigure{forest-red}{The edges in red form a forest.}
\paragraph{}The \eqncsref{forest-c1} limit the number of chosen \concepts{edge} with both \concept[endvertex]{endvertices} in any given \concept{subset} $V'$ of the \concepts{node} to at most $\abs{V'}-1$. Since a \concept{cycle} contains as many \concepts{edge} as \concepts{vertice}, the \eqncsref{forest-c1} prevent the graph $\tupl{V,E'}$ from containing \concepts{cycle}.
\paragraph{}
Notice that the number of \concepts{subset} of $V$ of size $2$ or bigger is $2^{\abs{V}}-\abs{V}-1$. Thus, the number of \eqnref{forest-c1} is exponential in the size of the problem. Thus, the size of the formulation is exponential in the size of the input, although the problem is trivially solvable by a \concept{greedy algorithm}.
\paragraph{}
The \concept{spanning tree} problem is a variant of the \concept{spanning forest} problem, where the set of feasible solutions is restricted to those \concept[acyclic graph]{acyclic subgraphs} that are \concept[connected graph]{connected}. It is well-known that for an \concept[acyclic graph]{acyclic subgraph} the requirement of \concept[connected graph]{connectedness} is equivalent to the requirement of having $\abs{V}-1$ \concepts{edge}, i.e., $\abs{E'}=\abs{V}-1$. Thus, by adding the constraint $\isumdomain[e]{E}{x_e}=n-1$ to \eqnnrefr{forest-c1}{forest-c2}, a correct \concept{formulation} of the \concept{minimum weight spanning tree} problem arises.

\subsection{The Knapsack Problem}
\ssclab{knapsack}

We start this section with the definition of the \concept{knapsack problem}:

\begin{definition}[Knapsack problem]
We are given a set of $n$ \concepts{item}, each with a \concept{weight} $a_j$ and a \concept{value} $c_j$ for $\rangei[j]{1}{n}$. \concepts{Feasible solution} are the \concepts{subset} of the set of \concepts{item} with \concept{cumulative weight} at most $b$. The objective is to find a \concept{feasible solution} of maximum value.
\end{definition}

The problem \concept{formulation} contains \concept{binary variable} $x_j$ which indicate whether \concept{item} $j$ with $\rangei[j]{1}{n}$ is in the \concept{knapsack}:

\begin{equation}
\semboolvar{x_j}{if item $j$ is selected in the knapsack;}{otherwise.}
\end{equation}

And here is the \concept{integer program}:

\begin{eqnarray}
\mbox{maximize}&\sumieqb[j]{1}{n}{c_j\cdot x_j}\eqnlab{knapsack-m}\\
\mbox{subject to}&\sumieqb[j]{1}{n}{a_j\cdot x_j}\leq b\eqnlab{knapsack-c1}\\
&\forall\rangei[j]{1}{n}:x_j\in\accl{0,1}\eqnlab{knapsack-c2}
\end{eqnarray}
In this text, we will come back extensively to the \concept{knapsack problem}. There is a book devoted to this problem, see Kellerer et al.\cite{KelPfePis04}.

\section{Formulations and difficulty}
\ssclab{difficulty}
Does the \concept{formulation} of a problem tell us anything about the problem's difficulty? The answer is no, it doesn't. Consider for instance the \concept{matching} problem (\sscref{matching}) and the \concept{stable set} problem (\sscref{independentset}). These problems look similar, since the roles of the \concepts{edge} and \concept[vertex]{vertices} are interchanged. However, there is a striking difference between them. The \concept{matching problem} can be solved in \concept{polynomial time}, but the \concept{node packing problem} is \concept{NP-hard} (see \chpref{complexity}). In other words, whereas, for the \concept{matching problem}, fast and efficient \concepts{algorithm} exist, and have been designed, no such \concepts{algorithm} have been found for the \concept{node packing problem}. Indeed, this does not rule out the possibility that fast \concepts{algorithm} could exist for \concept{node packing}, however, no one has ever found such an \concept{algorithm}. In fact, it is widely suspected that such \concepts{algorithm} do not exist, but a proof of this hypothesis is lacking. All this boils down to the famous, 1 million-dollar worth, P=NP question. In practice, this means that we can find an \concept{optimal solution} to a \concept{matching problem} on a \concept{graph} with, say 10.000 \concepts{node} and 50.000 \concepts{edge} within seconds, while there is no \concept{algorithm} known that would return an \concept{optimal solution} with one hour for the \concept{stable set} instance on the same \concept{graph}. Therefore: a \concept{formulation} does not give an indication of the solvability of a problem. Also, the number of variables and/or constraints is no clue concerning the difficulty of a problem. For instance, the ``natural'' \concept{formulation} of the \concept{minimum spanning tree problem} (see \sscref{spanningforest}) has an exponential number of constraints, while the problem is easily solvable by a \concept{greedy algorithm}.

\section{Multiple formulations of a combinatorial optimization problem}
\seclab{multipleformulations}

\subsection{Traveling Salesman Problem}
\ssclab{tsp}
Probably the most well-known problem in \concept{combinatorial optimization} is the \concept{traveling salesman problem} (\concept{TSP}). The \concept{TSP} is the prototype \concept{combinatorial optimization problem}. No other problem has received as much attention as the \concept{TSP}, and no other problem has captured the imagination as an easy-to-describe, yet hard-to-solve problem. A description of the problem is as follows.

\begin{definition}[Traveling salesman problem]
Given is a set of $n$ ``\concept[city]{cities}'' and a \concept{distance} $c_{i,j}$ between each pair of them. The goal is to find a \concept{tour} of \concept{minimum length}, that is to start in some \concept{city}, visit each other \concept{city} once, and to return to the \concept{city} where the \concept{tour} was started. More formally, given an $n\times n$ matrix $C=c_{i,j}$, find a permutation $\pi$ of $\accl{1,2,\ldots,n}$ such that $c_{\fun{\pi}{n},\fun{\pi}{1}}+\isumieqb[i]{1}{n-1}{c_{\fun{\pi}{i},\fun{\pi}{i+1}}}$ is minimum.
\end{definition}

Different \concepts{formulation} of the \concept{TSP} exist. Here is a conventional one, using \concepts{binary variable} $x_{i,j}$ indicating whether \concept{city} $j$ is visited directly after \concept{city} $i$:

\begin{eqnarray}
\mbox{minimize}&\sumieqb[i]{1}{n}{\sumieqb[j]{1}{n}{c_{i,j}\cdot x_{i,j}}}\eqnlab{tsp-m}\\
\mbox{subject to}&\forall\rangei[j]{1}{n}:\sumieqb[i]{1}{n}{x_{i,j}}=1\eqnlab{tsp-c1}\\
&\forall\rangei[i]{1}{n}:\sumieqb[j]{1}{n}{x_{i,j}}=1\eqnlab{tsp-c2}\\
&\forall S\subsetneq V:2\leq\abs{S}:\sumdomain[i]{S}{\sumdomain[j]{S}{x_{i,j}}}\leq\abs{S}-1\eqnlab{tsp-c3}\\
&\forall\rangei[i,j]{1}{n}:x_{i,j}\in\accl{0,1}\eqnlab{tsp-c4}
\end{eqnarray}

\paragraph{}
\eqncsref{tsp-c3} are called the \concept{subtour elimination constraints}. An equivalent way of writing them is:
\begin{equation}
\forall S\subsetneq V:\abs{S}\geq 2:\sumdomain[i]{S}{\sumndomain[j]{S}{x_{i,j}}}\geq 1
\end{equation}

\paragraph{}
Notice that this formulation has a polynomial number of \concepts{variable} ($n^2$), and an exponential number of \concepts{constraint} (\bigoh{2^n}). The latter fact might be considered a disadvantage of \concept{formulation} \eqnnrefr{tsp-m}{tsp-c4}.

\paragraph{}
There is, however, an alternative for this formulation which uses additional \concepts{real variable} $u_i$ , one such \concept{variable} for each \concept{city} $i$. The interpretation of this \concept{variable} is the \concept{position} of \concept{city} $i$ in the \concept{tour}, while putting the \concept{position} of city $1$ first. Next, by replacing \eqncsref{tsp-c3} by the following constraints:

\begin{eqnarray}
\forall \rangei[i,j]{2}{n}:i\neq j:u_i-u_j+n\cdot x_{i,j}\leq n-1\eqnlab{mtz-c1}\\
u_1=1\eqnlab{mtz-c2}
\end{eqnarray}

we arrive at a formulation that is called the \concept{Miller-Tucker-Zemlin (MTZ) formulation} of the TSP. It is an interesting exercise to verify the \concept{correctness} of the \concept{MTZ-formulation}. More concrete: why is it that \eqnnref{mtz-c1} exclude \concepts{subtour}? The answer lies in noticing that: any solution satisfying \eqnnref{tsp-c1}, \eqnnref{tsp-c2}, and \eqnnref{tsp-c4} consists of a collection of \concepts{subtour} (or a \concept{feasible solution}, that is, a single \concept{tour}). In case there are \concepts{subtour}, then there is a \concept{subtour} that does not contain \concept{city} $1$. Let us now, for each pair of consecutive \concept[city]{cities} $i$ and $j$ in this \concept{subtour}, sum the corresponding \concept[inequality]{inequalities} \eqnnref{mtz-c1}. The $u$-variables will cancel out, and the resulting lefthand side will be larger than the resulting righthand side, which means that this \concept{subtour} will be forbidden by \eqnnref{mtz-c1}. Thus, any \concept{subtour} not containing city $1$ will be forbidden, and hence, the only possible solution is a single \concept{tour}.

\paragraph{}
There is more than one book devoted to the TSP. We mention \emph{Applegate et al.}\cite{Applegate.2006} and \emph{Lawler et al.}\cite{Lawler85}.

\paragraph{}
We will end this section with a problem for which we have two natural \concepts{formulation}. Both \concepts{formulation} have the same sets of \concepts{variable}, but they have different sets of \concepts{constraint}. Later we will see different \concepts{formulation} of problems where the sets of \concepts{variable} differ.

\subsection{The Uncapacitated Facility Location Problem}
\ssclab{facilitylocation}

We start with defining the \concept{Uncapacitated facility location problem}:

\begin{definition}[Uncapacitated facility location problem]
We are given a set of $m$ \concept[facility]{facilities} and $n$ \concepts{client}. Let us call the set of \concept[facility]{facilities} $M\equiv\accl{1,2,\ldots,m}$, and let us call the set of \concepts{client} $N\equiv\accl{1,2,\ldots,n}$. Each of the \concept[facility]{facilities} can (but does not need to be) be opened to serve \concepts{client}. Each \concept{client} must be served by a \concept{facility}. The \concept{cost} for opening \concept{facility} $i$ is $f_i$, $i\in M$; the cost for serving \concept{client} $j$ by \concept{facility} $i$ is $c_{i,j}$, $i\in M$, $j\in N$. The aim is to minimize the total costs, but serve every \concept{client}.
\end{definition}

This problem can be formulated in two ways with the following sets of \concepts{variable}, defined for each $i\in M$, $j\in N$:

\begin{eqnarray}
\semboolvar{x_{i,j}}{if facility $j$ serves client $j$;}{otherwise.}\\
\semboolvar{y_i}{if facility $i$ is open;}{otherwise.}
\end{eqnarray}

\paragraph{}
The first \concept{formulation} makes use of the fact that there is an \concept{upper bound} on the number of \concepts{client} that are served by a \concept{facility}, namely the total number of \concepts{client} $n$.

\begin{eqnarray}
\mbox{minimize}&\sumdomain[i]{M}{\brak{f_i\cdot y_i+\sumdomain[j]{N}{c_{i,j}\cdot x_{i,j}}}}\eqnlab{ufl-ma}\\
\mbox{subject to}&\forall j\in N:\sumdomain[i]{M}{x_{i,j}}=1\eqnlab{ufl-ca1}\\
&\forall i\in M:\sumdomain[j]{N}{x_{i,j}}\leq n\cdot y_i\eqnlab{ufl-ca2}\\
&\forall i\in M,j\in N:x_{i,j},y_i\in\accl{0,1}\eqnlab{ufl-ca3}
\end{eqnarray}

\paragraph{}
In the second \concept{formulation}, each of the \eqncsref{ufl-ca2} is disaggregated into $n$ new \concepts{constraint}, leading to \concepts{constraint} \eqncsref{ufl-cb2}.

\begin{eqnarray}
\mbox{minimize}&\sumdomain[i]{M}{\brak{f_i\cdot y_i+\sumdomain[j]{N}{c_{i,j}\cdot x_{i,j}}}}\eqnlab{ufl-mb}\\
\mbox{subject to}&\forall j\in N:\sumdomain[i]{M}{x_{i,j}}=1\eqnlab{ufl-cb1}\\
&\forall i\in M,j\in N:x_{i,j}\leq y_i\eqnlab{ufl-cb2}\\
&\forall i\in M,j\in N:x_{i,j},y_i\in\accl{0,1}\eqnlab{ufl-cb3}
\end{eqnarray}

Which of these \concepts{formulation} is preferable is not a matter of comparing the number of \concepts{constraint} or \concepts{variable}. In fact, as will be shown in the sequel, large \concepts{formulation} with many \concepts{constraint} and/or \concepts{variable} are usually better from a \concept{solver}'s point of view. This depends on the techniques that are used to solve the problem. Since these techniques very often rely heavily on \concept{linear programming}, the quality of the \concept{formulation} depends almost always on the accuracy of the \concept{linear programming relaxation}, i.e., the problem which results when the \concepts{integrality constraint} are removed. For the \concept{uncapacitated facility location} problem \concept{formulation} \eqnnrefr{ufl-mb}{ufl-cb3} is better than \eqnnrefr{ufl-ma}{ufl-ca3}, since the \concepts{constraint} \eqncsref{ufl-cb2} imply the \concepts{constraint} \eqncsref{ufl-ca2}.

\section{Combinatorial Optimization: a General Formulation}
\seclab{generalform}
In this section we (informally) argue that each \concept{combinatorial optimization problem} can be formulated as an \concept{integer program}. In a \concept{combinatorial optimization problem} a finite \concept{ground set} $E$ is given. To each \concept{element} $e\in E$ a \concept{weight} $w_e$ is attached. A family $\calS$ of subsets of $E$ is identified as the set of \concept{feasible solution}. This family depends on the \concept{particular problem}. The \concept{weight} of a set $E'\subseteq E$ is the \concept{cumulative weight} of its \concepts{element}, i.e., $\fun{w}{E'}=\isumdomain[e]{E'}{w_e}$. The associated \concept{optimization problem} is to find the \concept[maximum weight feasible solution]{maximum} (or \concept[minimum weight feasible solution]{minimum}) weight feasible solution $E'\in S$, i.e.,

\begin{equation}
\displaystyle\max_{E'\in\calS}\accl{\fun{w}{E'}}
\end{equation}

\paragraph{}
The set of \concepts{feasible solution} $\calS$ is usually given implicitly. It is described by the properties of \concepts{feasible solution}; it may be very large. For instance, in case of the \concept{matching problem}, the \concept{ground set} equals the set of \concepts{edge}, and the set $\calS$ is the collection of edge-sets that are \concepts{matching}. In case of the \concept{knapsack problem}, the \concept{ground set} equals the set of items, and the set $\calS$ equals the collection of item-sets that can be put together in the \concept{knapsack}.

\paragraph{}
Many examples of problems that fit in the above formulation are found in \concept{graph theory} (\secref{formulations}). Among them are well-known problems like the \concept{shortest path problem}, the \concept{minimum spanning tree problem}, and the \concept{traveling salesman problem}. We define some graph theory problems below.

\begin{definition}[Shortest path problem]
In a \concept{graph} $G=\tupl{V,E}$ the \concepts{feasible solution} are the subsets of the \concepts{edge} that form \concepts{path} between two specified \concept[vertex]{vertices} $s$ and $t$. Among the \concepts{path} between $s$ and $t$ we want to find the one with a minimum number of \concepts{edge}, or if a \concept{length function} is given on the \concepts{edge}, we want to find a \concept{path} of \concept{minimum total length}.
\end{definition}

\begin{definition}[Minimum spanning tree problem]
In the \concept{minimum spanning tree problem} a \concept{graph} $G=\tupl{V,E}$ is given together with a \concept{weight function} on the \concepts{edge}. A \concept{feasible solution} is a set of \concepts{edge} that forms a \concept{tree}. Among the \concepts{tree} we want to find one with \concept{minimum weight}. This problem is easily solvable by a \concept{greedy algorithm} like \concept{Prim's algorithm} or \concept{Kruskal's algorithm}.

\begin{definition}[Hamiltonian path problem]
If we restrict the \concept{minimum spanning tree problem} such that the set of \concepts{feasible solution} only contains \concepts{tree} that form \concepts{path}, the problem becomes the \concept{Hamiltonian path problem}, which is highly \concept{intractable}. Thus, in general, problems do not become simpler when the set of \concepts{feasible solution} is reduced.
\end{definition}

\paragraph{}
To formulate a \concept{combinatorial optimization problem} in mathematical terms, we introduce \concepts{decision variable} for all elements of the \concept{ground set} $E$. Each \concept{decision variable} denotes a choice, namely whether the corresponding element is chosen or not. So, a \concept{variable} can have two values, which are usually taken from $\accl{0,1}$. A set $E'\subseteq E$ can be described by a \concept{binary vector} $\vec{x}_{E'}=\brak{x_e}_{e\in E}$ with $n=\abs{E}$ components as follows:

\begin{equation}
\semboolvar{x_e}{if element $e$ is in $E'$;}{otherwise.}
\end{equation}

\paragraph{}
The finite set of \concepts{vector} $X\subseteq\RRR^n$ corresponding to \concepts{feasible solution} from $\calS$ can then be described by means of \concepts{constraint}. In many cases, these \concepts{constraint} are \concept[linear constraint]{linear} and involve \concepts{binary variable}. The \concept{objective function} $\vec{w}$ is usually a \concept{linear function} of the components of $\vec{x}\in X$. The problem is then

\begin{equation}
\max\condset{\vec{w}\cdot\vec{x}}{A\cdot\vec{x}\leq\vec{b}\wedge \vec{x}\in\accl{0,1}^n}
\end{equation}
where $A$ and $\vec{b}$ depend on the problem at hand. This type of \concept{formulation} is often called an \concept[Integer linear programming]{Integer (or Binary) Linear Program}, (\concept{ILP}).

\section{General Combinatorial Optimization Problems}
\seclab{generalcombinatorial}

In the first section, we introduced \concepts{binary variable} to model \concepts{decision} of the yes-no type, more specifically, to decide whether an element is in a set or not. These variables were used further to describe the \concepts{constraint} that determine the \concepts{feasible solution}. In all the examples these constraints could be written as \concepts{linear function} with a \concept{bound} imposed on them. Similarly, the \concept{objective} could be formulated as a \concept{linear function} of the \concepts{variable}. In this section we will generalize the \concepts{formulation} of \concepts{combinatorial problem}, with respect to all three items, i.e., the \concepts{variable}, the \concepts{constraint}, and the \concept{objective}.

\paragraph{}
The \concepts{decision} in \concepts{combinatorial problem} may be more complicated than simple yes/no decisions. One may have to introduce \concepts{integer variable} or even \concepts{real variable}, like in \concept{linear programming}, to model certain \concepts{decision}. And, of course, combinations of these types of variables are possible in a single problem \concept{formulation}. The combinatorial nature (finite or countable number of feasible solutions) may not be so evident in these problems. However, the number of ``\concepts{interesting feasible solution}'' is usually still finite in such problems. For instance, in \concept{linear programming}, the \concepts{interesting feasible solution} are the \concept[vertex]{vertices} of a \concept{polyhedron}. The number of \concept[vertex]{vertices} is usually finite.

\paragraph{}
One may consider any function of the \concepts{variable} with a \concept{bound} imposed on it as a \concept{constraint}. Moreover, \concepts{logical composition} of \concepts{constraint}, like \concepts{implication} and \concepts{disjunction} (logical ``or'') can be used to model the restrictions of a problem. In abstracto, any relation on the \concepts{variable} that restricts the set of \concepts{feasible solution} can be used as a \concept{constraint}.

\paragraph{}
The \concept{objective} of a problem can be any function of the \concepts{variable}, thus it is not restricted to \concepts{linear function}.

\paragraph{}
The above observations lead to the following \concept{abstract formulation of optimization problems}. We distinguish between the \concepts{real variable}, denoted by the \concept{vector} $\vec{x}$, and the \concepts{integral variable}, denoted by the \concept{vector} $\vec{y}$. A \concept{formulation} of a \concept{combinatorial optimization problem} contains the following items:
\begin{itemize}
 \item a vector of $n$ real decision variables:
 \begin{equation}
  \vec{x}=\tupl{x_1,x_2,\ldots,x_n}
 \end{equation}
 \item a vector of $m$ integral decision variables:
 \begin{equation}
  \vec{y}=\tupl{y_1,y_2,\ldots,y_m}
 \end{equation}
 \item a set $C$ of $k$ constraints:
 \begin{equation}
  C=\accl{C_1,\ldots,C_k}
 \end{equation}
 \item an objective function on the variables:
 \begin{equation}
  \fun{f}{\vec{x},\vec{y}}
 \end{equation}
\end{itemize}

\paragraph{}
The set of \concepts{feasible solution} consists of \concepts{vector} $\tupl{\vec{x},\vec{y}}$ that satisfy all the \concepts{constraint}. Each \concept{variable} has a \concept{domain} $D$, usually the set of \concepts{real number} $\RRR$, or the set of \concepts{integer} $\ZZZ$. The \concept{solution space} of the problem is the \concept{Cartesian product} of the \concepts{domains} of the \concepts{variable}, i.e., $\RRR^n\times\ZZZ^m$.

\paragraph{}
For many \concepts{solution technique}, especially the ones that we are going to discuss, it is of eminent importance to restrict the \concepts{domain} of the \concepts{variable} as much as possible. Some \concepts{constraint} imply \concept[lower bound]{lower} and \concepts{upper bound} on the value of a \concept{variable}, directly or indirectly. For instance, \concepts{binary variable} have an \concept[explicit lower bound]{explicit lower} and \concept[explicit upper bound]{upper bound}. If this is the case, we usually take this into account in the problem \concept{formulation} explicitly, i.e., if a \concept{real variable} $x_i$ has a \concept{lower bound} $l_i$ and an \concept{upper bound} $u_i$ , then we describe its \concept{domain} as $\fbrk{l_i,u_i}$; if $y_j$ is an \concept{integral variable}, with a \concept{lower bound} $l_j$ and an \concept{upper bound} $u_j$ , then we describe its domain as $\accl{l_j,l_j+1,\ldots,u_j}$.

\section*{Exercises}
\begin{exercise}
Consider the \concept{stable set problem} on an \concept{undirected graph} $G=\tupl{V,E}$. Show that the \concept{formulation} \eqnnrefr{stableset-m}{stableset-c2} is a correct \concept{formulation} of the \concept{stable set problem}.
\end{exercise}
\begin{exercise}
Consider the \concept{matching} depicted in \figref{matching-red}. Is it \concept[maximal solution]{maximal}? Is it \concept[maximum solution]{maximum}?
\end{exercise}
\begin{exercise}
Consider the \concept{stable set} depicted in \figref{stableset-red}. Is it \concept[maximal solution]{maximal}? Is it \concept[maximum solution]{maximum}?
\end{exercise}
\begin{exercise}
Consider an \concept{undirected graph} $G=\tupl{V,E}$. An \concept{edge cover} $E'$ is a subset of the \concepts{edge} such that each \concept{node} is \concept{incident} to at least one \concept{edge} in $E'$. Formulate the problem of finding a \concept{minimum cardinality edge cover} as an \concept{integer linear program}.
\end{exercise}
\begin{exercise}
Consider an \concept{undirected graph} $G=\tupl{V,E}$. A \concept{node cover} $V'$ is a subset of the \concepts{node} such that each \concept{edge} is incident to at least one \concept{node} in $V'$. Formulate the problem of finding a \concept{minimum cardinality node cover} as an \concept{integer linear program}.
\end{exercise}
\begin{exercise}
\exclab{clique-partitioning}
A \concept{clique partitioning} of an \concept[undirected graph]{undirected}, \concept{complete graph} $G$ is a \concept{partitioning} of the \concept[vertex]{vertices} into subsets $V_1,V_2,\ldots,V_k$ such that the \concept{subgraph} induced by each $V_i$ is a \concept{complete graph} itself ($\rangei[i]{1}{k}$). Consider now a \concept{complete graph} $G$ and an arbitrary \concept{weight} $w_{i,j}$ for each \concept{edge} of the \concept{graph} (notice that $w_{i,j}$ can be negative; in fact, the problem is only interesting when there are both positive and negative \concept{edge} \concepts{weight}). The \concept{objective} is to find a \concept{clique partitioning} of \concept{maximal weight}, that is, a \concept{clique partitioning} such that the \concept{cumulative weight} of the \concepts{edge} that have both \concept[vertex]{vertices} in one and the same component is maximum. Formulate this problem as an \concept{integer linear programming problem}.
\begin{hint}
Use a \concept{variable} for each \concept{edge}.
\end{hint}
\end{exercise}
\begin{exercise}
Consider the \concept{TSP}.
\begin{itemize}
 \item When the \concept{distance matrix} $C$ is known to be \concept[symmetric distance matrix]{symmetric}, can you simplify \concept{formulation} \eqnnrefr{tsp-m}{tsp-c4} by ``merging'' \eqncsref{tsp-c1} and \eqnnref{tsp-c2}?
 \item Which of the two \concepts{formulation} given in \sscref{tsp} is stronger?
\end{itemize}
\end{exercise}
\begin{exercise}\exclab{rectanglestabbing}
Given is a set of \concepts{axis-aligned rectangle} in the \concept{plane}. Each of these \concepts{rectangle} needs to be stabbed, either by a \concept{horizontal line} or by a \concept{vertical line}. The problem is to find a set \concept[horizontal line]{horizontal} and \concepts{vertical line}, stabbing each \concept{rectangle} at least once, with \concept{minimum cardinality}. We call this the \concept{rectangle stabbing problem}. Formulate this problem as an \concept{integer linear programming problem}.
\begin{hint}
First, give a formulation for the instance depicted in \figref{rectanglestabbing}).
\end{hint}
\end{exercise}
\begin{exercise}
Given an instance of the \concept{rectangle stabbing problem} as depicted in \figref{rectanglestabbing}, what \concept{fractional solution} is the optimal \concept{linear programming relaxation} (of the integer program that you just wrote down) for this instance?

\importtikzfigure{rectanglestabbing}{A rectangle stabbing instance.}
\end{exercise}

\chapter{From linear to integer optimization}
\chplab{lintointopt}

\section{Two equivalent definitions of a polyhedron}
\seclab{defpolyhed}

Consider the \concept{feasible region} depicted in \figref{polyhedron}. How to describe this object mathematically?

\importtikzfigure{polyhedron}{A feasible region.}

Here is one way: we can view the \concept{feasible region} as the \concept{intersection} of as a number of \concepts{halfspace}, each \concept{halfspace} defined by a \concept{linear inequality}. For instance, in the case of the figure above, we can write
\begin{equation}
\accl{\tupl{x_1,x_2}\in\RRR^2: -x_1+5\cdot x_2\leq 20 \wedge 2\cdot x_1+x_2\leq 6\wedge 6\cdot x_1-x_2\leq 10\wedge x_1\geq 0\wedge x_2\geq 0}.
\end{equation}
So we need five \concept[inequality]{inequalities} to precisely describe the \concept{feasible region}.

\paragraph{}
An alternative is to focus on the \concept[extreme vertex]{extreme vertices} of the \concept{feasible region} in \figref{polyhedron}. Indeed, we can alternatively write
\begin{equation}
\accl{\tupl{x_1,x_2}\in\funma{conv}{\tupl{0,0},\tupl{0,4},\tupl{\frac{10}{11},\frac{46}{11}},\tupl{2,2},\tupl{\frac{5}{3},0}}}.
\end{equation}

\paragraph{}
The first approach can be seen as the \concept{linear programming} approach; here, we simply list all the \concept[inequality]{inequalities} that jointly define the \concept{feasible region}. The second approach can be seen as the \concept{integer programming} approach; then, we list the \concept{extreme vertices} of the \concept[feasible region]{region}, and define the \concept{feasible region} as anything that is in the \concept{convex hull} of these given vertices. The two approaches are equivalent: anything that can be written using \concept[inequality]{inequalities}, can be written as the \concept{convex hull} of a number of \concepts{point}, and vice versa. Let us now take a more general point of view. Thus, a \concept{polyhedron} $P$ can be defined in two equivalent ways. First, as the set of points in $\RRR^n$ that satisfy a finite set of \concepts{linear constraint}, i.e.,
\begin{equation}
P=\condset{\vec{x}\in\RRR^n}{A\cdot\vec{x}\leq b}.
\end{equation}

\paragraph{}
Second, $P$ can be defined as the set of \concepts{point} in $\RRR^n$ that are \concepts{convex combination} of \concepts{point} of a finite set $X=\accl{\vec{x}_1,\vec{x}_2,\ldots,\vec{x}_K}$ plus \concepts{nonnegative combination} of \concepts{point} of a finite set $Y=\accl{\vec{y}_1,\vec{y}_2,\ldots,\vec{y}_L}$, i.e.,
\begin{equation}
P=\funm{conv}{X}+\funm{cone}{Y}
\end{equation}
where
\begin{equation}
\funm{conv}{X}=\condset{\sumieqb[k]{1}{K}{\alpha_k\cdot\vec{x}_k}}{\forall k:\alpha_k\in\RRR^+\wedge\sumieqb[k]{1}{K}{\alpha_k}=1}
\end{equation}
and
\begin{equation}
\funm{cone}{X}=\condset{\sumieqb[l]{1}{L}{\beta_l\cdot\vec{x}_l}}{\forall l:\alpha_l\in\RRR^+}
\end{equation}

\paragraph{}
The \concept{representation theorem of Farkas, Minkowski, and Weyl} (see e.g. \emph{Nemhauser and Wolsey}\cite{citeulike:2212037}) proves that both definitions of a \concept{polyhedron} are equivalent. Moreover, if we have a description of $P$ in one form, then we know that there is a description in the other form.

\paragraph{}
In the sequel, we will consider only \concept[polyhedron]{polyhedra} which are subsets of the \concept{positive orthant} in $\RRR^n$ , i.e., $P=\condset{\vec{x}\in\RRR^n}{A\cdot\vec{x}\leq\vec{b}\wedge\vec{x}\geq\vec{0}}$. This guarantees, if $P$ is not \concept{empty}, the existence of \concepts{extreme point}. If $X$ and $Y$ are minimal, then $X$ contains the \concepts{extreme point}, and $Y$ contains the \concepts{extreme ray} of $P$. A \concept{polytope} is a \concept{bounded polyhedron}, i.e., $Y=\emptyset$ and $\funm{cone}{Y}=\accl{\vec{0}}$. With a few exceptions, we will only consider \concepts{polytope} in the sequel. In fact, most of the \concepts{polytope} that we consider lie in the $n$-dimensional unit cube $\BBB^n=\condset{\vec{x}\in\RRR^n}{\vec{0}\leq\vec{x}\leq\vec{1}}$.

\paragraph{}
The way the representation theorem is used in \concept{combinatorial optimization} is different from the way it is used in \concept{linear programming}. In \concept{linear programming} we are given a set of \concepts{feasible solution} by means of a system of \concepts{linear constraint} $\condset{\vec{x}\in\RRR^n}{A\cdot\vec{x}\leq\vec{b}\wedge\vec{x}\geq\vec{0}}$. The representation theorem is used to show that the \concept{optimization problem} $\max\condset{\vec{c}\cdot\vec{x}}{A\cdot\vec{x}\leq\vec{b}\wedge\vec{x}\geq\vec{0}}$ has an \concept{extreme point} that is the \concept{optimum}. In \concept{combinatorial optimization we are given, implicitly, a description of a finite set of feasible solutions $\calS$. The set $\calS$ is usually described as the set of subsets of a certain ground set $E$, where these subsets satisfy certain properties. The elements from $\calS$ are described by vectors $X=\accl{\vec{x}_1,\vec{x}_2,\ldots,\vec{x}_K}$. The representation theorem is used to conclude that $\funm{conv}{X}$ can be described by linear constraints, i.e., $\funm{conv}{X}=\condset{\vec{x}\in\RRR^n}{A\cdot\vec{x}\leq\vec{b}\wedge\vec{x}\geq\vec{0}}$ for some matrix $A$ and some vector $\vec{b}$. The main goal of this chapter is to find a way to obtain this set of linear constraints.

\paragraph{}
The set of feasible solutions $X$ of a combinatorial optimization problem is usually described through a formulation with linear restrictions and integrality constraints on the variables, i.e., $X=\condset{x\in\RRR^n}{A\cdot\vec{x}\leq\vec{b}\wedge\vec{x}\geq\vec{0}\wedge\vec{x}\in\ZZZ^n}$. Notice that the formulation of $X$ as an Integer Linear Program is not unique. The linear programming relaxation of this formulation, denoted by $\funm{LPR}{A,\vec{b}}$, is $\condset{\vec{x}\in\RRR^n}{A\cdot\vec{x}\leq\vec{b}\wedge\vec{x}\geq\vec{0}}$. Clearly, $\funm{conv}{X}$ is a subset of $\funm{LPR}{A,\vec{b}}$. A first question is: how well does the linear programming relaxation of a formulation describe the convex hull of $X$? This gives a criterion to decide on which formulation is best for a certain problem. In general we choose the formulation that defines the smallest polyhedron containing $\funm{conv}{X}$. A second question is: how can we find constraints that improve the linear programming relaxation? Then, we are looking for constraints that are satisfied by all feasible solutions in $X$, but that cut off part of the polyhedron defined by the relaxation.

\paragraph{}
We will give (partial) answers to both questions. In the following section we identify formulations for problems that have the property that the linear programming relaxation is tight, i.e., it describes $\funm{conv}{X}$ exactly. Then we describe a method, first illustrated using the stable set problem, that improves the linear programming relaxation by adding linear constraints (called valid inequalities) that make the formulation tighter. Finally, we discuss a systematic way of obtaining these inequalities.

\section{Linear description of combinatorial problems}
\seclab{lindesccomb}
Consider the integer linear program
\begin{equation}
\max\condset{\vec{c}\cdot\vec{x}}{A\cdot\vec{x}=\vec{b}\wedge\vec{x}\geq\vec{0}\wedge\vec{x}\in\accl{0,1}^m},
\end{equation}
where $A$ is an $n\times m$ matrix of integers and $\vec{b}$ is an $n$-vector of integers. Suppose we solve the corresponding linear programming formulation. Of course, we would be quite fortuitous if the resulting values for the $\vec{x}$-variables would be integral. However, in some special cases to be discussed next, one can guarantee that the resulting solution is indeed integral.

\paragraph{}
When solving the linear programming formulation, we know from the simplex method that there is a regular $n\times n$ submatrix $B$ of $A$ such that $\overline{x}=B^{-1}\cdot\vec{b}$ is the optimum solution of the linear program. Denoting the columns of $B$ by $B_{\star,j}$ $j\in\accl{1,\ldots,n}$, Cramer's rule gives an explicit description of the $n$ basic variables $x_j$ $j\in\accl{1,\ldots,n}$ corresponding to the columns of $B$:

\begin{equation}
x_j=\displaystyle\frac{\funm{det}{B_{\star,1}|B_{\star,2}|\ldots|B_{\star,j-1}|\vec{b}|B_{\star,j+1}|\ldots|B_{\star,n}}}{\funm{det}{B}}.
\end{equation}

The upper determinant is integral. If the determinant of $B$ would be $\pm 1$, then $x_j$ is certainly integral. We call a matrix $B$ with that property unimodular (UM). Since the basis $B$ may vary with the objective $\vec{c}$ and the right-hand side $\vec{b}$, we would like to have a characterization of matrices $A$ for which each $n\times n$ submatrix $B$ is UM. Indeed, for such matrices, the polyhedron
\begin{equation}
P=\condset{\vec{x}\in\RRR^n}{A\cdot\vec{x}=\vec{b}\wedge\vec{x}\geq\vec{0}}
\end{equation}
has integral vertices.

\paragraph{}
Next, consider the integer linear program
\begin{equation}
\max\condset{\vec{c}\cdot\vec{x}}{A\cdot\vec{x}\leq\vec{b}\wedge\vec{x}\geq\vec{0}\wedge\vec{x}\in\accl{0,1}}.
\end{equation}
To put the associated linear programming problem in the form above we introduce a slack-vector $\vec{y}$ and we reformulate it as
\begin{equation}
\max\condset{\vec{c}\cdot\vec{x}}{A\cdot\vec{x}+\vec{y}\leq\vec{b}\wedge\vec{x},\vec{y}\geq\vec{0}}.
\end{equation}
To obtain a similar result on the integrality of $\vec{x}$ (and $\vec{y}$), we can demand that each $n\times n$ submatrix of the matrix ($A|I$) be unimodular. This, however, is equivalent to stating that all regular square submatrices (not only those with $n$ rows and columns) of the matrix $A$ have a determinant $\pm 1$. Matrices with this property are called totally unimodular (TUM), see the next definition.

\begin{definition}[Totally unimodular matrix]
A matrix $A$ is called \emph{totally unimodular} if each square submatrix of $A$ has determinant equal to $-1$, $0$, or $+1$.
\end{definition}

\paragraph{}
We have the following important theorem for TUM matrices.

\begin{theorem}
If a matrix $A$ is TUM, then the polyhedron $P=\condset{\vec{x}\in\RRR^n}{A\cdot\vec{x}\leq\vec{b}\wedge\vec{x}\geq\vec{0}}$ has integral vertices.
\end{theorem}

\paragraph{}
Clearly, a TUM matrix consists only of entries $0,\pm 1$. In the following theorem we characterize two types of matrices that are totally unimodular.
\begin{theorem}
\thmlab{tum-matrix}
Let $A$ be a matrix with entries in $\accl{-1, 0, 1}$ such that there are at most two nonzeros in each column. If there exists a partition of the rows of $A$ in two sets $R_1$ and $R_2$ such that
\begin{enumerate}
 \item each column with two nonzero entries of the same sign, has one of its entries in $R_1$ and one in $R_2$,
 \item each column with two nonzero entries of different sign, has either both of its entries in $R_1$ or both of its entries in $R_2$,
\end{enumerate}
then the matrix $A$ is TUM.
\begin{proof}
We use induction on the size of the submatrices. Each submatrix of size $1\times 1$ is trivially TUM. Let $C$ be a submatrix of size $k\times k$. We consider three cases.
\begin{enumerate}
 \item If $C$ has a column with no nonzeros it is singular;
 \item if $C$ has a column with at most one nonzero, its determinant can be expanded along that column, and total unimodularity follows from the induction hypothesis;
 \item finally, if $C$ has only columns with two nonzeroes, we split its rows in the subsets with the property described above. Adding up the rows in each subset results in two identical row vectors, which implies that the matrix is singular.
\end{enumerate}
\end{proof}
\end{theorem}

\paragraph{}
There are two important classes of matrices that satisfy the conditions of the above theorem, and therefore are TUM. The node-arc incidence matrix $A$ of a directed graph (digraph) has rows corresponding with
the nodes and columns corresponding with the arcs. The entry $A_{v,a}$ can obtain three possible values:
\begin{enumerate}
 \item $A_{v,a}=-1$ if vertex $v$ is the tail of arc $a$.
 \item $A_{v,a}=1$ if vertex $v$ is the head of arc $a$.
 \item $A_{v,a}=0$ if vertex $v$ and arc $a$ are not incident.
\end{enumerate}

\paragraph{}
The node-edge incidence matrix $A$ of a graph has rows corresponding with the nodes and columns corresponding with the edges. The entry $A_{v,e}$ can obtain two possible values:
\begin{enumerate}
 \item $A_{v,e}=1$ if vertex $v$ and edge $e$ are incident.
 \item $A_{v,e}=0$ if vertex $v$ and edge $e$ are not incident.
\end{enumerate}

\begin{corollary}
\collab{tum-graph}
The node-arc incidence matrix of a digraph is TUM. The node-edge incidence matrix of a bipartite graph is TUM.
\end{corollary}

\paragraph{}
This implies that many optimization problems on (di)graphs can be solved with linear programming. Among them are the shortest path problem, the max-flow problem, the min-cost flow problem, and the matching problem on bipartite graphs. Finally, we will show by an example that general node-edge incidence matrices need not be TUM. Consider the complete graph on three vertices $K_3$. Its node edge
incidence matrix is:
\begin{equation}
\brak{\begin{array}{ccc}
1&1&0\\
1&0&1\\
0&1&1
\end{array}}
\end{equation}
The determinant of this matrix is $-2$. Now let us compare the ILP formulation of the maximum cardinality matching problem with its linear programming relaxation on this graph. A maximum matching consists of one edge, and thus has value $1$. The LP-relaxation has an optimum value of $1.5$, since each of the three variables can obtain the value $0.5$, without violating any of the linear constraints.

\section{Valid inequalities}
\seclab{validineq}
Consider the stable set problem. Given is an arbitrary graph $G=\tupl{V,E}$. The feasible solutions are the stable sets in $G$. These solutions are represented by binary $n$-vectors, where the components correspond to the vertices. The integer linear programming formulation is the following, see also \eqnnrefr{stableset-m}{stableset-c2}.

\begin{eqnarray}
\mbox{maximize}&\sumdomain[v]{V}{x_v}\eqnlab{stablesetr-m}\\
\mbox{subject to}&\forall\tupl{v_1,v_2}\in E:x_{v_1}+x_{v_2}\leq 1\eqnlab{stablesetr-c1}\\
&\forall v\in V:x_v\in\accl{0,1}\eqnlab{stablesetr-c2}
\end{eqnarray}

The stable set polytope $\funm{conv}{\mbox{SS}}$ is the convex hull of the feasible solutions of the above formulation, i.e., the set of vectors corresponding to stable sets. We will derive two classes of so-called valid inequalities, the clique inequalities and the odd-cycle inequalities. These valid inequalities will allow us to improve the solution found by the linear programming relaxation.

\begin{example}
Consider the following graph depicted in \figref{clique-graph} on five vertices. The nodes 3, 4 and 5 form a clique, i.e., a subset of the vertices that induces a complete graph. At most one of these nodes can be present in a stable set. In other words, each feasible solution must satisfy the inequality $x_3+x_4+x_5\leq 1$. Thus, when solving the linear programming relaxation, one could add this constraint to the formulation, thereby making the constraints $x_3+x_4\leq 1$, $x_3+x_5\leq 1$ and $x_4+x_5\leq 1$ obsolete.

\importtikzfigure{clique-graph}{Graph with cliques.}

\paragraph{}
Notice that the inequality $x_3+x_4+x_5\leq 1$ can be derived from the constraints in the formulation by rounding in the following way.

\begin{equation}
\begin{array}{rcrcrcc}
x_3&+&x_4&&&\leq&1\\
&&x_4&+&x_5&\leq&1\\
x_3&+&&&x_5&\leq&1\\\hline
2\cdot x_3&+&2\cdot x_4&&2\cdot x_5&\leq&3
\end{array}
\end{equation}
This gives $x_3+x_4+x_5\leq 1$. This way we can get $3$-cliques from $2$-cliques. The clique of the nodes $1$, $2$, $3$ and $4$ can not be derived by rounding constraints of the formulation. However, we can derive them from the constraints generated by the four $3$-cliques contained in $\accl{1,2,3,4}$.
\end{example}
\begin{example}
Consider the following graph depicted in \figref{odd-cycle}: It is a cycle consisting of five vertices. The following inequality is valid. At most two nodes of the cycle can be in a packing.
\begin{equation}
x_1+x_2+x_3+x_4+x_5\leq 2
\end{equation}
In general, for a cycle $C$ with an odd number of vertices in a graph $G$, the inequality $\isumdomain[v]{C}{x_v\leq\floor{\frac{1}{2}\abs{C}}}$ valid. Unfortunately, adding such a constraint does not guarantee that the resulting solution is necessarily integral, as the graph depicted in \figref{odd-cycle-claw} shows.

\importtikzfigure{odd-cycle}{Odd cycle.}

\importtikzfigure{odd-cycle-claw}{Odd cycle with 5-claw.}

\paragraph{}
In this graph the solution $x=\tupl{\frac{2}{5},\frac{2}{5},\frac{2}{5},\frac{2}{5},\frac{2}{5},\frac{1}{5}}$ satisfies all the constraints that we have found for the stable set problem. However, the cumulative sum of the variables is larger than two, and the number of nodes in a packing in this graph is at most two. Therefore, we still do not have a description of $\funm{conv}{\mbox{SS}}$ for this graph. We will now strengthen the odd cycle inequality generated by the vertices $1$, $2$, $3$, $4$ and $5$. To do so, we look at the following inequality and we try to find a value for $\alpha$ as high as possible, such that the inequality maintains its validity.
\begin{equation}
x_1+x_2+x_3+x_4+x_5+\alpha\cdot x_6\leq 2
\end{equation}
If $x_6=0$, then the inequality reduces to the odd cycle inequality and is valid for any value of $\alpha$. If $x_6=1$, then node $6$ is in the packing. But this implies that nodes connected to $6$ can not be in the packing. Thus, $x_1=x_2=x_3=x_4=x_5=0$, and the inequality is valid for all $\alpha\leq 2$. Concluding, the inequality remains valid for $\alpha=2$.
\end{example}
\begin{application}[Generalized node packing]
In structured problems with many inequalities one can sometimes derive so-called implications, i.e., if one variable has value 1 that implies that some other variable should have value 0. Such implications can lead to strong inequalities based on node packing inequalities which tighten a formulation. This technique is often used as a preprocessing step in solving huge integer linear programming problems.
\end{application}
\begin{example}
Three binary variables $x_1$, $x_2$ and $x_3$ satisfy the following relations.

\begin{equation}
\begin{array}{rcrcrcr}
&&-3\cdot x_2&-&2\cdot x_3&\leq&-2\\
-4\cdot x_1&-&3\cdot x_2&-&3\cdot x_3&\leq&-6\\
2\cdot x_1&-&2\cdot x_2&+&6\cdot x_3&\leq&5
\end{array}
\end{equation}

\paragraph{}
By using the complementing variables $y_1=1-x_1$, $y_2=1-x_2$ and $y_3=1-x_3$, this system can be reformulated into a system consisting of knapsack constraints, as follows.

\begin{equation}
\begin{array}{rcrcrcr}
&&3\cdot y_2&+&2\cdot y_3&\leq&3\\
4\cdot y_1&+&3\cdot y_2&+&3\cdot y_3&\leq&4\\
2\cdot x_1&+&2\cdot y_2&+&6\cdot x_3&\leq&7
\end{array}
\end{equation}

\paragraph{}
The node packing graph defined by the variables and constraints consists of the following vertices and edges. There are two nodes for each variable, one representing the variable $x_i$ and the other its complement $y_i$ . Two nodes are connected by an edge if the corresponding pair of variables sums up to a value of at most $1$. In this example the edges are $\tupl{x_i,y_i}$ ($i=1,2,3$); $\tupl{y_i,y_j}$ ($1\leq i < j \leq 3$) (second constraint); $\tupl{x1,x3}$ (third constraint); $\tupl{y2,x3}$ (third constraint).

\importtikzfigure{gennodepacking}{Generalized Node Packing graph.}

\paragraph{}
There is a clique consisting of the vertices corresponding to $x_3$, $y_3$ and $y_2$. Thus, $x_3+y_3+y_2\leq 1$. Since $x_3+y_3=1$ this gives $y_2=0$ and $x_2=1$. Notice that the vector $\tupl{x_1,x_2,x_3}=\tupl{1,\frac{1}{2},\frac{1}{2}}$ satisfies the original linear programming relaxation, but is cut off by the generated clique constraint.
\end{example}

\section{Gomory-Chv\`atal rounding}
\seclab{gomorychvatal}
A general method for finding valid inequalities that really cut off a part from the linear programming relaxation is \concept{Gomory-Chv\`atal rounding}.

\paragraph{}
Suppose we have the following description of a set of points
\begin{equation}
X=\condset{\vec{x}\in\RRR^n}{A\cdot\vec{x}\leq\vec{b}\wedge\vec{x}\geq\vec{0}\wedge\vec{x}\mbox{ integral}}
\end{equation}

\paragraph{}
Then, for any vector $\vec{u}\geq\vec{0}$, the following constraint is valid:
\begin{equation}
\floor{\vec{u}\cdot A}\cdot\vec{x}\leq\floor{\vec{u}\cdot\vec{b}}
\end{equation}

\paragraph{}
By the nonnegativity of x we have that ⌊uA⌋x ≤ ub, and from the integrality of x it follows that the left hand side is integral, and therefore we are allowed to round down the right hand side. By adding these constraints for all nonnegative vectors u, we get the polyhedron $P^1=\condset{\vec{x}\in\RRR^n}{\vec{x}\geq\vec{0}\wedge\forall \vec{u}\in\RRR^n:\vec{u}\geq\vec{0}\Rightarrow\floor{\vec{u}\cdot\vec{A}}\cdot\vec{x}\leq\floor{\vec{u}\cdot\vec{b}}}$ which is really smaller than $P$. The problem is that there are infinitely many constraints from which only few are necessary. Moreover, the resulting polyhedron $P^1$ may still contain fractional vertices. Then the process of rounding can be repeated. Chv\`atal showed that this process gives the convex hull of all integral vectors in $P$ after a finite number of times. The number of times that these cuts must be added is called the \concept{Chv\`atal-rank}.

\paragraph{}
\begin{example}
Consider the following integer program:
\begin{equation}
\begin{array}{rrcrcr}
\mbox{maximize}&&&x_2\\
\mbox{subject to}&3\cdot x_1&+&2\cdot x_2&\leq&6\\
&-3\cdot x_1&+&2\cdot x_2&\leq&0\\
&x_1&,&x_2&\geq&0\\
&x_1&,&x_2&&\mbox{integral}
\end{array}
\end{equation}

\paragraph{}
The feasible region of the LP-relaxation of this integer program is depicted in \figref{feasibleregion-gc}.


\importtikzfigure{feasibleregion-gc}{Feasible region of LP-relaxation.}

Consider now the nonnegative vector $\vec{u}=\tupl{\frac{5}{12},\frac{1}{12}}$. Using $\vec{u}$, we can construct the following linear combination of the given constraints:
\begin{equation}
\begin{array}{rcrcrcrcr}
\tfrac{5}{12}&\times&[3\cdot x_1&+&2\cdot x_2&\leq&6]&&\\
\tfrac{1}{12}&\times&[-3\cdot x_1&+&2\cdot x_2&\leq&0]&&\\\hline\\
&&x_1&+&x_2&\leq&\floor{\frac{30}{12}}&=&2
\end{array}
\end{equation}

\paragraph{}
Another possibility is to set $\vec{u}=\tupl{\frac{1}{12},\frac{5}{12}}$. The resulting constraint can be found as follows:
\begin{equation}
\begin{array}{rcrcrcrcr}
\tfrac{1}{12}&\times&[3\cdot x_1&+&2\cdot x_2&\leq&6]&&\\
\tfrac{5}{12}&\times&[-3\cdot x_1&+&2\cdot x_2&\leq&0]&&\\\hline
&&-x_1&+&x_2&\leq&\floor{\tfrac{6}{12}}&=&0
\end{array}
\end{equation}

\paragraph{}
When both additional restrictions are added to the original constraints, a polytope depicted in \figref{feasibleregion-gc-mod} with integral vertices arises. Thus, when these constraints are added to the linear program, the problem can be solved with the simplex method.

\importtikzfigure{feasibleregion-gc-mod}{New polyhedron with additional constraints.}
\end{example}

\paragraph{}
A main question is of course: how to get hold of a ``right'' $\vec{u}$? That is, a $\vec{u}$ such that the resulting inequality cuts away fractional extreme points without eliminating integral extreme vertices? A constructive method has been designed by Gomory.

\begin{example}
We continue our example described above to illustrate the idea. When we solve the LP-relaxation of the formulation given in the example with the simplex-method (see e.g. Chv\`atal\cite{Chvatal/83/Linear}), we start with the following dictionair:

\begin{equation}
\begin{array}{rcrcrcr}
x_3&=&6&-&3\cdot x_1&-&2\cdot x_2\\
x_4&=&&&3\cdot x_1&-&2\cdot x_2\\\hline
z&=&&&&&x_2
\end{array}
\end{equation}

\paragraph{}
After performing two iterations, we find the following, final, dictionair:

\begin{equation}
\begin{array}{rcrcrcr}
x_1&=&6&-&3\cdot x_1&-&2\cdot x_2\\
x_2&=&&&3\cdot x_1&-&2\cdot x_2\\\hline
z&=&&&&&x_2
\end{array}
\end{equation}

\paragraph{}
Consider now the second equation of this last dictionair. We can rewrite it as follows:

\begin{equation}
x_2+\tfrac{1}{4}\cdot x_3+\tfrac{1}{4}\cdot x_4=\tfrac{3}{2}\eqnlab{cg-cut-eq2-ini}
\end{equation}

\paragraph{}
Since $x_3,x_4\geq 0$, it follows from the latter equality that:

\begin{equation}
x_2+\floor{\tfrac{1}{4}}\cdot x_3+\tfrac{1}{4}\cdot x_4\leq\floor{\tfrac{3}{2}}
\end{equation}

\paragraph{}
Moreover, since we know that the left-hand side of this expression is integral, we can safely round down the right-hand side, and obtain:

\begin{equation}
x_2+\floor{\tfrac{1}{4}}\cdot x_3+\tfrac{1}{4}\leq\cdot x_4=1
\end{equation}

This is the same as writing

\begin{equation}
x_2\leq 1\eqnlab{cg-cut-eq2-fin}
\end{equation}

\paragraph{}
We now continue our argument by subtracting inequality \eqnnref{cg-cut-eq2-fin} from equality \eqnnref{cg-cut-eq2-ini}. This gives us:

\begin{equation}
\tfrac{1}{4}\cdot x_3+\tfrac{1}{4}\cdot x_4\geq\tfrac{1}{2}
\end{equation}

This is called a \concept{Gomory's cut}. Substituting for $x_3$ and $x_4$ their defining equations (see the first simplex dictionair), we obtain, an equivalent inequality:

\begin{equation}
x_2\leq 1.
\end{equation}
\end{example}

\begin{note}
Notice that the current fractional solution violates this inequality. Repeatedly adding such a Gomory's cut gives in the end an integral solution.
\end{note}

\paragraph{}
We now show that the procedure outlined above not only works for this particular example, but in fact works in general. We denote by $B$ the set of basic variables, and use $x_{B_i}$ to denote the basic variable corresponding to row $i$ of the dictionair. Consider row $i$ from the last dictionair:
\begin{equation}
x_{B_i}+\sumndomain[j]{B}{d_{i,j}\cdot x_j}=d_i.\eqnlab{gc-theory1}
\end{equation}

Since $x_j\geq 0$ for all $j$, it follows that 
\begin{equation}
x_{B_i}+\sumndomain[j]{B}{\floor{d_{i,j}}\cdot x_j}=d_i.
\end{equation}

Next, due to integrality of the left-hand side, we can round down the right-hand side, and obtain:
\begin{equation}
x_{B_i}+\sumndomain[j]{B}{\floor{d_{i,j}}\cdot x_j}=\floor{d_i}.\eqnlab{gc-theory2}
\end{equation}

When we define $f_{i,j}=d_{i,j}-\floor{d_{i,j}}$, and $f_i=d_i−\floor{d_i}$, and when we subtract \eqnnref{gc-theory2} from \eqnnref{gc-theory1}, we get Gomory's cut:
\begin{equation}
\sumndomain[j]{B}{f_{i,j}\cdot x_j}=f_i.
\end{equation}

\begin{note}
Notice that when $d_i$ is not integral, then $f_i>0$, and hence Gomory's cut is violated by the current solution (which, of course, has $x_j=0$ for all $j\notin B$).
\end{note}

\paragraph{}
Unfortunately, there are some disadvantages to Gomory's method:
\begin{enumerate}
 \item The number of additional constraints is usually large. Moreover, Gomory's method does not always yield the best constraints possible.
 \item The additional constraints contain a lot of non-zero elements, in general. Therefore problems may arise with respect to the numerical stability of the simplex method.
\end{enumerate}
In order to cope with these disadvantages we will develop methods for generating better constraints. Such methods are usually problem specific, but they alleviate, to some extent, the problems mentioned.

\section*{Exercises}
\begin{exercise}\exclab{interval-matrix}
Prove \colref{tum-graph}, i.e., show that the node-arc incidence matrix of a digraph is TUM, and that the node-edge incidence matrix of a bipartite graph is TUM.
\begin{hint}
Use \thmref{tum-matrix}.
\end{hint}
\begin{answer}We solve this exercise in two parts:
\begin{description}
 \item [Node-arc incidence matrix] Since $A_{v,a}$ can only be $0$, $1$ or $-1$, matrix $A$ consists out of $\accl{-1,0,1}$. The columns of $A$ represent the arcs every arc has only one tail and one head and therefore there is one $1$ and one $-1$ in each column. We can assign all rows to one partition $R_1$ since it cannot occur that two elements in a column have the same sign, there is no constraint that a row should belong to a different partition.
 \item [Node-edge incidence matrix] By definition of a node-edge incidence matrix, the matrix only contains ones and zeros. Since each edges are incident to exactly two nodes, each column has exactly two ones. A bipartite graph can be split in two set of nodes such that there are only edges between the vertices of two different sets. By assigning the rows corresponding to the vertices of the first set to the first partition and the rows corresponding to the vertices of the second set, every two rows that have both one in a column are in a different column.
\end{description}
\end{answer}
\end{exercise}
\begin{exercise}
A 0-1 matrix has the consecutive ones property if each row has its entries with value 1 in neighboring columns. Such a matrix is also called an \concept{interval matrix}. Show that such matrices are TUM.
\begin{hint}
Use induction, plus the facts that:
\begin{enumerate}
 \item interchanging two rows does not change whether a matrix is TUM, and
 \item replacing row $A_i$ by the difference of rows $A_i$ and $A_j$ also does not change whether a matrix is TUM.
\end{enumerate}
\end{hint}
\begin{answer}
\footnote{Note: this answer does not actually use induction.}
Firstly, we remark that each square submatrix of an interval matrix again is an interval matrix. So to prove that each interval matrix is TUM, it is sufficient to prove that each square interval matrix is unimodular. Secondly, given a square interval matrix $M$, if $det(M)=0$, then $M$ is unimodular.\\
What is left to prove, is that each $n$-by-$n$ interval matrix $M$ for which $det(M)\neq 0$ is unimodular, which means it has a determinant of $+1$ or $-1$. We will do this by showing that we can transform $M$ into an identity matrix with the row operations defined above.\\
The first step in this proof is to define for each row $i \in 1..n$ the variables $s_i$ and $e_i$, denoting respectively the column where the interval of $1$'s starts and ends. For instance, the following interval matrix has three rows, where $s_1=1$, $s_2=1$, $s_3=2$, $e_1=3$, $e_2=2$, $e_3=3$:
\begin{equation}
\label{unimodular_interval}
\brak{\begin{array}{ccc}
1&1&1\\
1&1&0\\
0&1&1
\end{array}}
\end{equation}
Now, whenever two rows $i$ and $j$ have $s_i=s_j$ or $e_i=e_j$, we subtract the row with the fewest $1$'s from the row with the most $1$'s. This transformation preserves both the consecutive ones property and the unimodular property, so we end up with a new interval matrix $M'$ with the same unimodular property\footnote{This transformation can change the sign of the determinant though.}, but with less $1$'s. For example, matrix \ref{unimodular_interval} can be transformed into
\begin{equation}
\brak{\begin{array}{ccc}
0&0&1\\
1&1&0\\
0&1&1
\end{array}}
\end{equation}
by noting that $s_1=s_2$, and subsequently subtracting the row $2$ from the row $1$.\\
If we now continue to apply this row operation until no two rows have the same start value or end value, we end up with a matrix containing exactly one $1$ in each row, never in the same column.\footnote{Note that no empty rows or columns can be derived, since we could assume the determinant was non-zero.} One way to understand this, is by noting that the $n$ $s_i$'s of the final matrix must all be different, and since they can only take the values $1..n$ (being the colomn numbers), each column is associated to exactly one $s_i$. The same argument can be made for the $e_i$'s. Now, since no $e_i$ can be smaller than its corresponding $s_i$, $e_i=s_i$ for all $i \in 1..n$. Continuing the example yields \ref{unimodular_interval_2} after subtracting row $1$ from row $3$ and then row $3$ from row $2$:
\begin{equation}
\label{unimodular_interval_2}
\brak{\begin{array}{ccc}
0&0&1\\
1&0&0\\
0&1&0
\end{array}}
\end{equation}
Such a matrix can after some row-swap operations be transformed into the identity, e.g.:
\begin{equation}
\brak{\begin{array}{ccc}
1&0&0\\
0&1&0\\
0&0&1
\end{array}}
\end{equation}
And since the identity is unimodular, and since we only used unimodularity preserving row operations, $M$ must be unimodular, which concludes the proof.
\end{answer}
\end{exercise}
\begin{exercise}
Consider \excref{rectanglestabbing} of \chpref{formulations}, and let us modify that problem by only considering vertical lines as
potential stabbing lines. The resulting problem can be formulated as follows: given a set of intervals,
find a minimum number of vertical lines stabbing each interval at least once.
\begin{enumerate}
 \item Give an IP-formulation of the resulting problem.
 \item Is the corresponding constraint matrix an interval matrix (see \excref{interval-matrix})?
 \item How would you solve this problem?
 \item Can you use the same approach for the rectangle stabbing problem?
\end{enumerate}
\end{exercise}
\begin{answer}
The IP-formulation is the following:
\begin{eqnarray}
\mbox{minimize}&\sumieqb[i]{1}{n}{y_i}\eqnlab{rsv-m}\\
\mbox{subject to}&\forall I\in\calI:\sumdomain[i]{I}{x_{i}}\geq 1\eqnlab{rsv-c1}\\
&\forall\rangei[i]{1}{n}:x_i\in\accl{0,1}\eqnlab{rsv-c2}
\end{eqnarray}
The corresponding matrix is TUM since it is an interval matrix. You can solve this problem using linear programming since the matrix is TUM and thus the variables are guaranteed to be integral. It is not possible to solve the rectangle stabbing problem since such matrix is not guaranteed to be TUM.
\end{answer}
\begin{exercise}
Consider the problem formulated below, where in addition, the variables $x_1$ and $x_2$ need to be integral.

\begin{equation}
\begin{array}{rrcrcr}
\mbox{maximize}&2\cdot x_1&+&x_2\\
\mbox{subject to}&4\cdot x_1&+&x_2&\leq&8\\
&-3\cdot x_1&+&3\cdot x_2&\leq&6\\
&x_1&,&x_2&\geq&0\\
&x_1&,&x_2&&\mbox{integral}
\end{array}
\end{equation}
Solving the linear program by the simplex method gives the following final dictionary.
\begin{equation}
\begin{array}{rcrcrcr}
x_1&=&\tfrac{18}{11}&-&\tfrac{3}{11}\cdot x_3&+&\tfrac{1}{11}\cdot x_4\\
x_2&=&\tfrac{16}{11}&+&\tfrac{1}{11}\cdot x_3&-&\tfrac{4}{11}\cdot x_4\\\hline
z&=&\tfrac{52}{11}&-&\tfrac{5}{11}\cdot x_3&-&\tfrac{2}{11}\cdot x_4
\end{array}
\end{equation}
\paragraph{}
Find two Gomory cuts for the problem where the variables $x_1$ and $x_2$ need to be integral. Next, rewrite each of these cuts in terms of solely $x_1$ and $x_2$.
\end{exercise}
\begin{exercise}
Consider the following problem in integral variables $x_1$ and $x_2$. How many cutting planes do you have to add to solve the problem for $k=2$? For $k=4$? For general $k$? Can you use integer rounding to generate a valid inequality?
\begin{hint}
Draw a picture.
\end{hint}
\begin{equation}
\begin{array}{rrcrcr}
\mbox{maximize}&&&x_2\\
\mbox{subject to}&x_1&-&2\cdot k\cdot x_2&\geq&0\\
&2\cdot k\cdot x_1&+&x_2&\leq&2\cdot k\\
&x_1&,&x_2&\geq&0\\
&x_1&,&x_2&&\mbox{integral}
\end{array}
\end{equation}
\end{exercise}
\begin{exercise}
Let $G=\tupl{V,E}$ be an undirected graph. A matching in $G$ is represented by a vector $\vec{x}\in\BoolSet^{\abs{E}}$ in the following way. For each edge $e\in E$ we define

\begin{equation}
\semboolvar{x_e}{if edge $e$ is selected in the matching;}{otherwise.}
\end{equation}

The set of matchings is now given by

\begin{equation}
M=\condset{\vec{x}\in\BoolSet^{|E|}}{\forall v\in V:\sumdomain[e]{\fun{\delta}{v}}{x_e}\leq 1\wedge\forall e\in E:x_e\in\accl{0,1}}.
\end{equation}

\begin{enumerate}
 \item Derive with integer rounding that
\begin{equation}
\sumdomain[e=\tupl{v_1,v_2}]{E:v_1,v_2\in S}{x_e}\leq\displaystyle\frac{\abs{S}-1}{2}
\eqnlab{matching-introunding}
\end{equation}
is a valid inequality for each $S\subseteq V$ with $3\leq\abs{S}\leq\abs{V}$ and $\abs{S}$ odd. (These inequalities are called \concept{odd-set constraints}).
 \item Find, for the graph depicted in \figref{triangle}, a fractional solution violating the inequality \eqnref{matching-introunding} with $S=\accl{1,2,3}$.
\end{enumerate}
\importtikzfigure{triangle}{A triangle.}
\end{exercise}
\begin{exercise}
Consider the integer programming formulation of the clique partitioning problem (see \excref{clique-partitioning} of \chpref{formulations}). \figref{cliquepartition-integer} shows an instance of that problem. Notice that solid lines correspond to edges with weight $+1$, whereas dotted lines correspond to edges with weight $-1$. What is the value of the linear programming relaxation of the integer programming formulation corresponding to this instance? Can you find a violated inequality using integer rounding?
\importtikzfigure{cliquepartition-integer}{An instance of clique partitioning.}
\end{exercise}

\chapter{A glimpse at computational complexity (Yves Crama)}
\chplab{complexity}
In order to fully appreciate the field of combinatorial optimization, it is necessary to understand, at least at an intuitive level, some of the basic concepts of computational complexity. This part of theoretical computer science deals with fundamental, but extremely deep questions like: ``what tasks can be carried out by a computer?'', or ``how much time does a given computational task require?''. In this chapter, we attempt to introduce some elements of computational complexity, in a very informal and hand-waving way.

\section{Computational performance criteria}
What do we expect from an algorithm for a combinatorial optimization problem? Well, an obvious answer would be that this algorithm should always return an optimal solution of the problem. Is it the only game in town? Certainly not. We might also want it to be fast or efficient. Combining these two expectations is a crucial thing. Of course, the time required to solve a problem increases with the size of this problem,  where the size can be measured by the amount of data needed to describe a particular instance of theproblem.

\begin{example}
Let us take a look at an example. Suppose that we want to solve a 0-1 linear programming problem involving $n$ variables $x_j\in\accl{0,1}$, $j=1,\ldots,n$. We can certainly find an optimal solution by listing all possible vectors $\tupl{x_1,x_2,\ldots,x_n}$, checking for each of them whether it is feasible or not, computing the value of the objective function for each such feasible solution, and, finally, retaining the best solution found in the process. If we decide to go that way, then we must consider $2^n$ vectors. For $n=50$, that means $2^{50}\approx 10^{15}=1'000'000'000'000'000$ vectors! If our algorithm is able to enumerate one million ($1'000'000$) solutions per second, the whole procedure takes $10^9$ seconds, or about $30$ years. And for $n=60$, the enumeration of the $2^{60}$ solutions would take about $30'000$ years !!

\paragraph{}
Notice that adding $10$ variables to the problem increases the computing time by a multiplicative factor of $2^{10}\approx1'000$. So, with $n=80$ variables (a rather modest problem size), the same algorithm would run for $30$ billion years, which is about twice the age of the universe. Not really efficient, by any practical standards...
\end{example}

\paragraph{}
Let us look at this issue from another vantage point. Consider the well-known Moore's law: Gordon Moore, co-founder of the chips giant Intel, prophetized in 1965 that the number of transistors per square inch on integrated circuits would double every 18 months per year starting from 1962, the year the integrated circuit was invented (see the original paper for more details). In other words, your PC processor works twice faster every year and a half, meaning that its speed is multiplied by $100$ in $10$ years\footnote{This rate has slowed down since Moore made his claim. It is now generally admitted that the number of transistors doubles every two years.}. So, if you were able to enumerate $2^n$ solutions in one hour in 1993, you can enumerate $100\cdot2^n<2{n+7}$ solutions in 2003. This great increase in computing speed thus allows you to ``gain'' only $7$ variables in 10 years!! Conclusion: exhaustive enumeration is not feasible in practice for large-scale (or even medium-scale) combinatorial optimization problems. Furthermore, we should not count on technical progress alone to improve the situation in any significant way. Only algorithmic, or mathematical, advances can help in this respect.

\paragraph{}
So, how much progress can we expect on the theoretical front? Before we provide a tentative answer to this question, let us try to formulate more precisely some of the notions that have just been sketched.

\section{Problems and problem instances}
Formally speaking, a (computational) \concept{problem} is a generic question whose formulation includes a number of undetermined parameters. Here are some simple examples.

\begin{description}
 \item[Matrix addition problem] The parameters are $n$, $A$ and $B$ where $n\in\NNN$, and $A$ and $B$ are two $n\times n$ matrices. Question: what is the value of $A+B$?
 \item[Shortest path problem] The parameters are a graph $G=\tupl{V,E}$, two vertices $s,t\in V$, and the length $\fun{l}{e}\geq 0$ of every edge $e\in E$. Question: find a shortest path from $s$ to $t$.
 \item[Traveling salesman problem] The parameters are a graph $G=\tupl{V,E}$, and the length $\fun{l}{e}\geq 0$ of every edge $e\in E$. Question: find a shortest traveling salesman tour in $G$.
\end{description}

\paragraph{}
An \concept{instance} of a problem $P$ arises when the values of all undetermined parameters of $P$ are specified (or, more intuitively, by specifying the input file that contains the numerical data). So, a problem can also be viewed as a collection of instances. Notice that an instance admits an answer, but a problem does not (try to give the answer of the matrix addition problem above!). We will use the symbol $I$ for an instance.

\paragraph{}
An \concept{algorithm} for a problem $P$ is a step-by-step procedure that describes how to compute a solution for every instance $I$ of $P$. To compare the efficiency of different algorithms for a same problem $P$ , we can determine the time required by each algorithm to solve an instance of $P$. Notice that this obviously depends on the particular instance which is to be solved, but also on the speed of the computer, on the skills of the programmer, etc. Therefore, we need again to define this notion in more formal way.

\paragraph{}
The \concept{size of an instance} $I$, denoted by $\fun{s}{I}$, is the number of bits needed to represent $I$. It is determined both by the number of parameters and by their magnitude. (Intuitively, this can be viewed as the size of the input file of a computer program which solves $I$.)

\paragraph{}
The \concept{running time} of an algorithm $A$ on an instance $I$, denoted $\fun{t_A}{I}$, is the number of elementary operations (additions, multiplications, comparisons,...) performed by $A$ when it runs on the instance $I$. Determining the running time of an algorithm for each particular instance $I$ is not an easy task. However, it is often easier to estimate the running time as a function of the size of the instance.

\paragraph{}
Consider again the examples defined above.
\begin{description}
 \item[Matrix addition problem] Instance size: $\approx n^2$.
 \begin{enumerate}
  \item Naive addition: $\approx n^2$ (additions). We denote this by $\bigoh{n^2}$, meaning that the running time grows at most like $n^2$.
 \end{enumerate}
 \item[Shortest path problem] Instance size: $\bigoh{n^2}$ where $n=\abs{V}$.
 \begin{enumerate}
  \item Enumerate all possible paths between $s$ and $t$. There could be exponentially many paths, leading to $\bigoh{2^n}$.
  \item \concept{Dijkstra's algorithm}: $\bigoh{n^2}$ operations.
 \end{enumerate}
 \item[Traveling salesman problem] Instance size: $\bigoh{n^2}$ where $n=\abs{V}$.
 \begin{enumerate}
  \item Enumerate all possible tours: $\bigoh{n!}$.
 \end{enumerate}
\end{description}

\paragraph{}
Notice that, in all these examples, we chose to ignore the size of a number, that is, we did not take into account the number of bits needed to store some number. In view of these examples, we are led to the following concept: the complexity of an algorithm $A$ for a problem $P$ is the function 
\begin{equation}
\fun{f_A}{n}=\max\condset{\fun{t_A}{I}}{\mbox{$I$ is an instance of $P$ with size $\fun{s}{I}=n$}}.
\end{equation}
This is sometimes called the ``worst-case complexity'' of $A$: indeed, the definition focuses on the worst-case running time of $A$ on an instance of size $n$, rather than on its average running.

\section{Easy and hard problems}

\begin{figure}[hbt]
\centering
\importtikzsubfigure{cpx-lin}{Linear: $\fun{F}{n}=a\cdot n+b$.}
\importtikzsubfigure{cpx-exp}{exponential: $\fun{F}{n}=a\cdot 2^n$.}
\caption{Different types of complexity behavior.}
\figlab{cpx}
\end{figure}

\figref{cpx} represents different types of complexity behaviors for algorithms. The algorithm $A$ is polynomial if $\fun{F_A}{n}$ is a polynomial (or is bounded by a polynomial) in $n$ and exponential if $\fun{F_A}{n}$ grows faster than any polynomial process in $n$. Intuitively, we can probably accept the idea that a polynomial algorithm is more efficient than an exponential one.

\paragraph{}
For instance, the obvious algorithms for the addition and or the multiplication of matrices are polynomial. So is the Gaussian elimination algorithm for the solution of systems of linear equations. On the other hand, the simplex method (or at least, some variants of it) for linear programming problems is known to be exponential\footnote{Klee and Minty provide instances I of the LP problem such that $\fun{t_{\mbox{\tiny simplex}}}{I}\geq 2^{\fun{s}{I}}$.} while interior point methods are polynomial. This clearly illustrates the emphasis on the ``worst-case'' running time which was already underlined above: indeed, in an average sense, the simplex algorithm is an efficient method.

\paragraph{}
The complete enumeration approach for shortest path, matching, stable set or traveling salesman problems is exponential, since all these problems have an exponential number of feasible solutions. But polynomial algorithms exist for the shortest path problem and the matching problem.

\paragraph{}
For stable set, the traveling salesman problem, or for 0-1 integer programming problems, by contrast, only exponential algorithms are known. In fact, it is widely suspected that there does not exist any polynomial algorithm for these problems. This is a typical feature of so-called \concept{NP-hard} problems which we define (very informally again) as follows.

\begin{definition}[NP-hard]
A problem $P$ is NP-hard if it is as least as difficult as the 0-1 linear programming problem, in the sense that any algorithm for $P$ can be used to solve the 0-1 LP problem with a polynomial increase in running time.
\end{definition}

\paragraph{}
The next claim has resisted all proof attempts (and there have been many) since the early 70's, but the vast majority of computer scientists and operations researchers believe that it holds true.

\begin{conjecture}[$P\neq NP$]
If a problem is N P -hard, then it cannot be solved by a polynomial algorithm.
\end{conjecture}

\paragraph{}
The $P\neq NP$ conjecture, if true, expresses a deep and fundamental fact of complexity theory. Its implications are of enormous importance for the development of algorithms in operations research and related areas. Indeed, a large number of combinatorial optimization problems turn out to be NP-hard, and hence difficult to solve.

\begin{proposition}
\prplab{nphard-problems}
The following problems are NP-hard:
\begin{enumerate}
 \item \concept{traveling salesman};
 \item \concept{stable set};
 \item \concept{graph coloring};
 \item \concept{knapsack};
 \item \concept{assembly line balancing};
 \item \concept{three-dimensional assignment};
 \item \concept{facility location};
 \item \concept{jobshop scheduling};
 \item several hundred other combinatorial optimization problems.
\end{enumerate}
\end{proposition}

It is quite remarkable that most of the problems in the above list can in fact be formulated as special cases of the 0-1 LP problem. This is obvious, in particular, for the knapsack problem, but it is also true (though less obvious) for graph equipartitioning, or for the traveling salesman problem, or for graph coloring, etc. Thus, the actual meaning of \prpref{nphard-problems} is that all these NP-hard problems are somehow equivalent, in the sense that an efficient algorithm for any of them would immediately provide an efficient algorithm for all of them.

\paragraph{}
From a practical point of view, however, some N P -hard problems turn out to be more difficult than others. For instance, the knapsack problem is quite easy to solve as compared with the general 0-1 LP problems. Nevertheless, NP-hard problems seem to be intrinsically tougher than linear systems, LP problems or shortest path problems. As a consequence, for the solution of such difficult (or ``apparently difficult'') problems, heuristic algorithms are often used in practice. In addition, of course, not all combinatorial optimization problems are NP-hard. \concept{Minimal spanning tree}, \concept{min-cost flow}, \concept{linear optimization} are prime examples where polynomial-time algorithms suffice to find an optimal solution.

\paragraph{}
\begin{definition}[Heuristic]
A \concept{heuristic} for an optimization problem $P$ is an algorithm which is based on intuitively appealing principles, but which does not guarantee to provide an optimal solution of $P$.
\end{definition}

So, when running on a particular combinatorial optimization problem, a heuristic could for instance:
\begin{enumerate}
 \item return an optimal solution of the problem; or
 \item return a suboptimal solution; or
 \item return an infeasible solution; or
 \item fail to return any solution at all;
 \item etc.
\end{enumerate}
This very broad definition of a heuristic may seem rather amazing at first sight. It raises again the question of the criteria which can be applied to analyze the performance of a particular heuristic. We mention here two criteria which will be of particular concern in this course.

\subsection{Computational complexity}
\ssclab{ccpx}
Generally speaking, we want heuristics to be fast, at least when compared with the highly exponential running times mentioned above. In fact, the main reason for giving up optimality is that we want the heuristic to compute quickly a reasonably good solution. Thus, the basic trade-off that we want to achieve reads
\begin{quote}
Solution quality versus running time
\end{quote}

\subsection{Quality of approximation}
\ssclab{qoa}
The solution returned by the heuristic should provide a good approximation of the optimal solution. To understand how to measure this, let $x^H$ be the solution computed by heuristic $H$ for a particular instance and let $x^{\star}$ be an optimal solution for this instance. Also, $\funm{val}{x^H}$ denotes the value of the solution found by the heuristic, and $\funm{val}{x^{\star}}$ denotes the optimum solution value. Then
\begin{equation}
\fun{E}{x^H}=\displaystyle\frac{\funm{val}{x^H}-\funm{val}{x^{\star}}}{\funm{val}{x^{\star}}}\geq 0
\end{equation}
provides a relative error measure: the closer it is to $0$, the better the solution $x^H$. Notice that we assume that we are dealing with a minimization problem.

\paragraph{}
In general, however, $\funm{val}{x^{\star}}$ is unknown. So, suppose now that we know how to compute a \concept{lower bound} on $\funm{val}{x^{\star}}$, i.e. a number $v^-$ such that $v^-\leq\funm{val}{x^{\star}}$ (this is often much easier to compute than $\funm{val}{x^{\star}}$). Define
\begin{equation}
\fun{E^-}{x^H}=\displaystyle\frac{\funm{val}{x^H}-v^-}{v^-}
\end{equation}
Then we have
\begin{equation}
\fun{E}{x^H}=\displaystyle\frac{\funm{val}{x^H}}{\funm{val}{x^{\star}}}-1\leq\displaystyle\frac{\funm{val}{x^H}}{v^-}-1=\fun{E^-}{x^H}
\end{equation}
which means that $\fun{E^-}{x^H}$ overestimates the relative error $\fun{E}{x^H}$. So, if $\fun{E^-}{x^H}$ is small, we can certainly be happy with the quality of the solution provided by $H$.
\begin{note}
Notice also that if the lower bound $v^-$ is reasonably close to $\funm{val}{x^{\star}}$, then $\fun{E^-}{x^H}$ actually provides a good estimate of the error.
\end{note}

\paragraph{}
\begin{example}
For example, consider the traveling salesman instance described by the (symmetric) distance matrix $L$, where $l_{i,j}$ represents the distance from $i$ to $j$, with $i,j\in\accl{1,2,\ldots,6}$:
\begin{equation}
L=\mtrx{cccccc}{
0&4&7&2&6&3\\
4&0&3&5&5&7\\
7&3&0&2&6&5\\
2&5&2&0&9&8\\
6&5&6&9&0&5\\
3&7&5&8&5&0
}
\end{equation}


Assume now that a heuristic returns the tour $x^H=\tupl{1,2,3,4,5,6}$ (displayed in \figref{heuristic-tsp-ex}).

\importtikzfigure{heuristic-tsp-ex}{A feasible tour.}

The total length of this tour is $\funm{val}{x^H}=4+3+2+9+5+3=26$. On the other hand, an obvious lower bound on the optimal tour length is given by the sum of the $6$ shortest distances in $6$. Thus $v^−=2+2+3+3+4+5=19$, and, consequently, $\fun{E^-}{x^H}=\tfrac{26-19}{19}\approx 0.37$. We can therefore conclude that $x^H$ is at most $37\%$ longer than the optimal tour.
\end{example}

\paragraph{}
In order to compute lower bounds for combinatorial optimization problems, a simple but powerful principle can often be used: when a constraint of a minimization problem $P$ is relaxed (i.e., when the constraint is either removed or replaced by a weaker one), then the optimal value of the resulting ``relaxed'' problem provides a lower bound on the optimal value of $P$. This principle will be illustrated on the examples below.
\section{Exercises}
\begin{exercise}
\exclab{ch3ex1}
Consider again the traveling salesman problem. For every vertex $v\in V$, select the shortest edge $e_v$ incident to $v$. Show that $\isumdomain[v]{V}{\fun{l}{e_v}}$ is a lower bound on the length of the optimal tour.
\paragraph{}
Compute this lower bound for the numerical example in \sscref{ccpx}. Can you improve this lower bound by taking into account the two shortest edges incident to every vertex $v$? What bound do you obtain for the numerical example?
\end{exercise}
\begin{exercise}
\exclab{ch3ex2}
Consider the following problem: you want to save $n$ electronic files with respective sizes $s_1,s_2,\ldots,s_n\geq 0$ on the smallest possible number of storing devices (say, floppy disks) with capacity $C$.
This problem is known under the name of \concept{bin packing problem}, and it is NP-hard. Can you compute a lower bound on its optimal value?
\end{exercise}
\begin{exercise}
\exclab{ch3ex3}
Show that the optimal value of the \concept{linear programming} problem
\begin{eqnarray}
\mbox{minimize}&\vec{c}\cdot\vec{x}\\
\mbox{subject to}&A\cdot\vec{x}\leq\vec{b}\\
&\forall j\in1,2,\ldots,n:0\leq x_j\leq 1
\end{eqnarray}
provides a lower bound on the optimal value of the \concept{0-1 linear programming} problem
\begin{eqnarray}
\mbox{minimize}&\vec{c}\cdot\vec{x}\\
\mbox{subject to}&A\cdot\vec{x}\leq\vec{b}\\
&\forall j\in1,2,\ldots,n:x_j\in\accl{0,1}
\end{eqnarray}
\end{exercise}
\begin{exercise}
Show that the lower bounds obtained in \excrefr{ch3ex1}{ch3ex2} can all be viewed as optimal solutions of a relaxation of the original problem.
\end{exercise}
\begin{exercise}
Consider the \concept{Hamiltonian cycle} problem (HC). Given a graph $G=\tupl{V,E}$, does $G$ contain a Hamiltonian cycle, i.e., a cycle visiting each node exactly once? Show that the \concept{Traveling salesman} Problem is at least as hard as the Hamiltonian cycle problem.
\end{exercise}
\begin{exercise}
Consider the \concept{Partition} problem. Given are $n$ integers $s_1,s_2,\ldots,s_n$, does there exist a set $S\subseteq{1,2,\ldots,n}$ such that $\isumdomain[i]{S}{s_i}=\isumndomain[i]{S}{s_i}$? Show that the \concept{Knapsack} problem is at least as hard as Partition.
\end{exercise}
\begin{exercise}
Show that \concept{min-cost flow} is at least as hard as \concept{shortest path}.
\end{exercise}

\chapter{Separation and Lifting: Towards a Cutting-plane Algorithm}
\chplab{lifting}
In this chapter we show how \concept[valid inequality]{valid inequalities} can be used to develop \concepts{cutting plane algorithm}. A key issue in any \concept{cutting plane algorithm} is \concept{separation}. We illustrate how the \concept{separation problem} for specific classes of \concept[valid inequality]{valid inequalities} can be solved, by considering the \concept[cover inequality]{cover inequalities} for the \concept{knapsack problem} (\secref{knapsack}), and the \concept[subtour elimination inequality]{subtour elimination inequalities} for the \concept{traveling salesman problem} (\secref{salesman}). \concept{Lifting} is about strengthening \concept[valid inequality]{valid inequalities}; we apply this concept to the \concept[cover inequality]{cover inequalities} in \secref{coverineq}.

\section{Separation for the Knapsack Problem}
\seclab{knapsack}
Although one could argue that the \concept{knapsack problem} is a very specific problem featuring only a single \concept{constraint}, one should realize that any \concept{integer programming formulation} can be seen as an \concept{optimization problem} with many \concepts{knapsack constraint}. Thus, \concept[valid inequality]{valid inequalities} that we can derive from a \concept{knapsack constraint} are important since such an \concept{inequality} can be relevant for any \concept{integer program}. We restate the \concept{knapsack problem} (see \eqnnrefr{knapsack-m}{knapsack-c2} in \chpref{formulations}).

\begin{eqnarray}
\mbox{maximize}&\sumieqb[j]{1}{n}{c_j\cdot x_j}\eqnlab{knapsack4-m}\\
\mbox{subject to}&\sumieqb[j]{1}{n}{a_j\cdot x_j}\leq b\eqnlab{knapsack4-c1}\\
&\forall\rangei[j]{1}{n}:x_j\in\accl{0,1}\eqnlab{knapsack4-c2}
\end{eqnarray}

We assume, without loss of generality, that the parameters $a_1,a_2,\ldots,a_n$, and $b$ satisfy the following \concepts{condition}:

\begin{enumerate}
 \item selecting any single \concept{item} is a \concept{feasible solution}, or: $a_j\leq b$ for each $j$, and
 \item selecting all $n$ \concepts{item} is not a \concept{feasible solution}, or: $\sumieqb[j]{1}{n}{a_j}>b$.
\end{enumerate}

\begin{note}
Notice that these \concepts{condition} can be checked easily. In the sequel we will also assume (unless explicitly stated otherwise) that the \concepts{item} in the instance of the \concept{knapsack problem} are sorted according to decreasing weights $a_j$, that is, $a_1\geq a_2\geq\ldots\geq a_n$. In the following section, we will derive a set of \concept[valid inequality]{valid inequalities} for the \concept{knapsack problem} and we will use them to sketch a \concept{solution} approach for the \concept{knapsack problem}.
\end{note}

\subsection{Separation of Cover Inequalities}

\begin{example}
Consider the following \concept{linear programming relaxation} of an instance of the \concept{knapsack problem}.

\begin{eqnarray}
\mbox{maximize}&206\cdot x_1+180\cdot x_2+176\cdot x_3+170\cdot x_4+146\cdot x_5+110\cdot x_6\\
\mbox{subject to}&83\cdot x_1+75\cdot x_2+70\cdot x_3+68\cdot x_4+59\cdot x_5+45\cdot x_6\leq 170\\
&\forall\rangei[i]{1}{6}:0\leq x_i\leq 1
\end{eqnarray}

A solution of the \concept{linear programming relaxation} of this instance is $\vec{x}^{\star}=\tupl{0,0,1,1,\dfrac{32}{59},0}$. We can conclude from this \concept{LP-solution} that \concepts{item} $3$, $4$, and $5$ cannot simultaneously be part of a \concept{knapsack} solution. Hence, $x_3=x_4=x_5=1$ is impossible. Therefore, the following \concept{constraint} is a \concept{valid inequality}:
\begin{equation}
x_3+x_4+x_5\leq2.
\end{equation}
In fact, this \concept{constraint} is a \concept{violated valid inequality}, that is, the given solution $\vec{x}^{\star}$ violates this \concept[valid inequality]{inequality}.
\end{example}

\paragraph{}
In general, for a subset $C\subseteq\accl{1,2,\ldots,n}$ of the \concepts{item} with the property that $\sumdomain[j]{C}{a_j}\geq b$, we have the following \concept{valid inequality}:

\begin{equation}
\sumdomain[j]{C}{x_j}\leq\abs{C}-1
\end{equation}

Such an \concept{inequality} is called a \concept{cover inequality}. The set $C$ is called a \concept{cover}. More precisely, any set of \concepts{item} whose total \concept{weight} exceeds $b$ is a \concept{cover} with respect to this \concept{constraint}. Thus, as an aside, earlier in this section, we in fact assumed that $\accl{1,2,\ldots,n}$ is a \concept{cover}. The \concept[valid inequality]{inequality} above is clearly \concept[valid inequality]{valid}: any \concept{feasible solution} to the \concept{knapsack problem} cannot select all \concepts{item} from $C$ when $C$ is a \concept{cover}. A \concept{minimal cover} is a \concept{cover} with the property that the removal of any \concept{item} makes the \concept{total weight} of the remaining \concepts{item} drop to $b$ or less. In other words: $C$ is a \concept{minimal cover} if

\begin{equation}
\sumdomain[j]{C}{a_j}\leq b\xand\forall i\in C:\sumdomain[j]{C\setminus\accl{i}}{a_j}\leq b
\end{equation}

The problem we now face is to find, given some \concept{point} $\vec{x}^{\star}\in\RRR^n$, a \concept{cover inequality} for the \concept{knapsack problem} that is not satisfied by the given point. More formally, the \concept{separation problem} for the \concept[cover inequality]{cover inequalities} is:

\begin{definition}[Separation problem for cover inequalities]
Given $\vec{x}^{\star}\in\RRR^n$, find a \concept{cover} $C$ such that $\sumdomain[j]{C}{x_j^{\star}}>\abs{C}-1$, or establish that no such \concept{cover} exists.
\end{definition}

Such an \concept{inequality} is called a \concept{violated cover inequality}. In general, the \concept{separation problem} is hard for \concepts{hard problem}. However, if we restrict the search of a \concept{violated inequality} to a subset of the \concept[valid inequality]{valid inequalities}, the \concept{separation problem} restricted to this subset of \concept[inequality]{inequalities}, may be easier to solve. We will restrict ourselves here to the \concept[separation problem]{separation} of the \concept[cover inequality]{cover inequalities} for the \concept{knapsack problem}.

\paragraph{}
The \concept{separation problem} for the \concept[cover inequality]{cover inequalities} is now restated as follows. Let the \concept{point} $\vec{x}^{\star}\in\RRR^n$ be given. Is there a \concept{cover inequality} that is \concept[violated inequality]{violated} by $\vec{x}^{\star}$? In mathematical terms this problem can be described as follows.

\paragraph{}
Given $\vec{x}^{\star}$: Is there a $C\subseteq\accl{1,2,\ldots,n}$ with

\begin{eqnarray}
&&\sumdomain[j]{C}{x_j^{\star}}>\abs{C}-1\xand\eqnlab{sep-c1}\\
&&\sumdomain[j]{C}{a_j}>b\eqnlab{sep-c2}
\end{eqnarray}
To solve this problem we introduce \concepts{binary variable} $z_j$ ($j=1,\ldots,n$):

\begin{equation}
\semboolvar{z_j}{if $j$ is chosen in the cover $C$;}{otherwise.}
\end{equation}

These variables will determine which \concepts{item} are present in the \concept{cover}. The \concept{solution} $\vec{z}$ to be found should satisfy the following conditions.

\begin{equation}
1+\sumieqb[j]{1}{n}{x_j^{\star}\cdot z_j}>\sumieqb[j]{1}{n}{z_j}\eqnlab{setcoversep}
\end{equation}

Clearly, this \concept{inequality} follows from reformulating \eqnnref{sep-c1} by taking the newly defined $\vec{z}$-\concepts{variable} into account (notice that, by definition, $\abs{C}=\sumieqb[j]{1}{n}{z_j}$). Let us rewrite \concept{inequality} \eqnnref{setcoversep} by bringing all terms with $\vec{z}$-\concepts{variable} to the left hand side:

\begin{equation}
\sumieqb[j]{1}{n}{1-x_j^{\star}}\cdot z_j<1
\end{equation}

\paragraph{}
Consider now \concept{inequality} \eqnnref{sep-c2}. When we plug in the $\vec{z}$-\concepts{variable} into this \concept{inequality}, the following \concept{inequality} results:

\begin{equation}
\sumieqb[j]{1}{n}{a_j\cdot z_j}\geq b+1
\end{equation}

\paragraph{}
Summarizing all \concepts{condition}, we can formulate the \concept{separation problem} for the \concept[cover inequality]{cover inequalities} as the following \concept{minimization problem}.

\begin{eqnarray}
\mbox{minimize}&\eta=\sumieqb[j]{1}{n}{\brak{1-x_j^{\star}}\cdot z_j}\eqnlab{sepco-m}\\
\mbox{subject to}&\sumieqb[j]{1}{n}{a_j\cdot z_j}\geq b+1\eqnlab{sepco-c1}\\
\forall\rangei[j]{1}{n}&z_j\in\accl{0,1}\eqnlab{sepco-c2}
\end{eqnarray}

Suppose that we are able to solve \eqnnrefr{sepco-m}{sepco-c2}. Then, if $\eta<1$, a \concept{cover inequality} violated by $\vec{x}^{\star}$ has been found. If, on the other hand, $\eta\geq1$, then we haven't found a \concept{cover inequality} violated by $\vec{x}^{\star}$ (and for good reason: none exists!). Although the \concept{separation problem} for \concept[cover inequality]{cover inequalities}, i.e., solving \eqnnrefr{sepco-m}{sepco-c2} \concept{NP-hard} in general, the problem can become fairly easy for a given instance. This is due to the fact that there are usually only few \concepts{fractional value} in the \concept{solution} $\vec{x}^{\star}$. Indeed, items whose corresponding $\vec{x}^{\star}$-\concepts{variable} equal $1$ can always be added to the \concept{cover}; this follows from observing that \concept{objective function} \eqnnref{sepco-m} will not increase when selecting such an \concept{item} in the \concept{cover}. Also, \concepts{item} whose corresponding $\vec{x}^{\star}$-\concept{variable} equal $0$ can be safely excluded from the \concept{cover}, since selecting such an item will result in $\eta\geq 1$. In other words: if $x_j^{\star}=1$ then $z_j=1$, and if $x_j^{\star}=0$ then $z_j=0$. We illustrate this with the next example.

\begin{example}
\begin{equation}
79\cdot x_1+53\cdot x_2+53\cdot x_3+47\cdot x_4+45\cdot x_5\leq178
\end{equation}

A \concept{fractional solution} for this example: $\vec{x}^{\star}=\tupl{1,1,\frac{46}{53},0,0}$. To find a \concept{violated cover} we have to solve the problem (verify this!):

\begin{eqnarray}
\mbox{minimize}&\eta=\frac{7}{53}\cdot z_3+z_4+z_5\eqnlab{sepco-ex-m}\\
\mbox{subject to}&79\cdot z_1+53\cdot z_2+53\cdot z_3+47\cdot z_4+45\cdot z_5\geq 179\eqnlab{sepco-ex-c1}\\
\forall\rangei[j]{1}{5}&z_j\in\accl{0,1}\eqnlab{sepco-ex-c2}
\end{eqnarray}

As argued above, if $x_j^{\star}$, then $z_j=1$ as well (since the the coefficient of $z_j$ in \eqnnref{sepco-ex-m} equals zero). And, since we are only interested in finding \concept[violated cover inequality]{violated cover inequalities}, $x_j^{\star}=0$ implies $z_j=0$ (since $z_j=1$ gives $\eta\geq 1$). Now, observe that the solution $\tupl{z_1,z_2,z_3,z_4,z_5}=\tupl{1,1,1,0,0}$ is feasible in \eqnnrefr{sepco-ex-c1}{sepco-ex-c2}, and has \concept{objective function} value less than $1$. Thus, we have identified a \concept{violated cover inequality}:

\begin{equation}
x_1+x_2+x_3\leq2.
\end{equation}

In the following iterations of a \concept{cutting plane algorithm}, the number of \concepts{fractional variable} may increase since the number of \concepts{constraint} increases. Therefore, identifying \concept[violated inequality]{violated inequalities} may become more complicated in forthcoming steps.
\end{example}

Consider the following example.

\begin{example}
\begin{eqnarray}
\mbox{maximize}&77\cdot x_1+6\cdot x_2+3\cdot x_3+6\cdot x_4+33\cdot x_5+13\cdot x_6+110\cdot x_7+21\cdot x_8+47\cdot x_9\\
\mbox{subject to}&774\cdot x_1+76\cdot x_2+22\cdot x_3+42\cdot x_4+21\cdot x_5+760\cdot x_6+818\cdot x_7+62\cdot x_8+785\cdot x_9\leq 1500\\
&67\cdot x_1+27\cdot x_2+794\cdot x_3+53\cdot x_4+234\cdot x_5+32\cdot x_6+797\cdot x_7+97\cdot x_8+435\cdot x_9\leq 1500\\
&\forall\rangei[i]{1}{9}:0\leq x_i\leq 1
\end{eqnarray}

In the following table we see how the \concept{linear programming solution} and its value develop after adding \concept[violated cover inequality]{violated cover inequalities}. In the first table we give the solutions that are generated in the iterations. In the second table the \concept[violated cover inequality]{violated cover inequalities} are given. Notice that we can derive \concept[cover inequality]{cover inequalities} from both \concepts{constraint}.

\importtabulartable{lift-exa}{Linear programming relaxation for the leading example.}

\importtabulartable{cove-exa}{Cover for the leading example.}

Notice that the \concept{inequality} $x_1+x_7+x_9\leq1$ is not a \concept{cover inequality}. Indeed, although $\tupl{1,7,9}$ is a \concept{cover} with respect to the first \concept{inequality}, the right-hand side does not equal $\abs{C}-1$. This \concept{inequality} is a so-called \concept{extended cover inequality}. The \concept[extended cover inequality]{extended cover inequalities} are the subject of \secref{liftingeci}.

\paragraph{}
From the first table one can see that the number of \concepts{fractional variable} is small compared to the total number of \concepts{variable}. Thus, the \concept{separation problem} for the \concept[cover inequality]{cover inequalities} is fairly easy here. In general, the number of \concepts{fractional variable} is bounded by the number of \concepts{constraint} of the problem at hand. Although this number may increase, the number of \concepts{fractional variable} tends to remain limited. This observation is true for many problems.
\end{example}


\section{Separation for the Traveling Salesman Problem}
Consider the following \concept{formulation} of the \concept{symmetric traveling salesman problem}, which uses a \concept{binary variable} $x_e$ indicating whether \concept{edge} $e$ is selected or not.

\begin{eqnarray}
\mbox{minimize}&\sumdomain[e]{E}{c_e\cdot x_e}\\
\mbox{subject to}&\forall v\in V:\sumdomain[e]{\fun{\delta}{v}}x_e=2\\
&\forall S\subseteq V:\abs{S}\geq2\rightarrow\sumdomain[e]{\fun{\delta}{S}}x_e\geq 2\\
&\forall e\in E:x_e\in\accl{0,1}
\end{eqnarray}

Obviously, \concept{constraints} (4.14) ensure that each node must be \concept{incident} to two \concepts{edge}; these \concepts{constraint} are referred to as \concepts{degree constraint}. Constraints (4.15) state that, for each \concept{nodeset} S ⊂ V , there at least two
edges with one endpoint in S and one endpoint not in S. These constraints are the subtour elimination
constraints.
When we replace the integrality constraints (4.16) by 0 ≤ xe ≤ 1, we have created a linear program that
we refer to as the subtour LP. Clearly, solving this subtour LP gives a lower bound to the optimal value
of the TSP instance. Notice however, that this subtour LP contains exponentially many constraints. We
will show how to solve the separation problem for the subtour elimination constraints efficiently, i.e., in
polynomial time. This implies that we can optimize over the subtour LP in polynomial time (despite the
exponential number of constraints).
So, let us assume we are given a vector x∗e , found, for instance, by optimizing (4.13) over the constraints
(4.14) and 0 ≤ xe ≤ 1. The question now is: does there exist a violated subtour elimination constraint?
In other words, is there a set S ⊂ V such that the left-hand side of (4.15) is less than 2? To answer this
question we build a network G consisting of nodes, edges, and a capacity for each edge as follows. There
is a node in G for each city in the TSP-instance, and there is an edge in the network for each e with
x∗e > 0. The capacity of this edge equals x∗e .
Proposition 4.2.1 The value of a minimum cut in G is less than 2 if and only if there exists a violated
subtour elimination constraint.
53

We leave it to the reader to verify this proposition. Concluding, by solving a minimum cut problem we
solve the separation problem for the subtour elimination constraints.

\section{Lifting: Extended Cover Inequalities}
\seclab{liftingeci}

Consider the example in Section 4.1. From the first constraint we could have obtained the valid inequality
x1 + x6 + x7 + x9 ≤ 1
. Indeed, since the coefficients of x1 , x6 , x7 and x9 are each larger than 750, no two of these variables
can both equal 1. Thus, this inequality is valid, and in fact, it is stronger that the corresponding cover
inequalities. Addition of this particular inequality to the linear programming relaxation would have
given the integral solution immediately. Unfortunately, this inequality is not part of the class of cover
inequalities. But we can derive it from a cover inequality with a technique called lifting.
Example Let S be the set of feasible solutions of the problem in the previous section. Consider the inequality x1 +x7 ≤ 1. This inequality is valid in the set of solutions restricted to the items {1, 2, 3, 4, 5, 7, 8},
i.e., it is valid for the set S ∩ {x ∈ IB 9 |x6 = x9 = 0}. We will now derive an inequality
x1 + αx6 + x7 ≤ 1
where α is chosen such that this inequality is valid for S ∩ {x ∈ IB 9 |x9 = 0}. In order to do so, we have
to evaluate two situations.
x6 = 0:

in this case the inequality remains valid trivially.

x6 = 1:

in this case each of the items 1 and 7 can not be taken into the knapsack because constraint 1
would be violated. Therefore, taking α = 1 maintains validity of the constraint.

This process can be repeated for item 9. It will lead to the inequality
x1 + x6 + x7 + x9 ≤ 1.
Notice that the resulting inequality is not a cover inequality (there are four variables on the lefthand side,
whereas the righthand side does not equal 3). This inequality belongs to the class of so-called extended
cover inequalities.
54

Recall that we have assumed that the items are indexed according to decreasing weight. Consider a
minimal cover C = {j1 , . . . , jk }. Then the corresponding extended cover E(C) is defined as
E(C) = {1, . . . , j1 − 1} ∪ C.
Thus, each item that is heavier than the heaviest item in the cover C is added in order to construct the
set of items E(C). We claim:
Theorem 4.1

j∈E(C)

xj ≤ |C| − 1 is a valid inequality for the knapsack problem.

The derivation of an extended cover inequality from a cover inequality can be seen as a special case
of a process called lifting. Given some valid inequality (say, a cover inequality) lifting refers to raising
coefficients of variables in this inequality in order to obtain a stronger, yet valid, new inequality (the
extended cover inequality). Indeed, the coefficients of variables x1 , x2 , . . . , xj1 −1 have value 0 in the cover
inequality, and have been raised to value 1 in the extended cover inequality. This principle, however, can
be applied in a more general way. In fact, lifting is a technique that can be used in many problems to
strengthen known classes of valid inequalities. The following theorems can be used repeatedly for this
purpose. In the theorems stated below, we assume that x is a n-dimensional {0, 1} vector, and that S
denotes the set of feasible x-vectors.
Theorem 4.2 Suppose that
Then αx1 +

n
j=2

n
j=2

πj xj ≤ π0 is valid for each x ∈ S with x1 = 0.

πj xj ≤ π0 is valid for each x ∈ S as long as α ≤ π0 − maxx∈S|x1 =1

Theorem 4.3 Suppose that
Then βx1 +

n
j=2

n
j=2

n
j=2

πj xj .

πj xj ≤ π0 is valid for each x ∈ S with x1 = 1.

πj xj ≤ π0 +β is a valid inequality for each x ∈ S as long as β ≥ maxx∈S|x1 =0

n
j=2

πj xj −

π0 .
Notice that applying Theorem 4.2 does not change the righthand side, whereas applying Theorem 4.3
may change the righthand side. In the remainder of this section we give an example in which the two
theorems above are applied repeatedly.
Example Consider the knapsack constraint
3x1 + x2 + x3 + x4 + x5 ≤ 4.
55

(4.17)

As before, let S be the set of (binary) feasible solutions of this constraint. Suppose that we fix the values
of the first three variables as follows:
x1 = 1, x2 = x3 = 0.
Plugging in these values in (4.17) reduces the inequality to x4 + x5 ≤ 1. Thus x4 + x5 ≤ 1 is valid for
each x ∈ S with x1 = 1, x2 = x3 = 0.
We now lift variable x3 by applying Theorem 4.2. This gives (using α3 as coefficient for x3 ): find a
maximal α3 such that α3 x3 + x4 + x5 ≤ 1 is valid for each in x ∈ S with x1 = 1, x2 = 0, x3 = 1. This is
equivalent to determining a maximal α3 such that
α3 ≤ 1 −

max

x∈S|x1 =1,x2 =0,x3 =1

x4 + x5 .

It follows that α3 = 1. In other words, we can conclude that x3 + x4 + x5 ≤ 1 is valid for all x ∈ S with
x1 = 1, x2 = 0.
We now proceed by lifting variable x1 using Theorem 4.3. This gives (using β1 as coefficient for x1 ): find
a minimal β1 such that β1 x1 + x3 + x4 + x5 ≤ 1 + β1 is valid for each x ∈ S with x1 = x2 = 0. This is
equivalent to determining a minimal β1 such that
β1 ≥

(x3 + x4 + x5 ) − 1.

max

x∈S|x1 =x2 =0

It follows that β1 = 2. Thus, we conclude that 2x1 + x3 + x4 + x5 ≤ 3 is valid for all x ∈ S with x2 = 0.
Finally, we lift variable x2 using Theorem 4.2. This gives (using α2 as coefficient for x2 ): find a maximal
α2 such that 2x1 + α2 x2 + x3 + x4 + x5 ≤ 3 is valid for each x ∈ S with x2 = 1. This is equivalent to
determining a maximal α2 such that
α2 ≤ 3 −

max

x∈S|x2 =1

2x1 + x3 + x4 + x5 .

It follows that α2 = 0. We arrive at the conclusion that the resulting inequality
2x1 + x3 + x4 + x5 ≤ 3

(4.18)

is a valid inequality for each x ∈ S. Notice that we have derived a new inequality (4.18) that is stronger
than the original one (4.17). As an example, consider the solution x = ( 13 , 0, 1, 1, 1), and observe that
this particular solution is cut off by (4.18), while it is not violated by (4.17).
56

Here we have lifted the variables x1 , x2 and x3 in the sequence x3 , x1 , x2 . It is interesting to notice that
different sequences may yield different inequalities. For instance, the sequence x3 , x1 , x2 produces another
inequality. Thus, the order of lifting influences the final inequality we get, see Exercise 4.

\section{Applications}

Knapsack constraints are embedded in many structured problems. Two obvious examples are the multiknapsack problem and the generalized assignment problem. We will give their respective formulations
below.
The multi-knapsack problem

n

cj xj

(M KP ) max

(4.19)

j=1
n

aij xj ≤ bi

s.t.

for all i = 1, 2, . . . , m

(4.20)

for all j = 1, 2, . . . , n

(4.21)

j=1

xj ∈ {0, 1}

The generalized assignment problem

m

n

cij xij

(GAP ) max

(4.22)

i=1 j=1
n

s.t.

aij xij ≤ bi

for all i = 1, 2, . . . , m

(4.23)

for all j = 1, 2, . . . , m

(4.24)

for all i = 1, 2, . . . , m, for all j = 1, 2, . . . , n.

(4.25)

j=1
m

xij ≤ 1
i=1

xij ∈ {0, 1}

57

\section*{Exercises}
\begin{exercise}
Find all inequalities, induced by minimal covers, of the following knapsack inequality:
79x1 + 53x2 + 53x3 + 45x4 + 45x5 ≤ 178.
Can you find an extended cover inequality?
\end{exercise}

\begin{exercise}
Consider the following knapsack problem.
max

3x1 + 2x2 + x3

s.t.

4x1 + 3x2 + 2x3 ≤ 6
x1 , x2 , x3 ∈ {0, 1}

Solve this problem with a cutting-plane algorithm.
\end{exercise}

\begin{exercise}
It is clear that one can use the simplex-method to find the linear programming relaxation of the knapsack
problem as formulated by (1.10)-(1.12). However, we claim that there is a much easier way:
Sort the items such that
c1
c2
cn
≥
≥ ... ≥
,
a1
a2
an
and next fill the knapsack with items 1, 2, . . . , k − 1, until item k can only be fractionally added to the
knapsack. In other words, we claim that the following solution is an optimal linear programming solution
to the knapsack problem.



1


k−1
xj =
(b − j=1 aj )/ak



 0

for all j = 1, 2, . . . , k − 1
if j = k
for all j = k + 1, . . . , n.

Can you prove this?

Apply this to the example in Section 4.1.
\end{exercise}

\begin{exercise}
Consider the example in Section 4.3. What inequality results if the variables are lifted in the sequence
x2 , x1 , x3 ? And what inequality results when using x1 , x2 , x3 as lifting sequence?
\end{exercise}

\chapter{Column generation and branch-and-price}

In this chapter we describe a technique that is used for solving linear programs with a huge number of variables, namely \concept{column generation} (see \secref{columngen}). The use of this technique within an \concept{enumerative framework} is next discussed in \secref{branchprice}. We illustrate these issues on the \concept{cutting stock problem} (\concept{bin packing}) and the \concept{crew scheduling problem}.

\section{Column generation}
\seclab{columngen}

In this section we show how column generation procedures can work. We emphasize that column generation is a technique for solving (large) linear programming problems. The main idea is \begin{inlineenumerate}\item to work in iterations, and deal, at all times during the course of the method, with only a limited number of variables, and \item to \concept{verify optimality} of the outcome of each iteration using \concept{complementary slackness}\end{inlineenumerate}. In \sscref{cuttingstock} we show how this method can be applied to the \concept{cutting stock problem}, and in \sscref{crewscheduling}, we show how it works for a crew scheduling problem.

\subsection{The cutting stock problem}
\ssclab{cuttingstock}

We start by illustrating this on the \concept{cutting stock problem}. This problem can be described as follows.
\begin{definition}[Cutting stock problem]
Given are $n$ item types; there is a demand of $d_i$ items for type $i$, and each item of type $i$ has size $w_i$, $i=1,2,\ldots,n$. Let the total number of items be denoted by $D=\sum_i d_i$. Given are a (large enough) number of rolls of size $L$. The problem is to fulfill demand exactly, using a minimum number of rolls.
\end{definition}

\paragraph{}
A straightforward formulation is as follows, where we use, with $i=1,\ldots,D$, $j=1,2,\ldots$. binary variables:
\begin{equation}
\semboolvar{x_{ij}}{if item $i$ is cut from roll $j$;}{otherwise.}
\end{equation}
and
\begin{equation}
\semboolvar{y_j}{if roll $j$ is used;}{otherwise.}
\end{equation}
\paragraph{}
The formulation is now:
\begin{eqnarray}
\mbox{maximize}&\sumieqb[j]{1}{\infty}{y_j}\eqnlab{cutting-m}\\
\mbox{subject to}&\forall j\in\accl{1,2,\ldots}:\sumieqb[i]{1}{D}{w_i\cdot x_{ij}}\leq L\cdot y_j\eqnlab{cutting-c1}\\
&\forall i\in\accl{1,2,\ldots,D}:\sumieqb[j]{1}{\infty}{x_{ij}}=1\eqnlab{cutting-c2}\\
&\forall i\in\accl{1,2,\ldots,D},j\in\accl{1,2,\ldots}:x_{ij}\in\accl{0,1}\eqnlab{cutting-c3}\\
&\forall j\in\accl{1,2,\ldots}:y_j\in\accl{0,1}\eqnlab{cutting-c4}
\end{eqnarray}

\paragraph{}
There are a number of reasons why this formulation is not suited for solving large instances of the \concept{cutting stock problem}. First of all, the linear programming bound is very weak, since the solution $x_{ij}=\dfrac{1}{D}$ is feasible to the linear programming relaxation (verify that this leads to a linear programming solution with value ($\sumieqb[i]{1}{\infty}{w_i/L}$)). Secondly, symmetry of solutions (the rolls are interchangeable) will also affect the performance of a \concept{branch-and-bound algorithm} based on this formulation. Therefore, another formulation based on another concept is welcome. Instead of focusing on individual items, let us focus on a possible way to cut a roll. More precisely, we refer to a possible way of cutting a roll as a \concept{pattern}. Obviously, given the input described above, we could, in principle enumerate all possible patterns. Then, we could write down the following formulation that uses an integral variable $z_j$ for the number of times a roll is cut according to \concept{pattern} $j=1,2,\ldots$:

\begin{eqnarray}
\mbox{maximize}&\sumieqb[j]{1}{\infty}{z_j}\eqnlab{cuttingb-m}\\
\mbox{subject to}&\forall i\in\accl{1,2,\ldots,n}:\sumieqb[j]{1}{\infty}{a_{ij}\cdot z_j}=d_i\eqnlab{cuttingb-c1}\\
&\forall j\in\accl{1,2,\ldots}:z_j\in\NNN\eqnlab{cuttingb-c2}
\end{eqnarray}

Notice that $a_ij$ denotes the number of times that an item of type $i$ is used in pattern $j$; this is a known number. Now although this formulation does not suffer from the disadvantages mentioned above, there is a disadvantage to the current formulation: the number of variables. Indeed, this number can be astronomically high.

\paragraph{}
Let us therefore first focus on solving the linear programming relaxation of formulation \eqnnrefr{cuttingb-m}{cuttingb-c2}. How to overcome the obstacle posed by the number of variables? The crucial idea of column generation is based on the observation that in an optimal solution of the linear programming relaxation of \eqnnrefr{cuttingb-m}{cuttingb-c2}, only very few of these variables will have a value different from $0$. Indeed, the theory of the simplex-algorithm tells us that if there is an optimal solution, this solution will consist of $n$ basic variables that may have a non-zero value, while all other variables will be non-basic and have value $0$. How to make use of this observation? Of course, we do not know a priori which variables are basic in an optimal solution and which variables are not. However, given a feasible basic solution, we can determine (using complementary slackness and duality) whether it is optimal, and if not, which variable should be included in the basis to improve the current solution. This idea suggests an iterative procedure for solving the LP-relaxation of \eqnnrefr{cuttingb-m}{cuttingb-c2} as follows:

\begin{stepenum}
 \item \stplab{colgen-a}Start with a subset of the variables (that contains a feasible solution). All other variables have implicitly the value $0$. This is called the \concept{restricted master problem}.
 \item \stplab{colgen-b}Solve the LP-relaxation, and arrive at a \concept{feasible primal solution}.
 \item \stplab{colgen-c}Question: does there exist a variable (a pattern) with \concept{negative reduced costs}? That is, does there exist a variable that should enter the basis? This question is called the \concept{pricing problem}. If no, the current LP-solution is optimal, and we STOP, else
 \item \stplab{colgen-d}Identify that variable and add it to the current set of variables. Go to \stpref{colgen-a}.
\end{stepenum}

\paragraph{}
At first sight, there may be no reason why this would be more efficient than solving the linear program with all variables. Indeed, the crux of this approach lies in solving the \concept{pricing problem}: if explicitly determining the \concept{reduced cost} of each individual variable by enumeration was the only way to solve the pricing problem, we would not have gained much. However, in a lot of cases, solving the \concept{pricing problem} can be done very efficiently, for instance by solving a \concept{shortest path problem}, or by solving a \concept{knapsack problem}. Then, \concept{column generation} is a very efficient way of solving the master problem. Let us proceed by illustrating this idea on the \concept{cutting stock problem}. Consider a column in the matrix $A$ of the second cutting stock formulation, and let us denote this column by $\vec{a}$. It describes a \concept{pattern}, that is, the entries of that column equal the \concept{multiplicity} of each item type in that \concept{pattern}. Of course, for a column $\vec{a}=\tupl{a_1,a_2,\ldots,a_n}$ of $A$ to be a \concept{feasible pattern} it must be true that $\sumieqb[i]{1}{\infty}w_i\cdot a_i\leq L$. But the converse is true as well, that is, any $n$ nonnegative integers $\tupl{a_1,a_2,\ldots,a_n}$ that satisfy $\sumieqb[i]{1}{\infty}w_i\cdot a_i\leq L$ is a \concept{feasible pattern}. Let us denote pattern $j$ by nonnegative integers $\tupl{a_{1j},a_{2j},\ldots,a_{nj}}$.

\paragraph{}
Now suppose we are given a \concept{primal solution} $\vec{z}=\tupl{z_1,z_2,\ldots}$, and we are also given the values of the \concept{dual variables} $\tupl{u_1,u_2,\ldots,u_n}$ associated to the constraints \eqnnref{cuttingb-c1}. How to determine whether $\vec{z}$ is an \concept{optimal solution} to the LP-relaxation of \eqnnrefr{cuttingb-m}{cuttingb-c2}? Let us consider the constraints of the dual of the LP-relaxation of \eqnnrefr{cuttingb-m}{cuttingb-c2}. We know that these constraints are of the form: $\sumieqb[i]{1}{\infty}u_i\cdot a_ij\leq1$ for each \concept{possible pattern} $j$ (verify this!). Thus, if the current $\vec{u}$-values satisfy all these constraints (i.e., for each \concept{pattern} $j$), we have an \concept{optimal LP-solution}. Or, alternatively formulated, we are looking for a \concept{pattern} $\tupl{a_1,a_2,\ldots,a_n}$ such that $\sumieqb[i]{1}{\infty}w_i\cdot a_i\leq L$ and $\sumieqb[i]{1}{\infty}u_i\cdot a_i>1$. If we can find such a \concept{pattern}, we know that the current LP-solution is not optimal, and hence we add this \concept{pattern} to the current set of columns and start a new iteration. Otherwise, if we cannot find such a \concept{pattern}, the current LP-solution is optimal. Summarizing, the \concept{pricing problem} boils down to solving the following problem:

\begin{eqnarray}
\mbox{maximize}&\sum_[i]{u_i\cdot a_i}\eqnlab{pricing-m}\\
\mbox{subject to}&\sum_[i]{w_i\cdot a_i}\leq L\eqnlab{pricing-c1}\\
&\forall i:a_i\in\NNN\eqnlab{pricing-c2}
\end{eqnarray}

This problem is (a variant of) the well-known \concept{knapsack problem} (which, in spite of its \concept[NP-hard]{NP-hardness}, can be solved in reasonable computing times for very large instances, see \chpref{lifting}). Concluding, the \concept{pricing problem} can be solved relatively fast, and hence the LP-relaxation of \eqnnrefr{cuttingb-m}{cuttingb-c2} can be solved fast as well.

\subsection{The hierarchical crew scheduling problem}
\ssclab{crewscheduling}

In this subsection we illustrate the \concept{column generation} technique on the \concept{hierarchical crew scheduling
problem}. This problem can be described as follows.

\begin{definition}[hierarchical crew scheduling problem]
Given are $m$ crews who have to perform $n$ tasks. For each \concept{crew} $t$ a \concept{cost-rate} $r_t$ is known, $t=1,2,\ldots,m$. Each \concept{task} is characterized by a \concept{starting time} $s_i$ and a \concept{processing time} $p_i$. Each task $i$ has to start at $s_i$ and must be carried out \concept{nonpreemptively} by some crew until $s_i+p_i$. Also, a \concept{distance} $d_{ij}$ for every pair of tasks $i$ and $j$ is given. Finally, the crews are \concept{hierarchically ordered} in the following way: for each job a 'maximal' crew is given such that each crew with an index higher than the index of the maximal crew is not capable of performing that job; all other crews are capable of performing that job. The cost of a crew is the product of its cost rate $r_t$ and the distance traveled by that crew. The problem is to find an assignment of tasks to crews against minimal costs.
\end{definition}

\begin{example}
Here is an instance with $2$ crews and $3$ jobs:
\begin{equation}
m=2, n=3, r_1=3, r_2=1, s_1=20, s_2=40, s_3=60, p_1=p_2=p_3=0
\end{equation}
job $1$ can only be carried out by crew $1$, jobs $2$ and $3$ can be carried out by each of the two crews and with distances $d_{ij}$ as in \figref{hcsp-distances}.
\importtikzfigure{hcsp-distances}{The distance of the hierarchical crew scheduling problem example.}
\end{example}

\paragraph{}
One (of the~$2$) optimal solution(s) for this instance can be described as follows. Starting at time~$0$, crew~$1$ travels to job~$1$, waits~$4$ time units, performs the job, travels to job~$3$, waits~$28$ time units, performs it and returns to the depot. Distance traveled is~$37$, so its costs are~$111$. Crew~$2$ simply travels to job~$2$, performs it and returns for a total cost of~$20$. Thus, the value of an optimal solution to this instance is~$131$.

\paragraph{}
We model the \concept{hierarchical crew scheduling problem} as a problem on a \concept{weighted, directed graph} $G=\tupl{V,A}$ as follows. Construct a \concept{vertex} for each task $i=1,\ldots,n$ and let $V=\accl{1,\ldots,n}\cup\accl{s,f}$. The vertices $s$ and $f$ can be regarded as the \concept{source} and \concept{sink} of the \concept{graph} $G$. There is an \concept{arc} from \concept{vertex} $i$ to $j$ in $A$ if $s_i+p_i+d_{ij}\leq s_j$ for all $i,j\in V\setminus\accl{s,f}$. Further, there is an \concept{arc} $\tupl{s,i}$ and $\tupl{i,f}$ in $A$ for all $i\in V\setminus\accl{s,f}$. Finally, there is a cost vector $c_{ij}^t$ associated to each \concept{arc} $\tupl{i,j}\in A$. We compute this \concept{vector} as follows:

\begin{equation}
\semvalvar{c_{ij}^t}{r_t\cdot d_{ij}}{if crew $t$ is able to do jobs $i$ and $j$,}{M}{otherwise, for all $t$, for all $\tupl{i,j}\in A$}
\eqnlab{semcijt}
\end{equation}

where $M$ is a large number.

\paragraph{}
Consider now the following formulation using the following parameters:
\begin{itemize}
 \item R is the set of paths in $G$ from $s$ to $f$,
for $i=1,2,\ldots,n$, $r\in R$:

\begin{equation}
\semboolvar{\delta_{ir}}{if vertex $i$ is in path $r$, and}{otherwise}
\end{equation}

 \item for $t=1,2,\ldots,n$, $r\in R$, $c_{tr}$ is the cost incurred when crew $t$ takes path $r$, (observe that, for a given $r$, $t$, we can easily compute this quantity using \eqnref{semcijt}; notice that if a path $r$ contains a vertex that cannot be served by crew $t$, we set the corresponding $c_{tr}$ to a large number), and using, for $t=1,\ldots,m$, $r\in R$, the decision variables
 
\begin{equation}
\semboolvar{y_{tr}}{if crew $t$ takes path $r$, and}{otherwise}
\end{equation}

\end{itemize}
We now introduce a new formulation based on the introduced variables:

\begin{eqnarray}
\mbox{minimize}&\sumieqb[t]{1}{m}{\sumdomain[r]{R}{c_{tr}\cdot y_{tr}}}\eqnlab{cghcsp-m}\\
\mbox{subject to}&\forall i\in\accl{1,2,\ldots,n}:\sumieqb[t]{1}{m}{\sumdomain[r]{R}{\delta_{ir}\cdot y_{tr}}}=1\eqnlab{cghcsp-c1}\\
&\forall t\in\accl{1,2,\ldots,m}:\sumdomain[r]{R}{y_{tr}}\leq1\eqnlab{cghcsp-c2}\\
&\forall t\in\accl{1,2,\ldots,m},r\in R:y_{tr}\in\accl{0,1}\eqnlab{cghcsp-c3}
\end{eqnarray}

Constraints \eqnnref{cghcsp-c1} state that each vertex must occur once in a selected path, inequalities \eqnnref{cghcsp-c2} express that a crew can do at most one path and constraints \eqnref{cghcsp-c3} are the \concept{integrality constraints}. The LP relaxation of this model is found by replacing constraints \eqnref{cghcsp-c3} by $y_{tr}\geq0$ for all $t,r$.

\paragraph{}
Given a \concept{feasible basis} for some LP the question determining whether this basis is an optimal one is: do there exist variables with negative \concept{reduced costs}? Using \concept{dual variables} $u_i$ attached to constraints \eqnnref{cghcsp-c1} and \concept{dual variables} $e_t$ to constraints \eqnnref{cghcsp-c2} we can deduce the following expression for the reduced costs of variable $y_{tr}$:

\begin{equation}
c_{tr}-\sumieqb[i]{1}{n}{\delta_{ir}\cdot u_i-e_t}
\end{equation}

Thus, given some LP-solution and its associated \concept{dual variables} the \concept{pricing problem} boils down to the following question:

\begin{equation}
\exists t,r\in R:c_{tr}-\sumieqb[i]{1}{n}{\delta_{ir}\cdot u_i-e_t}<0
\end{equation}

This question can be answered as follows:
\begin{lemma}
The \concept{price problem} can be solved by solving $m$ \concepts{shortest path problem} on a \concept{directed acyclic graph}.
\begin{proof}
We claim that for a fixed $t$, the \concept{price problem} boils down to a \concept{shortest path problem} which implies the lemma. This can be seen as follows: consider the \concept{graph} corresponding to the instance and consider only those nodes that can be visited by crew $t$. Observe that no \concepts{cycle} occur in this network. Modify the existing \concepts{arc cost} $c_{ij}^t$ by setting $h_{ij}:=c_{ij}^t-u_j$ for all $\tupl{i,j}\in A$. Observe that the cost of a path $P$ in this network with respect to costs $h_{ij}$ from $s$ to $f$ equals: $\sumdomain[\tupl{i,j}]{P}{h_{ij}}=\sumdomain[\tupl{i,j}]{P}{c_{ij}^t-u_j}=c_{tr}-\sum_i{\delta_{ir}\cdot u_i}$. If this expression is smaller than $e_t$, there is a \concept{profitable column} (\concept[profitable path]{path}) for crew $t$, otherwise not.
\end{proof}
\end{lemma}
This shows that the \concept{pricing problem}, and hence the associated \concept{LP-relaxation}, can be solved efficiently.

\section{Branch-and-price}
\seclab{branchprice}

Of course, it is nice that one can solve a linear program with (exponentially) many variables efficiently
using column generation. However, this leaves us with a potentially fractional solution. We are interested
in integral solutions. Can we embed column generation within an enumerative framework so that we are
guaranteed to find an integral optimum? In this section we describe such an approach.
Let us consider the example of the hierarchical crew scheduling problem. A simple idea would be to
argue as follows: given a fractional solution, pick a variable with a fractional value. We know that
in an optimal solution this variable will have either value 0 or value 1. Create two subproblems, one
67

subproblem in which that variable equals 1, and another subproblem where this variable is constrained to
be 0. Unfortunately, this idea is too simple. Setting a variable to 1 in the crew scheduling context poses
no problem: indeed the problem becomes smaller since we postulate that, in this branch this crew t takes
path r. However, setting a variable to 0 causes a difficulty: how can we guarantee that when we solve
the pricing problem this specific variable does NOT come out as the variable to be added? One might
think: OK let us find the shortest path, and if it corresponds to this variable, find the second shortest
path. But that idea is bound to cause difficulties when we have a branching tree that can have many
levels.
We need another partitioning of the solution space. Instead of branching by setting a variable to 1 versus
setting it to 0, we need a branching rule that does not destroy the efficient solvability of the pricing
problem. For the hierarchical crew scheduling problem, an example of such a rule is as follows: Here
we propose a branching rule that leaves the structure of the problem intact, allowing for the efficient
solvability of the pricing problem throughout the search tree.
Suppose that y is a fractional feasible LP-solution, and let us call a path r from s to f a fractional path
(with respect to y) if there exists a t with 0 < ytr < 1. We claim that y has the following property: there
exist two vertices i and j (that differ from source and sink) that lie consecutively on a fractional path
such that the sum of all ytr such that r contains arc {i, j} is greater than 0 and smaller than 1. Let us
formally phrase this claim in the following lemma:
Lemma: If y is fractional, there exist two nodes i, j ∈ V \ {s, f }, {i, j} ∈ A with
is contained in r ytr < 1.
Proof: Observe that if y is fractional there are at least two different fractional paths. Now, consider the
0<

t

r:{i,j}

first node in each fractional path. If this set of nodes has cardinality more than 1, the claim is easily seen
to be true. So assume that each fractional path has the same first node. However, then we can repeat
this argument replacing the sink s by this first node. Since there must be at least two fractional paths,
a pair i, j as described in the lemma must exist.
✷

Thus, by this Lemma, when y is fractional there exist nodes i and j (that differ from source and sink)
that are connected by an arc whose sum of fractional values lies between 0 and 1. Now, in an optimal
solution either these nodes are visited consecutively, or they are not. More specifically, given a fractional
68

solution we identify two nodes i and j having the property described above. Then we branch as follows.
In one branch we modify G into G1 by deleting all arcs {i, p} for p = j and all arcs {p, j} for p = i. Thus,
in any feasible solution there is a path that contains arc {i, j}. In the other branch we modify G into G2
by simply deleting arc {i, j}. In this case it is obvious that no feasible solution has a path with arc {i, j}
in it. Notice that the current solution is excluded by this rule. Let us illustrate this branching rule on
the example.
Example (continued): Consider the instance described in the example. The graph corresponding to
this instance is depicted in Figure 5.2.

Figure 5.2: The graph G
An optimal solution to the LP-relaxation of (5.12)-(5.14) for this instance is described by y2,s−2−3−f =
y1,s−1−2−f = y1,s−1−3−f = 12 , and all other variables 0. Now, let 1,2 be a pair of nodes that we branch
on. The resulting graphs G1 and G2 are depicted in Figure 5.3:



Figure 5.3: The graphs G1 (left) and G2 (right)
Notice that in G1 the arcs {s, 2}, {1, 3} and {1, f } have been deleted. When solving the LP-relaxation
corresponding to G1 we find an integral solution with value 132. G2 is constructed by deleting arc {1, 2}.
In this branch we also find an integral solution with value 131, which is therefore optimal.
Concluding, we have found a branching rule for the hierarchical crew scheduling problem that preserves
69

the structure of the original problem. In this way, the efficient solvability of the pricing problem remains
intact throughout the nodes in the branching tree. In general, this is the challenge when devising
branching rules for integer programming problems whose LP-relaxation is solved by column generation.

Exercises
Exercise 1
Consider the following problem occurring in combinatorial auctions. Given are m items, and n bidders.
Each bidder j, 1 ≤ j ≤ n, specifies a (nonnegative) bid bj (S) for subsets of the items S ⊆ {1, 2, . . . , m}.
(Notice that the value of a bid that a bidder specifies for a pair of items need not be the same as the sum
of the valuations for the two individual items; this is, in fact, the defining property of a combinatorial
auction). Clearly, each item can be allocated to at most one bidder, and each bidder receives at most
one set of items.
(i) Give an integer programming formulation for this problem that maximizes the total auction revenue.
(ii) How many variables are there?
(iii) What is the dual of the linear programming relaxation of your formulation?
Exercise 2
Consider the following problem occurring in production management. Given are N jobs which have to be
processed by a single machine. Each job needs some (prespecified) tools to be processed; in total there
are M tools available. The machine can hold at most C tools simultaneously (and of course, each job
does not need more than C tools). We call a subset of the jobs a feasible group if these jobs together
require at most C tools. The job grouping problem consists in finding a minimum number of feasible
groups such that each job is contained in at least one group.
(i) Formulate this problem as an integer programming problem.
(ii) Describe a column generation approach for this problem.
Exercise 3
Consider the following problem occurring in routing applications. Given are n + 1 locations, among
70

which a depot, and distances d between each pair of locations. The depot harbors K vehicles that serve
the locations. No vehicle can serve more than C locations. The problem is to select K paths (one for
each vehicle), each starting at the depot, such that each location is in exactly one path. The goal is to
minimize the total length of the K paths.
(i) Formulate
 this problem as a ‘traditional' integer program using variables
 1 if vehicle k travels from location i to location j
xijk =
 0 otherwise

(ii) Formulate this problem as an integer programming problem using a formulation that involves
exponentially many variables.
(iii) Discuss how a column generation approach for this formulation would work.

\chapter{Approximation algorithms}
\chplab{approx}

\section{Introduction}
Instances of \concepts{combinatorial optimization problem} cannot always be solved to optimality within a reasonable amount of computing time. Indeed, some combinatorial problems are so-called \concept{NP-hard} (e.g. \concept[knapsack problem]{knapsack}, \concept[stable set problem]{stable set}, \concept[node cover problem]{node cover}) which implies that the best-known algorithms that guarantee an optimal solution have an \concept{enumerative character} (like the \concept{branch-and-price} approach).

\paragraph{}
This chapter deals with \concepts{approximation algorithm}. These are algorithms that return a \concept{feasible solution} fast (i.e. within \concept{polynomial time}), but sacrifice the \concept{guarantee of optimality}. Thus, whereas \concepts{branch-and-price algorithm} always output an \concept{optimum solution} at the expense of potentially enormous computing times, \concepts{approximation algorithm} form a \concept{dual approach} to solving \concepts{combinatorial optimization problem}. An \concept{approximation algorithm} guarantees a fast solution, but not necessarily an optimal one. Of course,
one still wants an \concept{approximation algorithm} to produce solutions that have a value that does not differ too much from the \concept{optimum value}. In order to be able to judge approximation algorithms, the concept of \concept{worst case analysis} is discussed in \secref{worstcase}. Further, we presents \concept{approximation algorithm} for two specific problems, namely \concept[node cover problem]{node cover} (\secref{nodecoverapprox}) and the \concept{traveling salesman problem} (\secref{tspapprox}).

\section{Worst case analysis}
\seclab{worstcase}

The \concept{worst case ratio} (WCR) of an algorithm $A$ for a \concept{minimization problem} $P$ is defined as follows:

\begin{definition}[Worst case ratio]
Given an \concept{instance} $I$ of \concept{problem} $P$, let the value of the \concept{solution} generated by \concept{algorithm} $A$ be $\fun{A}{I}$, and let the \concept{optimum value} be denoted by $\funm{opt}{I}$; we will sometimes simply write $\mbox{OPT}$, and ignore the ``$\fun{}{I}$''-part. The ratio

\begin{equation}
\fun{r}{I}\equiv\displaystyle\frac{\fun{A}{I}}{\funm{opt}{I}}
\end{equation}

is a \concept{measure} of the \concept{quality of the solution} found. (Notice that other \concept{measure} are certainly possible as well!). Now, the \concept{smallest upper bound} on $\fun{r}{I}$, measured over all \concepts{instance} $I$ of $P$ is called the \concept{worst case ratio} (WCR) of \concept{algorithm} $A$, i.e.,

\begin{equation}
\funm{wcr}{I}\equiv\fun{\sup_I}{\fun{r}{I}}=\fun{\sup_I}{\displaystyle\frac{\fun{A}{I}}{\funm{opt}{I}}}
\end{equation}

Notice that this \concept{ratio} is at least $1$. For a \concept{maximization problem} $P$ we define a similar \concept{quality measure} as
follows.

\begin{equation}
\funm{wcr}{I}\equiv\fun{\inf_I}{\fun{r}{I}}=\fun{\inf_I}{\displaystyle\frac{\fun{A}{I}}{\funm{opt}{I}}}
\end{equation}
This \concept{ratio} is at most $1$ (and not smaller than $0$).
\end{definition}

\paragraph{}
How can one find the \concept{worst case ratio} of a certain \concept{algorithm}? Almost always this amounts to applying a problem specific analysis. In such an analysis two things must
be argued:
\begin{enumerate}
 \item one has to argue that for all \concepts{instance} $I$ of $P$ it is true that $\dfrac{\fun{A}{I}}{\funm{opt}{I}}\leq R$, and that
 \item there exists an \concept{instance} $I$ of $P$ for which the ratio $\dfrac{\fun{A}{I}}{\funm{opt}{I}}$ is equal (or arbitrarily close) to $R$.
\end{enumerate}

Then one can conclude that $\funm{WCR}{A}=R$. Notice that the \concept{worst case ratio} depends on the \concept{algorithm}, thus different \concepts{algorithm} for the same problem can yield different \concepts{worst case ratio} (as we shall see in this chapter). Of course, one could ask the question: ``How well can we \concept{approximate} a certain \concept{problem} when using only \concept{polynomial time} algorithms?''. More formal, let $\funm{\textsc{PolyTime}}{P}=\condset{A}{\mbox{$A$ is a polynomial time algorithm for $P$}}$, then one is interested in something that can be formulated as:

\begin{equation}
\fun{\inf_{A\in\funm{\textsc{\small{PolyTime}}}{P}}}{\funm{wcr}{A}}
\end{equation}

This is certainly a valid question, and in recent years a number of results in this direction have been obtained; however, we will not go into this issue.

\section{Node Cover}
\seclab{nodecoverapprox}

Recall that given a \concept{graph} $G=\tupl{V,E}$, a node cover is a \concept{subset} of the \concept[vertex]{vertices} $W\subseteq V$ with the property that each \concept{edge} in $E$ is \concept{incident} to at least one \concept{vertex} in $W$. The \concept{objective} in this problem is to find a \concept{node cover} with a minimum number of \concepts{node}. This problem is \concept{NP-hard}. Let us now describe two \concepts{approximation algorithm} for node cover.

\importalgorithmicalgorithm{nodecover1}{First proposed approximating algorithm for the node cover problem.}
\importalgorithmicalgorithm{nodecover2}{Second proposed approximating algorithm for the node cover problem.}

\algoref{nodecover1} is a \concept[Greedy algorithm]{greedy type of algorithm}. At first glance it seems to be better than \algoref{nodecover2}, as it uses at least part of the structure of the \concept{graph}. And indeed, in instances arising from practical applications, it turns out that \algoref{nodecover1} outperforms \algoref{nodecover2} with respect to the number of \concepts{node} in the \concept[node cover]{cover} found. However, as we are about to discover, the WCR of \algoref{nodecover2} is much better than the WCR of \algoref{nodecover1}.

\paragraph{}
Consider the following instance depicted in \figref{nodecover-approx-example}.

\importtikzfigure{nodecover-approx-example}{An instance for node cover.}

What is the \concept{optimum}? It is not hard to figure out that one needs at least $5$ nodes for a \concept{node cover} in the instance above, and moreover, that taking all nodes from layer $B$ indeed constitutes a \concept{feasible node cover}. Thus, this is an \concept{optimum solution}. Let us now apply \algoref{nodecover1} to this instance and let us break ties by choosing nodes in the lowest possible layer first (that is nodes of layer $A$ enjoy priority over nodes in layer $B$ which enjoy priority over nodes in layer $C$). What happens? \algoref{nodecover1} selects first the nodes from layer $A$ and then the nodes from layer $B$ concluding with a node cover consisting of $8$ nodes. Even worse, we can generalize this instance as follows: let the number of nodes in layer $B$ be $n$, then there are also $n$ nodes in layer $C$ and there are at most $n-1$ nodes in the bottom layer. Again, each node in layer $A$ is connected to each node in layer $B$ and a node in layer $C$ is connected only to its ``\concept{companion node}'' directly beneath it. Thus the \concept{optimum node cover} consists of all nodes in layer $B$ with optimal value $n$, whereas \algoref{nodecover1} will find a solution consisting of all nodes of layer $A$ and $B$, with a value equal to almost twice the optimum value. What can we deduce from this example concerning the \concept{worst case ratio} of \algoref{nodecover1}? Well, we can only say that it is at least $2$. One cannot conclude that the \concept{worst case ratio} equals $2$ since there may exist instances on which \algoref{nodecover1} fares even worse.

\paragraph{}
Such examples exist. Consider \figref{nodecover-approx-exampld}. In this instance we see that the \concept{optimum node} cover still consists of all middle nodes (i.e. value $6$), whereas the \concept{node cover} constructed by \algoref{nodecover1} will consist of all nodes in the two bottom layers (i.e. a solution with value $13$). From this instance alone we can conclude that the worst case ratio of \algoref{nodecover1} must be worse than $2.1666$.

\importtikzfigure{nodecover-approx-exampld}{Another instance for node cover.}

\paragraph{}
By generalizing this instance an even more dramatic statement can be made:
\begin{theorem}
\thmlab{wcr-nodecover-approx1}
For each $r\geq 0$, there exist \concepts{instance} $I$ of \concept{node cover} such that $\fun{A}{I}\geq r\cdot\funm{opt}{I}$.
\end{theorem}
In other words, the \concept{worst case ratio} of \algoref{nodecover1} is \concept[unbounded worst case ratio]{unbounded}; it is impossible to find a constant $r$ such that
\begin{equation}
\displaystyle\frac{\fun{A}{I}}{\funm{opt}{I}}\leq r.
\end{equation}

\paragraph{}
Let us motivate this theorem by generalizing the instance in \figref{nodecover-approx-exampld}. The structure of this instance is as follows. In case of $6$ middle nodes we do the following: \concept{partition} the nodes from $B$ into $3$ pairs and join the nodes in each pair with a node from layer $A$. Then we partition the nodes in layer $B$ into two triples, and again join all nodes in a triple with a new node from layer $A$. Repeat this for quadruples and quintuples, and so on, possibly leaving out some nodes from layer $B$, and always adding a new node in layer $A$. Then, when applying \algoref{nodecover1}, there is always a node from the bottom layer with highest degree, consequently \algoref{nodecover1} will find a \concept{node cover} consisting of $\fun{L}{n}+n$ \concepts{node}, where $\fun{L}{n}$ is the number of \concepts{node} in layer $A$. How large is $\fun{L}{n}$? Observe that

\begin{equation}
\fun{L}{n}=\sumieqb[j]{2}{n-1}{\floor{\displaystyle\frac{n}{j}}}.
\end{equation}
We leave the exact proof of \thmref{wcr-nodecover-approx1} as an exercise.

What about \algoref{nodecover2}? It is easy to establish a \concept{lower bound} of $2$ on the \concept{worst case ratio}. Indeed, when simply taking a \concept{graph} consisting of $2\cdot n$ \concept[vertex]{vertices} and $n$ \concepts{edge} forming a \concept{perfect matching}, one observes that $n$ is the value of a \concept{minimum node cover}, whereas \algoref{nodecover2} selects all $2\cdot n$ nodes. However, it turns out that this is the worst that can happen for \algoref{nodecover2}:

\begin{theorem}
\thmlab{wcr-nodecover-approx2}
$\funm{wcr}{\algoref{nodecover1}}=2$.
\end{theorem}

The argument is as follows. Of course, any \concept{node cover} must cover all \concepts{edge} chosen by \algoref{nodecover2}. However, these \concepts{edge} do not have a \concept{node} in common, and therefore each \concept{edge} must be covered by a different node in any \concept{node cover}. Thus no \concept{node cover} can be smaller than half the size of the \concept{node cover} found by \algoref{nodecover2}. Together with the example sketched above \thmref{wcr-nodecover-approx2} now follows.

\section{The Traveling Salesman Problem (TSP)}
\seclab{tspapprox}
In a sense the \concept{TSP} is a harder problem to solve than \concept{node cover}. This follows from the well known fact that, unless $\cclass{P}=\cclass{NP}$, no polynomial time algorithm for the TSP exists that has a bounded \concept{WCR} (which contrasts with \algoref{nodecover2} in \secref{nodecoverapprox}). What we can do is to restrict our instances. Recall that the input to the \concept{TSP} is a \concept{distance matrix} $D$. In the sequel we restrict ourselves to instances for which the distances satisfy the \concept{triangle inequality}.

\begin{definition}[Triangle inequality]
The \concept{triangle inequality} is a restriction on a distance metric $\delta:X\times X\rightarrow\RRR$ that states that for every three items $x_1,x_2,x_3\in X$, the distance $\fun{\delta}{x_1,x_3}$ is less than or equal to the sum of the distance from $x_1$ to $x_2$ and the distance from $x_2$ to $x_3$. More formally:
\begin{equation}
\forall x_1,x_2,x_3\in X:\fun{\delta}{x_1,x_3}\leq\fun{\delta}{x_1,x_2}+\fun{\delta}{x_2,x_3}
\end{equation}
\end{definition}

Observe that many practical problems satisfy this restriction. Indeed, \concept{TSP} instances coming from actual ``\concept{travel settings}'' are quite likely to obey the \concept{triangle inequality}.

\subsection{The double tree algorithm (DT)}
\ssclab{doubletree}

The \concept{double tree algorithm} consists of four \concepts{phase}. In the first three \concepts{phase}, we construct an \concept{Eulerian cycle} that we convert into a \concept{Hamilton circuit} in \phsref{tsp-dt-d}:

\begin{phasenum}
 \item \phslab{tsp-dt-a} Construct a \concept{minimum spanning tree} with respect to the \concept{distance function} $d$.
 \item \phslab{tsp-dt-b} Double all \concepts{edge} in the \concept{tree}. Notice that the resulting graph is \concept[Eulerian graph]{Eulerian}.
 \item \phslab{tsp-dt-c} Determine an \concept{Eulerian cycle} in the \concept{Eulerian graph} determined in \phsrf{tsp-dt-b}. Since the \concept{Eulerian graph} is \concept[connected graph]{connected} (it contains the \concept{minimum spanning tree} as a subgraph), the \concept{cycle} contains each \concept{vertex} at least once.
 \item \phslab{tsp-dt-d} Convert the \concept{Eulerian cycle} into a \concept{Hamiltonian cycle} by applying \concepts{shortcut}, i.e., replace a pair of \concepts{consecutive edge} $\tupl{i,j}$ and $\tupl{j, k}$ in the \concept{Eulerian cycle} by $\tupl{i,k}$. We are only allowed to do this if $j$ appears somewhere else in the \concept{Eulerian cycle}.
\end{phasenum}

\begin{example}
Consider the $7$-\concept{city} \concept{TSP} instance depicted \tblref{tsp-approx-dst} and \figref{tsp-approx-ph1}.

\importtabulartable{tsp-approx-dst}{The distance matrix of the Traveling Salesman Problem example.}

\begin{figure}[hbt]
\centering
\importtikzsubfigure{tsp-approx-ph1}{Phase $1$.}
\importtikzsubfigure{tsp-approx-ph2}{Phase $2$.}
\caption{Phase $1$ and $2$ of the \concept{double tree algorithm}.}
\end{figure}
\end{example}

\paragraph{}
The \concept{Eulerian cycle} generated in \concept{phase} $3$ is $1-2-3-2-4-5-6-7-6-5-4-2-1$.

\paragraph{}
The operation described in \phsref{tsp-dt-d} can be performed on any \concept{cycle}. Its effect is that a new \concept{cycle} is constructed with one edge less; in this cycle, there is one \concept{vertex} that is visited one time less in comparison with the previous \concept{cycle}. Due to the assumption that the \concept{distance function} satisfies the \concept{triangle inequality}, it follows that applying a \concept{shortcut} does not increase the \concept[length of a cycle]{length of the cycle}. Let us formally record this observation in a lemma.

\begin{lemma}
\lemlab{minimalitycycle}
Let $G=\tupl{V,E}$ be a \concept{complete graph}, and let $\funsig{d}{E}{\RRR^+}$ be a \concept{distance function} on the \concepts{edge}, which satisfies the \concept{triangle inequality}. Let $C$ be a \concept{cycle} in this \concept{graph} with total \concept[length of a cycle]{length} $\fun{d}{C}$. If $C'$ is a \concept{cycle} constructed from $C$ by applying \concepts{shortcut}, then we have that the total length of $C'$ is no more than $\fun{d}{C}$.
\end{lemma}

\paragraph{}
Let us now be more specific concerning \phsref{tsp-dt-d}. We apply a \concept{shortcut} on each second appearance of a \concept{vertex} $j$. Therefore, each \concept{vertex} remains in the \concept{cycle} at least once. After the \concept{shortcut} has been applied to all second appearances of the \concept[vertex]{vertices}, the \concept{cycle} contains each \concept{vertex} exactly once, that is, we have constructed a \concept{Hamiltonian cycle}.

\importtikzfigure{tsp-approx-ph4}{Phase $4$: replacing $3-2-4$ by $3-4$}

\importtikzfigure{tsp-approx-ph5}{Phase $4$: further replacements}

\paragraph{}
We now show that the \concept{double tree algorithm} will never produce a solution with length more than twice the length of an optimal solution. In other words:
\begin{theorem}
\thmlab{wcrdtleq2}
$\funm{wcr}{\mbox{double tree}}\leq2$.
\end{theorem}

Given an instance $I$, let $\funm{opt}{I}$ denote the length of an \concept{optimal tour}, and let $\fun{z_{dt}}{I}$ denote the length of the \concept{tour} constructed by \concept{double tree algorithm}. Finally, let $\fun{z_T}{I}$ denote the length of a \concept{minimum spanning tree}. We first prove that $\fun{z_T}{I}\leq\funm{opt}{I}$ for all instances $I$;
\begin{subproof}
Consider any \concept{optimal tour}. If we delete an arbitrary \concept{edge} from it, then we obtain a... \concept{spanning tree}. By definition, the \concept[length of a spanning tree]{length} of a \concept{spanning tree} does not exceed the \concept{length of a minimal spanning tree}, and hence, we know that its length amounts to at least $\fun{z_T}{I}$. Concluding, we have that $\fun{z_T}{I}\leq\funm{opt}{I}$.
\end{subproof}
We then complete the proof by showing that $\fun{z_{dt}}{I}\leq2\cdot\fun{z_T}{I}$:
\begin{subproof}
The total length of the \concepts{edge} in the \concept{Eulerian graph} that is constructed by doubling the \concept{minimum spanning tree} is equal to $2\cdot\fun{z_T}{I}$. From \lemref{minimalitycycle}, it follows that the length $\fun{z_{dt}}{I}$ of the \concept{Hamiltonian circuit} that is obtained by applying \concepts{shortcut}, amounts to no more than the length of the \concept{Eulerian cycle}, which is equal to $2\cdot\fun{z_T}{I}$.
\end{subproof}
Combining both results, we get $\fun{z_{dt}}{I}\leq2\cdot\fun{z_T}{I}\leq2\cdot\funm{opt}{I}$.

\subsection{The tree-matching algorithm (TM)}
\ssclab{treematching}

From the analysis above, it follows that, if we want to improve our \concept{worst case ratio}, then we can try to decrease the \concept[length of a Eulerian cycle]{length} of the \concept{Eulerian cycle}. In the sequel we will do so. This means that we use the same structure of the \concept{double tree algorithm}. In fact, \phsrefe{tsp-dt-a,tsp-dt-b,tsp-dt-c,tsp-dt-d} in the \concept{tree-matching algorithm} are identical to the corresponding phases in the \concept{double tree algorithm}. Thus, we only change the phase in which the \concept{Eulerian cycle} is constructed.

\paragraph{}
Recall that a graph is \concept[Eulerian graph]{Eulerian} if and only if it is \concept[connected graph]{connected}, and each \concept{vertex} has even \concept[vertex degree]{degree}. It seems obvious to start with a \concept{minimum spanning tree} to make sure that the \concept{graph} is \concept[connected graph]{connected}. The only problem left is to take care of the \concept[vertex]{vertices} with odd \concept[vertex degree]{degree}, which we denote by $V_0$; notice that the number of \concept[vertex]{vertices} with odd \concept[vertex degree]{degree} is even. We see that we can get even \concept[vertex degree]{degree} in each of these \concept[vertex]{vertices} by adding a \concept{perfect matching} on these \concept[vertex]{vertices} $V_0$ (see \chpref{formulation} for the definition of a \concept{perfect matching}). This is exactly what happens in Phase $2$ of the \concept{tree-matching algorithm}, where we will compute a \concept{minimum weight perfect matching} $M$ on the \concept[vertex]{vertices} in $V_0$.

\paragraph{}
\begin{example}
Consider the example again. The initial \concept{minimum spanning tree} contains $4$ \concept[vertex]{vertices} of odd \concept[vertex degree]{degree}, namely $1$, $2$, $3$, and $7$. The \concept{minimum weight perfect matching} of these \concept[vertex]{vertices} consists of the \concepts{edge} $\tupl{1,2}$ and $\tupl{3,7}$. Thus, we get the \concept[Minimum spanning tree extension]{extension} of the \concept{minimum spanning tree} as depicted in \figref{tsp-approx-pi2}.

\importtikzfigure{tsp-approx-pi2}{Phases $1$ and $2$ of the \concept{tree-matching algorithm}}.
\end{example}

\paragraph{}
An \concept{Eulerian cycle} in this \concept{graph} is $1-2-3-7-6-5-4-2-1$, where $2$ is the only \concept{vertex} that appears more than once, and therefore we remove one of its occurrences.

\importtikzfigure{tsp-approx-pi4}{Replacing $4-2-1$ by $4-1$.}

\paragraph{}
This small change with respect to the \concept{double tree algorithm} results in a better \concept{worst case behaviour}. For any instance of the \concept{TSP} (satisfying the \concept{triangle inequality}), the \concept{tree-matching algorithm} constructs a \concept{tour} with \concept[tour length]{length} no more than $\dfrac{3}{2}$ times the \concept[tour length]{length} of an optimal \concept{tour}. To prove this, it suffices to show that the length $\fun{z_M}{I}$ of the \concept{matching} $M$ is no more than $\dfrac{1}{2}$ times the optimum. Namely, then $\fun{z_{tm}}{I}\leq\fun{z_T}{I}+\fun{z_M}{I}\leq\funm{opt}{I}+\dfrac{1}{2}\cdot\funm{opt}{I}=\dfrac{3}{2}\cdot\funm{opt}{I}$.

\paragraph{}
The proof makes use of a \concept{shrinking argument}. Consider any \concept{optimal tour} with value $\funm{opt}{I}$. Apply \concepts{shortcut} such that only the \concept[vertex]{vertices} in $V_0$ remain in the \concept{cycle}; this yields a \concept{cycle} $C$ on the \concept[vertex]{vertices} in $V_0$ with length no more than $\funm{opt}{I}$. $C$ can be \concept[partitioning]{partitioned} into two \concept{perfect matchings} $M_1$ and $M_2$ on $V_0$ by ``walking'' along the \concept{circuit} and putting the first \concept{edge} in $M_1$, the second \concept{edge} in $M_2$, the third edge in
$M_1$, etc. Notice that the \concept{cycle} contains an even number of \concepts{edge}, because $\abs{v_0}$ is even.

\paragraph{}
Since $M_1$ and $M_2$ are \concepts{perfect matching} on $V_0$, we have that $\fun{d}{M}\leq\fun{d}{M_1}$ and $\fun{d}{M}\leq\fun{d}{M_2}$. Hence, we have that $2\cdot\fun{d}{M}\leq\fun{d}{M_1}+\fun{d}{M_2}=\fun{d}{C}\leq\funm{opt}{I}$, which was to be proved.

\paragraph{}
Notice that for both algorithms, our analysis of the \concept{worst case ratio} depends heavily on the assumption that the \concept{triangle inequality} holds. It can be shown that, in case the \concept{triangle inequality} fails to hold, the \concept{worst case ratio} is \concept[unbounded worst case ratio]{unbounded} (as could be inferred from the beginning of this section).

\section*{Exercises}
\begin{exercise}
Prove \thmref{wcr-nodecover-approx1}.
\end{exercise}

\begin{exercise}
Consider the following \concept{greedy algorithm} for the \concept{knapsack problem}. Sort the \concepts{object} by decreasing ratio of \concept{profit} and \concept{size}, and \concept{reindex} the \concepts{item} such that the \concept{order} of \concepts{object} is $1,2,\ldots,n$. Next, greedily pick \concepts{object} in this order while ensuring that the \concept{capacity of the knapsack} is not exceeded.
\begin{enumerate}
 \item Show that this method can behave arbitrarily bad,
 \item Modify this method by identifying the smallest $k$ such that the total size of the first $k$ \concepts{object} exceeds the capacity. Next, find the best of the following two solutions: $\accl{1,2,\ldots,k-1}$ and $\accl{k}$.
Show that this algorithm is a \concept[$r$-approximation]{2-approximation}.
\end{enumerate}
\end{exercise}

\begin{exercise}
Consider the \concept{node packing problem}. What can you tell about the \concept{worst case ratio} of \algoref{nodepacking1} and \algoref{nodepacking2}?
\importalgorithmicalgorithm{nodepacking1}{First proposed algorithm for the node packing problem.}
\importalgorithmicalgorithm{nodepacking2}{Second proposed algorithm for the node packing problem.}
\end{exercise}

\begin{exercise}
Consider the \concept{matching problem}. What can you tell about the \concept{worst case ratio} of \algoref{matching1}?
Algorithm (Input: a graph G = (V, E); output: a matching M )
M = ∅, G′ = (V ′ , E ′ ) := G
while E ′ = ∅
do
Choose an arbitrary edge {v, w} ∈ E ′ ;
M := M ∪ {v, w};
Update G′ , that is V ′ := V ′ \ {v, w} and remove from E ′ all edges incident to v or w;
od
\end{exercise}

\begin{exercise}
An \concept{edge coloring} of a \concept{graph} is a coloring of the \concepts{edge} such that no two \concepts{edge} connected to the same \concept{vertex} have the same \concept{color}. The \concept{edge coloring problem} is to minimize the number of \concepts{color} used to color a \concept{graph}. The \concept{greedy algorithm} for \concept{edge coloring} colors \concept{edge} after \concept{edge}. When coloring an edge it will first consider \concepts{color} that are already in use before assigning a new \concept{color}. What can you say concerning the \concept{worst case ratio} of this \concept{algorithm}?
\end{exercise}

\begin{exercise}
Can you find instances of the \concept{TSP} which imply that (together with \thmref{wcrdtleq2}) that $\funm{wcr}{\mbox{double tree}}=2$?
\end{exercise}

\begin{exercise}
A \concept{TSP} instance is called \concept{geometric} if each of the \concept[city]{cities} can be represented by a \concept{point} in the plane, i.e. each \concept{city} lies at \concepts{coordinate} $\tupl{x_i,y_i}$ for $i=1,2,\ldots,n$. What do the result in the previous exercise tell you about $\funm{wcr}{\mbox{dt}}$ for the \concept{geometric TSP}?
\end{exercise}

\chapter{Lagrangian Relaxation}
\chplab{lagrange}
In this chapter we study a concept called \concept{Lagrangian relaxation}. The formulation of many practical \concepts{combinatorial optimization problem} contains several sets of \concepts{constraint}. \concept{Lagrangian relaxation} exploits this property by disregarding one or more sets of \concepts{constraint}. It turns out that this \concept{relaxation} allows one to obtain \concepts{lower bound} (\concepts{upper bound}) for difficult minimization (maximization) problems. In \secref{lagrangeterminology} we introduce some terminology, \secref{lagrangeresult} presents some basic results, in \secref{lagrangeapplication} we describe an application, and \secref{lagrangedual} concludes this chapter by presenting two ways of strengthening the \concept{Lagrangian dual}.

\section{Terminology}
\seclab{lagrangeterminology}

Consider an \concept{integer program}:

\begin{equation}
z_{ip}=\funcs{\max}{\vec{c}\cdot\vec{x}}{x\in S},\mbox{ where }S=\condset{\vec{x}\in\ZZZ^n}{\vec{x}\geq\vec{0}\wedge A\cdot\vec{x}\leq\vec{b}},
\end{equation}

which can be rewritten as an \concept{integer problem} (IP):

\begin{eqnarray}
\mbox{maximize}&z_{ip}=\vec{c}\cdot\vec{x}\eqnlab{lagragea-m}\\
\mbox{subject to}&A_1\cdot\vec{x}\leq\vec{b}_1\eqnlab{lagragea-c1}\\
&A_2\cdot\vec{x}\leq\vec{b}_2\eqnlab{lagragea-c2}\\
&\vec{x}\geq\vec{0}\eqnlab{lagragea-c3}\\
&\vec{x}\in\ZZZ^n\eqnlab{lagragea-c4}
\end{eqnarray}

We are going to assume that $A_2\cdot\vec{x}\leq\vec{b}_2$ are $m-m_1$ ``\concepts{nice constraint}'', say those of an \concept[assignment problem]{assignment} or a \concept{network problem}. By simply dropping the $m_1$ ``\concepts{complicating constraint}'' $A_1\cdot\vec{x}\leq\vec{b}_1$, we obtain a \concept{relaxation} of the \concept{integer problem} (\concept{IP}) that is obviously easier to solve than the problem itself. There are many problems for which the constraints can be \concept[constraint partitioning]{partitioned} in this way. An example will be given in \secref{lagrangedual}.

\paragraph{}
The idea of dropping constraints can be embedded in a more general framework called \concept{Lagrangian relaxation}. It is convenient to consider a generalization of problem (IP) called \concept{IP(Q)}, which we formulate as follows:

\begin{eqnarray}
\mbox{maximize}&z_{ip}=\vec{c}\cdot\vec{x}\eqnlab{lagrageb-m}\\
\mbox{subject to}&A_1\cdot\vec{x}\leq\vec{b}_1\eqnlab{lagrageb-c1}\\
&\vec{x}\in Q\eqnlab{lagrageb-c2}
\end{eqnarray}

However, when we are discussing results that are specific to \concept{IP}, it is assumed that $Q=\condset{x\inZZZ^n}{\vec{x}\geq\vec{0}\wedge A_2\cdot\vec{x}=\vec{b}_2}\neq\emptyset$. Of course, the problem obtained from \concept{IP(Q)} by dropping the \concepts{complicating constraint}, $A_1\cdot\vec{x}\leq\vec{b}_1$ is much easier to solve than \concept{IP(Q)}. Now, for any $\lambda\in\brak{\RRR^+}^{m_1}$, consider the problem \concept{LR($\vec{\lambda}$)}:

\begin{equation}
\fun{z_{lr}}{\vec{\lambda}}=\funcs{\max}{\fun{z}{\vec{\lambda},\vec{x}}}{x\in Q},\mbox{ where }\fun{z}{\vec{\lambda},\vec{x}}=\vec{c}\cdot\vec{x}+\vec{\lambda}\cdot\brak{\vec{b}_1-A_1\cdot\vec{x}}.
\end{equation}

The problem \concept{LR($\vec{\lambda}$)} is called the \concept{Lagrangian relaxation} of \concept{IP(Q)} with respect to $A_1\cdot\vec{x}\leq\vec{b}_1$. This terminology is used because the vector $\lambda$ plays a role in \concept{LR($\vec{\lambda}$)} similar to the role of \concepts{Lagrange multiplier} in \concepts{constrained optimization problem}. By our choice, \concept{LR($\vec{\lambda}$)} does not contain the \concepts{complicating constraint}. Instead we have included these constraints in the \concept{objective function} with the ``\concept{penalty term}'' $\vec{\lambda}\cdot\brak{\vec{b}_1-A_1\cdot\vec{x}}$. Since $\vec{\lambda}\geq\vec{0}$, violations of $A_1\cdot\vec{x}\leq\vec{b_1}$ make the \concept{penalty term} negative, and thus, intuitively speaking, for suitably large values of $\vec{\lambda}$, one would expect that $A_1\cdot\vec{x}\leq\vec{b}_1$ will be satisfied.

\paragraph{}
Let us formally state the relation between $z_{ip}$ and $\fun{z_{lr}}{\vec{\lambda}}$:
\begin{theorem}
$\fun{z_{lr}}{\vec{\lambda}}\geq z_{ip}$ for all $\vec{\lambda}\geq\vec{0}$.
\begin{proof}
If $\vec{x}$ is feasible in \concept{IP(Q)}, then $\vec{x}\in Q$ and hence $\vec{x}$ is feasible for \concept{LR($\vec{\lambda}$)}. Also, $\fun{z}{\vec{\lambda},\vec{x}}=\vec{c}\cdot\vec{x}+\vec{\lambda}\cdot\brak{\vec{b}_1-A_1\cdot\vec{x}}\geq\vec{c}\cdot\vec{x}$ for all $\vec{x}$ feasible in \concept{IP(Q)} since $A_1\cdot\vec{x}\leq\vec{b}_1$ and $\vec{\lambda}\geq\vec{0}$.
\end{proof}
\end{theorem}

\paragraph{}
Obviously, one is interested in the \concept{least upper bound} from the infinite family of \concept[lagrange relaxation]{relaxations} $\accl{\funm{LR}{\vec{\lambda}}}_{\vec{\lambda}\geq\vec{0}}$,
denoted here by $\fun{z_{lr}}{\vec{\lambda}^{\star}}$, where $\vec{\lambda}^{\star}$ is an \concept{optimal solution} to the problem called \concept{LD}:

\begin{equation}
z_{ld}=\fun{\min_{\vec{\lambda}\geq\vec{0}}}{\fun{z_{lr}}{\vec{\lambda}}}.
\end{equation}

Problem \concept{LD} is called the \concept{Lagrangian dual} of \concept{IP(Q)} with respect to the \concepts{constraint} $A_1\cdot\vec{x}\leq\vec{b_1}$.

\section{Some results}
\seclab{lagrangeresult}
In this section we illustrate the terminology introduced in the previous section with the following example and use this example to derive some results.

\begin{example}
Consider the following problem.

\begin{eqnarray}
\mbox{maximize}&7\cdot x_1+2\cdot x_2		\eqnlab{lagragec-m}\\
\mbox{subject to}&-x_1+2\cdot x_2\leq4		\eqnlab{lagragec-c1}\\
&6\cdot x_1+x_2\leq24				\eqnlab{lagragec-c2}\\
&-2\cdot x_1-2\cdot x_2\leq-7			\eqnlab{lagragec-c3}\\
&-x_1\leq-2					\eqnlab{lagragec-c4}\\
&x_2\leq4					\eqnlab{lagragec-c5}\\
&x_1,x_2\in\ZZZ^+				\eqnlab{lagragec-c6}
\end{eqnarray}

Let $Q=\condset{\vec{x}\in\brak{\ZZZ^+}^2}{\vec{x}\mbox{ satisfies \eqnnrefe{lagragec-c1,lagragec-c2,lagragec-c3,lagragec-c4,lagragec-c5,lagragec-c6}}}$. The \concept{Lagrangian relaxation} (see \secref{lagrangeterminology}) with respect to $-x_1+2\cdot x_2\leq 4$ is: $\fun{z_{lt}}{\lambda}=\funf{\max_{x\in Q}}{7\cdot x_1+2\cdot x_2+\lambda\cdot\brak{4+x_1-2\cdot x_2}}$ or equivalently:

\begin{eqnarray}
\mbox{maximize}&\brak{7+\lambda}\cdot x_1+\brak{2-2\cdot\lambda}\cdot x_2+4\cdot\lambda\eqnlab{lagraged-m}\\
\mbox{subject to}&6\cdot x_1+x_2\leq24		\eqnlab{lagraged-c1}\\
&-2\cdot x_1-2\cdot x_2\leq-7			\eqnlab{lagraged-c2}\\
&-x_1\leq-2					\eqnlab{lagraged-c3}\\
&x_2\leq4					\eqnlab{lagraged-c4}\\
&x_1,x_2\in\ZZZ^+				\eqnlab{lagraged-c5}
\end{eqnarray}

Notice that $Q$ is a finite set of points, which can be written as follows (see \figref{lagrange-region-ex}):
\begin{equation}
\tupl{q_1,q_2,q_3,q_4,q_5,q_6,q_7,q_8}=\tupl{\tupl{2,2},\tupl{2,3},\tupl{2,4},\tupl{3,1},\tupl{3,2},\tupl{3,3},\tupl{3,4},\tupl{4,0}}.
\end{equation}
\end{example}

\importtikzfigure{lagrange-region-ex}{The $Q$ region of the Lagrange relaxation example.}

The example suggests at least two different \concepts{viewpoint}. The first one is to see $\fun{z}{\vec{\lambda},\vec{x}}=\brak{\vec{c}-\lambda\cdot A_1}\cdot\vec{x}+\vec{\lambda}\cdot\vec{b}_1$ as a \concept{linear function} of $\vec{x}$ for fixed $\vec{\lambda}$. It then follows that $\fun{z_{lr}}{\vec{\lambda}}$ can be determined by solving the \concept{linear program}.

\begin{equation}
\fun{z_{lr}}{\vec{\lambda}}=\funcs{\max}{\fun{z}{\vec{\lambda},\vec{x}}}{\vec{x}\in\funm{conv}{Q}}.
\end{equation}

In this example,

\begin{equation}
\funm{conv}{Q}=\condset{\vec{x}\in\brak{\RRR^+}^2}{-x_1\leq-2\wedge x_2\leq4\wedge-x_1-x_2\leq-4\wedge4\cdot x_1+x_2\leq16}.
\end{equation}

In \figref{lagrange-region-ex}, solid lines indicate the original constraints, the dots correspond to the \concept[feasible integral vertex]{feasible integral vertices}, and the dashed lines correspond to constraints describing $\funm{conv}{Q}$.

\paragraph{}
Thus, computing $\fun{z_{lr}}{\lambda}$ for $\lambda=0$ and $\lambda=1$ gives:

\begin{eqnarray}
\fun{z_{lr}}{0}=\funcs{\max}{7\cdot x_1+2\cdot x_2}{\vec{x}\in\funm{conv}{Q}}=\fun{z}{0,q_7}=\fun{z}{0,\tupl{3,4}}=29\\
\fun{z_{lr}}{0}=\funcs{\max}{8\cdot x_1+4}{\vec{x}\in\funm{conv}{Q}}=\fun{z}{1,q_8}=\fun{z}{1,\tupl{4,0}}=36
\end{eqnarray}

As one increases $\lambda$ from $0$, $\fun{z_{lr}}{\lambda}$ first decreases until $\lambda=\dfrac{1}{9}$ and then it increases. In general we obtain


\begin{equation}
\fun{z_{lr}}{\lambda}=\acclguard{\fun{z}{\lambda,q_7}=29-\lambda&\xif0\leq\lambda\leq\dfrac{1}{9}\\\fun{z}{\lambda,q_8}=28+8\cdot\lambda&\xif\dfrac{1}{9}<\lambda\leq1}.
\end{equation}

Hence, we can deduce that $z_{ld}=\fun{z_{lr}}{\dfrac{1}{9}}=\fun{z}}{\dfrac{1}{9},q_7}=\fun{z}{\dfrac{1}{9},q_8}=\dfrac{260}{9}$ and $\lambda^{\star}=\dfrac{1}{9}$. Notice that, for $\lambda=\dfrac{1}{9}$, $q_7$ as well as $q_8$ are optimal with respect to the constraints determining $\funm{conv}{Q}$, and hence the \concept{objective function} - which equals $\dfrac{64}{9}\cdot x_1+\dfrac{16}{9}\cdot x_2$ - must be parallel to $4\cdot x_1+x_2\leq16$. All these calculations can be seen in \figref{lagrange-region-ex}.

\paragraph{}
The second \concept{viewpoint} is to consider $\fun{z_{lr}}{\lambda}$ to be determined by \concept{maximization} over a set of \concepts{discrete point}, that is, zLR (λ) = maxxi ∈Q z(λ, xi ).
Observe here that for a fixed xi , z(λ, xi ) = cxi + λ(b1 − A1 xi ) is a linear function of λ. See Figure 7.2,
where we have drawn the linear functions z(λ, xi ) for xi ∈ Q.
z(λ, xi )

(4)
✱
✱
(8)
36
✱
✱
 
 
✱
 
(5)
 
✱
✟✟
 
✱
✟
 
✟
 
✱
 
✟✟
 
✱
 
✱ ✟✟
 
 
30
✱
✭✭✭(6)
✟✟
 
 
✟ ✭✭✭✭✭✭
 
❤
✱
❤❤
❤❤❤
✟
✭✭
❤✭
✟
✭
 
❤✱
❤✭
✭
❤❤❤
✟
✭✭✟
✱
✭
❤❤❤❤
✭
✭
✟
❤❤❤❤
✟✱✱
❤(7)
✟✟✱
✟
✱
24 ✱
✏✏(1)
✱
✏✏
✏
✏✏
€€
✏✏
€€
✏
€€
✏✏
(2)
€✏
✏
€€
✏
✏
€€
✏
€€
✏✏
€€
18 ✏
λ
1
2 €€
€€ (3)
€
Figure 7.2: The lines: (1) 18 + 2λ; (2) 20; (3) 22 - 2λ; (4) 23 + 5λ; (5) 25 + 3λ; (6) 27 + λ; (7) 29 - λ;
(8) 28 + 8λ;
90

In Figure 7.2 one can read the values of zLR (λ) for any value of λ. We see that zLR (λ) is piecewise linear
and convex (the heavy lines in Figure 7.2) and that zLD = 28 98 . Formally, one solves the linear program

zLR (λ) = min {w| w ≥ z(λ, xi ) for i = 1, . . . , 8},

which shows that zLR (λ) is the maximum of a finite number of linear functions and is therefore piecewise
linear and convex.
We now study how the solution of the Lagrangian dual relates to the solution of the original problem
IP(Q). Returning to Figure 7.1, notice that when λ = 1/9 we obtain

28

8
9

1
1
= z( , x7 ) = z( , x8 )
9
9
1 8 7 1 8
= z( , x + x )
9 9
9
1
1 8
= z( , (3, 4) + (4, 0))
9 9
9
1 28 32
1
28 32
= z( , ( , ) = z( , x∗ ) with x∗ = ( , )
9 9 9
9
9 9
1
∗
∗
∗
= cx + (4 + x1 − 2x2 )
9
∗
= cx .

In other words, by taking a convex combination of points in Q (in this example x7 and x8 ), we obtain a
point x∗ in conv(Q) satisfying the complicating constraint, for which cx∗ = zLD . This shows that for the
example we get zLD = max {cx| A1 x ≤ b1 , x ∈ conv(Q)}. And in fact this holds in general as witnessed
by the following theorem which we state without proof.

Theorem 7.2
zLD = max {cx| A1 x ≤ b1 , x ∈ conv(Q)}.

An interesting question is of course: how good is the bound zLD ? In general, the difference between
zLD and zIP (called the duality gap) depends on the sizes of conv(S) (which determines zIP ), conv(Q) ∩
91

{x| A1 x ≤ b1 } (which determines zLD ) and the objective coefficients c. A duality gap of 0 can be
characterized as follows.

Theorem 7.3 zLD = zIP for all c if and only if
n
n
conv{Q ∩ {x ∈ R+
| A1 x ≤ b1 }} = conv(Q) ∩ {x ∈ R+
| A1 x ≤ b1 }.

Another interesting difference is the difference between zLD and the value of the LP-relaxation, denoted
n
by zLP . Notice that this only makes sense when Q = {x ∈ Z+
| A2 x ≤ b2 }. We can characterize the case

where zLP = zLD .

n
Theorem 7.4 zLD = zLP for all c if all the extreme points of {x ∈ R+
| A2 x ≤ b2 } are integral.

It is easily verified that the conditions mentioned in the two previous theorems are not fulfilled by our
2
example. Indeed, we have 28 = zIP < zLD = 28 98 < zLP = 30 11
. (Check this !).

In fact, a more natural choice of complicating constraints in our example would lead to different results
2
2
for zLD . If we set Q = {x ∈ Z+
| − x1 ≤ −2, x2 ≤ 4}, we find that {x ∈ R+
| − x1 ≤ −2, x2 ≤ 4} only has

integral extreme points so that by our latest theorem, this Lagrangian relaxation would terminate with
2
.
zLD = zLP = 30 11

Summarizing, we have

n
n
conv(S) ⊆ conv(Q) ∩ {x ∈ R+
| A1 x ≤ b1 } ⊆ {x ∈ R+
| Ax ≤ b}.

This implies that zIP ≤ zLD ≤ zLP . But because some faces of the respective polyhedra can coincide,
we may obtain zIP = zLD or zLD = zLP for a particular c even if the conditions of the two previous
theorems do not hold. Below, we give at table indicating the possibilities using four different objective
functions c1 , c2 , c3 and c4 .
92

Objective functions

objective values

c1

zIP = zLD = zLP

2

zIP < zLD = zLP

c3

zIP < zLD < zLP

4

zIP = zLD < zLP

c

c

\section{An application}
\seclab{lagrangeapplication}

Suppose there is a set of n jobs to be assigned to a set of n workers, with N = {1, . . . , n}. Suppose
further that
• cij is the value of assigning worker i to job j,
• tij is the cost of training worker i to do job j, and
• there is a training budget of b units.
We wish to maximize the total value of the assignment subject to the budget constraint, that is

max

cij

xij

(7.8)

xij

=

1 for i ∈ N

(7.9)

xij

=

1 for j ∈ N

(7.10)

tij xij

≤

b

(7.11)

x

∈

{0, 1}.

(7.12)

i∈N j∈N

j∈N

i∈N

j∈N i∈N

If we wish to use Lagrangian relaxation there are different options to consider. Notice that in each of
the following four options the relaxed problem LR(λ) is considerably easier to solve than the original
problem.
1
1. Lagrangian relaxation with respect to (7.11). Then LR1 (λ), λ ∈ R+
is an assignment problem with

objective function
93

(cij − λtij )xij .

λb +
i∈N j∈N

2. Lagrangian relaxation with respect to (7.9) and (7.10). Then LR2 (u, v), u ∈ Rn , v ∈ Rn is a
knapsack problem with objective function

ui +
i∈N

(cij − ui − vj )xij .

vj +
j∈N

i∈N j∈N

3. Lagrangian relaxation with respect to (7.9) or (7.10), say (7.9). Then LR3 (u), u ∈ Rn is a knapsack
problem with so-called generalized upper bound constraints and with objective function

(cij − ui )xij .

ui +
i∈N

i∈N j∈N

4. Lagrangian relaxation with respect to (7.9) or (7.10) and (7.11), say (7.9) and (7.11). Only gener1
with objective function
alized upper bound constraints remain. Thus, LR4 (u, λ), u ∈ Rn , λ ∈ R+

λb +

(cij − ui − λtij )xij ,

ui +
i∈N

i∈N j∈N

which is trivial to solve. (For each j, an i is chosen to maximize cij −ui −λtij , and the corresponding
xij is set to 1).

In choosing a relaxation there are two major questions to consider: how strong is the lower bound zLD
and how difficult to solve is the Lagrangian dual (LD)? Let us here only consider the bounds.
When Q is a set of assignment constraints or a set of generalized upper bound constraints, we have that
4
1
= zLP . Since
= zLD
zLD

94

2

Q3 = {x ∈ {0, 1}n |

xij = 1 for j ∈ N,
i∈N

tij xij ≤ b}
i∈N j∈N

2

⊂ Q2 = {x ∈ {0, 1}n |

tij xij ≤ b}
i∈N j∈N

2

n
and conv(Q2 ) ⊂ {x ∈ R+
|

tij xij ≤ b, xij ≤ 1 for i, j ∈ N },
i∈N j∈N

we have
3
2
1
4
zIP ≤ zLD
≤ zLD
≤ zLD
= zLD
= zLP ,

and each of the inequalities is strict for some objective function.

\section{Strengthening the Lagrangian dual}
\seclab{lagrangedual}

We now consider two ways of strengthening the Lagrangian dual of problem IP. The first approach yields
a dual whose optimal value equals
n
n
max{cx| x ∈ conv(x ∈ Z+
| A1 x ≤ b1 ) ∩ conv(x ∈ Z+
| A2 x ≤ b2 )}.

This dual is obtained by applying Lagrangian duality to a reformulation of IP, which is called RIP:
zIP = max cx1
A1 x1

≤ b1

A2 x2

≤ b2

x1 − x2

= 0

x1

∈

n
Z+

x2

∈

n
Z+
.

95

Taking now x1 − x2 = 0 as complicating constraints, we obtain the Lagrangian dual of RIP:
zCSD = minu {max{(c

−

u)x1 + ux2 }}

A1 x1

≤

b1

A2 x2

≤

b2

x1

∈

n
Z+

x2

∈

n
Z+

= minc1 +c2 =c {

max

c1 x1 + max c2 x2 }

A1 x1

≤

b1

A2 x2

≤

b2

x1

∈

n
Z+

x2

∈

n
Z+
,

where u = c2 .
A polyhedral interpretation of the dual is stated in the next theorem.

n
n
| A2 x ≤ b2 
| A1 x ≤ b1  ∩ convx ∈ Z+
Theorem 7.5 zCSD = maxcx| x ∈ convx ∈ Z+

and zCSD ≤ zLD .

The technique described is referred to here with CS since this technique has been called cost splitting.
The technique is useful when

n
n
| A1 x ≤ b1 , so for some objective functions c we obtain
| A1 x ≤ b1  ⊂ x ∈ R+
• convx ∈ Z+

zCSD < zLD .
• The sets of constraints Ai x ≤ bi are simple to deal with separately; that is, the difficulty is caused
by their interaction.

In our example, we could take A1 x ≤ b1 to be constraint set (7.9) and (7.11) and take A2 x ≤ b2 to be
3
with the inequality strict for some objective
constraint sets (7.10) and (7.11). This yields zCSD ≤ zLD

functions c.
96

Another approach that domiantes the Lagrangian dual is the “surrogate” dual. Starting from IP(Q),
m1
with weights λ ∈ R+
for the complicating constraints, consider the following problem called SD(λ).

zSD (λ) = max{cx| λA1 x ≤ λb1 , x ∈ Q}.

The problem SD(λ) is called the surrogate relaxation of IP(Q) with respect to A1 x ≤ b1 . SD(λ) contains
n
a single complicating constraint. For instance when Q = Z+
the surrogate relaxation is a knapsack

problem. The surrogate dual of IP(Q) is the problem denoted by SD.

zSD = min

λ≥0 zSD (λ).

Although the surrogate dual can be used computationally, it does not have such nice theoretical properties
as the Lagrangian dual.

\section*{Exercises}
Exercise 1
Consider the following problem.
max 2x1 + 5x2
4x1 + x2

≤

28

x1 + 4x2

≤

27

x1 − 5x2

≤

1

x

∈

2
Z+
.

(i) Show that if any two constraints are dualized, the value of the Lagrangian dual equals the value of
the LP-relaxation.
(ii) Find an objective function for which (i) is false.
97

(iii) Show that if any single constraint is dualized, the value of the Lagrangian dual is an improvement
compared to the value of the LP-relaxation.
(iv) Apply cost-splitting to get a better Lagrangian dual.

Exercise 2
Construct two Lagrangian duals for the generalized assignment problem and discuss their merits.

max
s.t.

j cij xij

i
j

xij ≤ 1 for i ∈ M

i li xij

≤ bj for j ∈ N
2

x ∈ {0, 1}n .

\appendix
\chapter{Mathematics and Notation}
\applab{math}

\bibliographystyle{alpha}
\bibliography{biblio}
\printindex
\glsaddall
\glossarystyle{listgroup}
\printglossaries

\end{document}
