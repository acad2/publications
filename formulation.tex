\chapter{Formulations in combinatorial optimization}
\chplab{formulations}
In this first chapter, we will give an informal notion of \concepts{combinatorial optimization problem}. The defining characteristic of a \concept{combinatorial optimization problem} is that it has a finite number of \concepts{feasible solution}. We now give \concepts{formulation} of a number of basic \concepts{combinatorial optimization problem}.

\section{Formulations of Basic Combinatorial Optimization Problems}
\seclab{formulations}
Below, we present mathematical formulations of some well-known \concepts{combinatorial optimization problem}. Some of these problems can be formulated in a straightforward way. Others, however, have more complicated formulations. Among the latter ones are problems that can be solved very easily by a \concept{greedy algorithm}, like the \concept{spanning tree} problem. Other problems are notoriously hard to solve. This already shows that a \concept{formulation} need not give insight into the problem's difficulty. Moreover, a problem may have several correct \concepts{formulation}. This leads us to a very important question: which \concepts{formulation} are good and which are bad? Clearly, an answer to this question depends on one's goal; we will adopt here a \concept{solver}'s point of view. Thus, a \concept{formulation} is better than another \concept{formulation} when it leads to better \concepts{solution}, or when it produces \concepts{solution} faster.

\paragraph{}
Almost all \concepts{formulation} will use \concept[binary variable]{binary} or \concepts{integral variable}. This shows that \concept{integer programming} is closely related to (or, as a matter of fact, is itself a problem from the domain of) \concept{combinatorial optimization}.

\subsection{The Matching Problem}
\ssclab{matching}
The \concept{Matching} problem (also known as the \conceptsee{Edge packing}{Matching} problem) is one of the fundamental problems in Combinatorial Optimization. It is described as follows.

\begin{definition}[Matching problem]
We are given an arbitrary \concept{graph} $G=\tupl{V,E}$. A subset of the \concepts{edge} $E'\subseteq E$ is called a \concept{matching} (or an \concept{edge packing}) if each \concept{vertex} of $V$ is \concept{incident} to at most one \concept{edge} of $E'$. In other words, $E'$ is a \concept{matching} if no two edges of $E'$ have a \concept{vertex} in common. The problem is to find a \concept{matching} in $G$ consisting of as many \concepts{edge} as possible (a \concept{maximum cardinality matching}).
\end{definition}

\importtikzfigure{matching-red}{The red edges form a matching.}

\paragraph{}
A matching is called \concept[maximum solution]{maximum} when no matching of larger cardinality exists. A matching is called \concept[maximal solution]{maximal} when it cannot be enlarged.

\paragraph{}
The \concept{matching problem} can be formulated as an \concept{integer program} as follows, where we use the symbol $\fun{\delta}{v}$ to denote the set of edges that are incident to vertex $v\in V$. For instance, in \figref{matching-red}, $\fun{\delta}{v}=\accl{\tupl{3,7},\tupl{4,7},\tupl{7,8}}$. We define a \concept{0-1 variable} for each edge $e\in E$ as follows:
\begin{equation}
\semboolvar{x_e}{if edge $e$ is selected in the matching;}{otherwise.}
\end{equation}
And here is the integer program:
\begin{eqnarray}
\mbox{maximize}&\sumdomain[e]{E}{x_e}\eqnlab{matching-m}\\
\mbox{subject to}&\forall v\in V:\sumdomain[e]{\fun{\delta}{v}}{x_e}\leq 1\eqnlab{matching-c1}\\
&\forall e\in E:x_e\in\accl{0,1}\eqnlab{matching-c2}
\end{eqnarray}

\paragraph{}
The \concept{weighted matching} problem is a generalization of the \concept{cardinality matching problem} above by assuming that there is a given \concept{weight function} $\funsig{w}{E}{\RRR}$ defined on the \concepts{edge}. Then, the \concept{objective} is to find a \concept{maximum weight matching}, and the \concept{objective} is changed accordingly to $\max\isumdomain[e]{E}{w_e\cdot x_e}$ (Obviously, it is a generalization, since the \concept{cardinality matching problem} arises when the \concept{weight} $w_e$ for each $e\in E$ is the same positive number). In the \concept{perfect matching} problem the set of \concepts{feasible solution} is restricted to \concepts{perfect matching}. These are \concepts{matching} such that each \concept{vertex} is \concept{incident} to precisely one \concept{edge} in the \concept{matching}; then, constraints \eqnnref{matching-c1} become \concept[equality]{equalities}. Notice that a \concept{perfect matching} may not exist (consider e.g. a \concept{triangle}), whereas there is always a \concept{feasible solution} to \eqnrefr{matching-m}{matching-c2}.

\paragraph{}
We emphasize that we distinguish between, on the one hand, the \concept{problem} itself, and, on the other hand, its \concept{formulation} as an \concept{integer program}. Indeed, these two are not the same! In fact, in some cases it is appropriate to show the \concept{correctness} of a \concept{formulation}. Such an \concept[correctness argument]{argument} is usually based on a correspondence between \concepts{feasible solution} to the problem, and vectors of \concepts{decision variable} satisfying the \concepts{constraint}.

\paragraph{}
Let us illustrate this matter for the \concept{matching problem}. Notice that there is a 1-1 correspondence between \concepts{subset} of the set of \concepts{edge} $E$ and the 0-1 \concepts{vector} defined by the \concepts{variable}, which are indexed by the \concepts{edge}. To prove that this \concept{formulation} is correct we must show that there is a 1-1 correspondence between the \concepts{subset} of the \concepts{edge} that define \concepts{matching} and the \concepts{feasible solution} of the above \concept{formulation}. Moreover, we must show that the \concept{matching} and its corresponding \concept{vector} have the same value. This can be done as follows. First, consider an arbitrary \concept{matching} $M$ in a \concept{graph}. By definition, this implies that each \concept{node} $v\in V$ of the \concept{graph} is \concept{incident} with at most one \concept{edge} of $M$ . Let us now construct a \concept{solution vector} $\vec{x}_M$ in a straightforward manner: we put a ``$1$'' in $\vec{x}_M$ when the corresponding \concept{edge} is in $M$, and otherwise we put a ``$0$'' in the \concept{vector} $\vec{x}_M$. Clearly, $\vec{x}_M$ is a \concept{0-1 vector}, and obviously satisfies \eqnref{matching-c2}. Also, the \concept{solution vector} $\vec{x}_M$ corresponding to $M$ satisfies the \concepts{constraint} \eqnref{matching-c1} since at most one of the \concepts{variable} in the left-hand side has value $1$. Thus, a \concept{matching} $M$ corresponds to a \concept{feasible solution} of the \concept{integer program}. Second, consider a \concept{subset} of the \concepts{edge} $E'$ that is \textbf{not} a \concept{matching}. Then there is a \concept{vertex}, say $v$, that is \concept{incident} to at least two \concepts{edge} in $E'$. But then, at least two of the \concepts{variable} in the left-hand side of \eqnref{matching-c1} have value $1$ for this particular \concept{vertex}. Thus, the \concept{vector} corresponding to $E'$ is not \concept{feasible} in the \concept{formulation}. Finally, we observe that the value of each \concept{set} $E'\subseteq E$ is equal to its number of \concepts{edge}, i.e., $\abs{E'}=\isumdomain[e]{E'}{1}=\isumdomain[e]{E}{x_e}$.

\paragraph{}
In general, it may not be trivial to prove the 1-1 correspondence of \concepts{feasible solution} of a \concept{combinatorial problem} to \concepts{solution} of its \concept{formulation}, i.e., solutions satisfying the \concepts{constraint}. For many problems, however, a \concept{correctness proof} is omitted because the problem is an (extension of) a well-known problem, and the correctness of a formulation is evident.

\subsection{The Independent Set Problem}
\ssclab{independentset}

Another basic problem within the field of combinatorial optimization is the \concept{Independent set} problem (also known as \conceptsee{Stable set}{independent set}, or as \conceptsee{Node packing}{Independent set}). It can be described as follows.

\begin{definition}[Independent set problem]
Consider an arbitrary \concept{graph} $G=\tupl{V,E}$. A \concept{subset} of the \concept[vertex]{vertices} $V'\subseteq V$ is called an \concept{independent set} (or a \concept{stable set}, or a \concept{node packing}) if each \concept{edge} of $E$ is \concept{incident} to at most one \concept{vertex} of $V'$. In other words, $V'$ is an \concept{independent set} if no two \concept[vertex]{vertices} of an \concept{edge} are selected both. The problem is to find an \concept{independent set} in $G$ containing as many \concept[vertex]{vertices} as possible (a \concept{maximum cardinality independent set}).
\end{definition}

The problem can be formulated as an \concept{integer program} as follows. We define a 0-1 \concept{variable} for each \concept{vertex} $v\in V$ as follows:
\begin{equation}
\semboolvar{x_v}{if vertex $v$ is selected in the independent set;}{otherwise.}
\end{equation}
And here is the \concept{integer program}:
\begin{eqnarray}
\mbox{maximize}&\sumdomain[v]{V}{x_v}\eqnlab{stableset-m}\\
\mbox{subject to}&\forall\tupl{v_1,v_2}\in E:x_{v_1}+x_{v_2}\leq 1\eqnlab{stableset-c1}\\
&\forall v\in V:x_v\in\accl{0,1}\eqnlab{stableset-c2}
\end{eqnarray}

\importtikzfigure{stableset-red}{A stable set.}

\subsection{The Spanning Forest Problem}
\ssclab{spanningforest}

We start with a definition for the \concept{spanning forest problem}:

\begin{definition}[Spanning forest problem]
Consider an \concept{undirected graph} $G=\tupl{V,E}$. A \concept{subset} of the \concepts{edge} $E'\subseteq E$ is called a \concept{forest} if the \concept{subgraph} of $G$ induced by $E'$ is \concept[acyclic graph]{acyclic}, see \figref{forest-red} for an example. In case a \concept{forest} consists of $\abs{V}-1$ edges it is called a \concept{tree}. The problem is to find a \concept{maximum weight forest} in $G$.
\end{definition}

The problem to find a \concept{maximum weight forest} can be formulated as an \concept{integer program} as follows. We define a 0-1 \concept{variable} for each \concept{edge} $e\in E$ as follows:
\begin{equation}
\semboolvar{x_e}{if edge $e$ is selected in the forest;}{otherwise.}
\end{equation}
And here is the \concept{integer program}:
\begin{eqnarray}
\mbox{maximize}&\sumdomain[e]{E}{w_e\cdot x_e}\eqnlab{forest-m}\\
\mbox{subject to}&\forall V'\subseteq V:2\leq\abs{V'}\leq\abs{V}:\sumdomain[e]{\fun{\delta}{v_1}\cap\fun{\delta}{v_2},v_1,v_2\in V}{x_e}\leq\abs{V'}-1\eqnlab{forest-c1}\\
&\forall e\in E:x_e\in\accl{0,1}\eqnlab{forest-c2}
\end{eqnarray}

\importtikzfigure{forest-red}{The edges in red form a forest.}
\paragraph{}The \eqncsref{forest-c1} limit the number of chosen \concepts{edge} with both \concept[endvertex]{endvertices} in any given \concept{subset} $V'$ of the \concepts{node} to at most $\abs{V'}-1$. Since a \concept{cycle} contains as many \concepts{edge} as \concepts{vertice}, the \eqncsref{forest-c1} prevent the graph $\tupl{V,E'}$ from containing \concepts{cycle}.
\paragraph{}
Notice that the number of \concepts{subset} of $V$ of size $2$ or bigger is $2^{\abs{V}}-\abs{V}-1$. Thus, the number of \eqnref{forest-c1} is exponential in the size of the problem. Thus, the size of the formulation is exponential in the size of the input, although the problem is trivially solvable by a \concept{greedy algorithm}.
\paragraph{}
The \concept{spanning tree} problem is a variant of the \concept{spanning forest} problem, where the set of feasible solutions is restricted to those \concept[acyclic graph]{acyclic subgraphs} that are \concept[connected graph]{connected}. It is well-known that for an \concept[acyclic graph]{acyclic subgraph} the requirement of \concept[connected graph]{connectedness} is equivalent to the requirement of having $\abs{V}-1$ \concepts{edge}, i.e., $\abs{E'}=\abs{V}-1$. Thus, by adding the constraint $\isumdomain[e]{E}{x_e}=n-1$ to \eqnnrefr{forest-c1}{forest-c2}, a correct \concept{formulation} of the \concept{minimum weight spanning tree} problem arises.

\subsection{The Knapsack Problem}
\ssclab{knapsack}

We start this section with the definition of the \concept{knapsack problem}:

\begin{definition}[Knapsack problem]
We are given a set of $n$ \concepts{item}, each with a \concept{weight} $a_j$ and a \concept{value} $c_j$ for $\rangei[j]{1}{n}$. \concepts{Feasible solution} are the \concepts{subset} of the set of \concepts{item} with \concept{cumulative weight} at most $b$. The objective is to find a \concept{feasible solution} of maximum value.
\end{definition}

The problem \concept{formulation} contains \concept{binary variable} $x_j$ which indicate whether \concept{item} $j$ with $\rangei[j]{1}{n}$ is in the \concept{knapsack}:

\begin{equation}
\semboolvar{x_j}{if item $j$ is selected in the knapsack;}{otherwise.}
\end{equation}

And here is the \concept{integer program}:

\begin{eqnarray}
\mbox{maximize}&\sumieqb[j]{1}{n}{c_j\cdot x_j}\eqnlab{knapsack-m}\\
\mbox{subject to}&\sumieqb[j]{1}{n}{a_j\cdot x_j}\leq b\eqnlab{knapsack-c1}\\
&\forall\rangei[j]{1}{n}:x_j\in\accl{0,1}\eqnlab{knapsack-c2}
\end{eqnarray}
In this text, we will come back extensively to the \concept{knapsack problem}. There is a book devoted to this problem, see Kellerer et al.\cite{KelPfePis04}.

\section{Formulations and difficulty}
\ssclab{difficulty}
Does the \concept{formulation} of a problem tell us anything about the problem's difficulty? The answer is no, it doesn't. Consider for instance the \concept{matching} problem (\sscref{matching}) and the \concept{stable set} problem (\sscref{independentset}). These problems look similar, since the roles of the \concepts{edge} and \concept[vertex]{vertices} are interchanged. However, there is a striking difference between them. The \concept{matching problem} can be solved in \concept{polynomial time}, but the \concept{node packing problem} is \concept{NP-hard} (see \chpref{complexity}). In other words, whereas, for the \concept{matching problem}, fast and efficient \concepts{algorithm} exist, and have been designed, no such \concepts{algorithm} have been found for the \concept{node packing problem}. Indeed, this does not rule out the possibility that fast \concepts{algorithm} could exist for \concept{node packing}, however, no one has ever found such an \concept{algorithm}. In fact, it is widely suspected that such \concepts{algorithm} do not exist, but a proof of this hypothesis is lacking. All this boils down to the famous, 1 million-dollar worth, P=NP question. In practice, this means that we can find an \concept{optimal solution} to a \concept{matching problem} on a \concept{graph} with, say 10.000 \concepts{node} and 50.000 \concepts{edge} within seconds, while there is no \concept{algorithm} known that would return an \concept{optimal solution} with one hour for the \concept{stable set} instance on the same \concept{graph}. Therefore: a \concept{formulation} does not give an indication of the solvability of a problem. Also, the number of variables and/or constraints is no clue concerning the difficulty of a problem. For instance, the ``natural'' \concept{formulation} of the \concept{minimum spanning tree problem} (see \sscref{spanningforest}) has an exponential number of constraints, while the problem is easily solvable by a \concept{greedy algorithm}.

\section{Multiple formulations of a combinatorial optimization problem}
\seclab{multipleformulations}

\subsection{Traveling Salesman Problem}
\ssclab{tsp}
Probably the most well-known problem in \concept{combinatorial optimization} is the \concept{traveling salesman problem} (\concept{TSP}). The \concept{TSP} is the prototype \concept{combinatorial optimization problem}. No other problem has received as much attention as the \concept{TSP}, and no other problem has captured the imagination as an easy-to-describe, yet hard-to-solve problem. A description of the problem is as follows.

\begin{definition}[Traveling salesman problem]
Given is a set of $n$ ``\concept[city]{cities}'' and a \concept{distance} $c_{i,j}$ between each pair of them. The goal is to find a \concept{tour} of \concept{minimum length}, that is to start in some \concept{city}, visit each other \concept{city} once, and to return to the \concept{city} where the \concept{tour} was started. More formally, given an $n\times n$ matrix $C=c_{i,j}$, find a permutation $\pi$ of $\accl{1,2,\ldots,n}$ such that $c_{\fun{\pi}{n},\fun{\pi}{1}}+\isumieqb[i]{1}{n-1}{c_{\fun{\pi}{i},\fun{\pi}{i+1}}}$ is minimum.
\end{definition}

Different \concepts{formulation} of the \concept{TSP} exist. Here is a conventional one, using \concepts{binary variable} $x_{i,j}$ indicating whether \concept{city} $j$ is visited directly after \concept{city} $i$:

\begin{eqnarray}
\mbox{minimize}&\sumieqb[i]{1}{n}{\sumieqb[j]{1}{n}{c_{i,j}\cdot x_{i,j}}}\eqnlab{tsp-m}\\
\mbox{subject to}&\forall\rangei[j]{1}{n}:\sumieqb[i]{1}{n}{x_{i,j}}=1\eqnlab{tsp-c1}\\
&\forall\rangei[i]{1}{n}:\sumieqb[j]{1}{n}{x_{i,j}}=1\eqnlab{tsp-c2}\\
&\forall S\subsetneq V:2\leq\abs{S}:\sumdomain[i]{S}{\sumdomain[j]{S}{x_{i,j}}}\leq\abs{S}-1\eqnlab{tsp-c3}\\
&\forall\rangei[i,j]{1}{n}:x_{i,j}\in\accl{0,1}\eqnlab{tsp-c4}
\end{eqnarray}

\paragraph{}
\eqncsref{tsp-c3} are called the \concept{subtour elimination constraints}. An equivalent way of writing them is:
\begin{equation}
\forall S\subsetneq V:\abs{S}\geq 2:\sumdomain[i]{S}{\sumndomain[j]{S}{x_{i,j}}}\geq 1
\end{equation}

\paragraph{}
Notice that this formulation has a polynomial number of \concepts{variable} ($n^2$), and an exponential number of \concepts{constraint} (\bigoh{2^n}). The latter fact might be considered a disadvantage of \concept{formulation} \eqnnrefr{tsp-m}{tsp-c4}.

\paragraph{}
There is, however, an alternative for this formulation which uses additional \concepts{real variable} $u_i$ , one such \concept{variable} for each \concept{city} $i$. The interpretation of this \concept{variable} is the \concept{position} of \concept{city} $i$ in the \concept{tour}, while putting the \concept{position} of city $1$ first. Next, by replacing \eqncsref{tsp-c3} by the following constraints:

\begin{eqnarray}
\forall \rangei[i,j]{2}{n}:i\neq j:u_i-u_j+n\cdot x_{i,j}\leq n-1\eqnlab{mtz-c1}\\
u_1=1\eqnlab{mtz-c2}
\end{eqnarray}

we arrive at a formulation that is called the \concept{Miller-Tucker-Zemlin (MTZ) formulation} of the TSP. It is an interesting exercise to verify the \concept{correctness} of the \concept{MTZ-formulation}. More concrete: why is it that \eqnnref{mtz-c1} exclude \concepts{subtour}? The answer lies in noticing that: any solution satisfying \eqnnref{tsp-c1}, \eqnnref{tsp-c2}, and \eqnnref{tsp-c4} consists of a collection of \concepts{subtour} (or a \concept{feasible solution}, that is, a single \concept{tour}). In case there are \concepts{subtour}, then there is a \concept{subtour} that does not contain \concept{city} $1$. Let us now, for each pair of consecutive \concept[city]{cities} $i$ and $j$ in this \concept{subtour}, sum the corresponding \concept[inequality]{inequalities} \eqnnref{mtz-c1}. The $u$-variables will cancel out, and the resulting lefthand side will be larger than the resulting righthand side, which means that this \concept{subtour} will be forbidden by \eqnnref{mtz-c1}. Thus, any \concept{subtour} not containing city $1$ will be forbidden, and hence, the only possible solution is a single \concept{tour}.

\paragraph{}
There is more than one book devoted to the TSP. We mention \emph{Applegate et al.}\cite{Applegate.2006} and \emph{Lawler et al.}\cite{Lawler85}.

\paragraph{}
We will end this section with a problem for which we have two natural \concepts{formulation}. Both \concepts{formulation} have the same sets of \concepts{variable}, but they have different sets of \concepts{constraint}. Later we will see different \concepts{formulation} of problems where the sets of \concepts{variable} differ.

\subsection{The Uncapacitated Facility Location Problem}
\ssclab{facilitylocation}

We start with defining the \concept{Uncapacitated facility location problem}:

\begin{definition}[Uncapacitated facility location problem]
We are given a set of $m$ \concept[facility]{facilities} and $n$ \concepts{client}. Let us call the set of \concept[facility]{facilities} $M\equiv\accl{1,2,\ldots,m}$, and let us call the set of \concepts{client} $N\equiv\accl{1,2,\ldots,n}$. Each of the \concept[facility]{facilities} can (but does not need to be) be opened to serve \concepts{client}. Each \concept{client} must be served by a \concept{facility}. The \concept{cost} for opening \concept{facility} $i$ is $f_i$, $i\in M$; the cost for serving \concept{client} $j$ by \concept{facility} $i$ is $c_{i,j}$, $i\in M$, $j\in N$. The aim is to minimize the total costs, but serve every \concept{client}.
\end{definition}

This problem can be formulated in two ways with the following sets of \concepts{variable}, defined for each $i\in M$, $j\in N$:

\begin{eqnarray}
\semboolvar{x_{i,j}}{if facility $j$ serves client $j$;}{otherwise.}\\
\semboolvar{y_i}{if facility $i$ is open;}{otherwise.}
\end{eqnarray}

\paragraph{}
The first \concept{formulation} makes use of the fact that there is an \concept{upper bound} on the number of \concepts{client} that are served by a \concept{facility}, namely the total number of \concepts{client} $n$.

\begin{eqnarray}
\mbox{minimize}&\sumdomain[i]{M}{\brak{f_i\cdot y_i+\sumdomain[j]{N}{c_{i,j}\cdot x_{i,j}}}}\eqnlab{ufl-ma}\\
\mbox{subject to}&\forall j\in N:\sumdomain[i]{M}{x_{i,j}}=1\eqnlab{ufl-ca1}\\
&\forall i\in M:\sumdomain[j]{N}{x_{i,j}}\leq n\cdot y_i\eqnlab{ufl-ca2}\\
&\forall i\in M,j\in N:x_{i,j},y_i\in\accl{0,1}\eqnlab{ufl-ca3}
\end{eqnarray}

\paragraph{}
In the second \concept{formulation}, each of the \eqncsref{ufl-ca2} is disaggregated into $n$ new \concepts{constraint}, leading to \concepts{constraint} \eqncsref{ufl-cb2}.

\begin{eqnarray}
\mbox{minimize}&\sumdomain[i]{M}{\brak{f_i\cdot y_i+\sumdomain[j]{N}{c_{i,j}\cdot x_{i,j}}}}\eqnlab{ufl-mb}\\
\mbox{subject to}&\forall j\in N:\sumdomain[i]{M}{x_{i,j}}=1\eqnlab{ufl-cb1}\\
&\forall i\in M,j\in N:x_{i,j}\leq y_i\eqnlab{ufl-cb2}\\
&\forall i\in M,j\in N:x_{i,j},y_i\in\accl{0,1}\eqnlab{ufl-cb3}
\end{eqnarray}

Which of these \concepts{formulation} is preferable is not a matter of comparing the number of \concepts{constraint} or \concepts{variable}. In fact, as will be shown in the sequel, large \concepts{formulation} with many \concepts{constraint} and/or \concepts{variable} are usually better from a \concept{solver}'s point of view. This depends on the techniques that are used to solve the problem. Since these techniques very often rely heavily on \concept{linear programming}, the quality of the \concept{formulation} depends almost always on the accuracy of the \concept{linear programming relaxation}, i.e., the problem which results when the \concepts{integrality constraint} are removed. For the \concept{uncapacitated facility location} problem \concept{formulation} \eqnnrefr{ufl-mb}{ufl-cb3} is better than \eqnnrefr{ufl-ma}{ufl-ca3}, since the \concepts{constraint} \eqncsref{ufl-cb2} imply the \concepts{constraint} \eqncsref{ufl-ca2}.

\section{Combinatorial Optimization: a general formulation}
\seclab{generalform}
In this section we (informally) argue that each combinatorial optimization problem can be formulated as an integer program. In a combinatorial optimization problem a finite ground set $E$ is given. To each
element $e\in E$ a weight we is attached. A family $\calS$ of subsets of $E$ is identified as the set of feasible solutions. This family depends on the particular problem. The weight of a set $E'\subseteq E$ is the cumulative weight of its elements, i.e., $\fun{w}{E'}=\isumdomain[e]{E'}{w_e}$. The associated optimization problem is to find the maximum (or minimum) weight feasible solution $E'\in S$, i.e.,

\begin{equation}
\displaystyle\max_{E'\in\calS}\accl{\fun{w}{E'}}
\end{equation}

\paragraph{}
The set of feasible solutions $\calS$ is usually given implicitly. It is described by the properties of feasible solutions; it may be very large. For instance, in case of the matching problem, the ground set equals the set of edges, and the set $\calS$ is the collection of edge-sets that are matchings.In case of the knapsack problem, the ground set equals the set of items, and the set $\calS$ equals the collection of item-sets that can be put together in the knapsack.

\paragraph{}
Many examples of problems that fit in the above formulation are found in graph theory (\secref{formulations}). Among them are well-known problems like the shortest path problem, the minimum spanning tree problem, and the traveling salesman problem. The shortest path problem is defined as follows. In a graph $G=\tupl{V,E}$ the feasible solutions are the subsets of the edges that form paths between two specified vertices $s$ and $t$. Among the paths between $s$ and $t$ we want to find the one with a minimum number of edges, or if a length function is given on the edges, we want to find a path of minimum total length. In the \concept{minimum spanning tree} problem a graph $G=\tupl{V,E}$ is given together with a weight function on the edges. A feasible solution is a set of edges that forms a tree. Among the trees we want to find one with minimum weight. This problem is easily solvable by a greedy algorithm, as is well known. However, if we restrict the set of feasible solutions to trees that form paths, the problem becomes the \concept{Hamiltonian path} problem, which is highly intractable. Thus, in general, problems do not become simpler when the set of feasible solutions is reduced.

\paragraph{}
To formulate a combinatorial optimization problem in mathematical terms, we introduce \concepts{decision variable} for all elements of the ground set $E$. Each decision variable denotes a choice, namely whether the corresponding element is chosen or not. So, a variable can have two values, which are usually taken from $\accl{0,1}$. A set $E'\subseteq E$ can be described by a binary vector $\vec{x}_{E'}=\brak{x_e}_{e\in E}$ with $n=\abs{E}$ components as follows:

\begin{equation}
\semboolvar{x_e}{if element $e$ is in $E'$;}{otherwise.}
\end{equation}

\paragraph{}
The finite set of vectors $X\subseteq\RRR^n$ corresponding to feasible solutions from $\calS$ can then be described by means of \concepts{constraint}. In many cases, these constraints are \concept[linear constraint]{linear} and involve binary variables. The objective function $\vec{w}$ is usually a linear function of the components of $\vec{x}\in X$. The problem is then

\begin{equation}
\max\condset{\vec{w}\cdot\vec{x}}{A\cdot\vec{x}\leq\vec{b}\wedge \vec{x}\in\accl{0,1}^n}
\end{equation}
where $A$ and $\vec{b}$ depend on the problem at hand. This type of formulation is often called an \concept[Integer linear programming]{Integer (or Binary) Linear Program}, (ILP).

\section{General Combinatorial Optimization Problems}
\seclab{generalcombinatorial}

In the first section, we introduced binary variables to model decisions of the yes-no type, more specifically, to decide whether an element is in a set or not. These variables were used further to describe the constraints that determine the feasible solutions. In all the examples these constraints could be written as linear functions with a bound imposed on them. Similarly, the objective could be formulated as a linear function of the variables. In this section we will generalize the formulations of combinatorial problems, with respect to all three items, i.e., the variables, the constraints, and the objective.

\paragraph{}
The decisions in combinatorial problems may be more complicated than simple yes/no decisions. One may have to introduce integer variables or even real variables, like in linear programming, to model certain decisions. And, of course, combinations of these types of variables are possible in a single problem formulation. The combinatorial nature (finite or countable number of feasible solutions) may not be so evident in these problems. However, the number of ``interesting'' feasible solutions is usually still finite in such problems. For instance, in linear programming, the interesting feasible solutions are the vertices
of a polyhedron. The number of vertices is usually finite.

\paragraph{}
One may consider any function of the variables with a bound imposed on it as a constraint. Moreover, logical compositions of constraints, like implications and disjunctions (logical ``or'') can be used to model
the restrictions of a problem. In abstracto, any relation on the variables that restricts the set of feasible solutions can be used as a constraint.

\paragraph{}
The objective of a problem can be any function of the variables, thus it is not restricted to linear functions.

\paragraph{}
The above observations lead to the following \concept{abstract formulation of optimization problems}. We distinguish between the real variables, denoted by the vector $\vec{x}$, and the integral variables, denoted by the
vector $\vec{y}$. A formulation of a combinatorial optimization problem contains the following items:
\begin{itemize}
 \item a vector of $n$ real decision variables:
 \begin{equation}
  \vec{x}=\tupl{x_1,x_2,\ldots,x_n}
 \end{equation}
 \item a vector of $m$ integral decision variables:
 \begin{equation}
  \vec{y}=\tupl{y_1,y_2,\ldots,y_m}
 \end{equation}
 \item a set $C$ of $k$ constraints:
 \begin{equation}
  C=\accl{C_1,\ldots,C_k}
 \end{equation}
 \item an objective function on the variables:
 \begin{equation}
  \fun{f}{\vec{x},\vec{y}}
 \end{equation}
\end{itemize}

\paragraph{}
The set of \concepts{feasible solution} consists of vectors $\tupl{\vec{x},\vec{y}}$ that satisfy all the constraints. Each variable has a domain $D$, usually the set of real numbers $\RRR$, or the set of integer $\ZZZ$. The solution space of the problem is the Cartesian product of the domains of the variables, i.e., $\RRR^n\times\ZZZ^m$.

\paragraph{}
For many solution techniques, especially the ones that we are going to discuss, it is of eminent importance to restrict the domains of the variables as much as possible. Some constraints imply \concept[lower bound]{lower} and \concepts{upper bound} on the value of a variable, directly or indirectly. For instance, binary variables have an explicit lower and upper bound. If this is the case, we usually take this into account in the problem formulation explicitly, i.e., if a real variable $x_i$ has a lower bound $l_i$ and an upper bound $u_i$ , then we describe its domain as $\fbrk{l_i,u_i}$; if $y_j$ is an integral variable, with a lower bound $l_j$ and an upper bound $u_j$ , then we describe its domain as $\accl{l_j,\ldots,u_j}$.

\section*{Exercises}
\begin{exercise}
Consider the stable set problem on an undirected graph $G=\tupl{V,E}$. Show that the formulation \eqnnrefr{stableset-m}{stableset-c2} is a correct formulation of the stable set problem.
\end{exercise}
\begin{exercise}
Consider the matching depicted in \figref{matching-red}. Is it maximal? Is it maximum?
\end{exercise}
\begin{exercise}
Consider the stable set depicted in \figref{stableset-red}. Is it maximal? Is it maximum?
\end{exercise}
\begin{exercise}
Consider an undirected graph $G=\tupl{V,E}$. An \concept{edge cover} $E'$ is a subset of the edges such that each node is incident to at least one edge in $E'$. Formulate the problem of finding a minimum cardinality edge cover as an integer linear program.
\end{exercise}
\begin{exercise}
Consider an undirected graph $G=\tupl{V,E}$. A \concept{node cover} $V'$ is a subset of the nodes such that each edge is incident to at least one node in $V'$. Formulate the problem of finding a minimum cardinality node cover as an integer linear program.
\end{exercise}
\begin{exercise}
\exclab{clique-partitioning}
A \concept{clique partitioning} of an undirected, complete graph $G$ is a partitioning of the vertices into subsets $V_1,V_2,\ldots,V_k$ such that the subgraph induced by each $V_i$ is a complete graph itself ($\rangei[i]{1}{k}$). Consider now a complete graph $G$ and an arbitrary weight $w_{i,j}$ for each edge of the graph (notice that $w_{i,j}$ can be negative; in fact, the problem is only interesting when there are both positive and negative edge weights). The objective is to find a clique partitioning of maximal weight, that is, a clique partitioning such that the cumulative weight of the edges that have both vertices in one and the same component is maximum. Formulate this problem as an integer linear programming problem.
\begin{hint}
Use a variable for each edge.
\end{hint}
\end{exercise}
\begin{exercise}
Consider the TSP.
\begin{itemize}
 \item When the distance matrix $C$ is known to be symmetric, can you simplify formulation \eqnnrefr{tsp-m}{tsp-c4} by ``merging'' \eqncsref{tsp-c1} and \eqnnref{tsp-c2}?
 \item Which of the two formulations given in \sscref{tsp} is stronger?
\end{itemize}
\end{exercise}
\begin{exercise}\exclab{rectanglestabbing}
Given is a set of axis-aligned rectangles in the plane. Each of these rectangles needs to be stabbed, either by a horizontal line or by a vertical line. The problem is to find a set horizontal and vertical lines, stabbing each rectangle at least once, with minimum cardinality. We call this the \concept{rectangle stabbing} problem. Formulate this problem as an integer linear programming problem.
\begin{hint}
First, give a formulation for the instance depicted in \figref{rectanglestabbing}).
\end{hint}
\end{exercise}
\begin{exercise}
Given an instance of the rectangle stabbing problem as depicted in \figref{rectanglestabbing}, what fractional solution is the optimal linear programming relaxation (of the integer program that you just wrote down) for this instance?
\end{exercise}

\importtikzfigure{rectanglestabbing}{A rectangle stabbing instance.}