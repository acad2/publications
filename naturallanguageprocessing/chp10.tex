\section{Discourse Analysis}
\begin{df}{Discourse analysis}
\sb{} aims to link a sequence of text units together such that the meaning becomes clear. It therefore uses for instance \flv{Entity based coherence} to find out to whom one is referring with ``she''. \flv{Rhetorical coherence} links text units together by rhetorical relations like ``because''. \sb{} thus aims to analyze the content spread over multiple sentences, paragraphs,...
\end{df}
\begin{df}{Coherence}
\sb{} is the relation in meaning between two textual units. In other words how these textual units can combined build a discourse meaning for the larger unit. Coherence comes in different flavors: \flv{Entity based coherence}, \flv{Rhetorical coherence},...
\end{df}
\begin{df}[Lexical coherence]{Entity based coherence}
\sb{} is a form of \flv{Coherence} by repeating terms, or using \flv{synonym}s and \flv{hypernym}s, the meaning becomes clear. This can be applied to noun-phrases, temporal references, spatial references and events. A problem with \sb{} is that people avoid using the same words. One uses references like ``he'' and ``she''. Determining what is meant by such references is called \flv{Coreference resolution}.
\end{df}
\begin{df}{Rhetorical coherence}
\sb{} is a form of \flv{Coherence} where certain aspects in the text unit link these text units together. For instance ``because'' acts as an indicator of \flv{Causality}.
\end{df}
\begin{df}{Cohesion}
\sb{} is the way textual units are linked together.
\end{df}
\begin{df}{Causality}
\sb{} is a relation between two facts stating that the second fact is true because the first fact was true. Causality also means the first fact took place earlier in time than the second one.
\end{df}
\begin{df}{Noun phrase coreferent resolution}
\sb{} is the combination of \flv{Coreference resolution} and \flv{Anaphor resolution}. The task is to detect which \flv{Noun phrase}s refer to the same item and to which item they are actually refer to. In most implementations the text is processed as one reads it (forwards) and references are searched backward while filtering non-referential pronouns.
\end{df}
\begin{df}{Coreference resolution}
\sb{} is a form of resolution where one aims to determine whether two \flv{Noun phrase}s refer to the same entity in the discourse. For instance ``Bill Clinton'' and ``The president''. Coreference resolution is governed by syntactic, semantic and discourse constraints.
\end{df}
\begin{df}{Anaphor resolution}
\sb{} is a form of resolution where one aims to identify an antecedent for an \flv{Anaphoric noun phrase}. This is often an \flv{anaphoric pronoun}.
\end{df}
\begin{df}{Pronoun resolution}

\end{df}
\begin{df}{Hobbs algorithm}
\sb{} is a syntactical algorithm for \flv{Pronoun resolution} to parse a discourse up to and including the current sentence. The algorithm is given a pronoun to be resolved and walks up the parse tree to the sentence (\postag{S}) level. For each \postag{NP} or \postag{S} node, breadth-first left-to-right search of the node's children is performed. If no candidates are found, the preceding sentence is attempted. A candidate is further verified by grammatical role constraints like gender, person and number. Gender constraints can be resolved by using ancestor information (with \flv{WordNet}) or a list of personal names with gender.
\end{df}
\begin{df}{Centering theory}
\sb{} states that the single entity is ``centered'' on at a given point of the discourse. In other words entities have a certain focus point in a discourse. Centering theory is thus a form of \flv{Entity based coherence}.
\end{df}
\begin{df}{Centering algorithm}
\sb{} is a syntactical algorithm that given a fully syntactically parsed text with gender, number and person information available. The algorithm looks for two adjacent utterances $U_i$ and $U_{i+1}$ first. And defines two centers: the \flvb{Backwards looking center} $\fun{c_b}{U_i}$ which is the \flv{entity} currently being focused on in the \flv{Discourse} after $U_i$ is interpreted; and the \flvb{Forward looking centers}\footnote{Notice the plural form.} $\fun{c_f}{U_i}$ which is an ordered list containing the entities mentioned in $U_i$. All the entities in $\fun{c_f}{U_i}$ could serve as the backwards looking center of $U_{i+1}$. $\fun{c_f}{U_i}$ is ordered based on a \flvb{Grammatical role hierarchy}: first \flv{Subject}s, then existential \flv{Predicate noun}s, objects, indirect objects or oblique and finally demarcated \flv{Adverbial possessive pronoun}s. $C_p=\argmax \fun{C_f}{U_i}$ is the highest ranked forward looking center.
\end{df}