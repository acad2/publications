\section{Discourse Analysis}
\begin{df}{Discourse analysis}
\sb{} aims to link a sequence of text units together such that the meaning becomes clear. It therefore uses for instance \flv{Entity based coherence} to find out to whom one is referring with ``she''. \flv{Rhetorical coherence} links text units together by rhetorical relations like ``because''. \sb{} thus aims to analyze the content spread over multiple sentences, paragraphs,...
\end{df}
\begin{df}{Coherence}
\sb{} is the relation in meaning between two textual units. In other words how these textual units can combined build a discourse meaning for the larger unit. Coherence comes in different flavors: \flv{Entity based coherence}, \flv{Rhetorical coherence},...
\end{df}
\begin{df}[Lexical coherence]{Entity based coherence}
\sb{} is a form of \flv{Coherence} by repeating terms, or using \flv{synonym}s and \flv{hypernym}s, the meaning becomes clear. This can be applied to noun-phrases, temporal references, spatial references and events. A problem with \sb{} is that people avoid using the same words. One uses references like ``he'' and ``she''. Determining what is meant by such references is called \flv{Coreference resolution}.
\end{df}
\begin{df}{Rhetorical coherence}
\sb{} is a form of \flv{Coherence} where certain aspects in the text unit link these text units together. For instance ``because'' acts as an indicator of \flv{Causality}.
\end{df}
\begin{df}{Cohesion}
\sb{} is the way textual units are linked together.
\end{df}
\begin{df}{Causality}
\sb{} is a relation between two facts stating that the second fact is true because the first fact was true. Causality also means the first fact took place earlier in time than the second one.
\end{df}
\begin{df}{Coreference resolution}

\end{df}
\begin{df}{Anaphor resolution}

\end{df}
